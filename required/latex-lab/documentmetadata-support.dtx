% \iffalse meta-comment
%
%% File: documentmetadata-support.dtx
% Copyright (C) 2021-2025 The LaTeX Project
%
% It may be distributed and/or modified under the conditions of the
% LaTeX Project Public License (LPPL), either version 1.3c of this
% license or (at your option) any later version.  The latest version
% of this license is in the file
%
%    https://www.latex-project.org/lppl.txt
%
%
% The development version of the bundle can be found below
%
%    https://github.com/latex3/latex2e
%
% for those people who are interested or want to report an issue.
%
%    \begin{macrocode}
\def\documentmetadatasupportversion{1.0q}
\def\documentmetadatasupportdate{2025-05-10}
%    \end{macrocode}
%
%
%<*driver>
\DocumentMetadata{tagging=on,pdfstandard=ua-2}
\documentclass[kernel]{l3doc}
\usepackage{latex-lab-testphase-l3doc}
\EnableCrossrefs
\CodelineIndex
\begin{document}
  \DocInput{documentmetadata-support.dtx}
\end{document}
%</driver>
%
% \fi
%
%
%\NewDocElement[macrolike = false ,
%		 toplevel  = true,
%                idxtype   =  key  ,
%                noprint   = true  ,
%                idxgroup  = metadata keys ,
%                printtype = \textit{key}
%               ]{Mkey}{mkey}
%
%
% \title{The \texttt{documentmetadata-support} code\thanks{This file
%      has version \documentmetadatasupportversion\ dated
%      \documentmetadatasupportdate,  \copyright\ \LaTeX\
%    Project.}}
% \author{Frank Mittelbach, Ulrike Fischer, \LaTeX{} Project}
%
% \maketitle
%
%
% \section{Introduction}
%
%    The kernel command \cs{DocumentMetadata}, which can be used as
%    the very first declaration in a document (i.e., before
%    \cs{documentclass}), defines metadata and other configuration
%    data that applies to the document as a whole (using a key/value
%    syntax).
%
%    While the underlying functionality is still under
%    development (e.g., further keys will be added over time and keys
%    marked temporary may vanish again) the code for
%    \cs{DocumentMetadata} is placed in a separate bundle, so that it
%    is easier to update it without the need to build a full \LaTeX{}
%    release. Over time the functionality will move fully into the
%    kernel.
%
%    \cs{DocumentMetadata} also loads and activates
%    the new PDF management code from \pkg{pdfmanagement-testphase}.
%    As this forces the loading of the \pkg{l3backend} files, a backend
%    that can't be detected automatically like |dvipdfmx|,
%    must be set in the first \cs{DocumentMetadata}.
%
%    From a process perspective \cs{DocumentMetadata} loads the
%    \pkg{pdfmanagement-testphase} code the first time it is called and then
%    redefines itself to only manage key/value pairs in case it is
%    called more than once. In particular, this means that a document
%    without a \cs{DocumentMetadata} declaration has no code available
%    for extended management of PDF output as needed for various
%    features developed as part of the multi-year ``Tagged PDF''
%    project~\cite{blueprint}.
%
% \section{The \cs{DocumentMetadata} command}
%
% \begin{function}{\DocumentMetadata}
% \begin{syntax}
% \cs{DocumentMetadata}\Arg{key-value list}
% \end{syntax}
%
% The command should be used as the first command in a document, before
% \cs{documentclass}. It takes a key-value argument.
% \end{function}

% \section{Currently supported key/values}
%
%
% Currently the following keys are implemented for \cs{DocumentMetadata}:
% \DescribeMkey{backend}
% \DescribeMkey{colorprofiles}
% \DescribeMkey{debug}
% \DescribeMkey{lang}
% \DescribeMkey{pdfstandard}
% \DescribeMkey{xmp}
% \DescribeMkey{pdfversion}
% \DescribeMkey{testphase}
% \DescribeMkey{uncompress}
%
% \begin{description}
%    \item[\texttt{backend}] Passes the backend name to expl3. This is
%      needed only if the needed backend can't be automatically
%      determined or if the workflow used requires a special backend.
%
%^^A    This will probably be extended to  pass the value also to
%^^A      packages.
%
%    \item[\texttt{pdfversion}] Sets the PDF version explicitly, e.g., \texttt{pdfversion=1.7}
%    \item[\texttt{uncompress}] (no value) Forces an uncompressed pdf
%      --- mainly for debugging purposes.
%
%    \item[\texttt{lang}] Explicitly sets the Lang entry in the Catalog,
%     e.g., \texttt{lang=de-DE}. If not given the default value used is |en-US|.
%
%    \item[\texttt{pdfstandard}] Choice key to set the pdf standard.
%      Currently |A-1b|, |A-2a|, |A-2b|, |A-2u|, |A-3a|, |A-3b|, |A-3u|, |A-4|, |A-4E|
%      and |A-4F| are accepted as A-standard.
%      values. The casing is irrelevant, |a-1b| works too. 
%      Note that using this
%      key doesn't mean that the document actually follows the standard. \LaTeX{}
%      can neither ensure nor check all requirements of a standard, and not everything
%      it can do theoretically has already been implemented.
%      When setting an A-standard a color profile is included and the
%      \texttt{/OutputIntent} is set and any javascript action in hyperref are suppressed.
%      The |u| variants do not enforce unicode,
%      but they will pass the information to hyperref. The |a| variants
%      do \emph{not} enforce (or even test) a tagged pdf yet.
%
%      Beside the A-standards it is also possible to use the values
%      |X-4|, |X-4p|, |X-5g|, |X-5n|, |X-5pg|, |X-6|, |X-6n|, |X-6p| for
%      a PDF/X and |UA-1|  and |UA-2| for PDF/UA standard.
%      |UA-2| should only be used together with PDF 2.0.
%      Currently these keys set \emph{only} the relevant XMP-metadata.
%      They do not validate or enforce special requirements
%      (e.g., the |UA| standards do not automatically activate tagging).
%
%      |pdfstandard| can be used more than once to set overlapping standards, e.g:\\
%      |pdfstandard=A-2b,pdfstandard=X-4,pdfstandard=UA-1|. It is also possible
%      to pass a list like |pdfstandard={UA-2,A-4F}|.
%
%      If XMP-metadata are added (see the following key \texttt{xmp})
%      the necessary conformance markers for the standards are set.
%
%      More information can be found in the documentation
%      of \pkg{l3pdfmeta}.
%
%    \item[\texttt{xmp}] A boolean, if set to false no XMP metadata are added to the PDF.
%    The initial value is true. Details are described in the documentation of \pkg{l3pdfmeta}.
%
%    \item[\texttt{colorprofiles}] This allows to load icc-colorprofiles. Details
%       are described in the documentation of \pkg{l3pdfmeta}.
%       
%    \item[\texttt{tagging}] This key allows to activate, deactivate or partially deactive the tagging support.
%    It accepts the three values \texttt{on}, \texttt{off} and \texttt{draft}. 
%    When used, the key loads the tagpdf package and all modules that are loaded with \texttt{testphase=latest}. 
%    |tagging=off| deactivates then in the \texttt{class/before} hook the tagging commands. 
%    |tagging=draft| leaves the
%    tagging commands active  but it deactivates the writing of the structure tree 
%    at the end of the compilation. This can save
%    time when drafting a longer document but preserves, e.g., MC-content marker 
%    in the PDF stream and warnings and errors from \pkg{tagpdf} if the structure is faulty.
%    
%    \item[\texttt{tagging-setup}] This key allows to configure the tagging. It accepts all
%    keys that can also be used in \cs{tagpdfsetup}; see the \pkg{tagpdf} documentation. 
%    Additionally,
%    it accepts two keys to change the list of testphase modules loaded:
%     \begin{description}
%        \item[\texttt{modules}] With this key it is possible to overwrite the list of
%        modules loaded with by the \texttt{tagging} key. 
%        It will always load the |phase-II| module, so |modules={block,title}| does the 
%        same as |testphase={phase-II,block,title}|. Using this key is generally not needed, but
%        if used the \texttt{tagging} key should only be used before the \texttt{tagging-setup}
%        key to avoid that it resets the list of modules to the \texttt{latest set} again.
%        
%        \item[\texttt{extra-modules}] This key allows to load modules that 
%        haven't been added to the \texttt{latest} set yet.        
%      \end{description} 
%    
%    \item[\texttt{testphase}] This key is used to load testphase
%    code.
%    With the introduction of the \texttt{tagging} and
%    \texttt{tagging-setup} keys it is no longer necessary for a user
%    to specify the \texttt{testphase} key; it is only retained for
%    internal testing and to support documents written during the
%    early days of the tagging project. 
%
%    The |testphase| key can only be used in the first \cs{DocumentMetadata}.
%    The values it accepts and their effect will change over time, when testphase packages are added or
%    removed or when the code is moved into the kernel. The key accepts a list of values
%    and it can be used more than once.
%
%    The value \texttt{latest} loads all modules that we recommend
%    so that it is not necessary to specify individual modules.
%
%
%    The \texttt{phase} keys bundle testphase modules. They also all activate tagging.
%     \begin{description}
%    \item[\texttt{phase-I}]
%       This value loads code implementing the first phase of the project~\cite{blueprint}, i.e., it
%       will load the tagpdf package. It will also activate tagging by issuing
%       |\tagpdfsetup{activate,activate/spaces}|. This phase
%       is frozen.
%    \item[\texttt{phase-II}]
%       It differs from \texttt{phase-I} only in one point: It will
%       additionally activate tagging of paragraphs with
%       |\tagpdfsetup{para/tagging}|. In the upcoming months it will
%       also enable automatic tagging of other basic document elements.
%    \item[\texttt{phase-III}]
%       This is the current development phase.
%       It differs from \texttt{phase-II} \emph{a lot}: It will
%       load new code for the tagging of lists, sectioning commands,
%       table of contents and similar lists, graphics, minipages and floats.
%       As it redefines many internals it is currently restricted to the use of
%       standard classes (article, report, and book)
%       and it supports only a limited number of add-on packages.
%   \end{description}
%
%   The various testphase modules can also be loaded individually (at least in theory,
%   there can be hidden dependencies). If loaded like this, the tagpdf package is not
%   loaded and tagging is not activated! The list of modules will change over time.
%   \begin{description}
%    \item[\texttt{new-or-1}] This patches a few commands related
%    to the output routine. The patches are needed for the tagging
%    of paragraphs, for the tagging of header and footer and
%    to allow the PDF management to insert code which avoids that
%    links happening at page breaks spills into the header and footer.
%    This code is automatically loaded if the \texttt{testphase} values
%    \texttt{phase-I}, \texttt{phase-II} or \texttt{new-or} are used.%
%    \item[\texttt{new-or}] This loads more changes to the output routine required for the
%    tagging. It is not compatible with every class! The code is also loaded by the
%    \texttt{phase-II} value.
%    \item[\texttt{sec}] This adapts commands related to sectioning to make them tagging aware.
%     The \texttt{sec} module is loaded by \texttt{phase-III}.
%    \item[\texttt{toc}] This adapts commands related to the table of contents and
%    similar list to make them tagging aware. The \texttt{toc} module is loaded by \texttt{phase-III}.
%    \item[\texttt{graphic}] This enables tagging support for
%    the \cs{includegraphics} command and the \texttt{picture} environment.
%    This code is also loaded by the \texttt{phase-III} key.
%    \item[\texttt{block}] This reimplements lists and blocks environments and
%    add tagging support.
%    This code is also loaded by the \texttt{phase-III} key.
%    \item[\texttt{minipage}] This adds tagging support to \texttt{minipage}
%    and \cs{parbox}.
%    This code is also loaded by the \texttt{phase-III} key.
%    \item[\texttt{float}] This adds tagging support to floats.
%    This code is also loaded by the \texttt{phase-III} key.
%    \item[\texttt{bib}] This adds tagging support to citations and bibliographies.
%    This code is also loaded by the \texttt{phase-III} key.
%    \item[\texttt{text}] This module adds tagging support to the \LaTeX{} logo and
%    to the \cs{emph} command.
%    This code is also loaded by the \texttt{phase-III} key.
%    \item[\texttt{marginpar}] This module adds tagging support to the \cs{marginpar}
%    command. This code is also loaded by the \texttt{phase-III} key.
%    \item[\texttt{title}] This module add tagging support to the \cs{maketitle}
%    command if a standard class is used. It also enhances the \cs{title}
%    and \cs{author} commands to fill the XMP-metadata and set the window title.
%    It is not compatible with packages and classes which redefine these commands
%    too. The module is currently not loaded by any \texttt{phase} key.
%    \item[\texttt{math}] This adapts math for tagging. This is only a prototype.
%    The module is currently not loaded by any \texttt{phase} key.
%    \item[\texttt{table}] This provides basic tagging for
%    \texttt{tabular}, \texttt{longtable} and similar table environments.
%    The module is currently not loaded by any \texttt{phase} key. Its use and
%    restrictions is documented in \texttt{latex-lab-table.pdf}.%
%    \item[\texttt{firstaid}] This contains small adjustments to external packages.
%    The module is currently not loaded by any \texttt{phase} key.
%    \end{description}
%
%
%    \item[\texttt{debug}] This key activates some debug options. It takes a list of key-values
%    as value. Currently the following keys are known:
%     \begin{description}
%     \item[\texttt{para}] with the default and only value |show|. It will activate the |paratagging-show|
%       option of \pkg{tagpdf},
%     \item[\texttt{log}]  with the values as described in the documentation \pkg{tagpdf},
%     \item[\texttt{uncompress}] which does the same as |uncompress| as main key
%     \item[\texttt{pdfmanagement}] a boolean which allows to deactivate the pdfmanagement.
%     \item[\texttt{firstaidoff}] This accepts a comma lists of keywords and disables the patches
%       related to them. More information can be found in the documentation of
%       \pkg{pdfmanagement-firstaid}.
%     \item[\texttt{xmp-export}] This will export the XMP-metadata to a file \verb+\jobname.xmpi+.
%     with \texttt{debug=\{xmp-export=filename\}} the file name can be changed.
%     More information can be found in the documentation of \pkg{l3pdfmeta} of the \pkg{pdfmanagement-testphase}
%     bundle.
%     \item[\texttt{tagpdf}] This loads the package tagpdf-debug which enhances
%     various commands from tagpdf with additional debugging options. This can slow down
%     the compilation!
%     \item[\texttt{BBox}] This helps to debug BBox values of graphics, see the documentation
%     of latex-lab-graphic. It does nothing if  the graphic testphase module is not loaded.
%    \end{description}
% \end{description}
%
% \bibliographystyle{plain}
%
% \begin{thebibliography}{9}
%
% \bibitem{blueprint} Frank Mittelbach and Chris Rowley:
%   \emph{\LaTeX{} Tagged PDF\,---\,A blueprint for a large project}.
%   \url{https://latex-project.org/publications/indexbyyear/2020/}
%
% \end{thebibliography}
%
%
%
% \MaybeStop{\setlength\IndexMin{200pt}  \PrintIndex  }
%
% \begin{implementation}
%
% \section{The Implementation}
%
%
%    \begin{macrocode}
%<@@=pdfmanagement>
%<*code>
%    \end{macrocode}
%
%    \begin{macrocode}

\RequirePackage{pdfmanagement-testphase}
%    \end{macrocode}
%
%    \begin{macrocode}
\ExplSyntaxOn\makeatletter
%    \end{macrocode}

% \subsection{Variables}
% These variable definitions are currently also done in ltdocinit.
% They can be removed from there once latex-lab has been updated to
% provide them too.
%  \begin{variable}{\g_@@_firstaidoff_clist}
%  A list to store the firstaid code which should be disabled
%    \begin{macrocode}
\clist_if_exist:NF \g_@@_firstaidoff_clist
  { \clist_new:N \g_@@_firstaidoff_clist }
%    \end{macrocode}
%  \end{variable}
%  \begin{variable}{\g_@@_testphase_tl}
%  a tl to store the testphase loading code so that we can load them at
%  the end of the command.
%    \begin{macrocode}
\tl_if_exist:NF \g_@@_testphase_tl
 { \tl_new:N \g_@@_testphase_tl }
%    \end{macrocode}
% \end{variable}
%
% \subsection{Kernel changes with \cs{DocumentMetadata}}
% \begin{macro}{\@kernel@before@DocumentMetadata}
%    \begin{macrocode}
\cs_new_protected:Npn \@kernel@before@DocumentMetadata
  {
    \ifx\Umathcode\@undefined
    \fontencoding{T1}
    \renewcommand\encodingdefault{T1}
    \fi
  }
%    \end{macrocode}
% \end{macro}

% \subsection{\cs{DocumentMetadata}}
%
%  \begin{macro}{\DocumentMetadata}
%
%    \cs{DocumentMetadata} should not be used after
%    \cs{documentclass} so we error in this case.
%    It can be used more than once
%     but follow-up calls should not do the initialization code.
%    \begin{macrocode}
\cs_set_protected:Npn \DocumentMetadata #1
  {
    \cs_if_eq:NNTF \documentclass \@twoclasseserror
      { \msg_error:nn { meta } { after-class } }
      {
%    \end{macrocode}
% Load general format changes
%    \begin{macrocode}
        \@kernel@before@DocumentMetadata
%    \end{macrocode}
%
%    The wanted backend must be detected first, we read the init
%    key and then force the loading of the backend.
%    The backend can contain management commands, so the boolean should
%    be set to true first.
%    \begin{macrocode}
        \bool_gset_true:N \g_@@_active_bool
        \keys_set_groups:nnn { document / metadata} {init}{ #1 }
        %if no backend has been loaded force it now:
        \str_if_exist:NF \c_sys_backend_str
          {
            \sys_load_backend:n {}
          }
%    \end{macrocode}
%    Now we load the extra backend code:
%    \begin{macrocode}
        \ExplSyntaxOn\makeatletter
          \file_input:n {l3backend-testphase-\c_sys_backend_str.def}
        \ExplSyntaxOff\makeatother
%    \end{macrocode}
% \changes{v1.0m}{2025/02/17}{Set PDF 2.0 as default}
%    \begin{macrocode}
          \pdf_version_gset:n { 2.0 }
%    \end{macrocode}
%    Process the non-init keys and setup the generic driver.
%    \begin{macrocode}
        \keys_set_exclude_groups:nnn  { document / metadata } { init } { #1 }
        \bool_if:NT \g_@@_active_bool
          {
            \PassOptionsToPackage{customdriver=hgeneric-testphase}{hyperref}
%    \end{macrocode}
% Finally we setup the language default.
% This is done after the begindocument hook so that it can pick up settings
% from babel. If the Catalog dictionary already contains a lang value we do nothing,
% otherwise we use the value stored in \cs{BCPdata}, either the main language
% (if its exists) or the fall back language.
% Note: if babel is loaded without a language this gives the language \texttt{und}.
%    \begin{macrocode}
            \g@addto@macro\@kernel@after@begindocument
             {
               \pdfdict_get:nnN {g__pdf_Core/Catalog}{Lang}\l_@@_tmpa_tl
               \quark_if_no_value:NT\l_@@_tmpa_tl
                {
                  \tl_if_empty:eTF { \BCPdata{main.language} }
                   { \tl_set:Ne \l_@@_tmpb_tl { \BCPdata{language} } }
                   { \tl_set:Ne \l_@@_tmpb_tl { \BCPdata{main.language} } }
                  \msg_warning:nne { meta } { lang-missing }{ \l_@@_tmpb_tl }
                  \exp_last_unbraced:Ne
                   \AddToDocumentProperties{[document]{lang}{\l_@@_tmpb_tl}}
                  \pdfmanagement_add:nne {Catalog} {Lang}{(\l_@@_tmpb_tl)}
                }
             }
          }
%    \end{macrocode}
%    \cs{pdfmanagement_add:nnn} has collected values in this hook.
%    \begin{macrocode}
        \hook_use_once:n {pdfmanagement/add}
%    \end{macrocode}
%    Now we redefine \cs{DocumentMetadata} so that it only process the
%    keys on any further calls.
%
%     We need to update the hyperref option if the active status changes.
%    \begin{macrocode}
        \cs_set_protected:Npn \DocumentMetadata ##1
          {
            \keys_set_exclude_groups:nnn  { document / metadata } { init } { ##1 }
            \str_remove_all:cn {opt@hyperref.sty}{customdriver=hgeneric-testphase}
            \bool_if:NT \g__pdfmanagement_active_bool
             {
               \PassOptionsToPackage{customdriver=hgeneric-testphase}{hyperref}
             }
          }
%    \end{macrocode}
% Load more modules, the testphase code and the firstaid code. The code is only
% loaded in the first \cs{DocumentMetadata} call!
%    \begin{macrocode}
        \g_@@_testphase_tl
        \RequirePackage{pdfmanagement-firstaid}
      }
  }
%    \end{macrocode}
%  \end{macro}

%\subsection{\cs{DocumentMetadata} keys}
%\changes{1.0k}{2024-12-17}{added latest value for testphase key.}
%\changes{1.0n}{2025/03/25}{Allow a list as pdfstandard value.}
%\changes{1.0n}{2025/03/25}{Add tagging key}
%\changes{1.0n}{2025/03/25}{Add tagging-setup}
%\changes{1.0o}{2025/04/03}{Add debug=BBox}
%\changes{1.0p}{2025/04/28}{Add tikz module to latest key}
%    \begin{macrocode}
\keys_define:nn { document / metadata }
  {
    backend .choices:nn =
      { dvipdfmx , dvips , dvisvgm , luatex , pdftex , pdfmode , xdvipdfmx , xetex }
      {
        \sys_load_backend:n {#1}
      },
    backend .groups:n = { init } ,
  }

\keys_define:nn { document / metadata }
  {
    ,pdfversion .code:n =
      {
        \pdf_version_gset:n { #1 }
        \AddToDocumentProperties[document]{pdfversion}{#1}
      }
    ,uncompress .code:n =
      {
        \pdf_uncompress:
      }
    ,uncompress .value_forbidden:n = true
    ,lang .code:n =
      {
        \pdfmanagement_add:nnn {Catalog} {Lang}{(#1)}
        \AddToDocumentProperties[document]{lang}{#1}
      }
    %,xmpmeta .bool_gset:N = \g_pdfmeta_xmp_bool %see pdfmeta unused and undefined for now!
    % this uses internal command from pdfmeta, it should probably move there ...
    ,pdfstandard .code:n =
      {
        \clist_map_inline:nn{#1}
         {           
           \keys_set:ne {document / metadata} {_pdfstandard=\str_uppercase:n{##1}}
         }  
      }
    ,_pdfstandard .choices:nn =
      {A-1B,A-2A,A-2B,A-2U,A-3A,A-3B,A-3U,A-4}
      {
        \prop_gset_eq:Nc \g__pdfmeta_standard_prop { g__pdfmeta_standard_pdf/#1 _prop }
        \AddToDocumentProperties [document]{pdfstandard}{#1}
      }
    ,_pdfstandard / A-4F .code:n =
      {
        \prop_gset_eq:Nc \g__pdfmeta_standard_prop { g__pdfmeta_standard_pdf/A-4F_prop }
        \AddToDocumentProperties [document]{pdfstandard}{A-4F}
      }
    ,_pdfstandard / A-4E .code:n =
      {
        \prop_gset_eq:Nc \g__pdfmeta_standard_prop { g__pdfmeta_standard_pdf/A-4E_prop }
        \AddToDocumentProperties [document]{pdfstandard}{A-4E}
      }
    ,_pdfstandard / unknown .code:n =
      {
        \msg_error:nnn{pdf}{unknown-standard}{#1}
      }
    ,testphase .multichoice:
    ,testphase / latest .code:n =
      {
       \keys_set:nn
        {document / metadata}
        {testphase={phase-III,title,table,math,firstaid,tikz}}
       \tl_gput_right:Nn\g_@@_testphase_tl 
        {
         \msg_note:nnn { meta } { testphase-latest-loading }
          { phase-III,~title,~table,~math,~firstaid,~tikz }
        }  
      }
    ,testphase / tagpdf .code:n =
      {
        \tl_gput_right:Nn\g_@@_testphase_tl
          {
            \file_if_exist_input:nF {tagpdf-latex-lab-testphase.ltx}
              {
                \RequirePackage{tagpdf}
                \AddToDocumentProperties [document]{testphase/tagpdf}{loaded}
                \tagpdfsetup{activate,para/tagging,activate/spaces}
                \AddToDocumentProperties [document]{tagging}{active}
                \AddToDocumentProperties [document]{tagging/para}{active}
                \AddToDocumentProperties [document]{tagging/interwordspace}{active}
              }
          }
      }
    ,testphase / unknown .code:n =
      {
        \tl_gput_right:Nn\g_@@_testphase_tl
           {
             \tl_if_empty:eT{\GetDocumentProperties{document/testphase/#1}}
              {
                \AddToDocumentProperties [document]{testphase/#1}{loaded}
                \file_if_exist_input:nF {#1-latex-lab-testphase.ltx}
                 {
                  \msg_warning:nnn{meta}{latex-lab-pkg-missing}{#1}
                  \AddToDocumentProperties [document]{testphase/#1}{missing}
                 }
              }   
           }
      }
    ,tagging .choice:,
    ,tagging / on .code:n = 
      {
        \keys_set:nn { document / metadata }{ testphase=latest }
      }  
    ,tagging / off .code:n = 
      {
        \keys_set:nn { document / metadata }{ testphase=latest }
        \AddToHook{class/before}
         {
           \tagpdfsetup{activate/mc=false,activate/tree=false,
                        activate/struct=false,activate/socket=false,
                        para/tagging=false,math/setup=}
           \AddToDocumentProperties [document]{tagging}{inactive}
           \AddToDocumentProperties [document]{tagging/para}{inactive}
           % perhaps, need to test
           %\tag_suspend:n {global}
           %\cs_set_eq:NN\tag_suspend:n\use_none:n
           %\cs_set_eq:NN\tag_resume:n \use_none:n
           %\cs_new_protected:Npn \tag_mc_begin:n #1 { \__tag_whatsits: \int_gincr:N \c@g__tag_MCID_abs_int }
           %\cs_new_protected:Nn \tag_mc_end:{ \__tag_whatsits: }
           %\cs_new_protected:Npn \tag_struct_begin:n #1 {\int_gincr:N \c@g__tag_struct_abs_int}
           %\cs_new_protected:Npn \tag_struct_end:{}
         }
      }  
    ,tagging / draft .code:n = 
      {
        \keys_set:nn { document / metadata }{ testphase=latest }
        \AddToHook{shipout/lastpage}[tagging-setup]{\tagpdfsetup{activate/tree=false}}
        \DeclareHookRule{shipout/lastpage}{tagging-setup}{before}{tagpdf} 
      }   
    ,tagging-setup .code:n = 
      {
        \tl_gclear:N\g_@@_testphase_tl
        \keys_set:nn { document / metadata }{ testphase=latest }  
        \keys_set_groups:nnn{__tag/setup}{load}{#1}
        \AddToHook{class/before}
         {\keys_set_exclude_groups:nnn {__tag/setup}{load}{#1}}
      }   
    ,debug .code:n =
      {
        \keys_set:nn { document / metadata / debug } {#1}
      }
    ,debug / para .code:n =
      {
        \AddToHook
          {
            package/tagpdf/after
          }
          {
             \tagpdfsetup{debug/show=para}
          }
      }
    ,debug / log .code:n =
      {
        \AddToHook
          {
           package/tagpdf/after
          }
          {
             \tagpdfsetup{debug/log=#1}
          }
      }
    ,debug / tagpdf .code:n =
      {
        \AddToHook
          {
           package/tagpdf/after
          }
          {
             \RequirePackage{tagpdf-debug}
          }
      }
    ,debug / uncompress .code:n =
      {
        \pdf_uncompress:
      }
    ,debug / pdfmanagement .bool_gset:N = \g_@@_active_bool
    ,debug / firstaidoff .clist_gset:N = \g_@@_firstaidoff_clist
    ,debug / BBox .code:n = 
      { 
        \AddToHook{package/latex-lab-testphase-graphic/after}
         { \bool_set_true:N \l__tag_graphic_debug_bool} 
      }    
  }
\keys_define:nn{__tag/setup}
 {
   ,modules .code:n =  
     {
      \tl_gclear:N\g_@@_testphase_tl 
      \keys_set:nn { document / metadata }{testphase={phase-II,#1}} 
     }
   ,modules .groups:n = { load }    
   ,extra-modules .code:n = 
     {
       \keys_set:nn { document / metadata }{testphase={#1}}    
     }
   ,extra-modules .groups:n = { load }  
 }                 
%    \end{macrocode}
% \subsection{Messages}
%    \begin{macrocode}
%UFi is meta the right module name here?
\prop_gput:Nnn \g_msg_module_type_prop { meta } { LaTeX }
\prop_gput:Nnn \g_msg_module_name_prop { meta } { DocumentMetadata }

\msg_new:nnn { meta } { after-class }
             {
                \token_to_str:N \DocumentMetadata \c_space_tl
                should~be~used~only~before~\token_to_str:N\documentclass
             }
\msg_new:nnn { meta } { latex-lab-pkg-missing }
             {
               LaTeX-lab~package~'#1'~not~found.
              }
\msg_new:nnn { meta } { lang-missing }
             {
               The~language~has~not~been~set~in~\token_to_str:N
               \DocumentMetadata.\\Setting~it~to~'#1'~as~fallback.
             }
\msg_new:nnn { meta } { testphase-latest-loading }
             {
               Loading~testphase~modules\\
               #1
             }
%    \end{macrocode}
%
%    \begin{macrocode}
\ExplSyntaxOff\makeatother
%    \end{macrocode}
%
%
%    \begin{macrocode}
%</code>
%    \end{macrocode}
%
% \end{implementation}
%
% \Finale
%
