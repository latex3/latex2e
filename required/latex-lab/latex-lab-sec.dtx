% \iffalse meta-comment
%
%% File: latex-lab-sec.dtx (C) Copyright 2022-2025 LaTeX Project
%
% It may be distributed and/or modified under the conditions of the
% LaTeX Project Public License (LPPL), either version 1.3c of this
% license or (at your option) any later version.  The latest version
% of this license is in the file
%
%    https://www.latex-project.org/lppl.txt
%
%
% The development version of the bundle can be found below
%
%    https://github.com/latex3/latex2e/required/latex-lab
%
% for those people who are interested or want to report an issue.
%
\def\ltlabsecdate{2025-02-18}
\def\ltlabsecversion{0.84i}
%<*driver>
\documentclass[kernel]{l3doc}
\EnableCrossrefs
\CodelineIndex
\begin{document}
  \DocInput{latex-lab-sec.dtx}
\end{document}
%</driver>
%
% \fi
%
%
% \title{The \textsf{latex-lab-sec} package\\
% Changes related to the tagging of sectioning commands}
% \author{\LaTeX{} Project\thanks{Initial implementation done by Ulrike Fischer}}
% \date{v\ltlabsecversion\ \ltlabsecdate}
%
% \maketitle
%
% \newcommand{\xt}[1]{\textsl{\textsf{#1}}}
% \newcommand{\TODO}[1]{\textbf{[TODO:} #1\textbf{]}}
% \newcommand{\docclass}{document class \marginpar{\raggedright document class
% customizations}}
%
% \providecommand\hook[1]{\texttt{#1}}
%
% \begin{abstract}
% The following code implements a first draft for the tagging of sectioning commands.
% \end{abstract}
%
% \section{Limitations}
%
% Sectioning commands are in not defined by the format but by the classes.
% Their implementation vary: some are defined with the help of \cs{@startsection},
% some are like \cs{chapter} handcrafted,
% some build with the help of extension packages or as in the KOMA classes
% with class code that extends the \cs{@startsection} functionality.
%
% The following code can therefore currently be used \emph{only} with the standard classes
% or with classes which do not overwrite the changed definitions.
%
%
%
% \section{Introduction}
%
%  Tagging of sectioning commands consist of two parts:
%
% \begin{itemize}
%  \item The heading/title text of the section should be surrounded by a
%  heading tag, typically \texttt{Hn} with some value of \texttt{n}.
%  In theory, one could put the number of the section command in an \texttt{Lbl}.
%  However, current AT doesn't handle this well, so
%  we use a tag \texttt{section-number} that is mapped to \texttt{Span}.
%  The number of the \texttt{Hn} tag should reflect the \enquote{natural} level.
%  So in an article \cs{section} will use  \texttt{H1}, in a book \cs{chapter} will use
%  \texttt{H1} and \cs{section} \texttt{H2}.
%  Titles of \cs{part} are a bit out of this system as they are normally
%  not part of the hierarchy: often only some chapters are grouped under a part.
%  Their title is therefore tagged as \texttt{Title}.
%  \item
%  The whole section should normally be surrounded by
%  a \texttt{Sect} tag. Parts should be surrounded by \texttt{Part}.
%  It is a bit unclear if the headings should be inside or outside of these
%  structures---the best practice guide puts them outside---but on the whole
%  it sounds more logical to group the heading with the text inside the \texttt{Sect}.
%  For the part this is actually required, as there can be only one \texttt{Title}
%  in a structure, so the part title can't be at the same level as the
%  document \texttt{Title}.
%
%  Starting such an enclosing \texttt{Sect} structure is rather easy,
%  but closing it requires code in various place,
%  for example the commands \cs{mainmatter}, \cs{backmatter},
%  \cs{frontmatter} and \cs{appendix} should typically close everything.
%  Following sectioning commands should close all previous structures
%  with a level equal or higher than their own level.
%  \end{itemize}
%
%  \section{Technical details and problems}
%
%  The implementation has to take care of various details.
%
%  \begin{itemize}
%
%  \item As sections in \LaTeX{} are not environments, the
%  \texttt{<Sect>} structures can be wrongly nested with other structures. For example
%  if a document puts a sectioning command into a list or a trivlist or
%  a minipage then it can no longer close previous \texttt{<Sect>} structures correctly.
%  The problem can be detected by checking the structure stack
%  and a warning can be issued, but the author then has to close the structures
%  manually before the list or minipage.
%
%  Thus there have to be user interfaces to handle such cases.
%  It should also be possible not to create all the \texttt{<Sect>} structures
%  automatically but to tag only the headings so that the author can handle special
%  cases manually.
%
%  \item If hyperref is used, targets for links should be inserted, either with
%  \cs{refstepcounter} or manually with \cs{MakeLinkTarget}. These targets must be
%  in the correct structure for the structure destinations. They replace some
%  of the current patches in hyperref.
%
%  \item With lualatex the mc-commands set attributes \emph{locally}, so the
%  commands must be at the right grouping level.
%  \end{itemize}
%
%  \subsection{Functions and keys}
%
% \begin{function}{\tag_tool:n,\tagtool}
%
% \end{function}
%
%  \subsection{TODO}
%
%  \begin{itemize}
%   \item A dedicated command to close a sectioning unit should be provided.
%
%   \item A dedicated command to open a sectioning unit should be provided too.
%
%   \item It should also be possible to suppress the sectioning unit in sectioning commands
%     to allow e.g. to put an epigraph or similar in front.
%
%  \item The number in \cs{part} and  \cs{chapter} is currently not correctly
%  tagged as a \texttt{section-number} as this requires to redefine the internal (class dependent)
%  commands too.
%
%  \end{itemize}
%
%    \begin{macrocode}
%<*package>
%    \end{macrocode}
%
% \section{Implementation}
%    \begin{macrocode}
\ProvidesExplPackage {latex-lab-testphase-sec} {\ltlabsecdate} {\ltlabsecversion}
  {Code related to the tagging of sectioning commands}
%    \end{macrocode}
%
% \subsection{Surrounding by \texttt{Sect} structures}
%  We use a stack to record the levels of the open \texttt{Sect}. The first item
%  has level -100. A sectioning command will take a record from the stack. If its level is
%  greater or equal it closes this structure and takes the next record from the stack.
%  If the record has a smaller level then it puts it back and stops.
%  The stack is compared with the main structure stack, if they don't match
%  it means we can't safely close the \texttt{Sect} and so we issue a warning
%  and do nothing.
%
% \subsubsection{Loading general kernel changes}
% [kernel?]
%  Also loaded in the toc-tagging code.
%    \begin{macrocode}
\RequirePackage{latex-lab-kernel-changes}
%    \end{macrocode}
%    \begin{macrocode}
%</package>
%    \end{macrocode}
% \subsubsection{Glyphtounicode improvements}
%
% As lualatex runs with legacy encodings in the test files, we enable and
% load glyphtounicode. For the math we load additional definitions.
%
%    \begin{macrocode}
%<*kernelchange>
\ifdefined\directlua
 \ifnum\outputmode > 0
   \pdfvariable gentounicode =1
   \protected\def\pdfglyphtounicode {\pdfextension glyphtounicode }
   \protected\edef\pdfgentounicode  {\pdfvariable gentounicode}
   \input{glyphtounicode}
 \fi
\fi
\ifdefined\pdfglyphtounicode
 % A subset of glyhtounicode-cmr.tex
%
% Copyright (c) 2008,  Han The Thanh <thanh@river-valley.org>
% Copyright (c) 2014,  Peter Selinger <selinger@mathstat.dal.ca>
% Copyright (c) 2018,  Ross Moore <ross.moore@mq.edu.au>
%                      

%% Glyphs from the cmex fonts:

\pdfglyphtounicode{angbracketleftBig}{27E8 FE02}
\pdfglyphtounicode{angbracketleftBigg}{27E8 FE04}
\pdfglyphtounicode{angbracketleftbig}{27E8 FE01}
\pdfglyphtounicode{angbracketleftbigg}{27E8 FE03}
\pdfglyphtounicode{angbracketrightBig}{27E9 FE02}
\pdfglyphtounicode{angbracketrightBigg}{27E9 FE04}
\pdfglyphtounicode{angbracketrightbig}{27E9 FE01}
\pdfglyphtounicode{angbracketrightbigg}{27E9 FE03}
\pdfglyphtounicode{arrowbt}{2193}
\pdfglyphtounicode{arrowdblbt}{21D3}
\pdfglyphtounicode{arrowdbltp}{21D1}
\pdfglyphtounicode{arrowhookleft}{21AA}
\pdfglyphtounicode{arrowhookright}{21A9}
\pdfglyphtounicode{arrowtp}{2191}
\pdfglyphtounicode{arrowvertex}{23D0}
\pdfglyphtounicode{arrowvertexdbl}{20E6}%  was {ED12}% PUA
\pdfglyphtounicode{backslashBig}{005C FE02}
\pdfglyphtounicode{backslashBigg}{005C FE04}
\pdfglyphtounicode{backslashbig}{005C FE01}
\pdfglyphtounicode{backslashbigg}{005C FE03}
\pdfglyphtounicode{braceex}{23AA}
\pdfglyphtounicode{bracehtipdownleft}{23DF}% was {ED17}% PUA
\pdfglyphtounicode{bracehtipdownright}{23DF}%  was {ED18}% PUA
\pdfglyphtounicode{bracehtipupleft}{23DE}%  was {ED19}% PUA
\pdfglyphtounicode{bracehtipupright}{23DE}%  was {ED1A}% PUA
\pdfglyphtounicode{braceleftBig}{007B FE02}
\pdfglyphtounicode{braceleftBigg}{007B FE04}
\pdfglyphtounicode{braceleftbig}{007B FE01}
\pdfglyphtounicode{braceleftbigg}{007B FE03}
\pdfglyphtounicode{braceleftbt}{23A9}
\pdfglyphtounicode{braceleftmid}{23A8}
\pdfglyphtounicode{bracelefttp}{23A7}
\pdfglyphtounicode{bracerightBig}{007D FE02}
\pdfglyphtounicode{bracerightBigg}{007D FE04}
\pdfglyphtounicode{bracerightbig}{007D FE01}
\pdfglyphtounicode{bracerightbigg}{007D FE03}
\pdfglyphtounicode{bracerightbt}{23AD}
\pdfglyphtounicode{bracerightmid}{23AC}
\pdfglyphtounicode{bracerighttp}{23AB}
\pdfglyphtounicode{bracketleftBig}{005B FE02}
\pdfglyphtounicode{bracketleftBigg}{005B FE04}
\pdfglyphtounicode{bracketleftbig}{005B FE01}
\pdfglyphtounicode{bracketleftbigg}{005B FE03}
\pdfglyphtounicode{bracketleftbt}{23A3}
\pdfglyphtounicode{bracketleftex}{23A2}
\pdfglyphtounicode{bracketlefttp}{23A1}
\pdfglyphtounicode{bracketrightBig}{005D FE02}
\pdfglyphtounicode{bracketrightBigg}{005D FE04}
\pdfglyphtounicode{bracketrightbig}{005D FE01}
\pdfglyphtounicode{bracketrightbigg}{005D FE03}
\pdfglyphtounicode{bracketrightbt}{23A6}
\pdfglyphtounicode{bracketrightex}{23A5}
\pdfglyphtounicode{bracketrighttp}{23A4}
\pdfglyphtounicode{ceilingleftBig}{2308 FE02}
\pdfglyphtounicode{ceilingleftBigg}{2308 FE04}
\pdfglyphtounicode{ceilingleftbig}{2308 FE01}
\pdfglyphtounicode{ceilingleftbigg}{2308 FE03}
\pdfglyphtounicode{ceilingrightBig}{2309 FE02}
\pdfglyphtounicode{ceilingrightBigg}{2309 FE04}
\pdfglyphtounicode{ceilingrightbig}{2309 FE01}
\pdfglyphtounicode{ceilingrightbigg}{2309 FE03}
\pdfglyphtounicode{circledotdisplay}{2A00 FE02}
\pdfglyphtounicode{circledottext}{2A00 FE01}
\pdfglyphtounicode{circlemultiplydisplay}{2A02 FE02}
\pdfglyphtounicode{circlemultiplytext}{2A02 FE01}
\pdfglyphtounicode{circleplusdisplay}{2A01 FE02}
\pdfglyphtounicode{circleplustext}{2A01 FE01}
\pdfglyphtounicode{contintegraldisplay}{222E FE02}
\pdfglyphtounicode{contintegraltext}{222E FE01}
\pdfglyphtounicode{coproductdisplay}{2210 FE02}
\pdfglyphtounicode{coproducttext}{2210 FE01}
\pdfglyphtounicode{floorleftBig}{230A FE02}
\pdfglyphtounicode{floorleftBigg}{230A FE04}
\pdfglyphtounicode{floorleftbig}{230A FE01}
\pdfglyphtounicode{floorleftbigg}{230A FE03}
\pdfglyphtounicode{floorrightBig}{230B FE02}
\pdfglyphtounicode{floorrightBigg}{230B FE04}
\pdfglyphtounicode{floorrightbig}{230B FE01}
\pdfglyphtounicode{floorrightbigg}{230B FE03}
\pdfglyphtounicode{hatwide}{02C6 FE01}
\pdfglyphtounicode{hatwider}{02C6 FE02}
\pdfglyphtounicode{hatwidest}{02C6 FE03}
\pdfglyphtounicode{integraldisplay}{222B FE02}
\pdfglyphtounicode{integraltext}{222B FE01}
\pdfglyphtounicode{intersectiondisplay}{22C2 FE02}
\pdfglyphtounicode{intersectiontext}{22C2 FE01}
\pdfglyphtounicode{logicalanddisplay}{22C0 FE02}
\pdfglyphtounicode{logicalandtext}{22C0 FE01}
\pdfglyphtounicode{logicalordisplay}{22C1 FE02}
\pdfglyphtounicode{logicalortext}{22C1 FE01}
\pdfglyphtounicode{mapsto}{21A6}
\pdfglyphtounicode{parenleftBig}{0028 FE02}
\pdfglyphtounicode{parenleftBigg}{0028 FE04}
\pdfglyphtounicode{parenleftbig}{0028 FE01}
\pdfglyphtounicode{parenleftbigg}{0028 FE03}
\pdfglyphtounicode{parenleftbt}{239D}
\pdfglyphtounicode{parenleftex}{239C}
\pdfglyphtounicode{parenlefttp}{239B}
\pdfglyphtounicode{parenrightBig}{0029 FE02}
\pdfglyphtounicode{parenrightBigg}{0029 FE04}
\pdfglyphtounicode{parenrightbig}{0029 FE01}
\pdfglyphtounicode{parenrightbigg}{0029 FE03}
\pdfglyphtounicode{parenrightbt}{23A0}
\pdfglyphtounicode{parenrightex}{239F}
\pdfglyphtounicode{parenrighttp}{239E}
\pdfglyphtounicode{productdisplay}{220F FE02}
\pdfglyphtounicode{producttext}{220F FE01}
\pdfglyphtounicode{radicalBig}{221A FE02}
\pdfglyphtounicode{radicalBigg}{221A FE04}
\pdfglyphtounicode{radicalbig}{221A FE01}
\pdfglyphtounicode{radicalbigg}{221A FE03}
\pdfglyphtounicode{radicalbt}{23B7}%  was {221A}
\pdfglyphtounicode{radicaltp}{231C}%  was {ED6A}% PUA
\pdfglyphtounicode{radicalvertex}{20D3}%  was {ED6B}% PUA
\pdfglyphtounicode{slashBig}{002F FE02}
\pdfglyphtounicode{slashBigg}{002F FE04}
\pdfglyphtounicode{slashbig}{002F FE01}
\pdfglyphtounicode{slashbigg}{002F FE03}
\pdfglyphtounicode{summationdisplay}{2211 FE02}
\pdfglyphtounicode{summationtext}{2211 FE01}
\pdfglyphtounicode{tie}{2040}
\pdfglyphtounicode{tildewide}{02DC FE01}
\pdfglyphtounicode{tildewider}{02DC FE02}
\pdfglyphtounicode{tildewidest}{02DC FE03}
\pdfglyphtounicode{uniondisplay}{22C3 FE02}
\pdfglyphtounicode{unionmultidisplay}{2A04 FE02}
\pdfglyphtounicode{unionmultitext}{2A04 FE01}
\pdfglyphtounicode{unionsqdisplay}{2A06 FE02}
\pdfglyphtounicode{unionsqtext}{2A06 FE01}
\pdfglyphtounicode{uniontext}{22C3 FE01}
\pdfglyphtounicode{vextenddouble}{20E6}%  was {ED79}% PUA
\pdfglyphtounicode{vextendsingle}{20D3}%%% was {23D0}

\fi
%</kernelchange>
%    \end{macrocode}
%
% \subsubsection{updating \cs{@currentHref}}
% [kernel?]
%
% We  must ensure that manual targets (e.g. in unnumbered sections)
% correctly update \cs{@currentHref}. For this we extend the kernel definition of
% \cs{MakeLinkTarget}
% TODO: remove after release 2024/11
%    \begin{macrocode}
%<*kernelchange>
\ExplSyntaxOn
\str_if_exist:cF { l__socket_tagsupport/recordtarget_plug_str }
 {
  \int_new:N\g__kernel_target_int
  \RenewDocumentCommand\MakeLinkTarget{sO{}m}
   {%
    \ifvmode
      \special{}%
    \else
      \@savsf\spacefactor
      \smash{}%
      \spacefactor\@savsf
    \fi
    \IfBooleanTF {#1}
     {
       \tl_gset:Ne \@currentHref {#3}
     }
     {
       \int_gincr:N\g__kernel_target_int
       \tl_gset:Ne \@currentHref {target*.\int_use:N\g__kernel_target_int}
     }
   }
 }
\ExplSyntaxOff
%</kernelchange>
%    \end{macrocode}
%
%    \begin{macrocode}
%<*package>
%<@@=tag>
%    \end{macrocode}
% \subsubsection{Tagging commands}
%
%
% \begin{variable}{\g_@@_sec_stack_seq}
% The stack holds the tag, the level and the structure number.
%    \begin{macrocode}
\seq_new:N   \g_@@_sec_stack_seq
\seq_gpush:Nn\g_@@_sec_stack_seq {{Document}{-100}{2}}
%    \end{macrocode}
% \end{variable}
%
% \begin{macro}{\@@_get_data_current_Sect:}
% This allows to retrieve the number of the current Sect structure (or
% Document if we are outside any Sect) with |\tag_get:n{current_Sect}|
%    \begin{macrocode}
\cs_new:Npn \@@_get_data_current_Sect:
 {
   \exp_last_unbraced:Ne\use_iii:nnn{\seq_item:Nn\g_@@_sec_stack_seq{1}}
 }
%    \end{macrocode}
% \end{macro}
% \begin{variable}{\l_@@_sec_Sect_bool}
% This boolean controls if a Sect structure is opened.
%    \begin{macrocode}
\bool_new:N     \l_@@_sec_Sect_bool
\bool_set_true:N\l_@@_sec_Sect_bool
%    \end{macrocode}
% \end{variable}
%
% \begin{macro}{\@@_sec_begin:nn}
% This starts a sectioning structure.
% Currently the tag is fix, either Sect or Part, depending on the level,
% but this will perhaps change. The second argument is currently unused.
%    \begin{macrocode}
\cs_new_protected:Npn\@@_sec_begin:nn #1 #2 %#1 level #2 keyval
  {
    \tag_struct_begin:n
      {
         tag= {\int_compare:nNnTF {#1}={-1}{Part}{Sect}}
        ,#2
      }
    \seq_gpush:Ne \g_@@_sec_stack_seq
      {{\g_@@_struct_tag_tl}{\int_eval:n{#1}}{\g_@@_struct_stack_current_tl}}
  }
%    \end{macrocode}
% \end{macro}
%
% \begin{macro}{\@@_sec_end:n}
%    \begin{macrocode}
\msg_new:nnn { tag } {wrong-sect-nesting}
  {
    The~structure~#1~can~not~be~closed.\\
    It~is~not~equal~to~the~current~structure~#2~on~the~main~stack
  }

\cs_new_protected:Npn\@@_sec_end:n #1 % #1 level
  {
    \seq_get:NN \g_@@_sec_stack_seq \l_@@_tmpa_tl
    \int_compare:nNnT {#1}<{\exp_last_unbraced:NV\use_ii:nnn\l_@@_tmpa_tl+1}
      {
        \seq_get:NN\g_@@_struct_tag_stack_seq \l_@@_tmpb_tl
        \exp_args:Nee
          \tl_if_eq:nnTF
            {\exp_last_unbraced:NV\use_i:nnn\l_@@_tmpa_tl}
            {\exp_last_unbraced:NV\use_i:nn\l_@@_tmpb_tl}
            {
              \seq_gpop:NN \g_@@_sec_stack_seq \l_@@_tmpa_tl
              \tag_struct_end:
              \@@_sec_end:n {#1}
            }
            {
              \msg_warning:nnee {tag}{wrong-sect-nesting}
               { \exp_last_unbraced:NV\use_i:nnn \l_@@_tmpa_tl }
               { \exp_last_unbraced:NV\use_i:nn \l_@@_tmpb_tl }
            }
      }
  }
%    \end{macrocode}
% \end{macro}

% \begin{macro}{\@@_tool_para_split:}
% Runin-sectioning command must separate the heading from the following text.
%
%    \begin{macrocode}
\cs_new_protected:Npn \@@_tool_para_split:
  {
    \tag_mc_end:
    \tag_struct_end:
    \tag_struct_begin:n{tag=\l_@@_para_tag_default_tl}
    \tag_mc_begin:n{}
    \@@_setup_restore_para_default:
  }
%    \end{macrocode}
% \end{macro}

% \begin{macro}{\@@_setup_restore_para_default:}
% We change the para tagging in the sectioning code.
% This here restores the default. Currently it only resets the
% the tag, but perhaps more will be needed later.
%    \begin{macrocode}
\cs_new_protected:Npn \@@_setup_restore_para_default:
  {
    \tl_set:Nn \l_@@_para_main_tag_tl {text-unit}
    \tl_set_eq:NN\l_@@_para_tag_tl\l_@@_para_tag_default_tl
  }
%    \end{macrocode}
% \end{macro}
%
% \begin{macro}{\@@_sec_end_display:}
%    \begin{macrocode}
\cs_new_protected:Npn \@@_sec_end_display:
  {
    \tag_struct_end: %P = Hn
    \@@_setup_restore_para_default:
  }
%    \end{macrocode}
% \end{macro}
%
% Open sec structures should be closed at the end of the document. This should
% be done before tagpdf closes the Document structure.
%    \begin{macrocode}
\hook_gput_code:nnn{tagpdf/finish/before}{tagpdf/sec}{\@@_sec_end:n{-10}}
\hook_gset_rule:nnnn {tagpdf/finish/before}{tagpdf/sec}{before}{tagpdf}
%    \end{macrocode}
%
% The commands \cs{mainmatter}, \cs{backmatter}, \cs{frontmatter} and
% \cs{appendix} close all \texttt{Sect} and \texttt{Part} structures.
% \changes{v0.84g}{2025/01/05}{ensure that paragraph is ended (tagging/777)}
%    \begin{macrocode}
\AddToHook{cmd/frontmatter/before}{\par\@@_sec_end:n{-10}}
\AddToHook{cmd/mainmatter/before} {\par\@@_sec_end:n{-10}}
\AddToHook{cmd/backmatter/before} {\par\@@_sec_end:n{-10}}
\AddToHook{cmd/appendix/before}   {\par\@@_sec_end:n{-10}}
%    \end{macrocode}
%
% \subsection{Tagging tools}
% We need to provide user and package level commands
%
%    \begin{macrocode}
\cs_if_free:NT \tag_tool:n
 {
   \cs_new_protected:Npn \tag_tool:n #1
    {
      \tag_if_active:T { \keys_set:nn {tag / tool}{#1} }
    }
   \cs_set_eq:NN\tagtool\tag_tool:n
 }
\keys_define:nn { tag / tool}
  {
    ,sec-start-part .code:n =
      {
        \bool_if:NT\l_@@_sec_Sect_bool
          {
            \@@_sec_end:n   {-1}
            \@@_sec_begin:nn{-1}{tag=Part}
          }
         \tag_struct_begin:n{tag=part,title=#1}
%    \end{macrocode}
% We remap here the text-unit from the paragraph to NonStruct.
% It would be better to suppress it completely as with the other
% sectioning commands, but this would require to redefine \cs{@spart}
% and \cs{@part}, as there is the grouping, and these commands are
% all slightly different in the standard classes. So this is delayed
% to the time when sectioning commands are redefined with templates.
%    \begin{macrocode}
         \tl_set:Nn\l_@@_para_main_tag_tl {NonStruct}
         \tl_set:Nn\l_@@_para_tag_tl {Span}
      }
    ,sec-stop-part .code:n = {\@@_sec_end_display:}
    ,sec-start-chapter .code:n =
     {
       \bool_if:NT\l_@@_sec_Sect_bool
         {
           \@@_sec_end:n   {0}
           \@@_sec_begin:nn{0}{tag=Sect}
         }
        \tag_struct_begin:n{tag=chapter,title=#1}
%    \end{macrocode}
% similar to part we remap to NonStruct for now ...
%    \begin{macrocode}
        \tl_set:Nn\l_@@_para_main_tag_tl {NonStruct}
        \tl_set:Nn\l_@@_para_tag_tl {Span}
     }
    ,sec-stop-chapter .meta:n = { sec-stop-part}
    ,sec-start .code:n = % #1 is a name like "section"
      {
        \bool_if:NT\l_@@_sec_Sect_bool
          {
            \@@_sec_end:n    {\cs_if_exist_use:c{toclevel@#1}+0}
            \@@_sec_begin:nn {\cs_if_exist_use:c{toclevel@#1}+0}{tag=Sect}
          }
        \tl_set:Nn\l_@@_para_tag_tl{#1}
      }
    ,sec-start .value_required:n = true
    ,sec-split-para .code:n = {\@@_tool_para_split:}
    ,restore-para .code:n = {\@@_setup_restore_para_default:}
    ,sec-stop .code:n =
      {
        \par\@@_sec_end:n   {\cs_if_exist_use:c{toclevel@#1}+0}
      }
    ,sec-stop .value_required:n = true
    ,sec-add-grouping .bool_set:N = \l_@@_sec_Sect_bool
  }
%    \end{macrocode}
%
%
% \section{Sectioning commands}
%
% \subsection{\cs{part} and \cs{chapter}}
%
% \cs{part} and \cs{chapter} are defined by the classes.
% To tag them we redefine the user commands.
% This will probably break with various classes and with titlesec.
% The tagging inside relies on the para tagging.
% We do not yet use keyval in the optional argument, as this requires latex-dev
% and the naming of the keys and their key family is unclear.
% \changes{v0.84f}{2024/10/04}{Added braces around optional arg (tagging/725)}
%    \begin{macrocode}
\AddToHook{class/after}
 {
  \@ifundefined{chapter}
    {
%    \end{macrocode}
% This redefines \cs{part} in article class.
%    \begin{macrocode}
     \@ifundefined{part}{}
      {
        \RenewDocumentCommand\part{ s O{#3} m }
         {
           \if@noskipsec \leavevmode \fi
           \par
           \addvspace{4ex}%
           \@afterindentfalse
%    \end{macrocode}
% This are the tagging commands needed at the begin. They open a Part structure
% and the structure for the title of the heading.
%    \begin{macrocode}
        % tagging start commands
          \tag_tool:n {sec-start-part=#2}
        % end tagging start commands
%    \end{macrocode}
% This adds a  manual target if the part is unnumbered or starred.
% It replaces the hyperref patches.
%    \begin{macrocode}
           \bool_lazy_any:nT
            {
              { #1 }
              {
                \int_compare_p:nNn {\c@secnumdepth}<{-1}
              }
            }
            {
              \MakeLinkTarget[part]{}
            }
%    \end{macrocode}
% The main call to the underlying commands.
%    \begin{macrocode}
          \IfBooleanTF
            {#1}
            { \@spart {#3} }
            { \@part [{#2}]{#3} }
%    \end{macrocode}
% and now the closing command for the tagging of the title.
%    \begin{macrocode}
         \tag_tool:n {sec-stop-part}
         }
       }
    }
%    \end{macrocode}
% Redefinitions for book and report
%    \begin{macrocode}
    {
     \RenewDocumentCommand\chapter{ s O{#3} m }
      {
        \if@openright\cleardoublepage\else\clearpage\fi
        \thispagestyle{plain}%
        \global\@topnum\z@
        \@afterindentfalse
%    \end{macrocode}
% This are the tagging commands needed at the begin. They open a Sect structure
% and the structure for the title of the heading.
%    \begin{macrocode}
        \tag_tool:n { sec-start-chapter= #2 }
%    \end{macrocode}
% This adds a  manual target if the chapter is unnumbered or starred.
% It replaces the hyperref patches.
%    \begin{macrocode}
        \bool_lazy_any:nT
          {
            { #1 }
            {
              \int_compare_p:nNn {\c@secnumdepth}<{0}
            }
            {
              %in book target also needed in frontmatter
              \bool_lazy_and_p:nn
                { \cs_if_exist_p:c { @mainmattertrue } }
                { ! \legacy_if_p:n { @mainmatter } }
            }
          }
          {
%    \end{macrocode}
% The relation target-struct is stored internally by the MakeLinkTarget commands
%    \begin{macrocode}
            \MakeLinkTarget[chapter]{}
          }
%    \end{macrocode}
% The main call to the underlying commands.
%    \begin{macrocode}
        \IfBooleanTF
          {#1}
          { \@schapter {#3} }
          { \@chapter [{#2}]{#3} }
%    \end{macrocode}
% and now the closing command for the tagging of the title.
%    \begin{macrocode}
        \tag_tool:n {sec-stop-chapter}
      }
%    \end{macrocode}
% and similar for \cs{part}
%    \begin{macrocode}
     \RenewDocumentCommand\part{ s O{#3} m }
      {
        \if@openright
          \cleardoublepage
        \else
          \clearpage
        \fi
        \thispagestyle{plain}%
        \if@twocolumn
          \onecolumn
          \@tempswatrue
        \else
          \@tempswafalse
        \fi
        \null\vfil
%    \end{macrocode}
% This are the tagging commands needed at the begin. They open a Part structure
% and the structure for the title of the heading.
%    \begin{macrocode}
       \tag_tool:n {sec-start-part=#2}
%    \end{macrocode}
% This adds a  manual target if the part is unnumbered or starred.
% It replaces the hyperref patches.
%    \begin{macrocode}
        \bool_lazy_any:nT
          {
            { #1 }
            {
              \int_compare_p:nNn {\c@secnumdepth}<{-1}
            }
            {
              %in book target also needed in frontmatter
              \bool_lazy_and_p:nn
                { \cs_if_exist_p:c { @mainmattertrue } }
                { ! \legacy_if_p:n { @mainmatter } }
            }
          }
          {
            \MakeLinkTarget[part]{}
          }
%    \end{macrocode}
% The main call to the underlying commands.
%    \begin{macrocode}
        \IfBooleanTF
          {#1}
          { \@spart {#3} }
          { \@part [{#2}]{#3} }
%    \end{macrocode}
% and now the closing command for the tagging of the title.
%    \begin{macrocode}
        \tag_tool:n{sec-stop-part}
      }
    }
 }
%    \end{macrocode}
%
% \subsection{Sectioning commands based on \cs{@startsection}}
%
% The tagging relies again on the para tagging:
% we simply exchange the tag name by the one given as \#1.
% This assumes that a tag with the name of the sectioning type is defined.
% We don't try to pass the title, this will be done together with
% the new keyval handling in the user command.
%
% \subsubsection{Hyperref code}
% hyperref has to insert anchors. If the sectioning is numbered this is done by
% \cs{refstepcounter} (and so in vmode). For unnumbered section hyperref
% injects the anchor in hmode before the text, it also inserts a
% kern to compensate the indent.
%
% This means that the target of numbered and unnumbered sectioning commands
% differ, both regarding the location and in relation to the
% tagging structure: The anchor from the \cs{refstepcounter} is outside of
% the structure created by the heading title if the para tags are used,
% while the other anchors are inside and so the structure destinations are different.
%
% We unify this by suppressing the anchor from the refstepcounter.
% Also we only go back if the indent is positive.
%
% At first suppress all hyperref patches related to sectioning:
%    \begin{macrocode}
\def\hyper@nopatch@sectioning{}
%    \end{macrocode}
%
% \begin{macro}{\@hyp@section@target@nnn}
% A simple internal command. There is no need for something public,
% as packages defining their own version of \cs{@startsection} will
% probably need something slightly different based on \cs{MakeLinkTarget}.
% \changes{v0.84i}{2025/02/13}{Wrapped all use inside \cs{NoCaseChange},
% tagging-project issue \#787}
%    \begin{macrocode}
\cs_new_protected:Npn \@hyp@section@target@nnn #1 #2 #3 %#1 optarg #2 name/counter, #3 indent
  {
    \makebox[0pt][l]
     {
       \skip_set:Nn \@tempskipa {#3}
       \dim_compare:nNnF {\@tempskipa}<{0pt}{\kern-\@tempskipa}
       \MakeLinkTarget#1{#2}
     }
  }
%    \end{macrocode}
% \end{macro}
%
% \subsection{Adaption of the heading commands}
% We add to \cs{@startsection} the commands to open the \texttt{Sect}
% structure and to change the para tag.
%
%    \begin{macrocode}
\def\@startsection#1#2#3#4#5#6{%
  \if@noskipsec \leavevmode \fi
  \par
  \@tempskipa #4\relax
  \@afterindenttrue
  \ifdim \@tempskipa <\z@
    \@tempskipa -\@tempskipa \@afterindentfalse
  \fi
  \if@nobreak
    \everypar{}%
  \else
    \addpenalty\@secpenalty\addvspace\@tempskipa
  \fi
  \tag_tool:n { sec-start=#1}%new
  \@ifstar
    {\@ssect{#3}{#4}{#5}{#6}}%
    {\@dblarg{\@sect{#1}{#2}{#3}{#4}{#5}{#6}}}}
%    \end{macrocode}
% To be able to correctly tag the number we need a special
% \cs{@hangfrom} variant. This is a bit tricky:
% As the paragraph starts after the \cs{setbox} luatex attributes
% are not set yet and numbers are unmarked if one doesn't pay attention.
% The code assumes that we are in vmode!
% 2025-01-20: Until the redefinition is gone from tagpdf we have to empty the hook:
%    \begin{macrocode}
\RemoveFromHook{package/latex-lab-testphase-sec/after}[tagpdf]
%    \end{macrocode}
%    \begin{macrocode}
\cs_new_protected:Npn \@kernel@tag@hangfrom #1
  {
     \tagstructbegin{tag=\l_@@_para_tag_tl}
     \cs_if_exist_use:N \@@_gincr_para_begin_int:
     \tagstructbegin{tag=section-number}
     \setbox\@tempboxa
       \hbox
         {
%    \end{macrocode}
% In lua mode we have to set the attributes inside the box!
%    \begin{macrocode}
           \bool_lazy_and:nnT
            {\tag_if_active_p:}
            {\g_@@_mode_lua_bool}
            {\tagmcbegin{tag=section-number}}
           {#1}
         }
%    \end{macrocode}
% We stop tagging now, to avoid that the \cs{noindent} triggers
% the paratagging. We do not disable paratagging completely, to
% avoid that the numbering goes wrong.
%    \begin{macrocode}
   \tag_suspend:n{hangfrom}
   \hangindent \wd\@tempboxa\noindent
%    \end{macrocode}
% Restart tagging and insert the box.
%    \begin{macrocode}
   \tag_resume:n{hangfrom}
   \tagmcbegin{}\box\@tempboxa\tagmcend\tagstructend\tagmcbegin{}}
%    \end{macrocode}
% This command is used to tag the numbers of runin. We do not try
% to avoid the empty container from the paratagging, this would require
% more changes.
%    \begin{macrocode}
\cs_new_protected:Npn \@kernel@tag@svsec
  {
    \tag_mc_end_push:
    \tag_struct_begin:n{tag=section-number}
    \tag_mc_begin:n{}
    \@svsec
    \tag_mc_end:
    \tag_struct_end:
    \tag_mc_begin_pop:n{}
  }
%    \end{macrocode}
% \cs{@sect} is only changed to replace the hyperref patches
% and to use the new \cs{@kernel@tag@hangfrom} and \cs{@kernel@tag@svsec}
%    \begin{macrocode}
%<@@=>
\def\@sect#1#2#3#4#5#6[#7]#8{%
  \ifnum #2>\c@secnumdepth
    \def\@svsec{\NoCaseChange{\@hyp@section@target@nnn{[section]}{}{#3}}}
  \else
    \LinkTargetOff
    \refstepcounter{#1}%
    \LinkTargetOn
    \protected@edef\@svsec{\NoCaseChange{\@hyp@section@target@nnn{}{#1}{#3}}\@seccntformat{#1}\relax}%
  \fi
  \@tempskipa #5\relax
  \ifdim \@tempskipa>\z@
    \begingroup
    \tagtool{para-flattened=true} % or \bool_set_true\l_@@_para_flattened_bool
      #6{%
         \ifnum #2>\c@secnumdepth
          \@hangfrom {\hskip #3\relax\@svsec}%
         \else
          \@kernel@tag@hangfrom{\hskip #3\relax\@svsec}%
         \fi
          \interlinepenalty \@M #8\@@par}%
    \endgroup
    \csname #1mark\endcsname{#7}%
    \addcontentsline{toc}{#1}{%
      \ifnum #2>\c@secnumdepth \else
        \protect\numberline{\csname the#1\endcsname}%
      \fi
      #7}%
  \else
    \def\@svsechd{%
      #6{\hskip #3\relax
      \ifnum #2>\c@secnumdepth
       \@svsec
      \else
       \@kernel@tag@svsec
      \fi  #8}%
      \csname #1mark\endcsname{#7}%
      \addcontentsline{toc}{#1}{%
        \ifnum #2>\c@secnumdepth \else
          \protect\numberline{\csname the#1\endcsname}%
        \fi
        #7}}%
  \fi
  \@xsect{#5}}
%    \end{macrocode}
% similar for \cs{@ssect}
%    \begin{macrocode}
\def\@ssect#1#2#3#4#5{%
  \@tempskipa #3\relax
  \ifdim \@tempskipa>\z@
    \begingroup
    \tagtool{para-flattened=true}
      #4{%
        \@hangfrom{\hskip #1\relax\NoCaseChange{\@hyp@section@target@nnn{[section]}{}{#1}}}%
          \interlinepenalty \@M #5\@@par}%
    \endgroup
  \else
    \def\@svsechd{#4{\hskip #1\relax\NoCaseChange{\@hyp@section@target@nnn{[section]}{}{#3}}\relax #5}}%
  \fi
  \@xsect{#3}}
%    \end{macrocode}
% At last \cs{@xsect} needs code in two places. For display headings it has to
% restore the default para code, for run in headings it has to separated the
% heading from the following text.
%    \begin{macrocode}
\def\@xsect#1{%
  \@tempskipa #1\relax
  \ifdim \@tempskipa>\z@
    \par \nobreak
    \vskip \@tempskipa
    \tag_tool:n {restore-para}
    \@afterheading
  \else
    \@nobreakfalse
    \global\@noskipsectrue
    \everypar{%
      \if@noskipsec
        \global\@noskipsecfalse
       {\setbox\z@\lastbox}%
        \clubpenalty\@M
        \begingroup \@svsechd \endgroup
        \unskip
        \tag_tool:n {sec-split-para}
        \@tempskipa #1\relax
        \hskip -\@tempskipa
      \else
        \clubpenalty \@clubpenalty
        \everypar{}%
      \fi}%
  \fi
  \ignorespaces}
%</package>
%    \end{macrocode}

%    \begin{macrocode}
%<*latex-lab>
\ProvidesFile{sec-latex-lab-testphase.ltx}
        [\ltlabsecdate\space v\ltlabsecversion\space latex-lab wrapper sec]
\RequirePackage{latex-lab-testphase-sec}
%</latex-lab>
%    \end{macrocode}
