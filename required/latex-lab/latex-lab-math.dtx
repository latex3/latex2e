% \iffalse meta-comment
%
%% File: latex-lab-math.dtx
%
% Copyright (C) 2022-2025 The LaTeX Project
%
% It may be distributed and/or modified under the conditions of the
% LaTeX Project Public License (LPPL), either version 1.3c of this
% license or (at your option) any later version.  The latest version
% of this license is in the file
%
%    https://www.latex-project.org/lppl.txt
%
%
% The development version of the bundle can be found below
%
%    https://github.com/latex3/latex2e/required/latex-lab
%
% for those people who are interested or want to report an issue.
%
%

\def\ltlabmathdate{2024-12-16}
\def\ltlabmathversion{0.6k}
%
%<*driver>
\documentclass{l3doc}
\EnableCrossrefs
\CodelineIndex

\usepackage{todonotes}

\begin{document}
  \DocInput{latex-lab-math.dtx}
\end{document}
%</driver>
%
% \fi
%
%
% \title{The \texttt{latex-lab-math} code\thanks{}}
% \author{Frank Mittelbach, Joseph Wright, \LaTeX{} Project}
% \date{v\ltlabmathversion\ \ltlabmathdate}
%
% \maketitle
%
% \newcommand\NEW[1]{\marginpar{\mbox{}\hfill\fbox{New: #1}}}
% \providecommand\class[1]{\texttt{#1.cls}}
% \providecommand\pkg[1]{\texttt{#1}}
%
% \providecommand\hook[1]{\texttt{#1\DescribeHook[noprint]{#1}}}
% \providecommand\socket[1]{\texttt{#1\DescribeSocket[noprint]{#1}}}
% \providecommand\plug[1]{\texttt{#1\DescribePlug[noprint]{#1}}}
%
% \NewDocElement[printtype=\textit{socket},idxtype=socket,idxgroup=Sockets]{Socket}{socketdecl}
% \NewDocElement[printtype=\textit{hook},idxtype=hook,idxgroup=Hooks]{Hook}{hookdecl}
% \NewDocElement[printtype=\textit{plug},idxtype=plug,idxgroup=Plugs]{Plug}{plugdecl}
%
% ^^A \car {...} for marginal comments
% ^^A \car*{...} for longer inline comments
%
% \NewDocumentCommand\car{sO{}m}
%   {\IfBooleanTF{#1}{\todo[inline,color=blue!10,#2]{#3}}^^A
%                    {\todo[color=blue!10,#2]{#3}}}
%
% \NewDocumentCommand\fmi{sO{}m}
%   {\IfBooleanTF{#1}{\todo[inline,#2]{#3}}^^A
%                    {\todo[#2]{#3}}}
%
% \begin{abstract}
%    This is an experimental prototype. It captures math material
%    (basically okay, but the interfaces for packages aren't yet
%    there) and tags the material (which is not yet anywhere near the
%    final state). That part is provided for experimentation and to
%    gather feedback, etc.
% \end{abstract}
%
% \tableofcontents
%
% \section{Introduction}
% \car*{Todo: update all the documentation! Both here and
%   (what little there is!) in the implementation section.}
%
% Tagging math involves a variety of tasks that require that math is captured before the
% typesetting
% \begin{itemize}
% \item When typesetting the math MC-tags and structure commands must
% be inserted at the begin and the end, and perhaps also around lines
% or other subparts of the equation.
% \item The source and/or a mathml-representation of the source must be available
% so that it can be (perhaps after some preprocessing) be used in an associated file
% or in an alternate text
% \item It must be possible to measure the math for, e.g., a bbox setting.
% \end{itemize}
%
% This file implements capture of all math mode material at the outer
% level, i.e., a formula is captured in its entirety with inner text
% blocks (possibly containing further math) absorbed as part of the
% formula. For example,
%\begin{verbatim}
%      \[ a \in A \text{ for all $a<5$}  \]
%\end{verbatim}
% would only result in a single capture of the tokens
% ``\verb*/a \in A \text{ for all $a<5$}/''.
%
%
% \section{Math capture} \label{sec:mathcapture}
% In the current setup
%  \begin{itemize}
%   \item |$|, |\(...\)| and |$$| grab (through a command in \cs{everymath}/cs{everydisplay})
%   if the boolean \cs{l_@@_collected_bool} is false.
%   If the boolean is true they behave normally and can for example contain verbatim.
%
%   \item All (registered) environments grab their body
%    regardless of the state of the boolean. For
%    |equation|, |equation*| and |math| this is a change as they no longer can
%    contain verbatim.
%
%   \item BUG: |\[...\]| grabs if \cs{l_@@_collected_bool} is false. If it is
%   true it falls back to |equation*| and then errors because this can't find the end.
%  \end{itemize}
%
% \subsection{Code level interfaces}
%
% \begin{function}{\math_register_env:n, \math_register_env:nn}
%   \begin{syntax}
%     \cs{math_register_env:n} \Arg{env}
%     \cs{math_register_env:nn} \Arg{env} \Arg{options}
%   \end{syntax}
%   Registers the \meta{env} as a math environment which should be captured
%   and made available. This is necessary for all top-level math mode
%   environments: low-level errors may result if these are not correct
%   set up. One or more key--value \meta{options} may also be given:
%   \begin{itemize}
%     \item[\texttt{arg-spec}] The argument specification taken by the
%       beginning of the environment; this is used to remove non-mathematical
%       material.
%   \end{itemize}
% \end{function}
%
% \begin{function}{\math_processor:n}
%   \begin{syntax}
%     \cs{math_processor:n} \Arg{tokens}
%   \end{syntax}
%   Declares that the captured math content should be passed to the
%   \meta{tokens}, which will receive the environment type as |#1| and
%   the content as |#2|. The processing is done before the typesetting. It is not
%   applied if \cs{ifmeasuring@} is true.
% \end{function}
%
% \subsection{Document level interfaces}
%
% \begin{function}{\RegisterMathEnvironment}
%   \begin{syntax}
%     \cs{RegisterMathEnvironment} \oarg{options} \Arg{env}
%   \end{syntax}
%   Registers the \meta{env} as a math environment which should be captured
%   and made available. This is necessary for all top-level math mode
%   environments: low-level errors may result if these are not correct
%   set up. One or more key--value \meta{options} may also be given:
%   \begin{itemize}
%     \item[\texttt{arg-spec}] The argument specification taken by the
%       beginning of the environment; this is used to remove non-mathematical
%       material.
%   \end{itemize}
% \end{function}
%
% \section{Math tagging}
%
% \subsection{Code requirements}
% The tagging code has to handle
% \begin{itemize}
% \item the embedding into the surrounding. This means
%   \begin{itemize}
%     \item closing and reopening MC-chunks
%     \item closing and reopening text/P-structures
%     \item handling interferences of the tagging code with penalties and spacing.
%   \end{itemize}
% \item the actual tagging which means to do some or all of the following tasks:
%   \begin{itemize}
%     \item setup content for an associated source file
%     \item setup content for an associated mathml file
%     \item setup content for the /Alt key
%     \item setup content for the /ActualText key
%     \item setup attributes
%     \item add associated files
%     \item add a Formula structure
%     \item surround subparts (e.g., lines) with Formula sub structures
%          (perhaps with their own set of additional content)
%     \item surround elements of the equation with mathml structure elements
%          (currently only luatex with luamml)
%    \end{itemize}
%  \end{itemize}
%
%  \subsection{Inline math}
%
%  The embedding code is added through
%  the sockets
%   \begin{itemize}
%    \item |tagsupport/math/inline/begin|
%    \item |tagsupport/math/inline/end|
%   \end{itemize}
%  The sockets simply push and pop the MC currently. Without
%  tagging they use the noop-plug.
%
%  The actual tagging is in done through the sockets
%   \begin{itemize}
%    \item |tagsupport/math/inline/formula/begin|
%     This socket takes the math as second argument and its code
%     should output it for typesetting.
%     The |default| plug of the socket calls these three internal sockets
%     for the tagging support:
%      \begin{itemize}
%      \item |tagsupport/math/content| This should set up the various
%      content variables (empty variables are ignored by the structure code
%       and so can be used to suppress a setting).
%      \item |tagsupport/math/struct/begin| This calls \cs{tag_struct_begin:n}.
%      It should also write the associated files if needed.
%      \item |tagsupport/math/substruct/begin| this handles subparts.
%      TODO: does it really make sense in inline math to have that??
%      \end{itemize}
%    \item |tagsupport/math/inline/formula/end|
%      This socket ends the formula structure(s). The |default|
%      plug calls these internal sockets:
%       \begin{itemize}
%        \item |tagsupport/math/substruct/end|
%        \item |tagsupport/math/struct/end|
%       \end{itemize}
%   \end{itemize}
%
%  \subsection{Display math}
%
%  \textit{to be written}
%
%  \subsection{Associated Files}
%
%  The current code allows the attachment of two types of associated file to the
%  Formula structure:
%  the \LaTeX\ source and a MathML representation.
%  Technically both can be attached---AF is an array
%  of file references-----in practice there can be problems with PDF consumers:
%  e.g., ngpdf used both and so showed the equation twice
%  (this has been corrected in the newest version) and
%  Foxit seems to see only the first AF in the array (so we attach the
%  mathml as first file).
%
%  The \LaTeX\ source can be (and is) attached automatically.
%  It can be suppressed by an option with
%  \texttt{math/tex/AF=false}, see below.
%
%  The MathML is attached if the files |\jobname-mathml.html| and/or
%  |\jobname-luamml-mathml.html| are found
%  and if they contains a suitable MathML snippet for the current formula.
%  If the files contain more than one suitable snippet (as identified by the hash)
%  the first one is used.
%  |\jobname-luamml-mathml.html| is automatically generated (see below section~\ref{sec:luamml})
%  and read after |\jobname-mathml.html|. This means that |\jobname-mathml.html| can contain
%  improved versions of a formula.
%
%  The MathML processing can be suppressed globally by emptying the list of
%  mathml files with |math/mathml/sources=|. Locally for a formula |math/mathml/AF=false|
%  can be used.
%
%  For a MathML representation a file with such representations must be provided.
%  If the equation is numbered the numbering should be part of the MathML as
%  the |Lbl| substructure is ignored if an MathML is used (see https://github.com/foxitsoftware/PDF_UA-2).
%
%  The MathML representation is given in a special format.
%  It is meant to be a valid html file
%  that can be viewed  in a browser.
%  For this it can start with |<!DOCTYPE html><html>| and end with |</html>|
%  It should have the extension \texttt{.html}. The \meta{mathml} content
%  is read with special catcodes, so can contain ambersands, hashes, comment chars
%  and unmatched braces such as |<mo>{</mo>|
%
%  The file should contain a number of representations in this format:
%  \begin{quote}
%  |<div>| \\
%  |  <h2>\mml| \meta{key}|</h2>|\\
%  |  <p>|\meta{source}|</p>| \\
%  |  <p>|\meta{hash}|</p>|   \\
%  |  <math | \meta{attributes} |>|\\
%  \meta{mathml}\\
%  |  </math>|\\
%  |</div>|
%  \end{quote}
%  The keywords |<div>|, |<h2>\mml|, |<p>|, |<math|, |</math>| |</div>| are required as
%  they are used to delimit the arguments by the \LaTeX{} code.
%
%  \meta{key} and \meta{source} are only used for debugging, they help to identify
%  the equation referred by this representation. The source should be used correctly escaped
%  |&| and |<| so that if gives valid html!
%
%  \meta{attributes} is not required either, but can, e.g., contain attributes
%  to improve the display in a browser:
%  \begin{verbatim}
%  <math alttext="\mathbf{G}" class="ltx_Math" display="inline">
%  \end{verbatim}
%  It can also contain the name space declaration: |xmlns="http://www.w3.org/1998/Math/MathML"|%
%  \footnote{But it is probably not needed and only blows up the PDF.}
%
%
%  By default the code tries at the begin of the document
%  to read a file |\jobname-mathml.html| in the |html|-format.
%  The file name can be changed with |mathml/setfiles={filename1,filename2}|
%  (without extension, |html| is added automatically).
%  If there is a list, all files are loaded.
%  If a file doesn't exist it is ignored, only an info is written to the log.
%
%  Currently every MathML-snippet from a file is embedded into the PDF,
%  it is not checked first if it is actually used (simply writing everything to the PDF
%  is a bit easier than keeping everything in memory and also means that
%  the snippets are one after the other in the PDF).
%
%  As mentioned above the MathML-AF can be suppressed for the equations in a group with
%  |math/mathml/AF=false|, or
%  completely by setting |math/mathml/sources=| in the preamble.
%
%  Files embedded in a PDF can be listed in the attachments panel of a PDF viewer.
%  This is probably not so useful for lots of small files (but one could create
%  collections), but as long as PDF editors or viewers don't offer
%  proper support to access the AF it can help so have them there. The MathML are
%  added by default, but the \LaTeX{} source not. This can be changed with
%  |viewer/pane/mathsource=true| (anywhere in the document) and |viewer/pane/mathml=false| (in the
%  preamble, before the external file is read).
%
%
%  \subsection{Automatic mathml creation with luamml}\label{sec:luamml}
%
%  If lualatex and the package \pkg{unicode-math} is used
%  the package \texttt{luamml} is loaded and
%  will automatically generate the file |\jobname-luamml-mathml.html|
%  with mathml representations of all math formulas.
%  This file is then used in subsequent compilations and works also with
%  pdflatex.
%
%  The generation of the file can be suppressed (in the preamble)
%  with |math/mathml/luamml/write=false|.
%
%  If the package \pkg{unicode-math} is not used,
%  the loading of \pkg{luamml} and with it the generation of the file can be forced
%  with |math/mathml/luamml/load=true| or |math/mathml/luamml/write=true|
%  but be aware that it is then possible that various symbols
%  are mapped to the wrong Unicode code points.
%
%  The package \pkg{luamml} is still quite experimental and the output should be checked.
%  The |\jobname-luamml-mathml.html| file may be previewed in a browser although
%  you may need to add additional css or javascript declarations
%  to enable browser support for all mathml constructs.
%
%
%  \subsection{Summary of math options}
%  The following options exist to make math more accessible:
%  \begin{description}
%  \item[ActualText] An \texttt{ActualText} can be placed on structure elements,
%  but can also be added in the stream on a \texttt{BDC} marker with a \texttt{Span}
%  tag (normally an independant marker without an MCID number, it is not clear yet
%  if it can be used on a MC-chunk).
%  The content is a text string, typically one or a few Unicode characters.
%  \texttt{ActualText} is meant to replaces the content
%  and should only be used on small entities,
%  e.g., to define the semantic or the Unicode code point of a symbol.
%  \texttt{ActualText} is not supported by all PDF reader.
%  It is also unknown where it should be used at best (in a structure element,
%  or on an independent Span-BDC) and what happens if it is used in more than
%  one place.
%   \begin{description}
%   \item[enabled by default?] False
%   \item[how to enable/disable] No interface yet.
%   \texttt{ActualText} can only be added on the Formula structure element by
%   changing the \texttt{tagsupport/math/content} or some other socket.
%   For a BDC marker one can, e.g., use
%   \begin{verbatim}
%   \pdf_string_from_unicode:nnN{utf16/hex}{€}\l_tmpa_tl
%   \pdf_bdc:ee{Span}{/ActualText\l_tmpa_tl}content\pdf_emc:
%   \end{verbatim}
%   There should be no pagebreak in the \meta{content} and the BDC should be correctly
%   nested into tagging, so, e.g., a \cs{leavevmode} should be issued before the bdc command.
%   \item[Consumer support] in part and in part buggy, needs tests \ldots
%   \end{description}
%
%  \item[Alt] Like \texttt{ActualText} the \texttt{Alt} key can be used on
%  structure elements and on \texttt{Span} in the stream. It should contain a description
%  of the content and is mainly meant for images. PDF/UA-1,
%  which views math formulas as illustrations, mandates the key
%  also for \texttt{Formula} structure elements.
%   \begin{description}
%    \item[enabled by default?] false unless PDF/UA-1 is detected,
%    then it is enabled in the begindocument/end hook
%    (this will reconsidered when it is clear, that
%    the use of \texttt{Alt} does not shadow mathml). It can be enabled for
%    all engines and PDF versions.
%   \item[enable/disable] \verb+\tagpdfsetup{math/alt/use}+ (local boolean,
%   so can be used on individual equations)
%   \item[default value] A template text (stored in \cs{l_@@_content_template_tl})
%   starting with \texttt{LaTeX formula starts}.
%   \item[user value] No interface currently provided. This needs optional arguments
%   or an external setup command.
%   See \url{https://github.com/latex3/tagging-project/discussions/717}.
%
%   \end{description}
%
%  \item[source-AF] The \LaTeX{}-source of the equation can
%  be attached as an associated file with mime-type
%  application/Fx-tex. The \texttt{AFRelationship} is \texttt{Source}.
%  The source is embedded without expansion. This means that targets of
%  references and macros are not resolved.
%  The files are by default not shown in the EmbeddedFiles pane,
%  this can be enabled with |viewer/pane/mathsource=true|.
%  If an A-standard is used, it must be one that allows embedded files, e.g., A-4f.
%
%  \begin{description}
%   \item[enabled by default?] true for all engines and PDF versions
%   \item[enable/disable] \verb+\tagpdfsetup{math/tex/AF}+ (local boolean, so can
%   be used on individual equations)
%   \item[default value] source code including dollars or environment name.
%   \item[consumer support] Currently only ngpdf makes
%  use of it: if there is no mathml it passes the source to mathjax.
%   \end{description}
%
%  \item[luamml] The following options make (with lualatex) use
%  of the \pkg{luamml} package. \pkg{luamml} is currently automatically
%  loaded (at the end of the preamble) if \pkg{unicode-math} has been detected.
%  The loading can be forced or suppressed
%  with \verb+\tagpdfsetup{math/mathml/luamml/load=true/false}+.
%
%  \pkg{luamml} affects all math, locally it can be stopped with |math/mathml/ignore|,
%  or by using the commands described in the package.
%
%  \item[mathml-AF] A mathml representation of the equation can be attached
%  to the structure. The configuration possibilities are rather complex as the
%  keys have to control three different  tasks:
%  The \emph{generation} of the file with the mathml fragments,
%  the \emph{reading} and \emph{embedding} of the mathml fragments,
%  and the \emph{association} of a mathml fragment to a specific equation.
%
%  \begin{description}
%  \item[generation]
%   With pdf\LaTeX{} mathml fragments can not be generated automatically,
%   but a file with dummy fragments for every equation will be written if
%   \verb+\tagpdfsetup{math/mathml/write-dummy}+ is issued in the preamble.
%
%   With lua\LaTeX{} a file with mathml fragments will be created automatically
%   if the package \pkg{luamml} has been loaded (see above).
%
%   \item[reading and embedding]
%    By default the code will read and embed
%    mathml from |\jobname-mathml.html| and |\jobname-luamml-mathml.html| in this order and
%    the first fragment with a new hash value will be inserted.%
%    The list of sources and their order can be changed with the key
%    |math/mathml/sources|, setting that to an empty value suppresses
%    the loading mathml associated files completely. For efficiency reasons
%    it embeds math fragments directly, there is no check yet if the fragment is
%    actually used.
%
%    The files are by default shown in the EmbeddedFiles pane,
%    this can be disabled with |viewer/pane/mathml=false|.
%
%   \item[attaching] A mathml fragment is currently
%   attached as an associated file to an Formula if the hash of
%   the source matches the hash of the fragment. This is not a perfect test:
%   equations with the same source and so the same hash
%   can have different mathml representation, e.g.,
%   if there are references or commands or counters in the equation. This
%   will change in a feature version.
%   The attachment can be suppressed locally with |math/mathml/AF=false|.
%   The mathml fragment will still be embedded in the PDF!
%
% % TODO: adapt test
%  \end{description}
%
%  \item[mathml structure elements]
%  Mathml structure elements can be used in PDF~2.0 directly.
%  In PDF~1.7. one could theoretically
%  use them if one declares a role mapping first,
%  (this can be done with \verb+\tagpdfsetup{role/mathml-tags}+)
%  which maps all to \texttt{Span}. But such a role mapping currently breaks reading,
%  e.g. in Adobe, and so it is not recommended.
%
%  Automatic generation of structure elements is only possible with
%  lualatex. It requires that the packages \pkg{luamml} and \pkg{tagpdf}
%  have been loaded.
%  \begin{description}
%   \item[enabled by default?] false
%   \item[enable/disable] \verb+\tagpdfsetup{math/mathml/structelem}+
%   (local setting, so can be used with grouping on individual equations).
%   \item[consumer support] Needs more tests.
%   \end{description}
%
%  \end{description}
%
%
%
% \section{Known current bugs, etc.}
%
% \subsection{Capture/grabbing problems}
%
% \begin{enumerate}
%   \item Incorrect grabbing of |$|-math when there is also
%      explicit |$|-math within a \textit{text environment}
%      that is itself within the math that should all be grabbed.
%      For example,
%       \begin{verbatim}
%        $a\begin{minipage}{1cm}$b$\end{minipage}$
%       \end{verbatim}
%      would only result in the capture of the tokens
%      ``\verb*/a\begin {minipage}{1cm}/''.
%      This can be avoided by an additional brace group:
%       \begin{verbatim}
%        $a{\begin{minipage}{1cm}$b$\end{minipage}}$
%       \end{verbatim}
%
%   \item Similar incorrect grabbing with |$$| also.
%
%   \item The grabbing, for all the display environments (and |\) \]|), needs
%       to deal with nesting: \pkg{amsmath} contains code for this.
%
%   \item The math can't contain verbatim and verbatim-like commands. This is
%   nothing new for the \pkg{amsmath} environments but changes |$| and |\[\]|
%   and |equation*| (see, e.g., tagging-project issue \#30).
%
%   \item Begin and end of the math or math environment can not be hidden in commands.
%   For example \verb+>{$}l<{$}+ in a tabular would lead to errors.
%   Defining |\[| to fall back to |equation*| doesn't work if |equation*| is
%   a grabbing environment.
%
%   \item The behaviour of |\[...\]| is faulty. See above.
% \end{enumerate}
%
% \subsection{Fake math}
% In a number of places in \LaTeX{} math commands (mainly |$|) is used
% only for technical reason, e.g., to access a math font, to setup a symbol
% or to use \cs{vcenter}.
%
% The code identifies such fake math mostly by making use of the \cs{m@th} command
% where two methods are used for the automatic detection:
%
% \begin{itemize}
%  \item After grabbing math content the code checks if the content contains the token
%  \cs{m@th} and if yes it doesn't call the processor before reinserting
%  the content and perhaps adding tagging code.
%  This method requires that the math can be grabbed (e.g. that the end dollar is visible)
%  and that the \cs{m@th} is visible. It applies for example in \cs{@iiiparbox} where the
%  code from |$\vcenter| to |\m@th$| is grabbed an put back. It does not work for
%  example for |tabular| where the dollars and the \cs{m@th} token are spread around
%  over three commands. |tabular| needs therefore manual intervention.
%
%  A look in the list of usages (in \texttt{usage-of-m@th.md}) justifies this approach.
%  All usages are either not math at all, or related to small elements that probably
%  shouldn't be grabbed and processed on their own.
%
%  \item \cs{m@th} is redefined so that it sets the boolean \cs{l_@@_collected_bool}
%  to true. If \cs{m@th} is used inside math that has been grabbed
%  this doesn't change much as the boolean is set by the grabbing anyway. For usages
%  outside math the benefit is not so clear: The setting avoids that in \cs{LaTeXe}
%  the epsilon is processed as math, but it also prevents that the content of the amsmath
%  command \cs{boxed} is processed as math.
%  It means that if one wants to reenable math processing inside some (fake) math
%  one has to do it after \cs{m@th} calls.
% \end{itemize}
%
%  \subsubsection{Open problems}
%
%  \begin{enumerate}
%   \item The grabbing code doesn't pass the info that it detected a \cs{m@th} token.
%   This means that the tagging code has to do the same check (and doesn't do this
%   in all cases yet).
%
%   \item Commands are missing to locally disable the grabbing and processing, e.g.,
%   to handle |tabular|.
%
%   \item It must be checked if setting the boolean in \cs{m@th} really makes sense
%   or if commands like \cs{LaTeXe} should be handled manually.
%
%  \end{enumerate}
%
% \subsection{Processor}
%
% The grabbed math is at first passed to the processor. The processor is not called
% in a measuring phase (from the amsmath \cs{ifmeasuring@}) and if the \cs{m@th}
% token is detected.
% It is not quite clear what purpose the processor has. As it is a public interface
% it can't be used for internal code. And typesetting happens later and the processor
% can't really change this. Currently it is mostly used for debugging and messages.
% If the \cs{m@th} is found the \cs{l_@@_fakemath_bool} is set, so if the code
% is changed this must be preserved.
%
% \subsection{Other problems}
%
% \begin{enumerate}
%   \item
%      The presence of \cs{m@th} in association with \cs{ensuremath}
%      does not necessarily indicate fakemath.  This is because
%      wanting mathsurround to be zero is very reasonable and common,
%      \emph{even when the math is genuine} (and hence needs to be collected).
%
%      TODO: this claim needs some examples.
%
%   \item User-defined environments can create problems; but this area, of
%      new, copied and changed environments, has not yet been developed.
%
%  \car*{Joseph wrote, inter alia:\\
%      My thinking [regarding] \cs{RegisterMathEnvironment}\\
%    - (New) Math environments should not be created-then-patched, but only
%    generated by a [(future)] dedicated command (\cs{DeclareMathEnvironment},
%    presumably)\\
%    - Math environments created with \pkg{ltcmd} [commands] should not be copied, . . .\\
%    - Package authors should be able to manually set up math environments with a public boolean.}
% \end{enumerate}
%
%
% \subsection{Other ToDos}
%
% \begin{enumerate}
%  \item Add (some of) the math display commands that were \enquote{lifted from
%    plain}, e.g., \cs{displaylines} \cs{eqalign}(??).
%  \item The breqn packages changes catcodes and that isn't yet covered
%    by our mechanism.
%  \item \cs{intertext} is not correctly taken into account by the
%  code splitting multiline math into subformulas.
% \end{enumerate}
%
%
% \car*{\cs{MaybeStop} (temporarily) not executed, as it is unknown on Chris' system.}
% \iffalse
%  \MaybeStop{\setlength\IndexMin{200pt}  \PrintIndex  }
% \else
%  \StopEventually{\setlength\IndexMin{200pt}  \PrintIndex  }
% \fi
%
% \section{The Implementation}
%
%    \begin{macrocode}
%<@@=math>
%    \end{macrocode}
%
%    \begin{macrocode}
%<*kernel>
%    \end{macrocode}
%
% \subsection{File declaration}
%
%    \car{Change description here?}
%    \begin{macrocode}
\ProvidesFile{latex-lab-math.ltx}
             [\ltlabmathdate\space
              v\ltlabmathversion\space
              Grab all the math(s) and tag it (experiments)]
%    \end{macrocode}
%
%    Temp loading \ldots
%    \begin{macrocode}
\AddToHook{begindocument/before}{\RequirePackage{latex-lab-testphase-block}}
%    \end{macrocode}
%
%    \begin{macrocode}
\ExplSyntaxOn
%    \end{macrocode}
%
% \subsection{Setup}
%
% Loading \pkg{amsmath} is an absolute requirement: this avoids needing to
% have conditional definitions and deals with how to define \cs{[}/\cs{]}
% neatly.
%    \begin{macrocode}
\AddToHook{begindocument/before}{ \RequirePackage { amsmath } }
%    \end{macrocode}
%
%
% \subsection{Data structures}
%
% \begin{variable}{\l_@@_collected_bool}
%   Tracks whether math mode material has been collected, which happens inside
%   \pkg{amsmath} environments as well as those handled directly here.
%   If true following math will not grab and/or process.
%   See \ref{sec:mathcapture} for details.
%    \begin{macrocode}
\bool_new:N \l_@@_collected_bool
%    \end{macrocode}
% \end{variable}
%
%
% \begin{variable}{\l_@@_fakemath_bool}
%   Tracks whether math mode material has been identified as fake math during
%   the grabbing phase, which happens currently if the
%   grabbed contents contains the \cs{m@th} token.
%
%    \begin{macrocode}
\bool_new:N \l_@@_fakemath_bool
%    \end{macrocode}
% \end{variable}
%
%
%  \car{Change first tl name below: `env' $=>$ `info'?\\
%        Or do we need an extra storage tl?}
%
% \begin{variable}{\g_@@_grabbed_env_tl, \g_@@_grabbed_math_tl}
% \cs{g_@@_grabbed_env_tl} contains the name of the math environment
% (\texttt{math} in the case of inline math,
% \cs{g_@@_grabbed_math_tl} the math content.
%    \begin{macrocode}
\tl_new:N \g_@@_grabbed_env_tl
\tl_new:N \g_@@_grabbed_math_tl
%    \end{macrocode}
% \end{variable}
%
% \begin{variable}{\l_@@_tmpa_tl,\l_@@_tmpa_skip,\l_@@_tmpa_str}
% Temporary variables
%    \begin{macrocode}
\tl_new:N \l_@@_tmpa_tl
\skip_new:N \l_@@_tmpa_skip
\str_new:N \l_@@_tmpa_str
%    \end{macrocode}
% \end{variable}
%
% \begin{variable}{\l_@@_content_alt_tl,
%  \l_@@_content_actual_tl,
%  \l_@@_content_AF_tl}
% Temporary variables to hold math content that should
% be used in actual or alt text and stored as AF.
%    \begin{macrocode}
\tl_new:N \l_@@_content_alt_tl
\tl_new:N \l_@@_content_actual_tl
\tl_new:N \l_@@_content_AF_source_tl
\tl_new:N \l_@@_content_AF_source_tmpa_tl
\tl_new:N \l_@@_content_AF_mathml_tl
%    \end{macrocode}
% \end{variable}
%
% \subsection{Tagging tools}
% The following commands implement small tagging code chunks.
% This should probably be collected and moved into tagpdf later.
% \begin{macro}{\__tag_tool_close_P:}
% This closes a P/text-chunk, both the MC and the structure and
% increases the counter manually.
%    \begin{macrocode}
\cs_new_protected:Npn \__tag_tool_close_P:
  {
    \tag_if_active:T
     {
       \tag_mc_end: %end P-chunk, should perhaps be \tag_mc_end_push: ...
         \__tag_gincr_para_end_int:
         \__tag_check_para_end_show:nn{red}{} %debug: show para
         \tag_struct_end:
     }
  }
%    \end{macrocode}
% \end{macro}
%
%  We add also an attribute.
%    \begin{macrocode}
\tl_new:N\l_@@_attribute_class_tl
\tagpdfsetup
      {role/new-attribute = {inline}    {/O /Layout /Placement/Inline},
       role/new-attribute = {display}   {/O /Layout /Placement/Block},
      }
%    \end{macrocode}
%
% \subsection{Code related to AF}
% Booleans to handle the options.
% \begin{variable}{
%   \l__tag_math_texsource_AF_bool,
%   \l__tag_math_texsource_pane_bool,
%   \l__tag_math_mathml_AF_bool,
%   \g__tag_math_mathml_AF_bool,
%   \l__tag_math_mathml_pane_bool,
%   \l__tag_math_alt_bool,
%   \g__tag_math_luamml_tl,
% }
% The variable \cs{g__tag_math_luamml_tl} is initially 0 and
% the user key can set it to -1 or 1. This allows to distinguish
% the unset case from a value set by the user.
%    \begin{macrocode}
\bool_new:N\l__tag_math_texsource_AF_bool
\bool_new:N\l__tag_math_texsource_pane_bool
\bool_new:N\l__tag_math_mathml_AF_bool
\bool_new:N\g__tag_math_mathml_AF_bool
\bool_new:N\l__tag_math_mathml_pane_bool
\bool_new:N\l__tag_math_alt_bool
\tl_new:N\g__tag_math_luamml_tl
\tl_gset:Nn\g__tag_math_luamml_tl {0}
%    \end{macrocode}
% \end{variable}
%
% \begin{variable}{
%    \g_@@_mathml_total_int,
%    \g_@@_mathml_int,
%    \g_@@_math_total_int,
%    \g_@@_mathml_AF_found_int,
%    \g_@@_mathml_AF_attached_int,
%    }
% \cs{g_@@_mml_total_int} records the mathml fragments read in.
% \cs{g_@@_mml_int} records the mathml fragments read in with a different hash.
% \cs{g_@@_AF_total_int} records the number of math structures that try to
% attach a mathml AF.
% \cs{g_@@_AF_found_int} records the number of math structures for which a fitting
% mathml is found.
% \cs{g_@@_AF_attached_int} records the number of math structures which got a mathml fragment
% (if mathml-AF are not disabled locally this should be the equal to the previous number.
%
%    \begin{macrocode}
\int_new:N\g_@@_mathml_total_int
\int_new:N\g_@@_mathml_int
\int_new:N\g_@@_math_total_int
\int_new:N\g_@@_mathml_AF_found_int
\int_new:N\g_@@_mathml_AF_attached_int
%    \end{macrocode}
% \end{variable}
%
% \begin{variable}{\l__tag_math_mathml_files_clist}
% A sequence to store the file list for the mathml.
% We also check the luamml file.
%    \begin{macrocode}
\clist_new:N\l__tag_math_mathml_files_clist
\clist_put_right:Ne\l__tag_math_mathml_files_clist
  {\c_sys_jobname_str-mathml,\c_sys_jobname_str-luamml-mathml}
%    \end{macrocode}
% \end{variable}
%
% This is the internal variant of the \cs{mml} command.
% \begin{macro}{\@@_AF_mml:nnnn}
%    \begin{macrocode}
\cs_new_protected:Npn \@@_AF_mml:nnnn #1 #2 #3 #4
%#1 number, #2 tex source for debugging, #3 hash, #4 mathml
  {
    \int_gincr:N \g_@@_mathml_total_int
%    \end{macrocode}
% mathml with the same hash should be included only once:
%    \begin{macrocode}
    \tl_if_exist:cF { g_@@_mathml_#3_tl }
     {
       \int_gincr:N \g_@@_mathml_int
%    \end{macrocode}
%  a simple Desc key, take care that it is a valid string!
%    \begin{macrocode}
       \pdfdict_put:nne {l_pdffile/Filespec} {Desc}{(mathml-#1)}
       \pdffile_embed_stream:nnN {#4}{mathml-#1.xml}\l_@@_tmpa_tl
%    \end{macrocode}
%  not strictly necessary but makes the files visible in the file attachment
%  page
%    \begin{macrocode}
       \bool_if:NT \l__tag_math_mathml_pane_bool
        {\pdfmanagement_add:nne {Catalog/Names}{EmbeddedFiles}{\l_@@_tmpa_tl}}
       \tl_new:c{g_@@_mathml_#3_tl}
       \tl_gset_eq:cN{g_@@_mathml_#3_tl}\l_@@_tmpa_tl
     }
  }
%    \end{macrocode}
% \end{macro}
%
% The html reader.
%    \begin{macrocode}
\cs_new_protected:Npn \@@_AF_html_reader:w#1</h2>#2<p>#3</p>#4<p>#5</p>#6<math{
  \begingroup
   \char_set_catcode_other:N\{
   \char_set_catcode_other:N\}
   \char_set_catcode_other:N\#
   \char_set_catcode_other:N\%
   \@@_AF_html_reader_verb:w{#1}{#3}{#5}<math
}
%    \end{macrocode}
%    \begin{macrocode}
\cs_new_protected:Npn\@@_AF_html_reader_verb:w#1#2#3#4~</div>{
  \endgroup
   \@@_AF_mml:nnnn{#1}{#2}{#3}{#4}
   }
%    \end{macrocode}
%
% As with luatex we write two files we define a few constants for
% the shared texts.
%
% \begin{macro}
%  {\c_@@_mathml_write_init_tl,\l_@@_mathml_write_before_tl,
%   \c_@@_mathml_write_after_tl,\c_@@_mathml_write_final_tl}
%    \begin{macrocode}
\tl_const:Nn \c_@@_mathml_write_init_tl
  {
    <!DOCTYPE~html>
    \iow_newline:
    <html~ xmlns="http://www.w3.org/1999/xhtml">
    \iow_newline:
  }
\tl_new:N \l_@@_mathml_write_before_tl
\tl_const:Nn \c_@@_mathml_write_after_tl
  {
    \iow_newline:
    </div>
    \iow_newline:
  }
\tl_const:Nn \c_@@_mathml_write_final_tl
  {
    </html>
  }
%    \end{macrocode}
% \end{macro}

% \begin{socketdecl}{tagsupport/math/mathml/write/prepare}
% To prepare the hash and the starting command we use a socket, so
% that both the dummy and luamml can make use of it.
%    \begin{macrocode}
\socket_new:nn {tagsupport/math/mathml/write/prepare}{0}
%    \end{macrocode}
% \end{socketdecl}
%
% \begin{plugdecl}{On}
%    \begin{macrocode}
\socket_new_plug:nnn{tagsupport/math/mathml/write/prepare}{On}
  {
    \str_set:NV\l_@@_tmpa_str\l_@@_content_AF_source_tl
    \str_replace_all:Nnn\l_@@_tmpa_str{&}{&amp;}
    \str_replace_all:Nnn\l_@@_tmpa_str{<}{&lt;}
    \tl_set:Nn \l_@@_mathml_write_before_tl
      {
        <div>
        \iow_newline:
        <h2>\c_backslash_str mml\c_space_tl \int_use:N \g_@@_math_total_int </h2>
        \iow_newline:
        <p>\l_@@_tmpa_str</p>
        \iow_newline:
        <p>\l_@@_content_hash_tl </p>
        \iow_newline:
      }
  }
%    \end{macrocode}
% \end{plugdecl}
%
% With luatex we automatically generate mathml with \pkg{luamml} if the package
% can be loaded and \pkg{unicode-math} is detected.
% We start the process in the begindocument/end hook
% so that the reading from a previous compilation can happen before!
%
% For other engines, for future name changes
% and in case luamml is not loaded we provide
% some commands
%    \begin{macrocode}
\cs_new_protected:Npn\@@_provide_luamml_commands:
  {
    \providecommand\luamml_flag_structelem:{}
    \cs_if_free:NT \luamml_structelem:
     {
       \cs_set_eq:NN\luamml_structelem:\luamml_flag_structelem:
     }
    \providecommand\luamml_flag_process:{}
    \cs_if_free:NT \luamml_process:
     {
       \cs_set_eq:NN\luamml_process:\luamml_flag_process:
     }
    \providecommand\luamml_flag_ignore:{}
    \cs_if_free:NT \luamml_ignore:
     {
       \cs_set_eq:NN\luamml_ignore:\luamml_flag_ignore:
     }
  }
%    \end{macrocode}
%    \begin{macrocode}
\sys_if_engine_luatex:TF
 {
%    \end{macrocode}
% Temporary (!) fixes for endarray, can be removed when array 2.6h is released.
%    \begin{macrocode}
  \socket_if_exist:nF {tagsupport/math/luamml/array/save}
   {
     \NewSocket{tagsupport/math/luamml/array/save}{0}
     \NewSocket{tagsupport/math/luamml/array/finalize}{0}
     \NewSocket{tagsupport/math/luamml/array/initcol}{0}
     \NewSocket{tagsupport/math/luamml/array/savecol}{0}
     \NewSocket{tagsupport/math/luamml/array/finalizecol}{1}
     \AssignSocketPlug{tagsupport/math/luamml/array/finalizecol}{noop}
   }
   \cs_new_protected:Npn \@@_correct_luamml_array_patches:
     {
       \AddToHook{package/array/after}
        {
          \cs_set:Npn \endarray
           {
             \tbl_crcr:n{endarray}
             \tag_socket_use_expandable:n { math/luamml/array/save }
             \egroup
             \UseTaggingSocket{tbl/finalize}
             \tbl_restore_outer_cell_data:
             \egroup
             \tag_socket_use:n { math/luamml/array/finalize }
             \@arrayright
             \gdef \@preamble {}
           }
         \cs_set:Npn \@classz
           {
            \@classx
            \@tempcnta \count@
            \prepnext@tok
            \@addtopreamble {
              \ifcase \@chnum
                \hfil
                \hskip 1sp
                \d@llarbegin
                \cs_if_eq:NNTF \d@llarbegin \begingroup {
                  \insert@column
                  \d@llarend
                } {
                  \tag_socket_use:n { math/luamml/array/initcol }
                  \insert@column
                  \tag_socket_use:n { math/luamml/array/savecol }
                  \d@llarend
                  \tag_socket_use:nn { math/luamml/array/finalizecol}{0}
                }
                \do@row@strut
                \hfil
              \or
                \hskip 1sp
                \d@llarbegin
                \cs_if_eq:NNTF \d@llarbegin \begingroup {
                  \insert@column
                  \d@llarend
                } {
                  \tag_socket_use:n { math/luamml/array/initcol }
                  \insert@column
                  \tag_socket_use:n { math/luamml/array/savecol }
                  \d@llarend
                  \tag_socket_use:nn { math/luamml/array/finalizecol}{1}
                }
                \do@row@strut
                \hfil
              \or
                \hfil
                \hskip 1sp
                \d@llarbegin
                \cs_if_eq:NNTF \d@llarbegin \begingroup {
                  \insert@column
                  \d@llarend
                } {
                  \tag_socket_use:n { math/luamml/array/initcol }
                  \insert@column
                  \tag_socket_use:n { math/luamml/array/savecol }
                  \d@llarend
                  \tag_socket_use:nn { math/luamml/array/finalizecol}{2}
                }
                \do@row@strut
              \or
                \setbox \ar@mcellbox \vbox \@startpbox { \@nextchar }
                  \insert@pcolumn
                \@endpbox
                \ar@align@mcell
                \do@row@strut
              \or
                \vtop \@startpbox { \@nextchar }
                  \insert@pcolumn
                \@endpbox
                \do@row@strut
              \or
                \vbox \@startpbox { \@nextchar }
                  \insert@pcolumn
                \@endpbox
                \do@row@strut
              \fi
            }
            \prepnext@tok
          }
        }
     }
%    \end{macrocode}
%
%    \begin{macrocode}
   \AddToHook{begindocument/before}
     {
       \str_case:on \g_@@_luamml_load_tl
         {
           { 1 } {
                   \RequirePackage  { luamml }
                   \@@_correct_luamml_array_patches:
                   \AddToHook{begindocument/end}
                    {
                      \@@_luamml_activate_write:
                    }
                 }
           {-1 } {
                   \AddToHook{begindocument/end}
                    {
                     \msg_note:nnnn { tag }
                     { luamml-status }{ disabled }{ not~create }
                    }
                 }
           { 0 }
           {
             \@ifpackageloaded { unicode-math }
              {
                \RequirePackage  { luamml }
                \@@_correct_luamml_array_patches:
                \AddToHook{begindocument/end}
                  {
                    \@@_luamml_activate_write:
                  }
              }
              { \msg_warning:nn { tag }{ unicode-math-missing } }
           }
         }
         \@@_provide_luamml_commands:
     }
 }
 {
   \@@_provide_luamml_commands:
 }
\msg_new:nnn { tag }{ luamml-status }
  {
    luamml~has~been~#1~and~will~#2~an~MathML~file.
  }

\msg_new:nnn { tag }{ unicode-math-missing }
  {
    The~package~unicode-math~is~missing\\
    luamml~will~not~create~an~MathML~file.\\
    To~avoid~this~warning~load~unicode-math~\\
    or~disable~luamml~with~\\
    \tl_to_str:n{\tagpdfsetup{math/mathml/luamml/load=false}}\\
    or~force~luamml~with~\\
    \tl_to_str:n{\tagpdfsetup{math/mathml/luamml/load=true}}
  }
\cs_new_protected:Npn \@@_luamml_activate_write:
 {
   \bool_if:NT \g_@@_luamml_write_bool
     {
%    \end{macrocode}
% to avoid that nothing is written in the first run, we must activate the sockets:
%    \begin{macrocode}
       \bool_gset_true:N\g__tag_math_mathml_AF_bool
       \AssignSocketPlug{tagsupport/math/struct/begin}{mathml-AF}
       \AssignSocketPlug{tagsupport/math/struct/end}{mathml-AF}
       \AssignSocketPlug{tagsupport/math/substruct/begin}{single}
       \AssignSocketPlug{tagsupport/math/substruct/end}{single}
       \int_set:Nn \l__luamml_pretty_int { 7 }
       \RegisterFamilyMapping\symsymbols{oms}
       \RegisterFamilyMapping\symletters{oml}
       \AssignSocketPlug{tagsupport/math/mathml/write/prepare}{On}
       \iow_new:N   \g_@@_luamml_iow
       \iow_open:Nn \g_@@_luamml_iow {\c_sys_jobname_str-luamml-mathml.html}
       \iow_now:Ne  \g_@@_luamml_iow { \c_@@_mathml_write_init_tl  }
       \cs_new:Npn  \@@_luamml_output_hook:n  ##1
         {
           \tl_if_empty:NF \l_@@_mathml_write_before_tl
             {
%    \end{macrocode}
% We check here if the current group level is equal to the one stored for the
% outer math. If not we add currently only add text to change the hash.
% \changes{0.6k}{2024-12-04}{Test for current group level}
%    \begin{macrocode}
              \int_compare:nNnF
               { \@math@level } = { 1 }
               { \tl_put_right:Ne \l_@@_content_hash_tl {-INNER} }
              \iow_now:Ne \g_@@_luamml_iow
               {
                 \l_@@_mathml_write_before_tl
                 ##1
                 \c_@@_mathml_write_after_tl
               }
             }
         }
       \__luamml_register_output_hook:N \@@_luamml_output_hook:n
%    \end{macrocode}
% At the end of the document we must finish and close the file:
%    \begin{macrocode}
      \AddToHook{enddocument/afterlastpage}
        {
          \iow_now:Ne \g_@@_luamml_iow
            { \c_@@_mathml_write_final_tl }
          \iow_close:N \g_@@_luamml_iow
        }
      \msg_note:nnnn { tag }
        { luamml-status }{ enabled }{ create }
     }
 }
%    \end{macrocode}
%  And now  keys to activate/deactivate luamml feature
% \begin{variable}{\g_@@_luamml_load_tl}
% This variable will be used to suppress the loading of luamml
% altogether.
%    \begin{macrocode}
\tl_new:N  \g_@@_luamml_load_tl
\tl_gset:Nn \g_@@_luamml_load_tl {0}
%    \end{macrocode}
% \end{variable}
% \begin{variable}{\g_@@_luamml_write_bool}
% This variable decides if luamml writes a mathml
% altogether.
%    \begin{macrocode}
\bool_new:N  \g_@@_luamml_write_bool
\bool_gset_true:N \g_@@_luamml_write_bool
%    \end{macrocode}
% \end{variable}
% \begin{macro}{\@@_luamml_ignore:,\@@_luamml_structelem:}
% Internal variants of the luamml commands, that can be remapped if needed.
%    \begin{macrocode}
\cs_new:Npn\@@_luamml_structelem:{}
\cs_new:Npn\@@_luamml_ignore:{}
%    \end{macrocode}
% \end{macro}
%
%    \begin{macrocode}
\msg_new:nnn { tag }{ PDF-2.0-recommended }
  {
    The~key~#1~will~not~work~properly~with~PDF~#2.\\
    Switching~to~PDF~2.0~is~recommended.
  }
\keys_define:nn { __tag / setup }
   {
%    \end{macrocode}
% At first a key to suppress the loading altogether
%    \begin{macrocode}
     math/mathml/luamml/load .choice: ,
     math/mathml/luamml/load/true  .code:n = {\tl_gset:Nn \g_@@_luamml_load_tl{1}},
     math/mathml/luamml/load/false .code:n = {\tl_gset:Nn \g_@@_luamml_load_tl{-1}},
     math/mathml/luamml/load .default:n = true,
     math/mathml/luamml/load .usage:n=preamble,
%    \end{macrocode}
% A key to activate math structure elements.
% \changes{v0.6j}{2024-11-19}{no longer enable globally for better fake math handling, issue \#764}
%    \begin{macrocode}
     math/mathml/structelem .choice:,
     math/mathml/structelem/true .code:n =
      {
        \pdf_version_compare:NnT < {2.0}
         {
          \msg_warning:nnne { tag }{ PDF-2.0-recommended }
           { math/mathml/structelem }{ \pdf_version: }
         }
        \cs_set:Npn\@@_luamml_structelem:{\luamml_structelem:}
        \cs_set:Npn\@@_luamml_ignore:{\luamml_ignore:}
      },
     math/mathml/structelem/false .code:n =
      {
        \cs_set_eq:NN\@@_luamml_structelem:\prg_do_nothing:
        \cs_set_eq:NN\@@_luamml_ignore:\prg_do_nothing:
      },
     math/mathml/structelem .default:n = true,
%    \end{macrocode}
% and a key to call the ignore flag. This should only be used locally.
%    \begin{macrocode}
     math/mathml/ignore .code:n = {\luamml_ignore:},
%    \end{macrocode}
%    \begin{macrocode}
     math/mathml/luamml/write .choice:,
     math/mathml/luamml/write/true .code:n =
      {
        \tl_gset:Nn \g_@@_luamml_load_tl{1}
        \bool_gset_true:N \g_@@_luamml_write_bool
      },
     math/mathml/luamml/write/false .code:n =
      {
        \bool_gset_false:N \g_@@_luamml_write_bool
      },
     math/mathml/luamml/write .default:n = true,
     math/mathml/luamml/write .usage:n=preamble,
%    \end{macrocode}
% alias keys for compatibility
%    \begin{macrocode}
     math/mathml/luamml .bool_gset:N = \g_@@_luamml_write_bool,
     math/mathml/luamml .usage:n=preamble
   }
%    \end{macrocode}
% \begin{socketdecl}{tagsupport/math/mathml/write}
% This writes a html-dummy with the hash and the math content.
% This should be optional, so it uses a socket that can be disabled
%
%    \begin{macrocode}
\socket_new:nn {tagsupport/math/mathml/write}{0}
%    \end{macrocode}
% \end{socketdecl}
%
% \begin{plugdecl}{On}
%    \begin{macrocode}
\socket_new_plug:nnn{tagsupport/math/mathml/write}{On}
 {
    \iow_now:Ne \g_@@_writedummy_iow
     {
      \l_@@_mathml_write_before_tl
      <math~ xmlns="http://www.w3.org/1998/Math/MathML"></math>
      \c_@@_mathml_write_after_tl
      }
 }
%    \end{macrocode}
% \end{plugdecl}
% And now a key to activate the socket.
%    \begin{macrocode}

\keys_define:nn { __tag / setup }
   {
     math/mathml/write-dummy .code:n =
       {
         \bool_gset_true:N \g__tag_math_mathml_AF_bool
         \tl_if_exist:NF\g_@@_writedummy_iow
          {
            \iow_new:N  \g_@@_writedummy_iow
            \iow_open:Nn \g_@@_writedummy_iow
             {
               \c_sys_jobname_str-mathml-dummy.html
             }
            \iow_now:Ne \g_@@_writedummy_iow
             {
               \c_@@_mathml_write_init_tl
             }
            \AssignSocketPlug {tagsupport/math/mathml/write/prepare}{On}
            \AssignSocketPlug {tagsupport/math/mathml/write}{On}
            \AddToHook{enddocument/afterlastpage}
             {
               \iow_now:Ne \g_@@_writedummy_iow
                 { \c_@@_mathml_write_final_tl }
               \iow_close:N \g_@@_writedummy_iow
             }
          }
       },
     math/mathml/write-dummy .usage:n=preamble
   }
%    \end{macrocode}
%
% \begin{macro}{\@@_AF_process_mathml_files:}
%    \begin{macrocode}
\box_new:N\l_@@_tmpa_box
\cs_new_protected:Npn \@@_AF_process_mathml_files:
 {
   \hbox_set:Nn \l_@@_tmpa_box
    {
      \pdfdict_put:nnn { l_pdffile/Filespec }{AFRelationship} { /Supplement }
      \pdfdict_put:nne
       { l_pdffile }{Subtype}
       { \pdf_name_from_unicode_e:n{application/mathml+xml} }
      \char_set_catcode_other:N \#
      \cs_set_eq:NN\mml \@@_AF_html_reader:w
      \clist_map_inline:Nn \l__tag_math_mathml_files_clist
        {
          \file_if_exist:nTF {##1.html}
            {
              \typeout{Info:~reading~mathml~file~##1}
              \file_input:n {##1.html}
              \bool_gset_true:N\g__tag_math_mathml_AF_bool
            }
            {
              \typeout{Info:~mathml~file~##1~does~not~exist}%info message
            }
        }
    }
    \bool_if:NT\g__tag_math_mathml_AF_bool
      {
        \typeout{Info:~Activating~mathml~support}
        \AssignSocketPlug{tagsupport/math/struct/begin}{mathml-AF}
        \AssignSocketPlug{tagsupport/math/struct/end}{mathml-AF}
%    \end{macrocode}
% mathml handling doesn't like subparts, so we disable them for now:
%    \begin{macrocode}
        \AssignSocketPlug{tagsupport/math/substruct/begin}{single}
        \AssignSocketPlug{tagsupport/math/substruct/end}{single}
        \AddToHook{enddocument/info}
         {
           \iow_term:n{MathML~statistic}
           \iow_term:n{================}
           \iow_term:e{==>~\int_use:N\g_@@_mathml_total_int\c_space_tl
           MathML~fragments~read}
           \iow_term:e{==>~\int_use:N\g_@@_mathml_int\c_space_tl
           different~MathML~fragments}
           \iow_term:e{==>~\int_use:N\g_@@_math_total_int\c_space_tl
           math~fragments~found}
           \iow_term:e{==>~\int_use:N\g_@@_mathml_AF_found_int\c_space_tl
           fitting~MathML~AF~found}
           \iow_term:e{==>~\int_use:N\g_@@_mathml_AF_attached_int\c_space_tl
           MathML~AF~attached}
         }
      }
 }
\AddToHook{begindocument}{\@@_AF_process_mathml_files:}
%    \end{macrocode}
% \end{macro}
%
% \subsection{Mathstyle detection}
% In some cases we need to detect the mathstyle used in a \cs{mathchoice}
% command and to disable/enable tagging in the unused branches.
% This is currently only used in the amstext command \cs{text}
% but is perhaps also needed in other cases, so we create a general command.
%
%\begin{macro}{\l_@@_mathstyle_int,\g_@@_mathchoice_int,mathstyle}
%    \begin{macrocode}
\int_new:N \l_@@_mathstyle_int
\int_new:N \g_@@_mathchoice_int
\property_new:nnnn{mathstyle}{now}{-1}{\int_use:N \l_@@_mathstyle_int }
%    \end{macrocode}
%\end{macro}
% For now internal, but perhaps will need a public version.
% The command should be used in every branch of a \cs{mathchoice}
% (with the correct mathstyle number) and with an unique label (which should
% be the same in every branch).
% \cs{g_@@_mathchoice_int} can be, e.g., increased before the mathchoice and
% then used.
% \begin{macro}{\@@_tag_if_mathstyle:nn}
%    \begin{macrocode}
\cs_new_protected:Npn \@@_tag_if_mathstyle:nn #1 #2
 %#1 refers to label
 %#2 is a number for the mathstyle (typically 0,2,4,6)
 {
   \int_set:Nn \l_@@_mathstyle_int {#2}
   \property_record:nn {#1} { mathstyle }
   \int_compare:nNnTF { \property_ref:nn {#1}{ mathstyle} } = { #2 }
    { \tag_resume:n{\mathchoice} }{ \tag_suspend:n{\mathchoice} }
 }
\cs_generate_variant:Nn \@@_tag_if_mathstyle:nn {en}
%    \end{macrocode}
% \end{macro}
%
% \subsection{Tagging options}
%    \begin{macrocode}
\keys_define:nn { __tag / setup }
  {
   math/mathml/sources .clist_set:N = \l__tag_math_mathml_files_clist,
   math/alt/use        .bool_set:N  = \l__tag_math_alt_bool,
   viewer/pane/mathml      .bool_set:N = \l__tag_math_mathml_pane_bool,
   viewer/pane/mathml      .initial:n = true,
   viewer/pane/mathsource  .bool_set:N = \l__tag_math_texsource_pane_bool,
   math/mathml/AF .bool_set:N = \l__tag_math_mathml_AF_bool,
   math/mathml/AF  .initial:n = true,
   math/tex/AF    .bool_set:N = \l__tag_math_texsource_AF_bool,
   math/tex/AF    .initial:n = true
  }
%    \end{macrocode}
% alt is required for pdf/UA-1.
% TODO: l3pdfmeta should support this test.
%    \begin{macrocode}
\AddToHook{begindocument/end}
 {
   \str_if_eq:eeT
    {1}
    {
        \exp_last_unbraced:Ne\use_i:nn
         {\GetDocumentProperties{document/pdfstandard-UA}}
         \c_empty_tl\c_empty_tl
    }
    {
      \bool_if:NF \l__tag_math_alt_bool
       {
         \typeout{PDF/UA-1~detected.~Enabling~alt~text~on~Formula}
       }
      \bool_set_true:N\l__tag_math_alt_bool
    }
 }
%    \end{macrocode}
%
% \subsubsection{Meta keys}
% The |math/setup| key accepts a list with the values |structelem|, |mathml-AF| and |tex-AF|.
% It is a fast way to set the main option. It at first disables them all, to get a clean state.
%    \begin{macrocode}
\keys_define:nn {__tag / setup}
  {
    math/setup .code:n =
     {
       %deactivate loading of luamml
       \tl_gset:Nn \g_@@_luamml_load_tl{-1}
       \keys_set:nn {__tag / setup}
        {
          %deactivate tex source AF
          math/tex/AF = false,
          %deactivate reading of mathml-AF
          math/mathml/sources=,
          math/mathml/AF=false,
          %deactivate structelem
          math/mathml/structelem=false,
          %handle value
        }
       \clist_map_inline:nn { #1}
        {
          \keys_set:nn {__tag/ setup}{math/__setup/##1}
        }
     },
    math/__setup / structelem .code:n =
     {
      \tl_gset:Nn \g_@@_luamml_load_tl{1}
      \keys_set:nn {__tag / setup}
        {
          math/mathml/structelem=true
        }
     },
    math/__setup / mathml-AF .code:n =
     {
      \tl_gset:Nn \g_@@_luamml_load_tl{1}
      \clist_put_right:Ne\l__tag_math_mathml_files_clist
       {\c_sys_jobname_str-mathml,\c_sys_jobname_str-luamml-mathml}
      \keys_set:nn {__tag / setup}
        {
          math/mathml/AF=true
        }
     },
    math/__setup / tex-AF .code:n =
     {
      \keys_set:nn {__tag / setup}
        {
          math/tex/AF =true
        }
     },
  }
%    \end{macrocode}
% \subsection{Sockets}
% \subsubsection{Main inline math sockets}
%
% \begin{socketdecl}
%   {
%     tagsupport/math/inline/begin,
%     tagsupport/math/inline/end,
%     tagsupport/math/inline/formula/begin,
%     tagsupport/math/inline/formula/end,
%   }
%   The first two sockets are meant to embed inline
%   math into the surrounding (so to close/reopen, e.g., MC-chunks).
%   The other two implement the actual formula structure.
%   The formula sockets are despite their naming not symmetric:
%   the begin socket is issued after the math has started, while
%   the end socket is after the math!
%   \changes{v0.6j}{2024-11-19}{change the socket to two arguments so that it can
%   be used as tagging socket}
%    \begin{macrocode}
\socket_new:nn {tagsupport/math/inline/begin}{0}
\socket_new:nn {tagsupport/math/inline/end}{0}
\socket_new:nn {tagsupport/math/inline/formula/begin}{2} %
\socket_new:nn {tagsupport/math/inline/formula/end}{0}
%    \end{macrocode}
%\end{socketdecl}
%
%
% \begin{plugdecl}{MC}
%    \begin{macrocode}
\socket_new_plug:nnn
  {tagsupport/math/inline/begin}
  {MC}
  {\tag_mc_end_push:}
\socket_new_plug:nnn
  {tagsupport/math/inline/end}
  {MC}
  {\tag_mc_begin_pop:n{}}
%    \end{macrocode}
% \end{plugdecl}
%
% We probably will want to test different tagging recipes.
% \begin{plugdecl}{default}
%    \begin{macrocode}
\socket_new_plug:nnn
  {tagsupport/math/inline/formula/begin}
  {default}
%    \end{macrocode}
% \changes{v0.6g}{2024-10-02}{disable paratagging, issue \#711}
% \changes{v0.6j}{2024-11-19}{activate structelem locally issue \#764}
% \changes{v0.6j}{2024-11-19}{change to two arguments}
%    \begin{macrocode}
  { \tagpdfparaOff
    \@@_luamml_structelem:
    \tag_socket_use:n{math/content}
    \tag_socket_use:n{math/struct/begin}
%    \end{macrocode}
% TODO: does inline math need subformula handling?
%    \begin{macrocode}
    % inner formula if multiple parts (not really implemented yet)
    \tag_socket_use:n{math/substruct/begin}
    #2
    \tag_socket_use:n{math/end}
  }
\socket_new_plug:nnn
  {tagsupport/math/inline/formula/end}
  {default}
  {
    \socket_use:n{tagsupport/math/substruct/end}
    \socket_use:n{tagsupport/math/struct/end}
  }
%    \end{macrocode}
% \end{plugdecl}
%
% \subsubsection{Main display math sockets}
%
% \begin{socketdecl}
%   {
%     tagsupport/math/display/begin,
%     tagsupport/math/display/end,
%     tagsupport/math/display/formula/begin,
%     tagsupport/math/display/formula/end,
%   }
%   The first two sockets are meant to embed display
%   math into the surrounding (so to close/reopen, e.g., MC-chunks and
%   P-structure).
%   The other two implement the actual formula structure.
%   The formula sockets are despite their naming not symmetric:
%   the begin socket is issued after the math has started, while
%   the end socket is after the math!
%   \changes{v0.6j}{2024-11-19}{change number of arguments of formula/begin socket}
%    \begin{macrocode}
\socket_new:nn {tagsupport/math/display/begin}{0}
\socket_new:nn {tagsupport/math/display/end}{0}
\socket_new:nn {tagsupport/math/display/formula/begin}{2} %
\socket_new:nn {tagsupport/math/display/formula/end}{0}
%    \end{macrocode}
%\end{socketdecl}

% \begin{plugdecl}{default}
%    \begin{macrocode}
\socket_new_plug:nnn
  {tagsupport/math/display/begin}
  {default}
  { \__tag_tool_close_P:  }
\socket_new_plug:nnn
  {tagsupport/math/display/end}
  {default}
  {
  }
%    \end{macrocode}
% \end{plugdecl}


% \begin{plugdecl}{default}
% \changes{v0.6j}{2024-11-19}{moved \cs{tagpdfparaOff} into the socket, tagging/765}
%    \begin{macrocode}
\socket_new_plug:nnn
  {tagsupport/math/display/formula/begin}
  {default}
  {
    \tagpdfparaOff
    \@@_luamml_structelem:
    \tag_socket_use:n{math/content}
    \tag_socket_use:n{math/struct/begin}
    \tag_socket_use:n{math/substruct/begin}
    #2
    \tag_socket_use:n{math/end}
  }
\socket_new_plug:nnn
  {tagsupport/math/display/formula/end}
  {default}
  {
    \socket_use:n{tagsupport/math/substruct/end}
    \socket_use:n{tagsupport/math/struct/end}
  }
%    \end{macrocode}
% \end{plugdecl}
%
% \subsubsection{Internal sockets}
%
% \begin{variable}{\l_@@_content_template_tl}
% The default text used as alt or actual text.
%    \begin{macrocode}
\tl_new:N\l_@@_content_template_tl
\tl_set:Nn \l_@@_content_template_tl
   {
       LaTeX~ formula~ starts~
       \exp_not:N\begin{\g_@@_grabbed_env_tl}
       \c_space_tl
       \exp_not:V\g_@@_grabbed_math_tl
       \c_space_tl
       \exp_not:N\end{\g_@@_grabbed_env_tl}
       \c_space_tl LaTeX~ formula~ ends~
   }
%    \end{macrocode}
% \end{variable}

% \begin{variable}{\l_@@_texsource_template_tl}
% The default text used as texsource
%    \begin{macrocode}
\tl_new:N\l_@@_texsource_template_tl
\tl_const:Nn\c_@@_inline_env_tl {math}
\tl_set:Nn \l_@@_texsource_template_tl
   {
     \tl_if_eq:NNTF\g_@@_grabbed_env_tl\c_@@_inline_env_tl
      {
       $
         \exp_not:V\g_@@_grabbed_math_tl
       $
      }
      {
       \exp_not:N\begin{\g_@@_grabbed_env_tl}
       \exp_not:V\g_@@_grabbed_math_tl
       \exp_not:N\end{\g_@@_grabbed_env_tl}
      }
   }
%    \end{macrocode}
% \end{variable}

%
% \begin{socketdecl}{tagsupport/math/content}
% The math content is stored in associated files and used for
% actual and alternative text. As the exact text is still
% unclear we use a socket to be able to test variants.
% The socket should set all four tl vars above, if needed
% to identical values. It can use the two variables
% \cs{g_@@_grabbed_env_tl} and \cs{g_@@_grabbed_math_tl}
%    \begin{macrocode}
\socket_new:nn {tagsupport/math/content}{0}
%    \end{macrocode}
% \end{socketdecl}
%
% Some default sockets to set the contents.
% TODO: think about naming convention.
% TODO: think how this should organized so that one
% has options to change from the outside and so that
% there are less repetitions.
% \begin{plugdecl}{actual+source}
%    \begin{macrocode}
\socket_new_plug:nnn
  {tagsupport/math/content}
  {actual+source}
  {
   \tl_set:Ne\l_@@_content_actual_tl
    {
       \l_@@_content_template_tl
    }
   \tl_set:Ne \l_@@_content_AF_source_tl
    {
      \l_@@_texsource_template_tl
    }
   \tl_set:Nn    \l_@@_content_AF_mathml_tl {}
   \tl_set:Nn    \l_@@_content_alt_tl    {}
  }
%    \end{macrocode}
% \end{plugdecl}
%
% \begin{plugdecl}{alt+source}
%    \begin{macrocode}
\socket_new_plug:nnn
  {tagsupport/math/content}
  {alt+source}
  {
   \tl_set:Ne\l_@@_content_alt_tl
    {
       \l_@@_content_template_tl
    }
   \tl_set:Ne \l_@@_content_AF_source_tl
    {
       \l_@@_texsource_template_tl
    }
   \tl_set:Nn    \l_@@_content_AF_mathml_tl {}
   \tl_set:Nn    \l_@@_content_actual_tl    {}
  }
%    \end{macrocode}
% \end{plugdecl}
%    \begin{macrocode}
\socket_assign_plug:nn {tagsupport/math/content}{alt+source}
%    \end{macrocode}
%
% \begin{socketdecl}{tagsupport/math/struct/begin,
%  tagsupport/math/struct/end}
% For the main structure we use a socket too.
% This allows, e.g., to create a special one for luamml
% which setups additional objects.
% The begin socket can use the two variables
% \cs{g_@@_grabbed_env_tl} and \cs{g_@@_grabbed_math_tl}
%    \begin{macrocode}
\socket_new:nn {tagsupport/math/struct/begin}{0}
\socket_new:nn {tagsupport/math/struct/end}{0}
%    \end{macrocode}
% \end{socketdecl}
%
% \begin{plugdecl}{default}
% TODO: think about some naming convention ...
%    \begin{macrocode}
\socket_new_plug:nnn
  {tagsupport/math/struct/begin}
  {default}
  {
    \bool_if:NTF\l__tag_math_texsource_AF_bool
     { \tl_set_eq:NN \l_@@_content_AF_source_tmpa_tl \l_@@_content_AF_source_tl }
     { \tl_clear:N \l_@@_content_AF_source_tmpa_tl }
    \tl_if_eq:NnTF\g_@@_grabbed_env_tl {math}
          {
            \tl_set:Nn\l_@@_attribute_class_tl{inline}
          }
          {
            \tl_set:Nn\l_@@_attribute_class_tl{display}
          }
    \bool_if:NF\l__tag_math_alt_bool
      { \tl_set:Nn \l_@@_content_alt_tl{} }
    \tag_struct_begin:n
     {
       tag=Formula,
       attribute-class=\l_@@_attribute_class_tl,
       texsource   = \l_@@_content_AF_source_tmpa_tl,
       title-o     = \g_@@_grabbed_env_tl,
       actualtext  = \l_@@_content_actual_tl,
       alt         = \l_@@_content_alt_tl
     }
  }
\socket_new_plug:nnn
  {tagsupport/math/struct/end}
  {default}
  { \tag_struct_end: }

\socket_assign_plug:nn {tagsupport/math/struct/begin}{default}
\socket_assign_plug:nn {tagsupport/math/struct/end}{default}
%    \end{macrocode}
% \end{plugdecl}
%
% \begin{plugdecl}{mathml-AF}
% This socket tries to add a mathml-AF to formula.
% It is activated if a mathml.html has been found and loaded.
% As it disturbs the reading of the AF
% it currently deactivates the /Alt key,
% unless it has been reenabled with |math/alt/use=true|
%    \begin{macrocode}
\cs_generate_variant:Nn \str_mdfive_hash:n {o}
\tl_new:N\l_@@_content_hash_tl
%    \end{macrocode}
% we need to save the grabbed math:
%    \begin{macrocode}
\tl_new:N\l_@@_grabbed_math_tl
%    \end{macrocode}
% the socket definition
%    \begin{macrocode}
\socket_new_plug:nnn
  {tagsupport/math/struct/begin}
  {mathml-AF}
  {
   \int_gincr:N\g_@@_math_total_int
   \tl_set:Ne\l_@@_content_hash_tl
    {\str_mdfive_hash:o { \l_@@_content_AF_source_tl }}
   \tl_set_eq:NN\l_@@_grabbed_math_tl\g_@@_grabbed_math_tl
   \tl_if_eq:NnTF\g_@@_grabbed_env_tl {math}
     {
       \tl_set:Nn\l_@@_attribute_class_tl{inline}
     }
     {
       \tl_set:Nn\l_@@_attribute_class_tl{display}
     }
   \bool_if:NF\l__tag_math_alt_bool
     { \tl_set:Nn \l_@@_content_alt_tl{} }
%    \end{macrocode}
% debugging option. TODO: hide in debug key.
%    \begin{macrocode}
   \tl_if_exist:cTF { g_@@_mathml_ \l_@@_content_hash_tl _tl }
     {
       \int_gincr:N\g_@@_mathml_AF_found_int
       \bool_if:NTF \l__tag_math_mathml_AF_bool
        {
          \int_gincr:N\g_@@_mathml_AF_attached_int
          \typeout {Inserting~mathml~with~Hash~\l_@@_content_hash_tl}
        }
        {
          \typeout {Ignoring~mathml~with~Hash~\l_@@_content_hash_tl}
        }
     }
     {
       \bool_if:NT \l__tag_math_mathml_AF_bool
        {
          \typeout {WARNING:~mathml~missing~for~hash~\l_@@_content_hash_tl}
        }
     }
   \socket_use:n {tagsupport/math/mathml/write/prepare}
   \socket_use:n {tagsupport/math/mathml/write} % write hash if request
    \bool_if:NTF\l__tag_math_texsource_AF_bool
     { \tl_set_eq:NN \l_@@_content_AF_source_tmpa_tl \l_@@_content_AF_source_tl }
     { \tl_clear:N \l_@@_content_AF_source_tmpa_tl }
   \tag_struct_begin:n
     {
       tag=Formula,
       attribute-class=\l_@@_attribute_class_tl, %
       AFref       =
        \bool_if:NT\l__tag_math_mathml_AF_bool
         {
           \cs_if_exist_use:c {g_@@_mathml_ \l_@@_content_hash_tl _tl}
         },
       texsource   = \l_@@_content_AF_source_tmpa_tl, % should be after mathml AF!
       title-o     = \g_@@_grabbed_env_tl,    %
       alt         = \l_@@_content_alt_tl
     }
   }
%    \end{macrocode}
% not really needed but looks more symmetric:
%    \begin{macrocode}
\socket_new_plug:nnn
  {tagsupport/math/struct/end}
  {mathml-AF}
  {
    \tag_struct_end:
  }
%    \end{macrocode}
% \end{plugdecl}
%
% \begin{socketdecl}{tagsupport/math/substruct/begin,
%  tagsupport/math/substruct/end}
%  This holds the code to handle subparts of the formula.
%    \begin{macrocode}
\socket_new:nn {tagsupport/math/substruct/begin}{0}
\socket_new:nn {tagsupport/math/substruct/end}{0}
%    \end{macrocode}
% \end{socketdecl}
%
% \begin{plugdecl}{default}
%    \begin{macrocode}
\socket_new_plug:nnn
  {tagsupport/math/substruct/begin}
  {default}
  { \grabaformulapartandstart }
\socket_new_plug:nnn
  {tagsupport/math/substruct/end}
  {default}
  {
    \tagmcend
    \if@subformulas
      \tagstructend
    \fi
 }
\socket_assign_plug:nn {tagsupport/math/substruct/begin}{default}
\socket_assign_plug:nn {tagsupport/math/substruct/end}{default}
%    \end{macrocode}
% \end{plugdecl}

% \begin{plugdecl}{single}
% We need an option to disable subparts as it is unclear
% if consumers can handle them:
%    \begin{macrocode}
\socket_new_plug:nnn
  {tagsupport/math/substruct/begin}
  {single}
  {
    \typeout{====>subpart~splitting~deactivated}
    \typeout{====>grabbed~math=\meaning\g_@@_grabbed_math_tl}
    \tag_mc_begin:n{}
  }
\socket_new_plug:nnn
  {tagsupport/math/substruct/end}
  {single}
  { \tag_mc_end:}
%    \end{macrocode}
% \end{plugdecl}
%
% \begin{socketdecl}{tagsupport/math/end}
%  A socket used at the end of the math (before the closing dollar(s))
%  which can, e.g., set a flag for luamml.
%    \begin{macrocode}
\socket_new:nn {tagsupport/math/end}{0}
%    \end{macrocode}
% \end{socketdecl}
%
%
%  \changes{v0.6j}{2024-11-19}{removed enable/disable command and assign tagging sockets directly}
%    \begin{macrocode}
\socket_assign_plug:nn {tagsupport/math/inline/begin}{MC}
\socket_assign_plug:nn {tagsupport/math/inline/end}{MC}
\socket_assign_plug:nn {tagsupport/math/inline/formula/begin}{default}
\socket_assign_plug:nn {tagsupport/math/inline/formula/end}{default}
\socket_assign_plug:nn {tagsupport/math/display/begin}{default}
\socket_assign_plug:nn {tagsupport/math/display/end}{default}
\socket_assign_plug:nn {tagsupport/math/display/formula/begin}{default}
\socket_assign_plug:nn {tagsupport/math/display/formula/end}{default}
%    \end{macrocode}
%
%
% \subsection{Interface commands}
%
% \begin{macro}
%   {\@@_process:nn, \@@_process:Vn, \@@_process_auxi:nn, \@@_process_auxii:nn}
%   A no-op place-holder; the internal wrapper means that it does not need to
%   be concerned with internals.
%    \begin{macrocode}
\cs_new_protected:Npn \@@_process:nn #1#2
  {
    \legacy_if:nF { measuring@ }
      {
        \tl_if_in:nnTF {#2} { \m@th }
          { \bool_set_true:N\l_@@_fakemath_bool }
          { \tl_trim_spaces_apply:nN {#2} \@@_process_auxi:nn {#1} }
      }
  }
\cs_generate_variant:Nn \@@_process:nn { V }
\cs_new_protected:Npn \@@_process_auxi:nn #1#2
  {
    \tl_gset:Nn \g_@@_grabbed_env_tl {#2}
    \tl_gset:Nn \g_@@_grabbed_math_tl {#1}
    \@@_process_auxii:nn {#2} {#1}
  }
\cs_new_protected:Npn \@@_process_auxii:nn #1#2 { }
%    \end{macrocode}
% \end{macro}
%
% \begin{macro}{\math_processor:n}
%   A simple installer
%    \begin{macrocode}
\cs_new_protected:Npn \math_processor:n #1
  { \cs_set_protected:Npn \@@_process_auxii:nn ##1##2 {#1} }
%    \end{macrocode}
% \end{macro}
%
% \subsection{Content grabbing}
%
% \begin{macro}{\@@_grab_dollar:w}
% \begin{macro}{\@@_grab_dollar:n}
% \changes{v0.6c}{2024-08-22}{Correct handling of empty math segments}
%   Top-level function to handle grabbing of inline math mode delimited by
%   |$| tokens. We provide two different ways to do that: a token-by-token
%   one that can be used everywhere, and a fast delimited one that does not
%   work anywhere that the end |$| token may be hidden, most obviously in
%   tabulars. The function here is therefore set up as a variable starting
%   point.
%    \begin{macrocode}
\cs_new_protected:Npn \@@_grab_dollar:w { \@@_grab_dollar_delim:w }
%    \end{macrocode}
%   After grabbing inline math material, there is again common processing
%   independent of mechanism of collection.
%    \begin{macrocode}
\cs_new_protected:Npn \@@_grab_dollar:n #1
  {
%    \end{macrocode}
%  We need to do processing first as this picks up \enquote{fake} math mode:
%  that information is needed below.
%    \begin{macrocode}
    \@@_process:nn { math } {#1}
%    \end{macrocode}
%  We do not want math tagging in fakemath or when measuring,
%  We also do not want math tagging if tagging has been suspended.
%    \begin{macrocode}
      \bool_lazy_any:nTF
        {
          {\legacy_if_p:n { measuring@ }}
          { \l_@@_fakemath_bool }
          { \tl_if_blank_p:n {#1} }
        }
        {
          \@@_luamml_ignore:
          #1 $ % $
        }
        {
            \tag_socket_use:n  {math/inline/begin} %end P-MC
%    \end{macrocode}
% We do no use a tagging socket here, so that the argument (the
% math) is not lost, tagging-project issue 661.
% \changes{v0.6j}{2024-11-19}{change socket to tagging socket}
%    \begin{macrocode}
            \tag_socket_use:nnn {math/inline/formula/begin}{}{#1}
            $ % $
            \tag_socket_use:n  {math/inline/formula/end}
            \tag_socket_use:n  {math/inline/end} % restart P-MC
        }
  }
%    \end{macrocode}
% \end{macro}
% \end{macro}
%
% \begin{macro}{\@@_grab_dollar_delim:w}
%   Grab up to a single |$|, for inline math mode, suppressing
%   any processing if the token is \tn{m@th} found in the content.
%    \begin{macrocode}
\cs_new_protected:Npn \@@_grab_dollar_delim:w #1 $ % $
  { \@@_grab_dollar:n {#1} }
%    \end{macrocode}
% \end{macro}
%
% \begin{macro}{\@@_grab_dollardollar:w}
%   And for the classical \TeX{} display structure.
%    \begin{macrocode}
\cs_new_protected:Npn \@@_grab_dollardollar:w % $$
  #1 $$
  {
    \tl_if_blank:nF {#1}
      {
        \@@_process:nn { equation* } {#1}
        \tag_socket_use:n {math/display/begin}
        \tag_socket_use:nn{math/display/formula/begin}{}{#1}
      }
    $$
  }
%    \end{macrocode}
%
% The end code is added through a \cs{aftergroup} so we
% store it inside a command.
% \changes{v0.6j}{2024-11-19}{removed unneeded \cs{tagpdfparaOn}}
%    \begin{macrocode}
\cs_new_protected:Npn \@@_tag_dollardollar_display_end:
  {
    %  \typeout{== tag dollarldollar display end}
    %  \ShowTagging{struct-stack}
    \para_raw_end:
%    \end{macrocode}
%    The \cs{postdisplaypenalty} was temporarily set
%    to 10000 inside the display and the \cs{belowdisplayskip} and the
%    \cs{belowdisplayshortskip} was negated, so whatever was inserted
%    it should have been a negative skip. Whatever
%    was added, we pick up the value up, so that we can correct
%    the spacing after the tagging code was inserted.
%    \begin{macrocode}
    \l_@@_tmpa_skip \lastskip
    \tag_socket_use:n{math/display/formula/end}
%    \end{macrocode}
%    Now we add a skip without introducing a page break possibility,
%    that should bring the current vertical position back to the point
%    where \TeX{} would add the penalty and the \enquote{below skip}.
% \changes{v0.6f}{2024-09-30}{Correct logic for inserting below skips
%                             after displays (tagging/721)}
%    \begin{macrocode}
    \nobreak
    \skip_vertical:n { -\l_@@_tmpa_skip  }
%    \end{macrocode}
%    Then we finally add the real stuff: the true \cs{postdisplaypenalty}
%    and then negated value of skip we saved from above. It may look
%    strange that we have two identical negated skips next to each
%    other, but if you think about it, that is correct: the first
%    cancels the \enquote{below skip} that \TeX{} had added and the
%    second puts the same amount after the penalty (which is where it
%    should be).
%    \begin{macrocode}
    \penalty \postdisplaypenalty
    \skip_vertical:n { -\l_@@_tmpa_skip  } % insert the correct skip
%    \end{macrocode}
%    As we are now in vertical mode the situation is different from
%    the way \TeX{} would handle things after a display: \TeX{} would
%    internally switch to horizontal mode without adding a
%    \cs{parskip}. But this is not possible to do on the macro
%    level. Therefore we have to neutralize the upcoming \cs{parskip}
%    since we can't  prevent it from being set.
%  \changes{v0.6k}{2024/12/01}{Handle \cs{parskip} after \texttt{\$\$}
%    display correctly (tagging/762)}
%    \begin{macrocode}
%    \typeout{------->~ add~ negative~ parskip~ (to~ cancel~ the~
%                       one~ that~ TeX~ will~ add)}
    \skip_vertical:n { -\tex_parskip:D }
%    \end{macrocode}
%
%    We also set the \texttt{@domathendpetrue} flag to signal that the
%    \begin{macrocode}
    \@domathendpetrue
    \@doendpe             % this has no \end{...} to take care of it
}

%    \end{macrocode}
% \end{macro}
%
%
% \begin{macro}{\@@_grab_inline:w}
%   Collect inline math content and deal with the need to move to math mode.
%    \begin{macrocode}
\cs_new_protected:Npn \@@_grab_inline:w % \(
  #1 \)
  {
    \tl_if_blank:nF {#1}
      {
        $ #1 $
      }
    \bool_set_false:N \l_@@_collected_bool
  }
%    \end{macrocode}
% \end{macro}
%
% \begin{macro}{\@@_grab_eqn:w}
%   For the most common use of \cs{[}/\cs{]}: turn into an environment.
%    \begin{macrocode}
\cs_new_protected:Npn \@@_grab_eqn:w % \[
  #1 \]
   {
%     \typeout{collected? = \bool_if:NTF \l_@@_collected_bool {true}{false}}
     \begin { equation* } #1 \end { equation* }
   }
%    \end{macrocode}
% \end{macro}
%
% \subsection{Token-by-token inline grabbing}
%
% Grabbing inline math token-by-token is more involved. The mechanism here
% is essentially a simplified version of that originally seen in
% \pkg{collcell} and refined in \pkg{siunitx}. We make use of the fact that
% in math mode spaces are ignored, so we have to deal with only \texttt{N}-type
% tokens and groups. Furthermore, there is no need to look inside groups, so
% the only special cases are a small selection of \texttt{N}-type tokens.
%
% \begin{variable}{\l_@@_grabbed_tl}
%   For collection of the material piecewise.
%    \begin{macrocode}
\tl_new:N \l_@@_grabbed_tl
%    \end{macrocode}
% \end{variable}
%
% \begin{variable}{\l_@@_grab_env_int}
%   Needed to count up the number of nested environments encountered.
%    \begin{macrocode}
\int_new:N \l_@@_grab_env_int
%    \end{macrocode}
% \end{variable}
%
% \begin{macro}{\@@_grab_dollar_loop:}
% \begin{macro}{\@@_grab_loop:}
%   The lead-off here establishes a group: we need that as we will have to
%   be careful in the way \tn{cr} is handled and ensure this is only
%   manipulated whilst grabbing. The main loop is then started.
%    \begin{macrocode}
\cs_new_protected:Npn \@@_grab_dollar_loop:
  {
    \group_begin:
      \tl_clear:N \l_@@_grabbed_tl
      \@@_grab_loop:
  }
\cs_new_protected:Npn \@@_grab_loop:
  {
    \peek_remove_spaces:n
      {
        \peek_meaning:NTF \c_group_begin_token
          { \@@_grab_loop_group:n }
          { \@@_grab_loop_token:N }
      }
  }
%    \end{macrocode}
% \end{macro}
% \end{macro}
%
% \begin{macro}{\@@_grab_loop_group:n}
% \begin{macro}{\@@_grab_loop_store:n}
%   Handling of grabbed groups is pretty easy.
%    \begin{macrocode}
\cs_new_protected:Npn \@@_grab_loop_group:n #1
  { \@@_grab_loop_store:n { {#1} } }
\cs_new_protected:Npn \@@_grab_loop_store:n #1
  {
    \tl_put_right:Nn \l_@@_grabbed_tl {#1}
    \@@_grab_loop:
  }
%    \end{macrocode}
% \end{macro}
% \end{macro}
%
% \begin{macro}{\@@_grab_loop_token:N}
% \begin{macro}
%   {
%      \@@_grab_loop_$:              ,
%      \@@_grab_loop_\\:             ,
%      \@@_grab_loop_\begin:         ,
%      \@@_grab_loop_\end:           ,
%      \@@_grab_loop_\ignorespaces:  ,
%      \@@_grab_loop_\unskip:        ,
%      \@@_grab_loop_\textonly@unskip:
%   }
%   Filter out the special cases: for performance reasons, use a hash table
%   approach rather than a loop (\emph{cf.}~\pkg{collcell}).
%    \begin{macrocode}
\cs_new_protected:Npn \@@_grab_loop_token:N #1
  {
    \cs_if_exist_use:cF
      { @@_grab_loop_ \token_to_str:N #1 : }
      { \@@_grab_loop_store:n {#1} }
  }
\cs_new_protected:cpn { @@_grab_loop_ \token_to_str:N $ : }
  { \@@_grab_loop_end: }
\cs_new_protected:cpn { @@_grab_loop_ \token_to_str:N \\ : }
  {
    \int_compare:nNnTF \l_@@_grab_env_int = 0
       { \@@_grab_loop_newline: }
       { \@@_grab_loop_store:n { \\ } }
  }
%    \end{macrocode}
%   In contrast to \pkg{collcell}, nesting is tracked by counting
%   \cs{begin}/\cs{end} pairs: this is needed in case there is a tabular-like
%   construct containing |\\| inside a cell. As a result, the end-of-tabular
%   can be detected without checking the name argument: if \cs{end} is
%   encountered at nesting level~0, we've hit the end of a cell. In that case,
%   end the row and leave the environment to clean up.
%    \begin{macrocode}
\cs_new_protected:cpn { @@_grab_loop_ \token_to_str:N \begin : }
  {
    \int_incr:N \l_@@_grab_env_int
    \@@_grab_loop_store:n { \begin }
  }
\cs_new_protected:cpn { @@_grab_loop_ \token_to_str:N \end : }
  {
    \int_compare:nNnTF \l_@@_grab_env_int = 0
       {
         \@@_grab_loop_newline:
         \end
       }
       {
         \int_decr:N \l_@@_grab_env_int
         \@@_grab_loop_store:n { \end }
       }
  }
\tl_map_inline:nn { \ignorespaces \unskip \textonly@unskip }
  {
    \cs_new_protected:cpn { @@_grab_loop_ \token_to_str:N #1 : }
      { \@@_grab_loop: }
  }
%    \end{macrocode}
% \end{macro}
% \end{macro}
%
% \begin{macro}{\@@_grab_loop_newline:}
%   To allow collection of tokens in the part of the \tn{halign} template after
%   |#|, we need \TeX{} to see the primitive with the loop token in the right
%   place. That is done by re-defining \tn{cr} at present. Ideally there would
%   be a socket in the definition of \texttt{tabular}, etc., to handle this:
%   there is also the need to examine in interaction with \pkg{longtable}, which
%   also redefines \tn{cr}.
%    \begin{macrocode}
\cs_new_protected:Npn \@@_grab_loop_newline:
  {
    \if_false: { \fi:
    \cs_set_protected:Npn \cr
      {
        \@@_grab_loop:
        \tex_cr:D
      }
    \if_false: } \fi:
    \\
  }
%    \end{macrocode}
% \end{macro}
%
% \begin{macro}{\@@_grab_loop_end:}
%   Clean up and pass on.
%    \begin{macrocode}
\cs_new_protected:Npn \@@_grab_loop_end:
  {
    \exp_args:NNV \group_end:
    \@@_grab_dollar:n \l_@@_grabbed_tl
  }
%    \end{macrocode}
% \end{macro}
%
% \subsection{Marking math environments}
%
% A general mechanism for math mode environments that do not grab their
% content (\emph{cf.}~most \pkg{amsmath} environments).
%
% \begin{variable}{\l_@@_env_name_tl}
%  To allow us to carry out \enquote{special effects}
%    \begin{macrocode}
\tl_new:N \l_@@_env_name_tl
%    \end{macrocode}
% \end{variable}
%
% Here we set up specialised handling of environments. The idea for the
% \texttt{arg-spec} key is that if an environment takes arguments, we
% don't worry during the main grabbing. Rather, we remove the arguments
% from the grabbed content and forward only the payload. That is done by
% (ab)using \pkg{ltcmd}.
%    \begin{macrocode}
\keys_define:nn { @@ }
  {
     arg-spec .code:n =
       {
         \ExpandArgs { c } \DeclareDocumentCommand
           { @@_env \l_@@_env_name_tl _aux: }
           {#1}
           { \@@_env_forward:w }
       }
  }
%    \end{macrocode}
%
% \begin{macro}{\math_register_env:nn}
% \begin{macro}{\math_register_env:n}
% \begin{macro}{\RegisterMathEnvironment}
%   Set up to capture environment content and make available.
% \changes{0.6k}{2024-12-04}{Store the current grouplevel for luamml }
%    \begin{macrocode}
\cs_new_protected:Npn \math_register_env:nn #1#2
  {
    \tl_set:Nn \l_@@_env_name_tl {#1}
    \keys_set:nn { @@ } {#2}
    \cs_gset_eq:cc { @@_env_ #1 _begin: } {#1}
    \cs_gset_eq:cc { @@_env_ #1 _end: } { end #1 }
%
    \ExpandArgs { nne } \RenewDocumentEnvironment {#1} { b }
      {
        \exp_not:N \bool_if:NTF \exp_not:N \l_@@_collected_bool
          {
%            \typeout{===>B1}
          }
          {
%            \typeout{===>B2}
            \cs_if_exist:cTF { @@_env #1 _aux: }
              {
                \exp_not:c { @@_env #1 _aux: }
                  ##1 \exp_not:N \@@_env_end: {#1}
              }
              { \exp_not:N \@@_process:nn {#1} {##1} }
            \exp_not:n { \@kernel@math@registered@begin }
            \bool_set_true:N \exp_not:N \l_@@_collected_bool
          }
%        \exp_not:N \tracingall
        \exp_not:c { @@_env_ #1 _begin: }
        ##1
        \exp_not:c { @@_env_ #1 _end: }
%        \exp_not:N \tracingnone
     }
     {
     }
  }
%    \end{macrocode}
%  \changes{v0.6k}{2024/12/04}{Add tagging sockets for luamml support}
%  \changes{v0.6k}{2024/12/04}{Store the current grouplevel for luamml }
%    \begin{macrocode}
\cs_new_protected:Npn \math_register_halign_env:nn #1#2
  {
    \tl_set:Nn \l_@@_env_name_tl {#1}
    \keys_set:nn { @@ } {#2}
    \cs_gset_eq:cc { @@_env_ #1 _begin: } {#1}
    \cs_gset_eq:cc { @@_env_ #1 _end: } { end #1 }
%
    \ExpandArgs { nnee } \RenewDocumentEnvironment {#1} { b }
      {
        \exp_not:N \bool_if:NTF \exp_not:N \l_@@_collected_bool
          {
%            \typeout{===>B1}
          }
          {
%            \typeout{===>B2}
            \cs_if_exist:cTF { @@_env #1 _aux: }
              {
                \exp_not:c { @@_env #1 _aux: }
                  ##1 \exp_not:N \@@_env_end: {#1}
              }
              { \exp_not:N \@@_process:nn {#1} {##1} }
            \exp_not:n { \@kernel@math@registered@begin }
            \bool_set_true:N \exp_not:N \l_@@_collected_bool
          }
%        \exp_not:N \tracingall
        \exp_not:c { @@_env_ #1 _begin: }
        ##1
%        \exp_not:N \tracingnone
     }
     {
       \exp_not:c { @@_env_ #1 _end: }
     }
  }
%    \end{macrocode}
% TODO: the following command is neither documented nor used. Is is needed?
% TODO: if it is ever used in needs to set the current group level too.
%    \begin{macrocode}
\cs_new_protected:Npn \math_register_odd_env:nn #1#2
  {
    \tl_set:Nn \l_@@_env_name_tl {#1}
    \keys_set:nn { @@ } {#2}
    \cs_gset_eq:cc { @@_env_ #1 _begin: } {#1}
    \cs_gset_eq:cc { @@_env_ #1 _end: } { end #1 }
%
    \ExpandArgs { nnee } \RenewDocumentEnvironment {#1} { b }
      {
        \exp_not:N \bool_if:NTF \exp_not:N \l_@@_collected_bool
          {
%            \typeout{===>B1}
          }
          {
%            \typeout{===>B2}
            \cs_if_exist:cTF { @@_env #1 _aux: }
              {
                \exp_not:c { @@_env #1 _aux: }
                  ##1 \exp_not:N \@@_env_end: {#1}
              }
              { \exp_not:N \@@_process:nn {#1} {##1} }
            \exp_not:n { \@kernel@math@registered@begin }
            \bool_set_true:N \exp_not:N \l_@@_collected_bool
          }
%        \exp_not:N \tracingall
        \exp_not:c { @@_env_ #1 _begin: }
        ##1
     }
      {
        \exp_not:c { @@_env_ #1 _end: }
% needed if we don't have $$...$$
%        \exp_not:n { \typeout{---> @kernel@math@registered@end }}
        \exp_not:n { \@kernel@math@registered@end }
      }
  }


%  FMi: compare with block change!
%
%  \DeclareRobustCommand*\begin[1]{%
%  \UseHook{env/#1/before}%
%  \@ifundefined{#1}%
%    {\def\reserved@a{\@latex@error{Environment #1 undefined}\@eha}}%
%    {\def\reserved@a{\def\@currenvir{#1}%
%        \edef\@currenvline{\on@line}%
%        \@execute@begin@hook{#1}%
%        \csname #1\endcsname}}%
%  \@ignorefalse
%  \begingroup
%  \@endpefalse  % tmp!!! is it ok to drop this here?
%  \reserved@a}


\cs_new:Npn \@kernel@math@registered@begin {
%  \ShowTagging{struct-stack}
%\typeout{==>A1}\ShowTagging{struct-stack,mc-current}
  \mode_if_vertical:TF
       {
%         \legacy_if:nTF { @endpe }
%           { \legacy_if_set_false:n { @endpe } }
%           { \__block_list_beginpar_vmode: }
%
%         \typeout{==>~ at:~ \g__tag_struct_tag_tl}
%
        \tag_if_active:T
          {
            \exp_args:Noo\str_if_eq:nnF \g__tag_struct_tag_tl { \l__tag_para_main_tag_tl }    % needs correction!
             {
%               \typeout{==>A2}
               \__block_beginpar_vmode:
             }              % needs correction!
          }
       }
       {
%         \typeout{==>A3}
         \__tag_tool_close_P:
       }
  \tag_socket_use:nn{math/display/formula/begin}{}{}
%  \typeout{==>MC1}\ShowTagging{mc-current}
}

\cs_new:Npn \@kernel@math@registered@end {
%  \typeout{==>MC2}\ShowTagging{mc-current}
  \para_raw_end:
  \tagpdfparaOn
  \tag_socket_use:n{tagsupport/math/display/formula/end}
%  \typeout{==>MC3}\ShowTagging{mc-current}
  \@endpetrue
}

\cs_new_protected:Npn \math_register_env:n #1
  { \math_register_env:nn {#1} { } }

\NewDocumentCommand \RegisterMathEnvironment { O{} m }
  { \math_register_env:nn {#2} {#1} }
%    \end{macrocode}
% \end{macro}
% \end{macro}
% \end{macro}
%
% \begin{macro}{\@@_env_forward:w}
%    \begin{macrocode}
\cs_new_protected:Npn \@@_env_forward:w #1 \@@_env_end: #2
  { \@@_process:nn {#2} {#1} }
%    \end{macrocode}
% \end{macro}
%
% \subsection{Document commands}
%
% \car*{Add one more here: \texttt{displaymath}, which
%        is equivalent to \cs{[} , \cs{]}\\
%        and hence to the basic \texttt{equation*}.\\
%       Added in more recent branch.}
%
% \begin{macro}
%   {\equation, \@@_equation_begin:, \equation*, \@@_equation_star_begin:}
% \begin{macro}
%   {\endequation, \@@_equation_end:, \endequation*, \@@_equation_star_end:}
%   These environments are not set up by \pkg{amsmath} to collect their body,
%   so we do that here. This has to be done \emph{after} we can be sure
%   \pkg{amsmath} is loaded.
%
% \car*{Note that with \pkg{amsmath} loaded, \texttt{equation*} and \texttt{equation}\\
%        are the two basics: they are used to define the other single-row\\
%        display environments, etc.}
%
%    \begin{macrocode}
\tl_gput_right:Nn \@kernel@before@begindocument
  {
    \math_register_env:n { equation }
    \math_register_env:n { equation* }
% at the moment register_env can only do display math
%    \math_register_env:n { math }
    \RenewDocumentEnvironment{math} {b}{$#1$}{}
% and this one doesn't work either
%    \math_register_env:n { displaymath }
    \RenewDocumentEnvironment{displaymath} {b}{\[#1\]}{}
  }
%    \end{macrocode}
% \end{macro}
% \end{macro}
%
% \begin{macro}{\(, \)}
%  If math mode has not been collected, we need to  do that; otherwise, worry
%  about whether we are in math mode or not. The closing command here can only
%  occur inside a collected math block: otherwise it will be simply used as
%  a delimiter.
%    \begin{macrocode}
\cs_gset_protected:Npn \( % \)
  {
    \bool_if:NTF \l_@@_collected_bool
      {
        \mode_if_math:TF
          { \@badmath }
          { $ }
      }
      {
        \@@_grab_inline:w
      }
  } % \(
\cs_gset_protected:Npn \)
  {
    \mode_if_math:TF
      { $ }
      { \@badmath }
  }
%    \end{macrocode}
% \end{macro}
%
% \begin{macro}{\[, \]}
%   Again, we need to watch for when \pkg{amsmath} is loaded after this code.
%   The flag usage here is to cover the case where \cs{[}/\cs{]} is hidden
%   inside another environment. In this case the grabbing happens on
%    the outer level and should not be repeated.
%    \begin{macrocode}
\tl_gput_right:Nn \@kernel@before@begindocument
  {
    \cs_gset_protected:Npn \[ % \]
       {
         \@@_grab_eqn:w
 %       \bool_if:NTF \l_@@_collected_bool
%          { \begin { equation* } }
%          { \@@_grab_eqn:w }
      } % \[
    \cs_gset_protected:Npn \]
      {
        \@badmath
%        \bool_if:NTF \l_@@_collected_bool
%          { \end{ equation* } }
%          { \@badmath }
      }
  }
%    \end{macrocode}
% \end{macro}
%
%
% \begin{macro}{\ensuremath}
%   A bit of nesting fun to make sure we collect only if required.
%   \fmi{why does ensuremath need handling at all?}
%
%   \car{Indeed!  Currently, this is setup to process the math that
%     it has anyways already captured as its argument; thus it is more
%     efficient than leaving the capture to be repeated by the \cs{everymath}}
%
%    \begin{macrocode}
%\cs_gset_protected:Npn \ensuremath #1
%  {
%    \mode_if_math:TF
%      {#1}
%      {
%        \bool_if:NTF \l_@@_collected_bool
%          { \@ensuredmath {#1} }
%          {
%            \bool_set_true:N \l_@@_collected_bool
%            \@@_process:nn { math } {#1}
%            \@ensuredmath {#1}
%            \bool_set_false:N \l_@@_collected_bool
%          }
%      }
%  }
%    \end{macrocode}
% \end{macro}
%
% \subsection{\cs{everymath} and \cs{everydisplay}}
%
% The business end for grabbing inline math and \enquote{raw} \TeX{}
% display. Most display math mode is actually handled elsewhere, as we
% have macro control.
%    \begin{macrocode}

\exp_args:No \tex_everymath:D
  {
    \tex_the:D \tex_everymath:D
    \bool_if:NF \l_@@_collected_bool
      {
        \bool_set_true:N \l_@@_collected_bool
        \@@_grab_dollar:w
      }
  }

\exp_args:No \tex_everydisplay:D
  {
    \tex_the:D \tex_everydisplay:D
    \iftrue  % this may have to be a settable flag!
%        \typeout{==>~ in~ everydisplay}
%    \end{macrocode}
% flipping the \cs{belowdisplay} values is done so that we get (assumption)
% a negative skip and not make the page bigger then we take that out,
% then we add the tagging code (in \cs{@@_tag_dollardollar_display_end} ) and
% then we put a real \cs{postdisplaypenalty} in and
% the right skip (of which we don't know if it is short or a
% normal \cs{belowdisplayskip}). This might need some refinement if that skip
% is actually negative from the start
% (not sure it ever is and is worth bothering about)
%    \begin{macrocode}
        \skip_set:Nn \belowdisplayskip      {-\belowdisplayskip}
        \skip_set:Nn \belowdisplayshortskip {-\belowdisplayshortskip}
        \int_set:Nn \postdisplaypenalty {10000}
        \group_insert_after:N \@@_tag_dollardollar_display_end:
    \fi
    \bool_if:NF \l_@@_collected_bool
      {
        \bool_set_true:N \l_@@_collected_bool
        \@@_grab_dollardollar:w
      }
  }
%    \end{macrocode}
%
% \subsection{Modifying kernel environments}
%
%   We need to cover this even though it is, of course, not encouraged.
%    \begin{macrocode}
\math_register_env:n { eqnarray }
\math_register_env:n { eqnarray* }
%    \end{macrocode}
%
% Tabulars currently contain a \$ that shouldn't trigger math
% tagging.
%    \begin{macrocode}
\RequirePackage{array}
\tl_if_in:NnT\@tabular{$}
 {
   \def\@tabular{%
   \leavevmode
   \UseTaggingSocket{tbl/hmode/begin}%
   \hbox \bgroup
   \bool_set_true:N \l_@@_collected_bool
   $
   \bool_set_false:N \l_@@_collected_bool
   \col@sep\tabcolsep \let\d@llarbegin\begingroup
                                    \let\d@llarend\endgroup
%    \end{macrocode}
%   A proper switching mechanism is needed: for the present, do directly.
%    \begin{macrocode}
   \cs_set_protected:Npn \@@_grab_dollar:w { \@@_grab_dollar_loop: }
   \@tabarray}
 }
%    \end{macrocode}
%
% \begin{macro}{\@@_m@th:, \m@th}
%   Handle non-math use of math mode. At present nesting isn't supported as
%   \cs{m@th} pops up in a few places that \emph{are} math mode!
%    \begin{macrocode}
\cs_new_eq:NN \@@_m@th: \m@th
\cs_gset_protected:Npn \m@th
  {
    \bool_set_true:N \l_@@_collected_bool
    \@@_m@th:
  }
%    \end{macrocode}
% \end{macro}
%
% \subsection{Disable math grabbing in the begindocument hook}
% For example amsart uses math to measure text there.
%
%    \begin{macrocode}
\tl_gput_right:Nn\@kernel@before@begindocument
  {
    \bool_set_true:N\l_@@_collected_bool
  }
\tl_gput_right:Nn\@kernel@after@begindocument
 {
   \bool_set_false:N\l_@@_collected_bool
 }
%    \end{macrocode}
%
% \subsection{Modifying \pkg{amsmath}}
%
%
% \begin{macro}{\@@_amsmath_align@:nn}
% \begin{macro}
%   {
%     \@@_amsmath_gather@:n ,
%     \@@_amsmath_multline@:n
%   }
% \begin{macro}{\align@}
% \begin{macro}{\gather@, \multline@}
%   Mark up all of the display environments as the content is captured anyway.
%   We then use an internal macro in each environment type to insert the
%   processing code. Each of these is slightly different, so we cannot use a
%   simple loop here. The test for \cs{split@tag} is required as the
%   \texttt{split} environment internally uses \texttt{gather} \emph{when not
%   within an \pkg{amsmath}} environment, for example inside \texttt{equation}.
%   Without the precaution, we'd get two copies of the grabbed math, the second
%   of which would start with \cs{split@tag}.
%
%   This socket is provided temporarly, and can be removed 2025-06-01 at latest.
%    \begin{macrocode}
\socket_if_exist:nF { tagsupport/math/luamml/mtable/finalize }
 {
   \NewSocket{tagsupport/math/luamml/mtable/finalize}{1}
   \AssignSocketPlug{tagsupport/math/luamml/mtable/finalize}{noop}
 }
%    \end{macrocode}
%
%    \begin{macrocode}
\tl_gput_right:Nn \@kernel@before@begindocument {
%
\renewenvironment{gather*}{%
  \start@gather\st@rredtrue
}
{%
% this redirection doesn't work if we alter "gather"!
  %  \endgather
% so replace it with its real meaning
  \math@cr \black@\totwidth@ \egroup
  $$\ignorespacesafterend
}
%    \end{macrocode}
%
%    \begin{macrocode}
\def\common@align@ending {
  \math@cr \black@\totwidth@
  \UseExpandableTaggingSocket {math/luamml/mtable/finalize} {align}
  \egroup
  \ifingather@
    \restorealignstate@
    \egroup
    \nonumber
    \ifnum0=`{\fi\iffalse}\fi
  \else
    $$%
  \fi
  \ignorespacesafterend
}

\renewenvironment{alignat}{%
  \start@align\z@\st@rredfalse
}{%
  \common@align@ending
}
\renewenvironment{alignat*}{%
  \start@align\z@\st@rredtrue
}{%
  \common@align@ending
}
\renewenvironment{xalignat}{%
  \start@align\@ne\st@rredfalse
}{%
  \common@align@ending
}
\renewenvironment{xalignat*}{%
  \start@align\@ne\st@rredtrue
}{%
  \common@align@ending
}
\renewenvironment{xxalignat}{%
  \start@align\tw@\st@rredtrue
}{%
  \common@align@ending
}
\renewenvironment{align}{%
  \start@align\@ne\st@rredfalse\m@ne
}{%
  \common@align@ending
}
\renewenvironment{align*}{%
  \start@align\@ne\st@rredtrue\m@ne
}{%
  \common@align@ending
}
\renewenvironment{flalign}{%
  \start@align\tw@\st@rredfalse\m@ne
}{%
  \common@align@ending
}
\renewenvironment{flalign*}{%
  \start@align\tw@\st@rredtrue\m@ne
}{%
  \common@align@ending
}
%
\renewenvironment{multline*}{\start@multline\st@rredtrue}
{%
  \iftagsleft@ \@xp\lendmultline@ \else \@xp\rendmultline@ \fi
  \ignorespacesafterend
}
%    \end{macrocode}
%    Also for "false?"
%    \begin{macrocode}
\def\measuring@true{\let\ifmeasuring@\iftrue\tag_suspend:n{\measuring}}
%    \end{macrocode}
%
%    \begin{macrocode}
%
  \math_register_halign_env:nn {align}{}
  \math_register_halign_env:nn {align*}{}
  \math_register_halign_env:nn {alignat}{}
  \math_register_halign_env:nn {alignat*}{}
  \math_register_halign_env:nn {flalign}{}
  \math_register_halign_env:nn {flalign*}{}
  \math_register_halign_env:nn {gather}{}
  \math_register_halign_env:nn {gather*}{}
  \math_register_halign_env:nn {multline}{}
  \math_register_halign_env:nn {multline*}{}
  \math_register_halign_env:nn {xalignat}{}
  \math_register_halign_env:nn {xalignat*}{}
  \math_register_halign_env:nn {xxalignat}{}
  %
  \@namedef{maketag @ @ @} #1{%
%    \typeout{--->maketag @ @ @}
    \ifmeasuring@
      \hbox{\m@th\normalfont#1}%
    \else
      \tagmcend \tagstructbegin{tag=Lbl}%
      \tagmcbegin{tag=Lbl}%
      \hbox{\m@th\normalfont#1}%
      \tagmcend \tagstructend \tagmcbegin{}%
    \fi
  }
%    \end{macrocode}
%
%
%    \begin{macrocode}
\@namedef{math@cr @ @ @ gather}{%
    \ifst@rred\nonumber\fi
   &\relax
    \make@display@tag
%
    \maybestartnewformulatag
%
    \ifst@rred\else\global\@eqnswtrue\fi
    \global\advance\row@\@ne
    \cr
}
%    \end{macrocode}
%
%    \begin{macrocode}
\@namedef{math@cr @ @ @ align}{%
  \ifst@rred\nonumber\fi
  \if@eqnsw \global\tag@true \fi
  \global\advance\row@\@ne
  \add@amps\maxfields@
  \omit
  \kern-\alignsep@
  \iftag@
    \setboxz@h{\@lign\strut@{\make@display@tag}}%
    \place@tag
  \fi
%
    \maybestartnewformulatag
%
  \ifst@rred\else\global\@eqnswtrue\fi
  \global\lineht@\z@
  \cr
}
%    \end{macrocode}
%
%    \begin{macrocode}
\def\restore@math@cr{\@namedef{math@cr @ @ @}{
%
    \maybestartnewformulatag
%
    \cr}}
\restore@math@cr
%    \end{macrocode}
%
%    \begin{macrocode}
}
%    \end{macrocode}
% \end{macro}
% \end{macro}
% \end{macro}
% \end{macro}
%
%
%

%  \begin{macro}{\@@_split_at_nl:NN}
%  This splits grabbed math at newlines.
%
%    \begin{macrocode}
\cs_new:Npn \@@_split_at_nl:NN #1#2 {
  \tl_set:Nf \l_@@_tmpa_tl {
      \exp_after:wN \@@_split_at_nl_first:w #1 \\ \q_nil \\ \s_stop }
  \exp_after:wN \@@_split_at_nl_aux:nnNN \l_@@_tmpa_tl #1 #2
}
%    \end{macrocode}
% and the auxiliary commands
%    \begin{macrocode}
\cs_new:Npn \@@_split_at_nl_first:w #1 \\ #2 \\ #3 \s_stop
  {
    \quark_if_nil:nTF {#2}
      { {#1} {  } }
      {
        \@@_split_chk_if_begin:ww #1 \begin \q_nil \s_mark
          #2 \\ #3 \s_stop
      }
  }

\cs_new_protected:Npn \@@_split_at_nl_aux:nnNN #1 #2 #3 #4
  {
    \tl_gset:Nn #4 {#1}
    \tl_gset:Nn #3 {#2}
  }

\cs_new:Npn \@@_split_chk_if_begin:ww
   #1 \begin #2 #3 \s_mark #4 \\ \q_nil \\ \s_stop
  {
    \quark_if_nil:nTF {#2}
      { {#1} {#4} }
      {
        \exp_after:wN \@@_split_collect_one_end:w
          \@@_split_cleanup_begin_q_nil:w #1 \begin{#2} #3 \\ #4 \s_stop
            { } { 1 }
      }
  }

\cs_new:Npn \@@_split_cleanup_begin_q_nil:w #1 \begin \q_nil {#1}

\cs_new:Npn \@@_split_collect_one_end:w #1 \end #2 #3 \s_stop #4 #5
  {
    \exp_args:Nf \@@_split_check_count_begins:nnnn
      { \@@_split_count_begins:n { #4 #1 } } {#5}
      { #4 #1 \end{#2} } {#3}
  }
\cs_new:Npn \@@_split_count_begins:n #1
  { \int_eval:n { 0 \@@_split_count_begins:w #1 \begin \q_nil } }

\cs_new:Npn \@@_split_count_begins:w #1 \begin #2
  { \quark_if_nil:nF {#2} { +1 \@@_split_count_begins:w } }

\cs_new:Npn \@@_split_check_count_begins:nnnn #1 #2 #3 #4
  {
    \int_compare:nNnTF {#1} = {#2}
      {
        \exp_last_unbraced:Nf \@@_split_final_cleanup:nn
          { \@@_split:n { \@@_split_guard:n {#3} #4 } }
      }
      {
        \exp_args:No \use_ii_i:nn
          { \exp_after:wN { \int_value:w \int_eval:n { #2 + 1 } } }
          { \@@_split_collect_one_end:w #4 \s_stop {#3} }
      }
  }
\cs_new:Npn \@@_split_final_cleanup:nn #1 #2
  {
      \exp:w \@@_split_final_cleanup:w #1
        \@@_split_guard:n \q_nil \s_mark { }
      {#2}
  }
\cs_new:Npn \@@_split_final_cleanup:w #1 \@@_split_guard:n #2 #3 \s_mark #4
  {
    \quark_if_nil:nTF {#2}
      { \exp_end: { #4 #1 } }
      { \@@_split_final_cleanup:w #3 \s_mark { #4 #1 #2 } }
  }

\cs_new:Npn \@@_split:n #1 {
    \@@_split_at_nl_first:w #1 \\ \q_nil \\ \s_stop }

% this looks unused.
%\NewDocumentCommand \splitnl { mm +m }
%  {
%    \tl_set:Nf \l_@@_tmpa_tl { \split:n {#3} }
%    \show \l_@@_tmpa_tl
%    \exp_after:wN \__splitnl_aux:nnNN \l_@@_tmpa_tl #1 #2
%  }
%    \end{macrocode}
%  \end{macro}
%
%
%  \begin{macro}{\maybestartnewformulatag}
%
%    \begin{macrocode}

\newif\if@subformulas
\tl_new:N \result

\cs_new_protected:Npn\grabaformulapartandstart {
  \@@_split_at_nl:NN  \g_@@_grabbed_math_tl \result
  \typeout{====>first-result=\meaning\result}
  \typeout{====>first-tmpmathcontent=\meaning\g_@@_grabbed_math_tl}
  \tl_if_empty:NTF \g_@@_grabbed_math_tl
     {
       \typeout{====>formula~ has~ no~ subparts}
       \global\@subformulasfalse
     }
     {
       \typeout{====>formula~ has~ subparts}
       \global\@subformulastrue
       \edef\resulttitle{\g_@@_grabbed_env_tl\space (part)}
       \tagstructbegin{tag=Formula,
%    \end{macrocode}
%    For now we don't put real content in /alt or /ActualText on subformulas
%    but we add a short text to satisfy the pdf/ua-2 validator
%    \begin{macrocode}
%         alt=\result,
         alt = subformula,
         title-o=\resulttitle
       }
    }
    \tagmcbegin{}
}

\cs_new_protected:Npn\grabaformulapartandmayberestart {
  \@@_split_at_nl:NN  \g_@@_grabbed_math_tl \result
  \typeout{====>result=\meaning\result}
  \typeout{====>tmpmathcontent=\meaning\g_@@_grabbed_math_tl}
%  \tl_if_empty:NTF \g_@@_grabbed_math_tl
%     {
%       \typeout{====>tmpmathcontent=empty}
%     }
%     {
%       \typeout{====>tmpmathcontent=not-empty}
       \edef\resulttitle{\g_@@_grabbed_env_tl\space (part)}
       \tagstructbegin{tag=Formula,
         alt=\result,
         title-o=\resulttitle
       }
%    }
    \tagmcbegin{}
}
%    \end{macrocode}
%  \end{macro}
%
%
%
%
%    \begin{macrocode}
\def\maybestartnewformulatag {
\if@subformulas
 \ifmeasuring@\else
%
  \tl_if_empty:NF \g_@@_grabbed_math_tl
     {
       \tagmcend
       \tagstructend
       \grabaformulapartandmayberestart
     }
 \fi
\fi
}
%    \end{macrocode}
%
%    The breqn packages changes catcodes and that isn't yet covered
%    by our mechanism.
%    \begin{macrocode}
%\AddToHook{package/breqn/after}{
%  \typeout{===>~ in~ hook}
%  \math_register_halign_env:nn {dmath}{}
%  \math_register_halign_env:nn {dgroup*}{}
%}
%    \end{macrocode}
%
%    \begin{macrocode}
\ExplSyntaxOff
%    \end{macrocode}
%
%    \begin{macrocode}
%<@@=>
%    \end{macrocode}
%
%    \begin{macrocode}
%</kernel>
%    \end{macrocode}
%
% \Finale
%
%
