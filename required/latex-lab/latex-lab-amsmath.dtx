% \iffalse meta-comment
%
%% File: latex-lab-amsmath.dtx
%
% Copyright (C) 2022-2025 The LaTeX Project
%
% It may be distributed and/or modified under the conditions of the
% LaTeX Project Public License (LPPL), either version 1.3c of this
% license or (at your option) any later version.  The latest version
% of this license is in the file
%
%    https://www.latex-project.org/lppl.txt
%
%
% The development version of the bundle can be found below
%
%    https://github.com/latex3/latex2e/required/latex-lab
%
% for those people who are interested or want to report an issue.
%
%<*driver>
\documentclass{l3doc}
\EnableCrossrefs
\CodelineIndex
\begin{document}
  \DocInput{latex-lab-amsmath.dtx}
\end{document}
%</driver>
%
% \fi
%
%
% \title{The \texttt{latex-lab-amsmath} code\thanks{}}
% \author{\LaTeX{} Project}
%
% \maketitle
%
% \newcommand\fmi[1]{\begin{quote} TODO: \itshape #1\end{quote}}
% \newcommand\NEW[1]{\marginpar{\mbox{}\hfill\fbox{New: #1}}}
% \providecommand\class[1]{\texttt{#1.cls}}
% \providecommand\pkg[1]{\texttt{#1}}
% \providecommand\hook[1]{\texttt{#1}}
%
% \begin{abstract}
% \end{abstract}
%
% \tableofcontents
%
% \section{Introduction}
%
% This file implements adaptations to the \pkg{amsmath} package needed for
% the tagging project.
%
% \MaybeStop{\setlength\IndexMin{200pt}  \PrintIndex  }
%
% \section{The Implementation}
%
%    \begin{macrocode}
%<@@=math>
%    \end{macrocode}
%
%    \begin{macrocode}
%<*kernel>
%    \end{macrocode}
%
% \subsection{File declaration}
%    \begin{macrocode}
\ProvidesFile{latex-lab-amsmath.ltx}
        [2024-12-03 v0.1d amsmath adaptions]
%    \end{macrocode}
% \subsection{Tagpdf support}
% To make the code independent from tagging being loaded and active
% we load the \pkg{tagpdf-base} package:
%
%    \begin{macrocode}
\RequirePackage{tagpdf-base}
%    \end{macrocode}
%
%    \begin{macrocode}
\ExplSyntaxOn
%    \end{macrocode}
%
% \subsection{\cs{intertext}}
% The \cs{intertext} command errors with active tagging as it is processed twice
% which leads to duplicated structures.
%    \begin{macrocode}
\AddToHook{package/amsmath/after}
 {
    \def\intertext@{%
      \def\intertext##1{%
        \ifvmode\else\\\@empty\fi
        \noalign{%
%    \end{macrocode}
% we have to flip the sign and use a negative \cs{belowdisplayskip}
% as we flipped the sign at the outside.
%    \begin{macrocode}
          \penalty\postdisplaypenalty\vskip-\belowdisplayskip
          \vbox{
%    \end{macrocode}
% Stop tagging when measuring:
%    \begin{macrocode}
           \ifmeasuring@\tag_suspend:n{\measuring}\fi
           \normalbaselines
            \ifdim\linewidth=\columnwidth
            \else \parshape\@ne \@totalleftmargin \linewidth
            \fi
%    \end{macrocode}
% End the previous mc:
%    \begin{macrocode}
            \tag_mc_end_push:
%    \end{macrocode}
% We are already in a par so we change now to text:
%    \begin{macrocode}
            \tagpdfsetup{para/tag=P}%
%    \end{macrocode}
% TODO why \cs{tagpdfparaOn} needed?
%    \begin{macrocode}
            \tagpdfparaOn
            \noindent\ignorespaces##1\par
%    \end{macrocode}
% Restart the MC
%    \begin{macrocode}
            \tag_mc_begin_pop:n{}}%
          \penalty\predisplaypenalty\vskip\abovedisplayskip%
        }%
    }}
  }
%    \end{macrocode}
%
% \subsection{\cs{text}}
%
% The \cs{text} command uses \cs{mathchoice} which \enquote{typesets} the argument
% four times. This makes it quite problematic for tagging. Without precautions
% structure objects would be created four times and would get MC-chunks as kids
% that doesn't really exist. amsmath contains a switch that allows to execute code
% only in the first (displaymath) branch, but that isn't usable here. At first because
% we don't know if the first branch creates the same structure as the one that is
% actually used. At second because the engines executes some commands like \cs{label}
% and \cs{pdfannot} only at shipout from the branch that really was used. So we would
% get structure data from one \cs{mathchoice}-branch and MC-labels and links from another
% one and that gets very messy.
%
% We therefore have to avoid that tagging is active in unused branches. In pdflatex it
% is not possible to detect the mathstyle before, so we use a label. With lualatex is
% is possible to redefine \cs{text} not to use \cs{mathchoice}
%
%    \begin{macrocode}
\AddToHook{package/amstext/after}
 {
%    \end{macrocode}
% currently amsmath is loaded in a begindocument hook, so this
% test is fine. If amstext is loaded earlier (in the kernel), this needs perhaps a change.
% \changes{v0.1d}{2024/12/03}{Add tagging socket for luamml support}
%    \begin{macrocode}
   \tag_if_active:T
    {
      \sys_if_engine_luatex:TF
       {
         \def\text@#1{
          \tag_socket_use:nnn {math/luamml/hbox}{}
          {{%
           \ifcase\mathstyle
           \hbox{{#1}}\or
           \hbox{{#1}}\or
           \hbox{{#1}}\or
           \hbox{{#1}}\or
           \hbox{{\let\f@size\sf@size\selectfont#1}}\or
           \hbox{{\let\f@size\sf@size\selectfont#1}}\or
           \hbox{{\let\f@size\ssf@size\selectfont#1}}\or
           \hbox{{\let\f@size\ssf@size\selectfont#1}}\or
           \ERROR
           \fi
           \check@mathfonts
         }}}
       }
       {
         \def\text@#1
          {{
           \int_gincr:N\g__math_mathchoice_int
           \tag_suspend:n{\text@}
           \mathchoice
            {
             \@@_tag_if_mathstyle:en{mathchoice-\int_use:N\g__math_mathchoice_int}{0}
             \textdef@\displaystyle\f@size{#1}
            }
            {
             \@@_tag_if_mathstyle:en{mathchoice-\int_use:N\g__math_mathchoice_int}{2}
             \textdef@\textstyle\f@size{\firstchoice@false #1}
            }
            {
             \@@_tag_if_mathstyle:en{mathchoice-\int_use:N\g__math_mathchoice_int}{4}
             \textdef@\textstyle\sf@size{\firstchoice@false #1}
            }
            {
              \@@_tag_if_mathstyle:en{mathchoice-\int_use:N\g__math_mathchoice_int}{6}
              \textdef@\textstyle \ssf@size{\firstchoice@false #1}
            }
            \check@mathfonts
          }}
       }
    }
 }
%    \end{macrocode}
%
% \subsection{\cs{pmb}}
% \cs{pmb} prints its argument three times. For tagging we must mark
% two of occurrences as artifact.
% For luatex the attributes in the box must be reset, for this
% we switch to expl3-boxes.
%    \begin{macrocode}
\AddToHook{package/amsbsy/after}
 {
   \def\pmb@@@@#1#2#3{\leavevmode\hbox_set:Nn\l_@@_tmpa_box{xxx#3}
      \dimen@-\box_wd:N\l_@@_tmpa_box
      \kern-.5\ex@\box_use:N\l_@@_tmpa_box
      \tag_mc_end:\tag_mc_begin:n{artifact}
      \tag_mc_reset_box:N\l_@@_tmpa_box
      \kern\dimen@\kern.25\ex@\raise.4\ex@\box_use:N\l_@@_tmpa_box
      \kern\dimen@\kern.25\ex@\box_use_drop:N\l_@@_tmpa_box
      \tag_mc_end:\tag_mc_begin:n{}
    }
   \def\pmb@#1#2{\hbox_set:Nn\l_@@_tmpa_box{$\m@th#1{#2}$}
     \setboxz@h{$\m@th#1\mkern.5mu$}\pmbraise@\wdz@
     \binrel@{#2}
     \dimen@-\box_wd:N\l_@@_tmpa_box
     \binrel@@@@{
       \mkern-.8mu\box_use:N\l_@@_tmpa_box
       \tag_mc_end:\tag_mc_begin:n{artifact}
       \tag_mc_reset_box:N\l_@@_tmpa_box
       \kern\dimen@\mkern.4mu\raise\pmbraise@\box_use:N\l_@@_tmpa_box
       \kern\dimen@\mkern.4mu\box_use_drop:N\l_@@_tmpa_box
       \tag_mc_end:\tag_mc_begin:n{}
       }
    }
  }
%    \end{macrocode}
%    \begin{macrocode}
\ExplSyntaxOff
%    \end{macrocode}
%
%    \begin{macrocode}
%</kernel>
%    \end{macrocode}
%
% \Finale
%
