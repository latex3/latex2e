% \iffalse meta-comment
%
%% File: latex-lab-l3doc-tagging.dtx
% Copyright (C) 2022-2025 The LaTeX Project
%
% It may be distributed and/or modified under the conditions of the
% LaTeX Project Public License (LPPL), either version 1.3c of this
% license or (at your option) any later version.  The latest version
% of this license is in the file
%
%    https://www.latex-project.org/lppl.txt
%
%
% The development version of the bundle can be found below
%
%    https://github.com/latex3/latex2e/required/latex-lab
%
% for those people who are interested or want to report an issue.
%
\def\ltlabdoctaggingdate{2025-05-07}
\def\ltlabdoctaggingversion{0.80b}

%<*driver>
\DocumentMetadata
  {
    lang=en,
    pdfstandard=ua-2,
    tagging=on
  }
\tagpdfsetup{table/header-rows=1}
\documentclass{l3doc}
\usepackage{latex-lab-testphase-l3doc}
\EnableCrossrefs
\CodelineIndex
\NewDocElement[envlike, idxtype=instance, idxgroup=instances,
   printtype=\textit{inst.}] {Instance}{instance}

\begin{document}
  \DocInput{latex-lab-l3doc-tagging.dtx}
\end{document}
%</driver>
%
% \fi
%
%
% \title{The \texttt{latex-lab-testphase-l3doc} package\thanks{}}
% \author{Ulrike Fischer, \LaTeX{} Project}
%
% \maketitle
%
%  \tableofcontents
%
% \begin{documentation}
% \section{Introduction}
%
% This package provides tagging support for the l3doc class.
% It is work in progress. It is neither garantied that every element
% is correctly tagged nor that the output looks identical to the untagged version.
%
% If a documentation is tagged with this code, it is important to check the log-file
% for warnings, the tag structure for missing parts and the output for deviations!
%
% Feedback at \url{https://github.com/latex3/tagging-project} is welcome!
%
% TODO: Index?
%
% TODO: update key names if block key names change
%
% TODO: once Sub is allowed in Code, switch to
% verbatim = Code, codeline = Sub, line number = Lbl
% \end{documentation}
%
% \begin{implementation}
% \section{implementation}
%
%    \begin{macrocode}
%<*package>
%    \end{macrocode}
%
% \subsection{Declaration}
%
%    \begin{macrocode}
\ProvidesExplPackage {latex-lab-testphase-l3doc} {2025-05-07} {0.80b}
  { Tagging Support for the l3doc class }
%    \end{macrocode}
% some minimal error checking
%    \begin{macrocode}
\IfClassLoadedF{l3doc}
 {\PackageError{latex-lab-testphase-l3doc}{This~package~requires~the~l3doc~class!}{}}
%    \end{macrocode}
%
%    \begin{macrocode}
\IfDocumentMetadataF
 {\PackageError{latex-lab-testphase-l3doc}{This~package~requires~the~PDF~management!}{}}
%    \end{macrocode}
%
% \subsection{Tag and roles}
%
% The exact tagging structure is not yet clear. Especially for verbatim and
% code material. So we define roles that we can exchange when needed.

%    \begin{macrocode}
\tagpdfsetup
 {
  role/new-tag=function/Sect,
  role/new-tag=functionnames/Caption,
  role/new-tag=functionname/Code,
  role/new-tag=functionnamepart/Span,
  role/new-tag=syntax/Sect,
  role/new-tag=function-description/Div,
  role/new-tag=macro/Sect,
  role/new-tag=macronames/Caption,
  role/new-tag=macroname/Span,
  role/new-tag=variable/Sect,
  role/new-tag=variablenames/Caption,
  role/new-tag=variablename/Span,
  role/new-tag=codelinenum/Span,
  role/new-tag=part/H1
 }
%    \end{macrocode}
%
% \subsection{Variables}
%
%
% \begin{variable}{\l__codedoc_current_names_struct_tl}
% The structure around the command names in the margin must be moved around.
% For this we define a variable that will hold the relevant current structure.
%    \begin{macrocode}
\tl_new:N\l__codedoc_current_names_struct_tl
%    \end{macrocode}
% \end{variable}
%
% \begin{variable}{\g__codedoc_macroname_structnum_seq}
% This holds the structure numbers of the
% structure around every command name in a macro environment. As the
% structures are created inside a box this is a global variable, which is cleared
% at the end of the outer macro environment.
%    \begin{macrocode}
\seq_new:N\g__codedoc_macroname_structnum_seq
%    \end{macrocode}
% \end{variable}
%
%
% \subsection{The \env{function} enviroment}
%
% \begin{macro}{\__codedoc_function_typeset_start:}
% This command starts the typesetting and opens the structure.
%    \begin{macrocode}
\cs_set_protected:Npn \__codedoc_function_typeset_start:
  {
    \par \bigskip
    \tagstructbegin{tag=function}
%    \end{macrocode}
% The paragraph structure should start after the building of the macronames.
% TODO: check if we can simply move ...
%    \begin{macrocode}
    \tagpdfparaOff
    \noindent
  }
%    \end{macrocode}
% \end{macro}
%
%
% \begin{macro}{\__codedoc_typeset_functions:}
% TODO: \cs{toprule} and \cs{bottomrule} should be artifacts with pdflatex
% (they are already with lualatex).
%    \begin{macrocode}
\cs_set_protected:Npn \__codedoc_typeset_functions:
  { %
    \small\ttfamily
    \__codedoc_target:
    \Hy@MakeCurrentHref { HD. \int_use:N \c@HD@hypercount }
%    \end{macrocode}
% The function names are typeset as a tabular. We use
% the table code to add the tagging.
%    \begin{macrocode}
    \tagstructbegin{tag=functionnames}
    \tagmcbegin{}
    \tagpdfsetup{table/tagging=div}
    \tl_set:Nn\l__tbl_rowtag_tl   {functionname}
    \tl_set:Nn\l__tbl_celltag_tl   {functionnamepart}
    \begin{tabular} [t] { @{} l @{} >{\hspace{\tabcolsep}} r @{} }
      \toprule
      \__codedoc_function_extra_labels:
      \__codedoc_names_typeset:
      \__codedoc_typeset_dates:
      \bottomrule
    \end{tabular}
    \tagmcend
    \tagstructend
    \normalfont\normalsize\par
  }
%    \end{macrocode}
% \end{macro}
%
% \begin{macro}{\__codedoc_function_descr_start:w}
% The function enviroment is mainly a box. We store the structure
% for the syntax environment.
%    \begin{macrocode}
\cs_set_protected:Npn \__codedoc_function_descr_start:w
  {
    \vcoffin_set:Nnw \l__codedoc_descr_coffin { \textwidth }
     \tagpdfparaOn
     \tagstructbegin{tag=function-description}
     \tl_set:Ne\l__codedoc_current_names_struct_tl{\tag_get:n{struct_num}}
     \noindent \ignorespaces
  }
%    \end{macrocode}
% \end{macro}
%
% \begin{macro}{\__codedoc_function_typeset_stop:}
% At the end we must close the two main structures.
%    \begin{macrocode}
\cs_set_protected:Npn \__codedoc_function_typeset_stop:
  {
    \par
    \tagstructend % function-description
    \tagstructend % function
    \dim_set:Nn \prevdepth { \coffin_dp:N \l__codedoc_descr_coffin }
    \allowbreak
  }
%    \end{macrocode}
% \end{macro}
%
% \subsection{The \env{syntax} environment}
%
% These environments are full of boxes that are moved around.
% Getting the mc-chunks right is not trivial.
%
% TODO: check if one should change/simplify the tagging of the inner minipage.
% \begin{macro}{\__codedoc_syntax:w}
%    \begin{macrocode}
\cs_set_protected:Npn \__codedoc_syntax:w
  {
    \box_if_empty:NF \g__codedoc_syntax_box
      { \msg_error:nn { l3doc } { multiple-syntax } }
    \dim_set:Nn \l__codedoc_syntax_dim
      {
        \textwidth
        \bool_if:NT \l__codedoc_long_name_bool
          { + \marginparwidth - \l__codedoc_trial_width_dim }
      }
    \tag_mc_end_push:
    \hbox_gset:Nw \g__codedoc_syntax_box
      \tl_if_empty:NTF\l__codedoc_current_names_struct_tl
       {
         \tagstructbegin{tag=syntax}
       }
       {
%    \end{macrocode}
% TODO: check if firstkid is the right thing.
%    \begin{macrocode}
         \tagstructbegin{tag=syntax,firstkid,parent=\l__codedoc_current_names_struct_tl}
       }
      \small \ttfamily
      \tagpdfsetup{table/tagging=false}
      \tagpdfparaOff
      \arrayrulecolor{white}
      \begin{tabular} { @{} p{\l__codedoc_syntax_dim} @{} }
        \toprule
        \begin{minipage}[t]{\l__codedoc_syntax_dim}
          \raggedright
          \obeyspaces
          \obeylines
  }
%    \end{macrocode}
% \end{macro}
%
%
% \begin{macro}{\__codedoc_syntax_end:}
%    \begin{macrocode}
\cs_set_protected:Npn \__codedoc_syntax_end:
  {
       \end{minipage}
      \end{tabular}
      \arrayrulecolor{black}
    \tagstructend
    \hbox_gset_end:
    \tag_mc_begin_pop:n{}
    \bool_if:NF \l__codedoc_in_function_bool
      {
        \begin{quote}
          \mode_leave_vertical:
          \box_use_drop:N \g__codedoc_syntax_box
        \end{quote}
      }
  }
%    \end{macrocode}
% \end{macro}
%
% \subsection{The \env{macro} environment}
% The macro environment is difficult: It collects various content
% in boxes and outputs everything at the end in \cs{__codedoc_macro_dump:},
% including the command names in the argument which are output in the margin.
% A macro environment can have more than one name in
% the argument and the environment be nested and the respective command names
% are then combined with the outer names.
% A command name can link to a \env{function} environment,
% which mean that it isn't simple text but contains a structure.
% This means that one has to collect structure numbers to insert them when needed.
%
%
% \begin{macro}{\__codedoc_macro_dump:}
% Here we have larger changes as we need also to replace the trivlist by a displayblock.
%    \begin{macrocode}
\cs_set_protected:Npn \__codedoc_macro_dump:
  {
    \int_compare:nNnF{\l__codedoc_nested_macro_int}>{1}
     {
       \begin{displayblock} [tag-name=macro,beginsep=\MacroTopsep]
%    \end{macrocode}
% we add here a container for the macronames which are built later and use
% the structures from the list.
%    \begin{macrocode}
       \tagstructbegin{tag=macronames}
       \tl_set:Ne \l__codedoc_current_names_struct_tl {\tag_get:n{struct_num}}
       \seq_map_inline:Nn \g__codedoc_macroname_structnum_seq
         {\tag_struct_use_num:n {##1}}
       \seq_gclear:N \g__codedoc_macroname_structnum_seq
       \tagstructend
     }
   \noindent\llap
          { \tagmcend
            \hbox_unpack_drop:N \l__codedoc_macro_index_box
            \vtop to \baselineskip
              {
                \vbox_unpack_drop:N \l__codedoc_macro_box
                \vss
              }
            \hspace{\labelsep}
            \tagmcbegin{}
          }
   }
%    \end{macrocode}
% \end{macro}
%
% \begin{macro}{\__codedoc_macro_typeset_one:nN}
% The \cs{__codedoc_macro_typeset_one:nN} command appends one macro name
% to the \cs{l__codedoc_macro_box}. We have to add a structure.
% If we are on the outer level we have to record the structure number.
% In the inner level we can use the existing structure.
% As the structures are used in another place, we have to push/pop the mc.
%    \begin{macrocode}
\cs_set_protected:Npn \__codedoc_macro_typeset_one:nN #1#2
  {
    \tag_mc_end_push:
    \vbox_set:Nn \l__codedoc_macro_box
      {
        \vbox_unpack_drop:N \l__codedoc_macro_box
        \int_compare:nNnTF { \l__codedoc_nested_macro_int } = { 1 }
         {
           \tagstructbegin{tag=macroname,stash}
            \seq_gput_right:Ne
             \g__codedoc_macroname_structnum_seq{\tag_get:n{struct_num}}
         }
         {
           \tagstructbegin{tag=macroname,parent=\l__codedoc_current_names_struct_tl}
         }
        \tagmcbegin{}
        \hbox { \llap { \__codedoc_print_macroname:nN {#1} #2
            \MacroFont \
        } }
        \tagmcend
        \tagstructend
      }
    \tag_mc_begin_pop:n{}
    \int_incr:N \l__codedoc_macro_int
  }
%    \end{macrocode}
% \end{macro}
%
% \begin{macro}{\__codedoc_macro_reset:}
% As we have no nested lists, we need to ignore spaces explicitly
%    \begin{macrocode}
\cs_set_protected:Npn \__codedoc_macro_reset:
  {
    \tl_set:Nn \l__codedoc_override_module_tl { \q_no_value }
    \ignorespaces
  }
%    \end{macrocode}
% \end{macro}
%
% \begin{macro}{\__codedoc_macro_end:}
%    \begin{macrocode}
\cs_set_protected:Npn \__codedoc_macro_end:
  {
    \__codedoc_macro_end_check_tested:
    \int_compare:nNnT \l__codedoc_nested_macro_int = 1
      {
        \par \__codedoc_macro_end_style:n { \__codedoc_print_end_definition: }
        \end{displayblock}
      }
  }
%    \end{macrocode}
% \end{macro}
%
% \subsection{The \env{macrocode} environment}
%
%
% \begin{macro}[no-user-doc]{\legacymacrocodesetup}
% We want to base the environment on the new template code.
% This is a counterpart to the \cs{legacyverbatimsetup}
% command in the latex-lab-block code.
%    \begin{macrocode}
\def\legacymacrocodesetup{%
  \macro@font
  \blank@linefalse \def\par{\ifblank@line
                             \leavevmode \else \fi
%    \end{macrocode}
% Similar to the case for verbatim we must group the
% \cs{@@par} so that we do not loose indentation on the first
% line
%    \begin{macrocode}
                             \blank@linetrue{\@@par}
                             \penalty\interlinepenalty}
   \obeylines
   \@noligs
   \let\do\@makeother \dospecials
   \global\@newlistfalse
   \global\@minipagefalse
   \ifcodeline@index

     \everypar{\global\advance\c@CodelineNo\@ne
               \llap{\tagmcend\tagstructbegin{tag=codelinenum}
               \tagmcbegin{}
               \theCodelineNo\pdffakespace
               \tagmcend\tagstructend
               \tagmcbegin{}%
               \ }%
               \check@module}%
   \else \everypar{\check@module}%
   \fi
   \tagtool{paratag=codeline}% check tagging
   }
%    \end{macrocode}
% \end{macro}
% \begin{instance}{blockenv macrocode}
%    \begin{macrocode}
\DeclareInstance{blockenv}{macrocode}{display}
{
  env-name       = macrocode,
  tag-name       = verbatim,
  tag-class      = ,
  tagging-recipe = standard,
  inner-level-counter  = ,
  level-increase = false,
  setup-code     = ,
  block-instance = verbatimblock ,
  inner-instance = ,
  final-code     = \legacymacrocodesetup ,
  para-instance  = justify,
  para-flattened= true
}
%    \end{macrocode}
% \end{instance}
%
%  \begin{environment}{macrocode}
%    \begin{macrocode}
\RenewDocumentEnvironment{macrocode}{!O{}}
  {\UseInstance{blockenv}{macrocode}
    {beginpenalty=\predisplaypenalty,
     begin-skip=\MacrocodeTopsep,
     leftmargin=\MacroIndent,
     parindent=0pt,#1}%
    \@setupverbinvisiblespace
    \init@crossref
    \frenchspacing \@vobeyspaces
    \xmacro@code
  }
  {
   \ifpm@module \endgroup \pm@modulefalse \fi
   \everypar{}
   \endblockenv
   \close@crossref
  }
%    \end{macrocode}
% \end{environment}
%
% \subsection{New doc elements}
% Various documentation elements are declare with the \cs{NewDocElement}
% command. Here we patch the relevant commands to get tagged versions.
%
% \begin{macro}[no-user-doc]{\@doc@env@}
%    \begin{macrocode}
\long\def\@doc@env@#1#2#3{%
%    \end{macrocode}
% we setup a new role:
%    \begin{macrocode}
   \tagpdfsetup{role/new-tag=#2/Sect}
   \int_incr:N\l__codedoc_nested_macro_int
   \int_compare:nNnF{\l__codedoc_nested_macro_int}>{1}
    {\UseInstance{blockenv}{displayblock}
     {level-increase=false,tag-name=#2,beginsep=\MacroTopsep}
%    \end{macrocode}
% A container for the function names.
%    \begin{macrocode}
    \tagstructbegin{tag=macronames}
    \tl_set:Ne \l__codedoc_current_names_struct_tl {\tag_get:n{struct_num}}
    \tagstructend
    }
  \edef\saved@macroname{\string#3}%
    \if #1%
      \edef\saved@indexname{\expandafter\@gobble\saved@macroname}%
      \expandafter\ifx
                  \csname Code#2Index\endcsname
                  \CodeMacroIndex
      \else
        \record@index@type@save
          {\saved@indexname}{#2}%
      \fi
    \else
      \let\saved@indexname\saved@macroname
    \fi
    \def\makelabel##1{\noindent\llap{
      \tagmcend
      \tagstructbegin{tag=macroname,parent=\l__codedoc_current_names_struct_tl}
      \tagmcbegin{}
      ##1\hspace{\itemsep}
      \tagmcend
      \tagstructend
      \tagmcbegin{}
      }}%
    \int_compare:nNnTF{\l__codedoc_nested_macro_int}>{1}
     {
      \let\@tempa\@empty
      \count@\macro@cnt
      \loop\ifnum\count@>\z@
        \edef\@tempa{\@tempa\hbox{\strut}}\advance\count@\m@ne
      \repeat
      \edef\makelabel##1{\noindent\llap{
      \tagmcend
      \tagstructbegin{tag=macroname,parent=\l__codedoc_current_names_struct_tl}
      \tagmcbegin{}
      \vtop to\baselineskip{\@tempa\hbox{##1\kern\labelsep}\vss}
      \tagmcend
      \tagstructend
      \tagmcbegin{}
      }}%
      \advance\macro@cnt\@ne
     }
     { \macro@cnt\@ne }
    \ifdoc@noprint
    \else
      \edef\@tempa{%
        \noexpand\makelabel{
        \noexpand\doc@providetarget
        \noexpand\strut
        \noexpand\@nameuse{Print#2Name}{\saved@macroname}}}%
      \@tempa
    \fi
    \ifdoc@noindex\else
      \global\advance\c@CodelineNo\@ne
      \csname SpecialMain#2Index\expandafter\endcsname
        \expandafter{\saved@macroname}\nobreak
      \global\advance\c@CodelineNo\m@ne
    \fi
    \if#1\expandafter\DoNotIndex \expandafter {\saved@macroname}\fi
    \ignorespaces}
%    \end{macrocode}
% \end{macro}
%
% \begin{macro}[no-user-doc]{\doc@createenv}
% We must use \cs{endblockenv}.
%    \begin{macrocode}
\def\doc@createenv#1#2#3{%
  \@namedef{#3}{%
    \@ifnextchar[%]
    {\doc@env{#1}{#2}}{\doc@env{#1}{#2}[]}}%
  \@namedef{end#3}{\endblockenv}%
}
%    \end{macrocode}
% \end{macro}
%
% We must renew the environment doc element so that is uses the new code:
%    \begin{macrocode}
\RenewDocElement[macrolike = false ,
                idxtype   = env.  ,
                idxgroup  = environments ,
                printtype = \textit{env.}
               ]{Env}{environment}
%    \end{macrocode}
%
% \subsection{\cs{maketitle}}
% The \pkg{doc} package redefines \cs{maketitle}. So we have to reinstate the
% version from the title module.
%
%    \begin{macrocode}
\cs_if_exist:NT \__tag_patch_maketitle:
 {
   \cs_set_eq:NN \maketitle \__tag_patch_maketitle:
 }
%    \end{macrocode}
%
% \subsection{Sectioning commands and table of contents}
%
% \cs{l@section} must get the hooks:

% \begin{macro}[no-user-doc]{\l@section}
%    \begin{macrocode}
\cs_gset:Npn \l@section #1#2
  {
    \ifnum \c@tocdepth >\z@
      \addpenalty\@secpenalty
      \addvspace{1.0em \@plus\p@}
      \setlength\@tempdima{2.5em}  % was 1.5em
      \begingroup
        \parindent \z@ \rightskip \@pnumwidth
        \parfillskip -\@pnumwidth
        \leavevmode \bfseries
        \advance\leftskip\@tempdima
        \hskip -\leftskip
        \UseHookWithArguments{contentsline/text/before}{4}
           {\toclevel@section}{#1}{#2}{\@contentsline@destination}%
        \csname contentsline@text@1@format\endcsname{#1}
        \UseHookWithArguments{contentsline/text/after}{4}
           {\toclevel@chapter}{#1}{#2}{\@contentsline@destination}%
        \nobreak\hfil \nobreak\hb@xt@\@pnumwidth{\hss
        \UseHookWithArguments{contentsline/page/before}{4}
        {\toclevel@section}{#1}{#2}{\@contentsline@destination}%
        #2
        \UseHookWithArguments{contentsline/page/after}{4}
        {\toclevel@section}{#1}{#2}{\@contentsline@destination}%
        }\par
      \endgroup
    \fi
  }
%    \end{macrocode}
% \end{macro}
%
% \subsection{Support \pkg{fancyvrb}}
% l3doc uses \pkg{fancyvrb}.
% As the block code redefines \env{verbatim} at begin document,
% we have to overwrite that again:
%    \begin{macrocode}
\AtBeginDocument
  {
    \cs_gset_eq:NN \verbatim \Verbatim
    \cs_gset_eq:NN \endverbatim \endVerbatim
  }
%    \end{macrocode}
%
%    \begin{macrocode}
%</package>
%    \end{macrocode}
% \end{implementation}
\endinput
