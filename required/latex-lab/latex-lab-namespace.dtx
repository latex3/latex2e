% \iffalse meta-comment
%
%% File: latex-lab-namespace.dtx 
%% 
% Copyright (C) 2023-2025 LaTeX Project
%
% It may be distributed and/or modified under the conditions of the
% LaTeX Project Public License (LPPL), either version 1.3c of this
% license or (at your option) any later version.  The latest version
% of this license is in the file
%
%    https://www.latex-project.org/lppl.txt
%
%<*driver>
\documentclass{l3doc}
\usepackage{amstext}
\EnableCrossrefs
\CodelineIndex

\usepackage{todonotes}

\begin{document}
  \DocInput{latex-lab-namespace.dtx}
\end{document}
%</driver>
% \fi
% \title{Prototype reimplementation of \LaTeXe{}'s role mapping}
% \author{\LaTeX{} Project, initial implementation Ulrike Fischer}
% \date{v0.8c 2025-01-09}
%
% \maketitle
%
% \begin{documentation}
%
%
%
% \section{Introduction}
%
% This short file collects tag names from the \LaTeX{} namespace and their role mapping used
% in the latex-lab files. The names and the roles are not necessarily the final ones!
% 
% Later settings of the same tag overwrite previous settings.
% The syntax of a line is a list of three strings ending with commas:
% \begin{verbatim}
% latex-tag, PDF-tag, name space,
% \end{verbatim}
% 
% The name space is normally pdf2 with the exception of the following PDF-tags 
% which are only in pdf namespace: 
% Art, BlockQuote, TOC, TOCI, Index, Private, Quote, Note,
%  Reference, BibEntry, Code.
% 
% \end{documentation}
% \begin{implementation}
% \changes{v0.8c}{2025/01/12}{changed mapping of verbatim to Code as Sub is now allowed inside Code}
%    \begin{macrocode}
%<*ns-latex-lab>
%% \ProvidesExplFile {tagpdf-ns-latex-lab.def} {2023-09-04} {0.8}
%% {latex} {https://www.latex-project.org/ns/lab/2022}{}
%    \end{macrocode}
% \subsection{Role from tagpdf}
% The following tags are already set by tagpdf, so deleting here will not remove them.
%    \begin{macrocode}
title,        Title,    pdf2,
part,         Title,    pdf2,
%    \end{macrocode}
%    Headings differ in article and book class. This here is for article:
%    \begin{macrocode}
section,      H1,       pdf2, 
subsection,   H2,       pdf2,
subsubsection,H3,       pdf2,
paragraph,    H4,       pdf2, 
subparagraph, H5,       pdf2,
%    \end{macrocode}
%    And this here is used in book:
%    \begin{macrocode}
%chapter,       H1,      pdf2,
%section,       H2,      pdf2,
%subsection,    H3,      pdf2,
%subsubsection, H4,      pdf2,
%paragraph,     H5,      pdf2,
%subparagraph,  H6,      pdf2,
%    \end{macrocode}
%
%    \begin{macrocode}
list,         L,        pdf2,
itemize,      L,        pdf2,
enumerate,    L,        pdf2,
description,  L,        pdf2,
quote,        BlockQuote,pdf,
quotation,    BlockQuote,pdf,
verbatim,     Code,     pdf,  % overwritten by block
item,         LI,       pdf2,
itemlabel,    Lbl,      pdf2,
itembody,     LBody,    pdf2,
footnote,     FENote,   pdf2,
footnotemark, Lbl,      pdf2,
footnotelabel,Lbl,      pdf2,
text-unit,    Part,     pdf2,
text,         P,        pdf2,
%    \end{macrocode}
% \subsection{From block code}
%    \begin{macrocode}
theorem-like, Sect,     pdf2,
verbatim,     Code,     pdf,
codeline,     Sub,      pdf2,
%    \end{macrocode}
% \subsection{From float code}
%    \begin{macrocode}
float,        Aside,    pdf2,
figures,      Sect,     pdf2,
tables,       Sect,     pdf2,
%    \end{macrocode}
%    \begin{macrocode}
%</ns-latex-lab>
%    \end{macrocode}
% \end{implementation}
\endinput
