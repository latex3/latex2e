% \iffalse meta-comment
%
%% File: latex-lab-tikz.dtx (C) Copyright 2025 LaTeX Project
%
% It may be distributed and/or modified under the conditions of the
% LaTeX Project Public License (LPPL), either version 1.3c of this
% license or (at your option) any later version.  The latest version
% of this license is in the file
%
%    https://www.latex-project.org/lppl.txt
%
%
% The development version of the bundle can be found below
%
%    https://github.com/latex3/latex2e/required/latex-lab
%
% for those people who are interested or want to report an issue.
%
\def\ltlabtikzdate{2024-10-09}
\def\ltlabtikzversion{0.80a}
%<*driver>
\documentclass{l3doc}
\EnableCrossrefs
\CodelineIndex
\begin{document}
  \DocInput{latex-lab-tikz.dtx}
\end{document}
%</driver>
%
% \fi
%
% \providecommand\tikzname{Ti\emph {k}Z}
% \title{The \textsf{latex-lab-tikz} package\\
% Support for the tagging of \tikzname\ pictures }
% \author{\LaTeX{} Project\thanks{Initial implementation done by Ulrike Fischer}}
% \date{v\ltlabtikzversion\ \ltlabtikzdate}
%
% \maketitle
%
% \newcommand{\xt}[1]{\textsl{\textsf{#1}}}
% \newcommand{\TODO}[1]{\textbf{[TODO:} #1\textbf{]}}
% \newcommand{\docclass}{document class \marginpar{\raggedright document class
% customizations}}
% \providecommand\hook[1]{\texttt{#1}}
%
% \NewDocElement[printtype=\textit{socket},idxtype=socket,idxgroup=Sockets]{Socket}{socketdecl}
% \NewDocElement[printtype=\textit{hook},idxtype=hook,idxgroup=Hooks]{Hook}{hookdecl}
% \NewDocElement[printtype=\textit{plug},idxtype=plug,idxgroup=Plugs]{Plug}{plugdecl}
%

%
% \begin{abstract}
% \end{abstract}
%
% \section{Introduction}
%
% Tagging of \tikzname\ (and other) pictures is non trivial.
%
% At first pictures generally can have various purposes:
% \begin{itemize}
% \item They can be purely ornamental and decorative, e.g. some page border.
% This should normally be tagged as artifact.
%
% \item They can show a illustrative figure, similar to png graphics included with
% \cs{includegraphics}.
% This should normally be tagged as a Figure structure with alternative text.
%
% \item They can be meant as normal text. For example the todonotes package uses a
% \tikzname\ picture to surround the text in a node with some colored background.
% In this case the text should be tagged e.g. as an Aside.
%
% \item They can represent a symbol. Then we want to tag as Span structure element
% with an /ActualText or perhaps even simply in the stream with a
% SPAN-BDC with an /ActualText.
%
% \item and naturally there can be all sort of mixtures of these elements.
% \end{itemize}
%
% At second \tikzname\ pictures uses lots of boxes and moves them around
% and that makes is not easy to get the tagging right --
% at least with pdflatex where one has to
% insert the literals at the right time.
%
% At third in some cases, e.g., when the \tikzname\ picture is tagged as Figure,
% one normally should calculate the BBox.
% This is currently done with some low-level hacking into the pgf code.
%
% The following is a first try to tag at least some of the \tikzname\ pictures.
% It is incomplete and should be used with care. Resulting structures and contents should
% be checked!
%
% The main idea of the implementation is to use sockets that allow to change the
% purpose of the \tikzname\ picture. This must be done before the actual environment as
% \tikzname\ processes the keys too late to allow to do this in the optional argument.
%
% \subsection{Tagging recipes}
%
% As \tikzname\ pictures have so varied purposes there are a number of
% \enquote{tagging recipes}. Currently some of the recipe must be set before the
% \tikzname\ picture with \verb+\tagpdfsetup{graphic/tagging=+\meta{recipe name}\verb+}+.
% The following recipes exist:
%
% \begin{description}
% \item[figure] This is the default recipe.
% It surrounds the picture with a \texttt{Figure} tag and adds a BBox.
% Inside the figure tagging is suspended.
% Such a figure should have an alternative text which describes the content. This alternative
% text can be set with the \texttt{alt} key:
% \begin{verbatim}
% \begin{tikzpicture}[alt=A duck]
% \duck
% \end{tikzpicture}
% \end{verbatim}
% This recipe is meant for meaningful pictures.
%
% \item[text] This surrounds the graphical parts with an artifact MC and activates
% tagging on node texts. It is meant for small pictures containing text in a node
% that should be part of the text flow, e.g. a todo.
%
% \item[artifact] This marks the picture as an artifact. This is meant for
% decorations.
%
% \item[symbol] This is meant for pictures where the drawing should
% represent a single symbol. Such pictures should then add an actualtext:
% \begin{verbatim}
% \begin{tikzpicture}[actualtext=A]
% % drawing of a A
% \end{tikzpicture}
% \end{verbatim}
%
% \end{description}
% \section{Todos}
% \begin{enumerate}
% \item
%
% \end{enumerate}
%
% \begin{implementation}
% \section{Implementation}
%    \begin{macrocode}
%<*package>
%<@@=tag>
%    \end{macrocode}
%    \begin{macrocode}
\ProvidesExplPackage {latex-lab-testphase-tikz} {\ltlabtikzdate} {\ltlabtikzversion}
  {Code related to the tagging of tikz pictures}
%    \end{macrocode}
%
% \subsection{Sockets}
%
% \begin{socketdecl}{tagsupport/tikzpicture/init,
% tagsupport/tikzpicture/begin,
% tagsupport/tikzpicture/end}
% Sockets at the begin and the end of a tikzpicture.
% The argument should process the keys of the picture and switch
% the plugs if needed.
%    \begin{macrocode}
\NewTaggingSocket{tikzpicture/init}{1}
\NewTaggingSocket{tikzpicture/begin}{0}
\NewTaggingSocket{tikzpicture/end}{0}
%    \end{macrocode}
% \end{socketdecl}
%
% \begin{socketdecl}{tagsupport/tikzpicture/textbegin,tagsupport/tikzpicture/textend}
% Sockets at the end and begin of text parts.
%    \begin{macrocode}
\NewSocket{tagsupport/tikzpicture/textbegin}{0}
\NewSocket{tagsupport/tikzpicture/textend}{0}
%    \end{macrocode}
% \end{socketdecl}
%%
%
% \subsection{Plugs}
%
% \begin{plugdecl}{default (tagsupport/tikzpicture/init)}
% The init socket takes a list of keys, processes the known keys to setup tagging options
% and then assigns the plugs.
% 
%    \begin{macrocode}
\NewTaggingSocketPlug{tikzpicture/init}{default}
 {
   \keys_set_known:nn { __tag / tikzpicture } {#1}  
 }
\AssignTaggingSocketPlug{tikzpicture/init}{default}
%    \end{macrocode}
% \end{plugdecl}
% \begin{plugdecl}{text (tagsupport/tikzpicture/begin),text (tagsupport/tikzpicture/end)}
% This plug handles the \tikzname\ picture as a text object. So the graphical parts
% are tagged as artifact, but when we encounter a node we activate tagging there.
% There is no Bbox.
%    \begin{macrocode}
\NewTaggingSocketPlug{tikzpicture/begin}{text}
 {
   \ifvmode
    {
     \UseTaggingSocket{para/begin}  %check
    }
   \fi
   \tag_mc_end_push:
   \tagmcbegin{artifact}
%    \end{macrocode}
% We hook into two pgf commands to add the tagging code.
% They are only used for postscript and svg so it should be
% safe inside a tagging socket for now.
% TODO: ask for an interface.
%    \begin{macrocode}
   \def\pgfsys@begin@text
    {
      \tag_resume:n{\tikzpicture}
      \tag_socket_use:n{tikzpicture/textbegin}
    }
   \def\pgfsys@end@text
    {
      \tag_socket_use:n{tikzpicture/textend}
      \tag_suspend:n{\tikzpicture}
    }
 }
\NewTaggingSocketPlug{tikzpicture/end}{text}
 {
   \tagmcend
   \tag_mc_begin_pop:n{}
 }
%    \end{macrocode}
% \end{plugdecl}
%
% \begin{plugdecl}{figure (tagsupport/tikzpicture/begin),figure (tagsupport/tikzpicture/end)}
% This plug handles the \tikzname\ picture as a figure.
% Around the graphic is a \texttt{Figure} environment which will
% use an alt text given in the optional argument and internally tagging is suspended.
% The Bbox will be set (after the second compilation) to the size of the bounding box.
%    \begin{macrocode}
\NewTaggingSocketPlug{tikzpicture/begin}{figure}
 {
   \ifvmode
    {
     \tag_socket_use:n{para/begin}
    }
   \fi
   \tag_mc_end_push:
   \tag_struct_begin:n{tag=Figure,
    alt=\l__tag_tikzpicture_option_tl%
    }
   \bool_set_false:N\l__tag_tikzpicture_usetext_bool
   \pgfrememberpicturepositiononpagetrue
   \tag_mc_begin:n{tag=Figure}
 }

\NewTaggingSocketPlug{tikzpicture/end}{figure}
 {
   \tag_mc_end:
   \cs_set:Npn\pgfqpoint##1##2
    {
      \dim_to_decimal_in_bp:n {##1+ \pgf@picminx}
      \c_space_tl
      \dim_to_decimal_in_bp:n {##2+ \pgf@picminy}
      \c_space_tl
      \dim_to_decimal_in_bp:n {##1+ \pgf@picmaxx}
      \c_space_tl
      \dim_to_decimal_in_bp:n {##2+ \pgf@picmaxx}
    }
   \cs_if_exist:cT { pgf@sys@pdf@mark@pos@pgfid\the\pgf@picture@serial@count }
    {
      \__tag_prop_gput:cne
        { g__tag_struct_ \g__tag_struct_stack_current_tl _prop }
        { A }
        {
           <<
             /O /Layout /BBox~
             [
               \use:c
                { pgf@sys@pdf@mark@pos@pgfid\the\pgf@picture@serial@count }
             ]
           >>
        }
     }
   \tag_struct_end:
   \tag_mc_begin_pop:n{}
  }
%    \end{macrocode}
% \end{plugdecl}
%
% \begin{plugdecl}{actualtext (tagsupport/tikzpicture/begin),actualtext (tagsupport/tikzpicture/end)}
% This plug handles the \tikzname\ picture as a symbol with an actualtext.
% It tags the content as a Span and expects and actual text.
% Internally tagging is suspended.
%    \begin{macrocode}
\NewTaggingSocketPlug{tikzpicture/begin}{actualtext}
 {
   \ifvmode
    {
     \tag_socket_use:n{para/begin}
    }
   \fi
   \tag_mc_end_push:
   \tag_struct_begin:n{tag=Span,actualtext=\l__tag_tikzpicture_option_tl}
   \tag_mc_begin:n{}
 }

\NewTaggingSocketPlug{tikzpicture/end}{actualtext}
 {
   \tag_mc_end:
   \tag_struct_end:
   \tag_mc_begin_pop:n{}
  }
%    \end{macrocode}
% \end{plugdecl}
%
% \begin{plugdecl}{artifact (tagsupport/tikzpicture/begin),artifact (tagsupport/tikzpicture/end)}
% This plug handles the \tikzname\ picture as an artifact, as decoration.
% So it is surrounded by an artifact MC and internal text does not restart tagging.
%    \begin{macrocode}
\NewTaggingSocketPlug{tikzpicture/begin}{artifact}
 {
   \ifvmode
    {
     \tag_socket_use:n{para/begin}
    }
   \fi
   \tag_mc_end_push:
   \tag_mc_begin:n{artifact}
 }

\NewTaggingSocketPlug{tikzpicture/end}{artifact}
 {
   \tag_mc_end:
   \tag_mc_begin_pop:n{}
  }
%    \end{macrocode}
% \end{plugdecl}
%  By default we use the text plugs
%    \begin{macrocode}
\AssignTaggingSocketPlug{tikzpicture/begin}{text}
\AssignTaggingSocketPlug{tikzpicture/end}{text}
%    \end{macrocode}
%
% We add the begin socket to the \cs{tikz@picture} and the \cs{tikz@opt} command. 
% This allows us to process the keys of the picture and then to suspend tagging.
%    \begin{macrocode}
\AddToHookWithArguments{cmd/tikz@picture/before}
 {
  \tag_socket_use:nn{tikzpicture/init}{#1}
  \tag_socket_use:n {tikzpicture/begin}
  \tag_suspend:n{\tikzpicture}
 }

\AddToHookWithArguments{cmd/tikz@opt/before}
 {
  \tag_socket_use:nn{tikzpicture/init}{#1}
  \tag_socket_use:n {tikzpicture/begin}
  \tag_suspend:n{\tikzpicture}
 }
%    \end{macrocode}
% The end socket is in the \cs{endpgfpicture} command.
%    \begin{macrocode}
\AddToHook{cmd/endpgfpicture/after}
 {
  \tag_resume:n{\tikzpicture}
  \tag_socket_use:n{tikzpicture/end}
 }
%    \end{macrocode}
%
% \begin{plugdecl}{text (tagsupport/tikzpicture/textbegin),
%  text (tagsupport/tikzpicture/textend)}
% These sockets are used inside the text
% plugs and ends the previous mc and restarts it after the text.
%    \begin{macrocode}
\NewTaggingSocketPlug{tikzpicture/textbegin}{text}
 {
  \tag_mc_end_push:
  \tag_mc_begin:n{}
 }
\NewTaggingSocketPlug{tikzpicture/textend}{text}
 {
  \tag_mc_end:
  \tag_mc_begin_pop:n{}
 }
\AssignTaggingSocketPlug{tikzpicture/textbegin}{text}
\AssignTaggingSocketPlug{tikzpicture/textend}{text}
%    \end{macrocode}
% \end{plugdecl}
%
% \subsection{Keys to change the tagging behaviour}
% These keys will be processed directly at the begin of the picture commands
% to change the tagging behaviour. They should also be usable for other 
% picture environments but for now we make the tikz specific.
% 
%    \begin{macrocode}
\tl_new:N \l__tag_tikzpicture_option_tl
\keys_define:nn { __tag / tikzpicture }
  {
    alt        .code:n = 
     {
       \tl_set:No \l__tag_tikzpicture_option_tl{#1}
       \AssignTaggingSocketPlug{tikzpicture/begin}{figure}
       \AssignTaggingSocketPlug{tikzpicture/end}{figure}
     },
    actualtext .code:n = 
     {
       \tl_set:No \l__tag_tikzpicture_option_tl{#1}
       \AssignTaggingSocketPlug{tikzpicture/begin}{actualtext}
       \AssignTaggingSocketPlug{tikzpicture/end}{actualtext}    
     },
    artifact   .code:n = 
     {
       \AssignSocketPlug{tagsupport/tikzpicture/begin}{artifact}
       \AssignSocketPlug{tagsupport/tikzpicture/end}{artifact}
     }
  }
%    \end{macrocode}
% 
%   
% \subsection{Hooking into \tikzname}
%    \begin{macrocode}
\AddToHook{package/tikz/after}
 {
%    \end{macrocode}
% we add an alt, actualtext and artifact key to avoid errors.
%    \begin{macrocode}
   \tikzset{alt/.code={},actualtext/.code={},artifact/.code={}}
 }
%    \end{macrocode}
%
% \subsection{tagpdf keys to switch the recipes}
%
%    \begin{macrocode}
\keys_define:nn { __tag / setup }
 {
   graphic/tagging .choice:,
   graphic/tagging/figure .code:n =
    {
      \AssignTaggingSocketPlug{tikzpicture/begin}{figure}
      \AssignTaggingSocketPlug{tikzpicture/end}{figure}
    },
   graphic/tagging/text .code:n =
    {
      \AssignTaggingSocketPlug{tikzpicture/begin}{text}
      \AssignTaggingSocketPlug{tikzpicture/end}{text}
    },
   graphic/tagging/actualtext .code:n =
    {
      \AssignTaggingSocketPlug{tikzpicture/begin}{actualtext}
      \AssignTaggingSocketPlug{tikzpicture/end}{actualtext}
    },
   graphic/tagging/artifact .code:n =
    {
      \AssignTaggingSocketPlug{tikzpicture/begin}{artifact}
      \AssignTaggingSocketPlug{tikzpicture/end}{artifact}
    },
 }
%    \end{macrocode}
%
% Handle todonotes. TODO This perhaps should go into firstaid instead
%    \begin{macrocode}
\AddToHook{package/todonotes/after}
{
\NewSocket{tagsupport/todonotes/todo}{0}
\NewSocketPlug{tagsupport/todonotes/todo}{default}
 {\tagpdfsetup{graphic/tagging=text}}
\AssignSocketPlug{tagsupport/todonotes/todo}{default}
%

\renewcommand{\todo}[2][]{%
  % Needed to output any dangling \item of a noskip section (see #36):
  \if@inlabel \leavevmode \fi
  \if@noskipsec \leavevmode \fi
  \if@todonotes@inlinepar
    \ifhmode
      \@bsphack
      \@todonotes@vmodefalse
    \else
      \@savsf\@m
      \@savsk\z@
      \@todonotes@vmodetrue
    \fi
     {\UseTaggingSocket{todonotes/todo}\@todo[#1]{#2}}%
    \@esphack%
    \if@todonotes@vmode \par \fi
  \else%
    {\UseTaggingSocket{todonotes/todo}\@todo[#1]{#2}}%
  \fi}
}
%
%    \end{macrocode}
%    \begin{macrocode}
%</package>
%    \end{macrocode}
%    \begin{macrocode}
%<*latex-lab>
\ProvidesFile{tikz-latex-lab-testphase.ltx}
        [\ltlabtikzdate\space v\ltlabtikzversion\space
         latex-lab wrapper tikz]

\RequirePackage{latex-lab-testphase-tikz}

%</latex-lab>
%    \end{macrocode}
% \end{implementation}
