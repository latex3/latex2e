% \iffalse meta-comment
%
%% File: latex-lab-tikz.dtx (C) Copyright 2025 LaTeX Project
%
% It may be distributed and/or modified under the conditions of the
% LaTeX Project Public License (LPPL), either version 1.3c of this
% license or (at your option) any later version.  The latest version
% of this license is in the file
%
%    https://www.latex-project.org/lppl.txt
%
%
% The development version of the bundle can be found below
%
%    https://github.com/latex3/latex2e/required/latex-lab
%
% for those people who are interested or want to report an issue.
%
\def\ltlabtikzdate{2025-04-28}
\def\ltlabtikzversion{0.80b}
%<*driver>
\documentclass{l3doc}
\EnableCrossrefs
\CodelineIndex
\begin{document}
  \DocInput{latex-lab-tikz.dtx}
\end{document}
%</driver>
%
% \fi
%
% \providecommand\tikzname{Ti\textit{k}Z}
% \title{The \textsf{latex-lab-tikz} package\\
% Support for the tagging of \tikzname\ pictures }
% \author{\LaTeX{} Project\thanks{Initial implementation done by Ulrike Fischer}}
% \date{v\ltlabtikzversion\ \ltlabtikzdate}
%
% \maketitle
%
% \newcommand{\xt}[1]{\textsl{\textsf{#1}}}
% \newcommand{\TODO}[1]{\textbf{[TODO:} #1\textbf{]}}
% \newcommand{\docclass}{document class \marginpar{\raggedright document class
% customizations}}
% \providecommand\hook[1]{\texttt{#1}}
%
% \NewDocElement[printtype=\textit{socket},idxtype=socket,idxgroup=Sockets]{Socket}{socketdecl}
% \NewDocElement[printtype=\textit{hook},idxtype=hook,idxgroup=Hooks]{Hook}{hookdecl}
% \NewDocElement[printtype=\textit{plug},idxtype=plug,idxgroup=Plugs]{Plug}{plugdecl}
%

%
% \begin{abstract}
% \end{abstract}
%
% \section{Introduction}
%
% This implements the tagging of \tikzname\ picture along the ideas
% described in latex-lab-graphic, please check its documentation for the
% background! 
% 
% Resulting structures and contents should be checked!
% 
% The code doesn't use the generic sockets documented in \texttt{latex-lab-graphic} 
% but instead setups sockets and keys that are \tikzname-specific. One reason for
% this approach is to avoid that the tagging \tikzname\ pictures has side effect on pictures
% but more importantly because as this implementation patches into
% tikz internals special code in the plugs is required.
% 
% The module is automatically loaded if \texttt{tagging=on} or \texttt{testphase=latest}
% is used. The patches of \tikzname\ commands can be suppressed by removing the
% code with the label \texttt{latex-lab-testphase-tikz} from the hook 
% \texttt{package/tikz/after} before loading \tikzname.
%
% \begin{implementation}
% \section{Implementation}
%    \begin{macrocode}
%<*package>
%<@@=tag>
%    \end{macrocode}
%    \begin{macrocode}
\ProvidesExplPackage {latex-lab-testphase-tikz} {\ltlabtikzdate} {\ltlabtikzversion}
  {Code related to the tagging of tikz pictures}
%    \end{macrocode}
%
% \subsection{Sockets}
%
% \begin{socketdecl}{
%   tagsupport/tikz/picture/init,
%   tagsupport/tikz/picture/begin,
%   tagsupport/tikz/picture/end}
% Sockets at the begin and the end of a tikzpicture.
% The argument in the \texttt{init} 
% should process the keys of the picture and switch
% the plugs if needed. Unlike the generic sockets the
% begin and end sockets here do not need arguments.
%    \begin{macrocode}
\NewTaggingSocket{tikz/picture/init}{1}
\NewTaggingSocket{tikz/picture/begin}{0}
\NewTaggingSocket{tikz/picture/end}{0}
%    \end{macrocode}
% \end{socketdecl}
%
% \begin{socketdecl}{tagsupport/tikz/picture/text/begin,tagsupport/tikz/picture/text/end}
% Sockets at the end and begin of text parts.
%    \begin{macrocode}
\NewSocket{tagsupport/tikz/picture/text/begin}{0}
\NewSocket{tagsupport/tikz/picture/text/end}{0}
%    \end{macrocode}
% \end{socketdecl}
%%
%
% \subsection{Plugs}
%
% \begin{plugdecl}{default (tagsupport/tikz/picture/init)}
% The init socket takes a list of keys, processes the known keys to setup tagging options
% and then assigns the plugs.
% TODO: correct module name? 
% TODO: tikz should probably use pgfkeys, but as l3keys are easier for me \ldots
%    \begin{macrocode}
\NewTaggingSocketPlug{tikz/picture/init}{default}
  {
    \keys_set_known:nn { tikz / tagging } {#1}  
  }
\AssignTaggingSocketPlug{tikz/picture/init}{default}
%    \end{macrocode}
% \end{plugdecl}
% 
% \begin{plugdecl}{text (tagsupport/tikz/picture/begin),text (tagsupport/tikz/picture/end)}
% This plug handles the \tikzname\ picture as a text object. So the graphical parts
% are tagged as artifact, but when we encounter a node we activate tagging there.
% There is no BBox. A \tikzname\ picture does not necessarly starts with a \cs{leavevmode}
% so we test for vmode.
%    \begin{macrocode}
\NewTaggingSocketPlug{tikz/picture/begin}{text}
  {
   \ifvmode
    {
     \if@inlabel \leavevmode \else \tag_socket_use:n{para/begin}\fi
    }
   \fi
   \tag_mc_end_push:
   \tagmcbegin{artifact}
%    \end{macrocode}
% We hook into two pgf commands to add the tagging code.
% They are only used for postscript and svg so it should be
% safe inside a tagging socket for now.
% TODO: ask for an interface.
%    \begin{macrocode}
   \def\pgfsys@begin@text
    {
      \tag_resume:n{\tikz/picture}
      \tag_socket_use:n{tikz/picture/text/begin}
    }
   \def\pgfsys@end@text
    {
      \tag_socket_use:n{tikz/picture/text/end}
      \tag_suspend:n{\tikzpicture}
    }
  }
\NewTaggingSocketPlug{tikz/picture/end}{text}
  {
   \tagmcend
   \tag_mc_begin_pop:n{}
  }
%    \end{macrocode}
% \end{plugdecl}
%
% \begin{plugdecl}{figure (tagsupport/tikz/picture/begin),figure (tagsupport/tikz/picture/end)}
% This plug handles the \tikzname\ picture as a figure.
% Around the graphic is a \texttt{Figure} environment which will
% use an alt text given in the optional argument and internally tagging is suspended.
% The Bbox will be set (after the second compilation) to the size of the bounding box.
%    \begin{macrocode}
\NewTaggingSocketPlug{tikz/picture/begin}{alt}
  {
    \ifvmode
     {
      \if@inlabel \leavevmode \else \tag_socket_use:n{para/begin}\fi
     }
    \fi
    \tag_mc_end_push:
%    \end{macrocode}
% TODO: The structure name should be adaptable, similar to the generic implementation
%    \begin{macrocode}
    \tag_struct_begin:n
     {
      tag=Figure,
      alt=\l__tikz_tagging_alt_tl
     }
    \pgfrememberpicturepositiononpagetrue
    \tag_mc_begin:n{tag=Figure}
  }

\NewTaggingSocketPlug{tikz/picture/end}{alt}
  {
   \tag_mc_end:
%    \end{macrocode}
% this is the code that calculates and sets the BBox attribute
%    \begin{macrocode}
   \cs_set:Npn\pgfqpoint##1##2
    {
      \dim_to_decimal_in_bp:n {##1+ \pgf@picminx}
      \c_space_tl
      \dim_to_decimal_in_bp:n {##2+ \pgf@picminy}
      \c_space_tl
      \dim_to_decimal_in_bp:n {##1+ \pgf@picmaxx}
      \c_space_tl
      \dim_to_decimal_in_bp:n {##2+ \pgf@picmaxx}
    }
   \cs_if_exist:cT { pgf@sys@pdf@mark@pos@pgfid\the\pgf@picture@serial@count }
    {
      \tag_struct_gput:ene
        {\tag_get:n{struct_num}}
        {attribute}
        {
          /O /Layout /BBox~
          [
            \use:c
             { pgf@sys@pdf@mark@pos@pgfid\the\pgf@picture@serial@count }
          ]
        }
     }
   \tag_struct_end:
   \tag_mc_begin_pop:n{}
  }
%    \end{macrocode}
% \end{plugdecl}
%
% \begin{plugdecl}{actualtext (tagsupport/tikz/picture/begin),actualtext (tagsupport/tikz/picture/end)}
% This plug handles the \tikzname\ picture as a symbol with an actualtext.
% It tags the content as a Span and expects an actualtext.
% Internally tagging is suspended.
%    \begin{macrocode}
\NewTaggingSocketPlug{tikz/picture/begin}{actualtext}
  {
   \ifvmode
    {
     \if@inlabel \leavevmode \else \tag_socket_use:n{para/begin}\fi
    }
   \fi
   \tag_mc_end_push:
   \tag_struct_begin:n{tag=Span,actualtext=\l__tikz_tagging_actualtext_tl}
   \tag_mc_begin:n{}
  }

\NewTaggingSocketPlug{tikz/picture/end}{actualtext}
  {
   \tag_mc_end:
   \tag_struct_end:
   \tag_mc_begin_pop:n{}
  }
%    \end{macrocode}
% \end{plugdecl}
%
% \begin{plugdecl}{artifact (tagsupport/tikz/picture/begin),artifact (tagsupport/tikz/picture/end)}
% This plug handles the \tikzname\ picture as an artifact, as decoration.
% So it is surrounded by an artifact MC and internal text does not restart tagging.
%    \begin{macrocode}
\NewTaggingSocketPlug{tikz/picture/begin}{artifact}
  {
    \ifvmode
     {
      \if@inlabel \leavevmode \else \tag_socket_use:n{para/begin}\fi
     }
    \fi
    \tag_mc_end_push:
    \tag_mc_begin:n{artifact}
  }

\NewTaggingSocketPlug{tikz/picture/end}{artifact}
  {
    \tag_mc_end:
    \tag_mc_begin_pop:n{}
  }
%    \end{macrocode}
% \end{plugdecl}
% 
%  By default we use the text plugs. This allow packages like todonotes to 
%  get a sensible tagging without changes.
%    \begin{macrocode}
\AssignTaggingSocketPlug{tikz/picture/begin}{text}
\AssignTaggingSocketPlug{tikz/picture/end}{text}
%    \end{macrocode}
%
% \begin{plugdecl}{default (tagsupport/tikz/picture/text/begin),
%  default (tagsupport/tikz/picture/text/end)}
% These sockets are used inside the text
% plugs and ends the previous mc and restarts it after the text.
%    \begin{macrocode}
\NewTaggingSocketPlug{tikz/picture/text/begin}{default}
  {
   \tag_mc_end:
   \tag_mc_begin:n{}
  }
\NewTaggingSocketPlug{tikz/picture/text/end}{default}
  {
   \tag_mc_end:
   \tag_mc_begin:n{artifact}
  }
\AssignTaggingSocketPlug{tikz/picture/text/begin}{default}
\AssignTaggingSocketPlug{tikz/picture/text/end}{default}
%    \end{macrocode}
% \end{plugdecl}
%
% \subsection{Patching tagging into tikz}
% We add the begin socket to the \cs{tikz@picture} command (it is also used
% by \cs{tikz}). 
% This allows us to process the keys of the picture, assign the plugs 
% and then to suspend tagging.
%    \begin{macrocode}
\AddToHook{package/tikz/after}
 {
   \AddToHookWithArguments{cmd/tikz@picture/before}
     {
       \tag_socket_use:nn{tikz/picture/init}{#1}
       \tag_socket_use:n {tikz/picture/begin}
       \tag_suspend:n{\tikzpicture}
     }
%    \end{macrocode}
% The end socket is in the \cs{endpgfpicture} command.
%    \begin{macrocode}
   \AddToHook{cmd/endpgfpicture/after}
     {
       \tag_resume:n{\tikzpicture}
       \tag_socket_use:n{tikz/picture/end}
     }
  }
%    \end{macrocode}
%
%
% \subsection{User interface: Keys to change the tagging behaviour}
% These keys will be processed directly at the begin of the picture commands
% to change the tagging behaviour. 
% They are specific to tikz and should probably better be implemented with
% pgfkeys \ldots.
%    \begin{macrocode}
\tl_new:N  \l__tikz_tagging_alt_tl
\tl_set:Nn \l__tikz_tagging_alt_tl {Alternative~text~missing!}
\tl_new:N  \l__tikz_tagging_actualtext_tl

\keys_define:nn { tikz / tagging }
  {
    alt        .code:n = 
     {
       \tl_if_empty:nF{#1}
         { \tl_set:Ne\l__tikz_tagging_alt_tl {\text_purify:n{#1}}}       
       \AssignTaggingSocketPlug{tikz/picture/begin}{alt}
       \AssignTaggingSocketPlug{tikz/picture/end}{alt}
       \def\pgfsys@begin@text{}
       \def\pgfsys@end@text{}
     },
    actualtext .code:n = 
     {
       \tl_if_empty:nF{#1}
         { \tl_set:Ne\l__tikz_tagging_actualtext_tl {\text_purify:n{#1}} }  
       \AssignTaggingSocketPlug{tikz/picture/begin}{actualtext}
       \AssignTaggingSocketPlug{tikz/picture/end}{actualtext}
       \def\pgfsys@begin@text{}
       \def\pgfsys@end@text{}    
     },
    artifact   .code:n = 
     {
       \AssignTaggingSocketPlug{tikz/picture/begin}{artifact}
       \AssignTaggingSocketPlug{tikz/picture/end}{artifact}
       \def\pgfsys@begin@text{}
       \def\pgfsys@end@text{}
     },
    tagging-setup .multichoice:,
    tagging-setup / text .code:n =
     {
       \AssignTaggingSocketPlug{tikz/picture/begin}{text}
       \AssignTaggingSocketPlug{tikz/picture/end}{text}
     },
    tagging-setup / unknown .code:n= 
      {
        \keys_set:nn { tag/graphic }{#1}
      }        
  }
%    \end{macrocode}
% 
%   
% \subsection{Key definitions for \tikzname}
%    \begin{macrocode}
\AddToHook{package/tikz/after}
  {
%    \end{macrocode}
% we add a dummy definition for alt, actualtext and artifact key to avoid errors,
% and to allow to set the variables outside the tikzpicture.
%    \begin{macrocode}
   \tikzset
     {
      alt/.code=
       {         
         \tl_set:Ne\l__tikz_tagging_alt_tl {\text_purify:n{#1}}
         \AssignTaggingSocketPlug{tikz/picture/begin}{alt}
         \AssignTaggingSocketPlug{tikz/picture/end}{alt}
         \def\pgfsys@begin@text{}
         \def\pgfsys@end@text{}  
       },
      actualtext/.code=
       {
         \tl_set:Ne \l__tikz_tagging_actualtext_tl {\text_purify:n{#1}}
         \AssignTaggingSocketPlug{tikz/picture/begin}{actualtext}
         \AssignTaggingSocketPlug{tikz/picture/end}{actualtext}
         \def\pgfsys@begin@text{}
         \def\pgfsys@end@text{}
       },
      artifact/.code=
       {
         \AssignTaggingSocketPlug{tikz/picture/begin}{artifact}
         \AssignTaggingSocketPlug{tikz/picture/end}{artifact}
         \def\pgfsys@begin@text{}
         \def\pgfsys@end@text{} 
       }
     }  
%    \end{macrocode}
%
% \subsection{\tikzname~ keys to switch the recipes}
% Todo: alt and actualtext should be able to handle a value, but how
% is that done in pgfkeys??
%    \begin{macrocode}
   \tikzset 
     {
       tagging-setup/.is~choice,
       tagging-setup/text/.code =
        {
          \AssignTaggingSocketPlug{tikz/picture/begin}{text}
          \AssignTaggingSocketPlug{tikz/picture/end}{text}
        }, 
       tagging-setup/alt/.code =
        {
          \AssignTaggingSocketPlug{tikz/picture/begin}{alt}
          \AssignTaggingSocketPlug{tikz/picture/end}{alt}
          \def\pgfsys@begin@text{}
          \def\pgfsys@end@text{}
        },       
       tagging-setup/actualtext/.code =
        {
           \AssignTaggingSocketPlug{tikz/picture/begin}{actualtext}
           \AssignTaggingSocketPlug{tikz/picture/end}{actualtext}
           \def\pgfsys@begin@text{}
           \def\pgfsys@end@text{}    
        },           
       tagging-setup/artifact/.code =
        {
          \AssignTaggingSocketPlug{tikz/picture/begin}{artifact}
          \AssignTaggingSocketPlug{tikz/picture/end}{artifact}
          \def\pgfsys@begin@text{}
          \def\pgfsys@end@text{}
        },
     }
 }
%    \end{macrocode}
%
% Todonotes works with the default text recipe quite ok, details can
% be handled later.
%    \begin{macrocode}
%</package>
%    \end{macrocode}
%    \begin{macrocode}
%<*latex-lab>
\ProvidesFile{tikz-latex-lab-testphase.ltx}
        [\ltlabtikzdate\space v\ltlabtikzversion\space
         latex-lab wrapper tikz]

\RequirePackage{latex-lab-testphase-tikz}

%</latex-lab>
%    \end{macrocode}
% \end{implementation}
