% \CheckSum{2

% \iff
%    Tell the \LaTeX\ system who we are and write an entry on
%    transcript. Nothing to write to the .cfg file, if genera
%<*
\ProvidesFile{frenchb.
%</
% \changes{v2.1d}{2008/05/04}{Argument of \cs{ProvidesLanguage} cha
%     from `french' to `frenchb', otherwise \cs{listfiles} pr
%     no date/version information.  The bug with \cs{listfi
%     (introduced in v.1.5!), was pointed out by Ulrike Fisch
%<code>\ProvidesLanguage{fren
%\ProvidesFile{frenchb.
%<*!
        [2012/05/08 v2.5j French support from the babel sys
%</!
%<*
%% frenchb.cfg: configuration file for frenchb
%% Daniel Flipo daniel.flipo at fre
%</
%%    File `frenchb.
%%    Babel package for LaTeX versio
%%    Copyright (C) 1989 -
%%              by Johannes Braams, TeX

%<*!
%%    Frenchb language Definition
%%    Copyright (C) 1989 -
%%              by Johannes Braams, TeX
%%                 Daniel Flipo, GUTen

%%    Please report errors to: Daniel Flipo, GUTen
%%                             daniel.flipo at fre
%</!

%    This file is part of the babel system, it provides the so
%    code for the French language definition f

%<*filedri
\documentclass[a4paper]{ltx
\DeclareFontEncoding{T1}
\DeclareFontSubstitution{T1}{lmr}{m
\DeclareTextCommand{\guillemotleft}{OT
  {\fontencoding{T1}\fontfamily{lmr}\selectfont\char
\DeclareTextCommand{\guillemotright}{OT
  {\fontencoding{T1}\fontfamily{lmr}\selectfont\char
\newcommand*\TeXhax{\TeX
\newcommand*\babel{\textsf{bab
\newcommand*\langvar{$\langle \mathit lang \rang
\newcommand*\note[
\newcommand*\Lopt[1]{\textsf{
\newcommand*\file[1]{\texttt{
\begin{docum
\setlength{\parindent}{
\begin{cen
  \textbf{\Large A Babel language definition file for French}\\[3mm]^
  Daniel \textsc{Flip
  \texttt{daniel.flipo@free
\end{cen
 \RecordCha
 \DocInput{frenchb.
\end{docum
%</filedri
%
% \GetFileInfo{frenchb.

%  \section{The French langu

%    The file \file{\filename}\footnote{The file described in
%    section has version number \fileversion\ and was last revise
%    \filedate.}, defines all the language definition macros for
%    French langu

%    Customisation for the French language is achieved following
%    book ``Lexique des r\`egles typographiques en usage
%    l'Imprimerie nationale'' troisi\`eme \'edition (19
%    ISBN-2-11-08107

%    First version released: 1.1 (1996/05/31) as par
%    \babel-3.6b

%    |frenchb| has been improved using helpful suggestions from
%    people, mainly from Jacques Andr\'e, Michel Bovani, Thierry Bou
%    and Vincent Jalby.  Thanks to all of t

%    This new version (2.x) has been designed to be used with \LaTe
%    and Plain\TeX{} formats only. \LaTeX-2.09 is no longer suppor
%    Changes between version 1.6 and \fileversion{} are liste
%    subsection~\ref{ssec-changes} p.~\pageref{ssec-chang

%    An extensive documentation is available in French her
%    |http://daniel.flipo.free.fr/fren

%  \subsection{Basic interf

%    In a multilingual document, some typographic rules are lang
%    dependent, i.e. spaces before `double punctuation' (|:| |;|
%    |?|) in French, others concern the general layout (i.e. layou
%    lists, footnotes, indentation of first paragraphs of sections)
%    should apply to the whole docum

%    Starting with version~2.2, |frenchb| behaves differently accor
%    to \babel's \emph{main language} defined as the \emph{l
%    option\footnote{Its name is kept in \texttt{\textbacks
%           bbl@main@language}.} at \babel's loading.  When Frenc
%    not \babel's main language, |frenchb| no longer alters the gl
%    layout of the document (even in parts where French is the cur
%    language): the layout of lists, footnotes, indentation of f
%    paragraphs of sections are not customised by |frenc

%    When French is loaded as the last option of \babel, |fren
%    makes the following changes to the global layout, \emph{bot
%    French and in all other languages}\footno
%       For each item, hooks are provided to reset stan
%       \LaTeX{} settings or to emulate the behavior of former vers
%       of \texttt{frenchb} (see com
%       \texttt{\textbackslash frenchbsetup\{
%       section~\ref{ssec-custom}
%    \begin{enumer
%    \item the first paragraph of each section is inde
%          (\LaTeX{} on
%    \item the default items in itemize environment are set to
%          instead of `\textbullet', and all vertical spacing and
%          is deleted; it is possible to change `--' to something
%          (`---' for instance) using |\frenchbsetup
%    \item vertical spacing in general \LaTeX{} list
%          shorte
%    \item footnotes are displayed ``\`a la fran\c{c}ais
%    \end{enumer

%    Regarding local typography, the command |\selectlanguage{fren
%    switches to the French language\footno
%      \texttt{\textbackslash selectlanguage\{francai
%      and \texttt{\textbackslash selectlanguage\{frenchb\}} are
%      for backward compatibility but should no longer be use
%    with the following effe
%    \begin{enumer
%    \item French hyphenation patterns are made act
%    \item `double punctuation' (|:| |;| |!| |?|) is
%           active%\footnote{Actually, they are active in the w
%           document, only their expansions differ in French
%           outside French.} for correct spacing in Fre
%    \item |\today| prints the date in Fre
%    \item the caption names are translated into Fr
%          (\LaTeX{} on
%    \item the space after |\dots| is removed in Fre
%    \end{enumer

%    Some commands are provided in |frenchb| to make typeset
%    eas
%    \begin{enumer
%    \item French quotation marks can be entered using the comm
%          |\og| and |\fg| which work in \LaTeXe and Plain\
%          their appearance depending on what is available to
%          them; even if you use \LaTeXe{} \emph{and} |T1|-encod
%          you should refrain from entering the
%          |<<~French quotation marks~>>|: |\og| and |\fg| pro
%          better horizontal spac
%          |\og| and |\fg| can be used outside French, they typ
%          then English quotes `` and
%    \item A command |\up| is provided to typeset superscripts
%          |M\up{me}| (abbreviation for ``Madame''), |1\up{er}|
%          ``premier'').  Other commands are also provided
%          ordinals: |\ier|, |\iere|, |\iers|, |\ieres|, |\ie
%          |\iemes| (|3\iemes| prints 3\textsuperscript{e
%    \item Family names should be typeset in small capitals and n
%          be hyphenated, the macro |\bsc| (boxed small caps)
%          this, e.g., |Leslie~\bsc{Lamport}| will pro
%          Leslie~\mbox{\textsc{Lamport}}. Note that composed n
%          (such as Dupont-Durant) may now be hyphenated on expl
%          hyphens, this differs from |frenchb|~v.
%    \item Commands |\primo|, |\secundo|, |\tertio| and |\qua
%          print 1\textsuperscript{o}, 2\textsuperscript
%          3\textsuperscript{o}, 4\textsuperscript
%          |\FrenchEnumerate{6}| prints 6\textsuperscript
%    \item Abbreviations for ``Num\'ero(s)'' and ``num\'ero(
%          (N\textsuperscript{o} N\textsuperscript
%          n\textsuperscript{o} and n\textsuperscript{o
%          are obtained via the commands |\No|, |\Nos|, |\no|, |\n
%    \item Two commands are provided to typeset the symbol
%          ``degr\'e'': |\degre| prints the raw character
%          |\degres| should be used to typeset temperatures (e
%          ``|20~\degres C|'' with an unbreakable space), or
%          alcohols' strengths (e.g., ``|45\degres|'' with \emph
%          space in Fren
%    \item In math mode the comma has to be surrounded
%          braces to avoid a spurious space being inserted after
%          in decimal numbers for instance (see the \TeX{}book p.~1
%          The command |\DecimalMathComma| makes the comma b
%          ordinary character \emph{in French only} (no space add
%          as a counterpart, if |\DecimalMathComma| is active
%          explicit space has to be added in lists and interv
%          |$[0,\ 1]$|, |$(x,\ y)$|. |\StandardMathComma| switches
%          to the standard behaviour of the co
%    \item A command |\nombre| was provided in 1.x versions to ea
%          format numbers in slices of three digits separated ei
%          by a comma in English or with a space in French; |\nom
%          is now mapped to |\numprint| from \file{numprint.sty},
%          \file{numprint.pdf} for more informat
%    \item |frenchb| has been designed to take advantage of the |xsp
%          package if present: adding |\usepackage{xspace}| in
%          preamble will force macros like |\fg|, |\ier|, |\ie
%          |\dots|, \dots, to respect the spaces you type after t
%          for instance typing `|1\ier juin|' will p
%          `1\textsuperscript{er} juin' (no need for a forced s
%          after |1\ie
%    \end{enumer

%  \subsection{Customisat
%  \label{ssec-cus

%     Up to version 1.6, customisation of |frenchb| was achi
%     by entering commands in \file{frenchb.cfg}.  This possibi
%     remains for compatibility, but \emph{should not longer be us
%     Version 2.0 introduces a new command |\frenchbsetup{}| u
%     the \file{keyval} syntax which should make it easier to ch
%     among the many options available. The command |\frenchbsetu
%     is to appear in the preamble only (after loading \bab

%     \vspace{.5\baselines
%     |\frenchbsetup{ShowOptions}| prints all available option
%     the \file{.log} file, it is just meant as a remainder of
%     list of offered options. As usual with \file{keyval} syn
%     boolean options (as |ShowOptions|) can be entere
%     |ShowOptions=true| or just |ShowOptions|, the `|=true|'
%     can be omit

%     \vspace{.5\baselines
%     The other options are listed below. Their default value is s
%     between brackets, sometimes followed be a `\texttt{
%     The `\texttt{*}' means that the default shown applies
%     |frenchb| is loaded as the \emph{last} option of \bab
%     ---\babel's \emph{main language}---, and is toggled otherw
%     \begin{item
%     \item |StandardLayout=true [false*]| forces |frenchb| no
%       interfere with the layout: no action on any kind of li
%       first paragraphs of sections are not indented (as in Engli
%       no action on footnotes. This option replaces the fo
%       command |\StandardLayout|.  It can be used to avoid confl
%       with classes or packages which customise lists or footno
%     \item |GlobalLayoutFrench=false [true*]| can be used, when Fr
%       is the main language, to emulate what prior version
%       |frenchb| (pre-2.2) did: lists, and first paragr
%       of sections will be displayed the standard way in o
%       languages than French, and ``\`a la fran\c{c}aise'' in Fre
%       Note that the layout of footnotes is language indepen
%       anyway (see below |FrenchFootnotes| and |AutoSpaceFootnote
%       This option replaces the former command |\FrenchLayo
%     \item |ReduceListSpacing=false [true*]|; |frenchb| norm
%       reduces the values of the vertical spaces used in
%       environment |list| in French; setting this option to |fa
%       reverts to the standard settings of |list|.  This op
%       replaces the former command |\FrenchListSpacingfal
%     \item |CompactItemize=false [true*]|; |frenchb| norm
%       redefines the |itemize| environmen
%       suppresse any vertical space between items of |itemize| l
%       in French; setting this option to |false| reverts to
%       standard settings of |itemize| lists.  This option repl
%       the former command |\FrenchItemizeSpacingfal
%     \item |StandardItemLabels=true [false*]| when set to |true|
%       option stops |frenchb| from changing the labels in |item
%       lists in Fre
%     \item |ItemLabels=\textemdash|, |\textbullet|, |\ding{4
%       \dots, |[\textendash*]|; when |StandardItemLabels=false|
%       default), this option enables to choose the label use
%       |itemize| lists for all levels.  The next three option
%       the same but each one for one level only. Note that
%       example |\ding{43}| requires |\usepackage{pifon
%     \item |ItemLabeli=\textemdash|, |\textbullet|, |\ding{4
%       \dots,|[\textendas
%     \item |ItemLabelii=\textemdash|, |\textbullet|, |\ding{4
%       \dots, |[\textendas
%     \item |ItemLabeliii=\textemdash|, |\textbullet|, |\ding{4
%       \dots, |[\textendas
%     \item |ItemLabeliv=\textemdash|, |\textbullet|, |\ding{4
%       \dots, |[\textendas
%     \item |StandardLists=true [false*]| forbids |frenchb
%       customise any kind of list. Do activate the op
%       |StandardLists| when using classes or packages that custo
%       lists too (|enumitem|, |paralist|, \dots{}) to avoid confli
%       This option is just a shorthand for |ReduceListSpacing=fa
%       and |CompactItemize=false| and |StandardItemLabels=tr
%     \item |IndentFirst=false [true*]|; |frenchb| normally fo
%       indentation of the first paragraph of secti
%       When this option is set to |false|, the first paragrap
%       will look the same in French and in English (not indent
%     \item |FrenchFootnotes=false [true*]| reverts to the stan
%       layout of footnotes. By default |frenchb| typesets lea
%       numbers as `1.\hspace{.5em}' instead of `$\hbox{}^1$',
%       has no effect on footnotes numbered with symbols (as in
%       |\thanks| command).  The former commands |\StandardFootno
%       and |\FrenchFootnotes| are still there, |\StandardFootno
%       can be useful when some footnotes are numbered with let
%       (inside minipages for instan
%     \item |AutoSpaceFootnotes=false [true*]| ; by default |fren
%       adds a thin space in the running text before the numbe
%       symbol calling the footnote.  Making this option |fa
%       reverts to the standard setting (no space add
%     \item |FrenchSuperscripts=false [true]| ;
%       |\up=\textsuperscript| (option added in version 2
%       Should only be made |false| to recompile older docume
%       By default |\up| now relies on |\fup| designed to pro
%       better looking superscri
%     \item |AutoSpacePunctuation=false [true]|; in French, the
%       \emph{should} input a space before the four characters `|:;
%       but as many people forget about it (even among native Fr
%       writers!), the default behaviour of |frenchb| i
%       automatically add a |\thinspace| before `|;|' `|!|' `|?|' a
%       normal (unbreakable) space before~`|:|' (this is recommende
%       the French Imprimerie nationale).  This is convenient in
%       cases but can lead to addition of spurious spaces in URLs o
%       MS-DOS paths but only if they are no typed using |\texttt
%       verbatim mode. When the current font is a monosp
%       (typewriter) font, |AutoSpacePunctuation| is locally swit
%       to |false|, no spurious space is added in that case, so
%       default behaviour of of |frenchb| in that area should be
%       in most circumstan

%       Choosing |AutoSpacePunctuation=false| will ensure
%       a proper space will be added before `|:;!?|' \emph{if and
%       if} a (normal) space has been typed in. Those who are un
%       about their typing in this area should stick to the def
%       option and type |\string;| |\string:| |\string!| |\stri
%       instead of |;| |:| |!| |?| in case an unwanted spac
%       added by |frenc
%     \item |ThinColonSpace=true [false]| changes the no
%       (unbreakable) space added before the colon `:' to a thin sp
%       so that the same amount of space is added before any of
%       four double punctuation characters. The default settin
%       supported by the French Imprimerie nation
%     \item |LowercaseSuperscripts=false [true]| ; by default |fren
%       inhibits the uppercasing of superscripts (for instance when
%       are moved to page headers). Making this option |fa
%       will disable this behaviour (not recommend
%     \item |PartNameFull=false [true]|; when true, |frenchb| num
%       the title of |\part{}| commands as ``Premi\`ere parti
%       ``Deuxi\`eme partie'' and so on. With some classes which ch
%       the|\part{}| command (AMS and SMF classes do so), you will
%       ``Premi\`ere partie~I'', ``Deuxi\`eme partie~II'' inst
%       when this occurs, this option should be set to |fal
%       part titles will then be printed as ``Partie I'', ``Partie I
%     \item |SuppressWarning=true [false]|; when true |frenchb| is
%       no warnings if |\@makecaption| has been redefined or if
%       bigfoot package is in
%     \item |og=|\texttt{\guillemotleft}, |fg=|\texttt{\guillemotrig
%       when guillemets characters are available on the keyb
%       (through a compose key for instance), it is nice to use
%       instead of typing |\og| and |\fg|. This option tells |fren
%       which characters are opening and closing French guille
%       (they depend on the input encoding), then you can type ei
%       \texttt{\guillemotleft{} guillemets \guillemotright}
%       \texttt{\guillemotleft{}guillemets\guillemotright} (wit
%       without spaces), to get properly typeset French quo
%       This option requires \file{inputenc} to be loaded with
%       proper encoding, it works with 8-bits encodings (lat
%       latin9, ansinew,  applemac,\dots) and multi-byte encod
%       (utf8 and utf
%     \end{item

%  \subsection{Hyphenation che
%  \label{ssec-hyp

%    Once you have built your format, a good precaution would b
%    perform some basic tests about hyphenation in French.
%    \LaTeXe{} I suggest t
%    \begin{item
%    \item run the following file, with the encoding suitable
%      your machine (\textit{my-encoding} will be |latin1|
%      \textsc{unix} machines, |ansinew| for PCs running~Wind
%      |applemac| or |latin1| for Macintoshs, or |utf8|\dots\\[3mm]^
%      |%%% Test file for French hyphenation
%      |\documentclass{article
%      |\usepackage[|\textit{my-encoding}|]{inputenc
%      |\usepackage[T1]{fontenc} % Use LM font
%      |\usepackage{lmodern}     % for Frenc
%      |\usepackage[frenchb]{babel
%      |\begin{document
%      |\showhyphens{signal container \'ev\'enement alg\`ebre
%      |\showhyphens{|\texttt{signal container \'ev\'ene
%                     alg\`ebre}|
%      |\end{docume
%    \item check the hyphenations proposed by \TeX{} in your log-f
%      in French you should get with both 7-bit and 8-bit encodin
%      \texttt{si-gnal contai-ner \'ev\'e-ne-ment al-g\`ebre
%      Do not care about how accented characters are displayed in
%      log-file, what matters is the position of the `|-|' hy
%      signs \emph{on
%    \end{item
%    If they are all correct, your installation (probably) works f
%    if one (or more) is (are) wrong, ask a local wizard to see wh
%    going wrong and perform the test again (or e-mail me about
%    happens
%    Frequent mismatc
%    \begin{item
%    \item you get |sig-nal con-tainer|, this probably means that
%    hyphenation patterns you are using are for US-English, not
%    Fre
%    \item you get no hyphen at all in \texttt{\'ev\'e-ne-ment},
%    probably means that you are using CM fonts and the m
%    |\accent| to produce accented charact
%    Using 8-bits fonts with built-in accented characters av
%    this kind of misma
%    \end{item

%    \textbf{Options' order} -- Please remember that options are
%    in the order they appear inside the |\frenchbsetup| comm
%    Someone wishing that |frenchb| leaves the layout of l
%    and footnotes untouched but caring for indentation of f
%    paragraph of sections could ch
%    |\frenchbsetup{StandardLayout,IndentFirst}| and get the expe
%    layout. Choosing |\frenchbsetup{IndentFirst,StandardLayo
%    would not lead to the expected result: option |IndentFirst| w
%    be overwritten by |StandardLayo

%  \subsection{Chan
%  \label{ssec-chan

%  \subsubsection*{What's new in version 2

%    Here is the list of all chan
%    \begin{item
%    \item Support for \LaTeX-2.09 and for \LaTeXe{} in compatibi
%      mode has been dropped. This version is meant for \LaTeXe{}
%      Plain based formats (like \file{bplain}). \LaTeXe{} for
%      based on ml\TeX{} are no longer supported either (plent
%      good 8-bits fonts are available now, so T1 encoding sh
%      be preferred for typesetting in French). A warning is is
%      when OT1 encoding is in use at the |\begin{documen
%    \item Customisation should now be handled by com
%      |\frenchbsetup{}|, \file{frenchb.cfg} (kept for compatibil
%      should no longer be used. See section~\ref{ssec-custom}
%      the list of available opti
%    \item Captions in figures and table have changed in French: fo
%      abbreviations ``Fig.'' and ``Tab.'' have been replaced by
%      names ``Figure'' and ``Table''.  If this leads to format
%      problems in captions, you can add the following two command
%      your preamble (after loading \babel) to get the former captio
%      |\addto\captionsfrench{\def\figurename{{\scshape Fig.}}
%      |\addto\captionsfrench{\def\tablename{{\scshape Tab.}
%    \item The |\nombre| command is now provided by the \file{numpr
%      package which has to be loaded \emph{after} \babel{} with
%      option |autolanguage| if number formatting should depend on
%      current langu
%    \item The |\bsc| command no longer uses an |\hbox| to
%      hyphenation of names but a |\kern0pt| instead. This ch
%      enables \file{microtype} to fine tune the length of
%      argument of |\bsc|; as a side-effect, compound names
%      Dupont-Durand can now be hyphenated on  explicit hyph
%      You can get back to the former behaviour of |\bsc| by addi
%      |\renewcommand*{\bsc}[1]{\leavevmode\hbox{\scshape #1}
%      to the preamble of your docum
%    \item Footnotes are now displayed ``\`a la fran\c caise'' for
%      whole document, except with an explic
%      |\frenchbsetup{AutoSpaceFootnotes=false,FrenchFootnotes=false}
%      Add this command if you want standard footnotes. It is s
%      possible to revert locally to the standard layout of footn
%      by adding |\StandardFootnotes| (inside a |minipage| environ
%      for instan
%    \end{item

%  \subsubsection*{What's new in version 2

%      New command |\fup| to typeset better looking superscri
%      Former command |\up| is now defined as |\fup|, an op
%      |\frenchbsetup{FrenchSuperscripts=false}| is provided
%      backward compatibility.  |\fup| was designed using ideas
%      Jacques Andr\'e, Thierry Bouche and Ren\'e Fritz, thanks to t

%  \subsubsection*{What's new in version 2

%      Starting with version~2.2a, |frenchb| alters the layou
%      lists, footnotes, and the indentation of first paragraph
%      sections) \emph{only if} French is the ``main langua
%      (i.e. babel's last language option). The layout is global
%      the whole document: lists, etc. look the same in French an
%      other languages, everything is typeset ``\`a la fran\c cai
%      if French is the ``main language'', otherwise |frenchb| doe
%      change anything regarding lists, footnotes, and indentatio
%      paragra

%  \subsubsection*{What's new in version 2

%      Starting with version~2.3a, |frenchb| no longer inserts sp
%      automatically before `|:;!?|' when a typewriter font is in
%      this was suggested by Yannis Haralambous to pre
%      spurious spaces in computer source code or expressions
%      \texttt{C\string:/foo}, \texttt{http\string://foo.b
%      etc.  An option (|OriginalTypewriter|) is provided to get
%      to the former behaviour of |frenc

%      Another probably invisible change: lowercase conversio
%      |\up{}| is now achieved by the \LaTeX{} command |\MakeLowerc
%      instead of \TeX's |\lowercase| command.  This prevents e
%      messages when diacritics are used inside |\up{}| (diacri
%      should \emph{never} be used in superscripts thoug

%  \subsubsection*{What's new in version 2

%      A new option |SuppressWarning| has been added (desactivate
%      default) to suppress warnings if |\@makecaption| has
%      redefined or if the bigfoot package is in

%      French hyphenation patterns are now coded in Unicode, see
%      \file{hyph-fr.tex}.  Extra code has been added to deal
%      hyphenation of the French ``apostrophe'' with XeTeX and Lu
%      engi

%      Better compatibility with the enumitem pack

%      When typewriter fonts are in use (hence in verbatim mode
%      space is added after `\guillemotleft' and be
%      `\guillemotright' when they are entered as charac
%      (see |\frenchbsetu

%  \subsubsection*{What's new in version 2

%      The main change is that active characters are no longer
%      in French with (recent) Xe\TeX-based engines (they still
%      with \TeX-based engines).  All the functionalities (autom
%      insertion of missing spaces before |:;!?| or bare replace
%      of typed spaces with suitable unbreabable ones, tuning of
%      spaces width) remain available and the user interfac
%      unchanged. The use of active characters is replaced by
%      |\XeTeXinterchartoks| mechanism (as in pac
%      \file{polyglossi

%      A new command |\NoAutoSpacing| has been added. It should be
%      \emph{inside a group} instead of |\shorthandoff{;:!?}| when
%      active characters or automatic spacing of French punctuatio
%      quote characters conflict with other packages; it is designe
%      work with \TeX- and Xe\TeX-based engi

%      Bug correction: |\frenchspacing| and |\nonfrenchspacing| ar
%      longer messed up by \file{frenchb.l

% \StopEventual

%  \subsection{File frenchb.
%  \label{sec-

%    \file{frenchb.cfg} is now a dummy file just kept for compatibi
%    with previous versi

% \iff
%<*
%
%    \begin{macroc
%%%%%%%%%%%%%%%%%%%%%%%%%%%%%%%%%%%%%%%%%%%%%%%%%%%%%%%%%%%%%%%%%%
%%%%%%%%%  WARNING: THIS  FILE SHOULD  NO  LONGER  BE  USED  %%%%%
%% If you want to customise frenchb, please DO NOT hack into the c
%% Do no put any code in this file either, please use the new com
%% \frenchbsetup{} with the proper options to customise fren

%% Add \frenchbsetup{ShowOptions} to your preamble to see the lis
%% available options and/or read the documentat
%%%%%%%%%%%%%%%%%%%%%%%%%%%%%%%%%%%%%%%%%%%%%%%%%%%%%%%%%%%%%%%%%%
%    \end{macroc
% \iff
%</
%

%  \subsection{Initial se

% \changes{v2.1d}{2008/05/02}{Argument of \cs{ProvidesLanguage} cha
%     above from `french' to `frenchb' (otherwise \cs{listfiles} pr
%     no date/version information).  The real name of current lang
%     (french) as to be corrected before calling \cs{LdfIni

% \iff
%<*c
%

%    While this file was read through the option \Lopt{frenchb} we
%    it behave as if \Lopt{french} was specif
%    \begin{macroc
\def\CurrentOption{fre
%    \end{macroc

%    The macro |\LdfInit| takes care of preventing that this fil
%    loaded more than once, checking the category code of
%    \texttt{@} sign,

%    \begin{macroc
\LdfInit\CurrentOption\datefr
%    \end{macroc

% \changes{v2.1d}{2008/05/04}{Avoid warning ``\cs{end} occu
%   when \cs{ifx} ... incomplete'' with LaTeX-2.

%  \begin{macro}{\ifLaT
%    No support is provided for late \LaTeX-2.09: issue a war
%    and exit if \LaTeX-2.09 is in use. Plain is still suppor
%    \begin{macroc
\newif\ifLa
\let\bbl@tempa\r
\ifx\magnification\@undef
   \ifx\@compatibilitytrue\@undef
     \PackageError{frenchb.
        {LaTeX-2.09 format is no longer supported.\MessageB
         Aborting h
        {Please upgrade to LaTeX
     \let\bbl@tempa\endi
   \
     \LaTeXe


\bbl@t
%    \end{macroc
%  \end{ma

%    Check if hyphenation patterns for the French language have
%    loaded in language.dat; we allow for the names `fren
%    `francais', `canadien' or `acadian'. The latter two are
%    names used in Canada for variants of French that are in us
%    that coun

%    \begin{macroc
\ifx\l@french\@undef
  \ifx\l@francais\@undef
    \ifx\l@canadien\@undef
      \ifx\l@acadian\@undef
        \@nopatterns{Fre
        \adddialect\l@fre
      \
        \let\l@french\l@aca

    \
      \let\l@french\l@cana

  \
    \let\l@french\l@fran


%    \end{macroc
%    Now |\l@french| is always defi

%    The internal name for the French language is |fren
%    |francais| and |frenchb| are synonymous for |fren
%    first let both names use the same hyphenation patte
%    Later we will have to set aliases for |\captionsfren
%    |\datefrench|, |\extrasfrench| and |\noextrasfren
%    As French uses the standard values of |\lefthyphenmin|
%    and |\righthyphenmin| (3), no special setting is required h

%    \begin{macroc
\ifx\l@francais\@undef
  \let\l@francais\l@fr

\ifx\l@frenchb\@undef
  \let\l@frenchb\l@fr

%    \end{macroc
%    When this language definition file was loaded for one of
%    Canadian versions of French we need to make sure that a suit
%    hyphenation pattern register will be found by \
%    \begin{macroc
\ifx\l@canadien\@undef
  \let\l@canadien\l@fr

\ifx\l@acadian\@undef
  \let\l@acadian\l@fr

%    \end{macroc

%    This language definition can be loaded for different variant
%    the French language. The `key' \babel\ macros are only def
%    once, using `french' as the language name, but |frenchb|
%    |francais| are synonym
%    \begin{macroc
\def\datefrancais{\datefre
\def\datefrenchb{\datefre
\def\extrasfrancais{\extrasfre
\def\extrasfrenchb{\extrasfre
\def\noextrasfrancais{\noextrasfre
\def\noextrasfrenchb{\noextrasfre
%    \end{macroc

% \begin{macro}{\extrasfre
% \begin{macro}{\noextrasfre
%    The macro |\extrasfrench| will perform all the e
%    definitions needed for the French langu
%    The macro |\noextrasfrench| is used to cancel the action
%    |\extrasfrench
%    In French, character ``apostrophe'' is a letter in express
%    like |l'ambulance| (French  hyphenation patterns provide ent
%    for this kind of words).  This means that the |\lccode
%    ``apostrophe'' has to be non null in French for proper hyphena
%    of those expressions, and has to be reset to null when exi
%    Fre

%  \begin{macro}{\ifFBunic
%  \begin{macro}{\ifFLua
%  \begin{macro}{\ifFXe
%    French hyphenation patterns are now coded in Unicode, see
%    \file{hyph-fr.tex}.  XeTeX and LuaTeX engines require some e
%    code to deal with the French ``apostroph
%    Let's define three new `if': |\ifFBLuaTeX|, |\ifFBXeTeX|
%    |\ifFBunicode| which will be true for XeTeX and LuaTeX eng
%    and false for 8-bits engi

% \changes{v2.4a}{2009/11/23}{Added a new `if' \cs{FBunicode}
%    some \cs{lccode} definitions to \cs{extrasfrench}
%    \cs{noextrasfrenc

% \changes{v2.5d}{2011/01/17}{Added two new `if' \cs{FBXeTeX}
%    \cs{FBLuaTeX} as XeTeX and behave differently regarding the st
%     of the French ``apostrophe

%    \begin{macroc
\newif\ifFBuni
\newif\ifFBLu
\newif\ifFBX
\begingroup\expandafter\expandafter\expandafter\endg
\expandafter\ifx\csname luatexversion\endcsname\r
\
  \FBunicodetrue \FBLuaTeX

\begingroup\expandafter\expandafter\expandafter\endg
\expandafter\ifx\csname XeTeXrevision\endcsname\r
\
  \FBunicodetrue \FBXeTeX

%    \end{macroc
%    These |\lccode| changes will ensure correct hyphenatio
%    words like |d'aventure|, |l'utopie|, with all TeX eng
%    (XeTeX, LuaTeX, pdfTeX) using \file{hyph-fr.tex} patte
%    \begin{macroc
\@namedef{extras\CurrentOption}{\lccode`\'
                    \ifFBLuaTeX \lccode`\'="2019
                    \ifFBXeTeX  \lccode"2019=`\'
\@namedef{noextras\CurrentOption}{\lccode`
                    \ifFBXeTeX  \lccode"2019=0
%    \end{macroc
% \end{ma
% \end{ma
% \end{ma
% \end{ma
% \end{ma

%    One more thing |\extrasfrench| needs to do is to make sure
%    ``Frenchspacing'' is in effect.  |\noextrasfrench| will sw
%    ``Frenchspacing'' off again if necess
%    \begin{macroc
\addto\extrasfrench{\bbl@frenchspac
\addto\noextrasfrench{\bbl@nonfrenchspac
%    \end{macroc

%  \subsection{Punctuat
%  \label{sec-pu

% \changes{v2.5a}{2010/08/10}{Punctation is no longer made ac
%    with XeTeX-based engin

%    As long as no better solution is available, the `do
%    punctuation' characters (|;| |!| |?| and |:|) have to be
%    |\active| for an automatic control of the amount of spac
%    insert before them.  Xe\TeX{} provides an alternative to ac
%    characters and Lua\TeX{} will hopefully do so as well in
%    (near?) fut

%    Before doing so, we have to save the standard definitio
%    |\@makecaption| (which includes two ':') to compare it late
%    its definition at the |\begin{documen
%    \begin{macroc
\long\def\STD@makecaption#1
  \vskip\abovecaption
  \sbox\@tempboxa{#1:
  \ifdim \wd\@tempboxa >\h
    #1: #2
  \
    \global \@minipagef
    \hb@xt@\hsize{\hfil\box\@tempboxa\hf

  \vskip\belowcaptions
%    \end{macroc

% \changes{v2.5a}{2010/08/10}{Define \cs{Fthinspace} for those who
%   to customise the width of the space before ; ! and

%    According to the I.N. specifications, the `:' requires a no
%    space before it, but some people prefer a |\thinspace| (
%    like the other three). We define |\Fcolonspace| to hold
%    required amount of space (user customisable). In case some u
%    are not satisfied with |\thinspace|'s width, it is
%    customisa
%    \begin{macroc
\newcommand*{\Fcolonspace}{\sp
\newcommand*{\Fthinspace}{\thinsp
%    \end{macroc

%  \begin{macro}{\ifF@active@pu
%  \begin{macro}{\ifF@xetex@pu
%    Check the availability of |\XeTeXinterchartokenstate| and de
%    whether the `double punctuation' characters (|;| |!| |?| and
%    have to be made |\active| or
%    \begin{macroc
\newif\ifFB@active@punct  \FB@active@punct
\newif\ifFB@xetex@p
\begingroup\expandafter\expandafter\expandafter\endg
\expandafter\ifx\csname XeTeXinterchartokenstate\endcsname\r
\
  \FB@xetex@puncttrue\FB@active@punctf

%    \end{macroc
%  \end{ma
%  \end{ma

%    If |\XeTeXinterchartokenstate| is available, we use
%    ``inter char'' mechanism (as in  polyglossia,
%    \file{gloss-french.ldf}) to provide correct spacing in Fr
%    before the four characters |;| |!| |?| and |:|.  We use the
%    mechanism for French quotes (\texttt{\guillemotleft}
%    \texttt{\guillemotright}), when automatic spacing for quote
%    required by options \texttt{og=} and \texttt{fg=
%    |\frenchbsetup{}| (see section~\ref{sec-keyva

%    For every character used in French text-mode (except spac
%    |\XeTeXcharclass| value must b
%    |\XeTeXcharclass| value for spaces is assumed to be
%    Otherwise, the spacing before the `double punctuation' charac
%    and inside quotes might not be corr

%    We switch |\XeTeXinterchartokenstate| to 1 and change
%    |\XeTeXcharclass| values of |;| |!| |?| |:| |(|
%    \texttt{\guillemotleft} and \texttt{\guillemotright}
%    entering French.  Special care is taken to restore them to t
%    inital values when leaving Fre

% \changes{v2.5d}{2011/01/19}{Moved the \cs{newcount} command out
%    \cs{ifFB@xetex@punct} ... \cs{fi} (it broke Plain format

% \changes{v2.5g}{2012/01/02}{Add four \cs{newif} to con
%    spacing of quotes (characters and control sequence

% \changes{v2.5g}{2012/01/01}{Skip the message in Pl
%    \cs{PackageInfo} undefined for Plain forma

%    \begin{macroc
\newif\ifFBAutoSpaceGuill  \FBAutoSpaceGuill
\newif\ifFBguillo@adds
\newif\ifFBguillf@adds
\newif\ifFBog@addspace     \FBog@addspace
\newif\ifFBfg@addspace     \FBfg@addspace
\newcount\FB@interchartokenstat
\ifFB@xetex@p
   \ifLa
    \PackageInfo{frenchb.ldf}{No need for active punctuation charac
                    \MessageBreak with this version of XeTeX! repor

%    \end{macroc
%    We will need the following code (borrowed
%    \file{zhsusefulmacros.sty}) for lo
%    \begin{macroc
   \@ifundefined{@fo
     \def\@nnil{\@n
     \def\@empt
     \def\@fornoop#1\@@#2#
     \long\def\@for#1:=#2\do
       \expandafter\def\expandafter\@fortmp\expandafter{
       \ifx\@fortmp\@empty \
         \expandafter\@forloop#2,\@nil,\@nil\@@#1{#3}\
     \long\def\@forloop#1,#2,#3\@@#4#5{\def#4{#1}\ifx #4\@nnil \
       #5\def#4{#2}\ifx #4\@nnil \else#5\@iforloop #3\@@#4{#5}\fi\
     \long\def\@iforloop#1,#2\@@#3#4{\def#3{#1}\ifx #3\@
            \expandafter\@fornoop \
       #4\relax\expandafter\@iforloop\fi#2\@@#3{#
     \def\@tfor#1:={\@tf@r#
     \long\def\@tf@r#1#2\do#3{\def\@fortmp{#2}\ifx\@fortmp\space\
         \@tforloop#2\@nil\@nil\@@#1{#3}\
     \long\def\@tforloop#1#2\@@#3#4{\def#3{#1}\ifx #3\@
            \expandafter\@fornoop \
           #4\relax\expandafter\@tforloop\fi#2\@@#3{#

%    \end{macroc
% \changes{v2.5i}{2012/04/20}{Temporary fix: as lon
%    \file{xeCJK.sty} will not use \cs{newXeTeXintercharclass
%    allocate its classes, we will have to define 3 fake class

% \changes{v2.5j}{2012/05/08}{Previous fix removed: bug fixe
%    \file{xeCJK.sty} version 3.0.4 (06-May-201

%    Five new character classes are defined for |frenc
%    \begin{macroc
   \newXeTeXintercharclass\FB@punctt
   \newXeTeXintercharclass\FB@punct
   \newXeTeXintercharclass\FB@punc
   \newXeTeXintercharclass\FB@punctg
   \newXeTeXintercharclass\FB@punctg
%    \end{macroc
%    We define a command to store the |\XeTeXcharclass| values w
%    will be modified for French (as a comma separated list) a
%    command to retrieve t
%    \begin{macroc
   \def\FB@charclassesO
   \def\emp
   \def\FB@parse#1,#2\endparse{\def\FB@class{
        \def\FB@charclassesORI{#
%    \end{macroc

%  \begin{macro}{\FB@xetex@punct@fre
%    The following command will be executed when entering French
%    first saves the values to be modified, then fits them to
%    needs. It also redefines |\shorthandoff| and |\shorthan
%    (locally) to avoid error messages with XeTeX-based engi

% \changes{v2.5g}{2011/12/31}{XeTeXcharclass(es) for French quotes
%    be set to \cs{FB@punctguilo} and \cs{FB@punctguilf} by opt
%    `og' and `fg' in \cs{frenchbsetup}.  French quotes should be
%    as normal characters by default in XeLaTeX as in LaT

% \changes{v2.5i}{2012/04/20}{\file{xeCJK.sty} changes
%    \cs{XeTeXcharclass} of ASCII chars '-' ',' '.' ')' ']'
%    '\{' '\%' opening and closing single and double quo
%    We set their class to 0 in French and reset their c
%    to their original value when leaving French.
%    \cs{FB@xetex@punct@nonfrench} bel

%    \begin{macroc
   \newcommand*{\FB@xetex@punct@frenc
%    \end{macroc
%    Saving must not be repeated if saved values are already
%    \begin{macroc
     \ifx\FB@charclassesORI\e
       \FB@interchartokenstateORI=\XeTeXinterchartokens
       \@for\FB@char:={`\:,`\;,`\!,`\?,"AB,"BB,`\(,`\[,`\{,`\,,`
                       `\-,`\),`\],`\},`\%,"22,"27,"60,"2019
            {\edef\FB@charclassesORI{\FB@charclasses
                                    \the\XeTeXcharclass\FB@char
       \let\shorthandonORI\shortha
       \let\shorthandoffORI\shorthan

%    \end{macroc
%    Set the classes and interactions between clas
%    \begin{macroc
     \XeTeXinterchartokensta
     \XeTeXcharclass `\: = \FB@punctt
     \XeTeXinterchartoks \z@ \FB@punctthick
                     \ifhmode\FDP@colonspace\
     \XeTeXinterchartoks \FB@punctguilf \FB@punctthick
                     \FDP@colonspa
     \XeTeXinterchartoks 255 \FB@punctthick
                     \ifhmode\unskip\penalty\@M\Fcolonspace\
     \@for\FB@char:={`\;,`\!,`\?
          {\XeTeXcharclass\FB@char=\FB@punctth
     \XeTeXinterchartoks \z@ \FB@punctthin
                     \ifhmode\FDP@thinspace\
     \XeTeXinterchartoks \FB@punctguilf \FB@punctthin
                     \FDP@thinspa
     \XeTeXinterchartoks 255 \FB@punctthin
                     \ifhmode\unskip\penalty\@M\Fthinspace\
     \XeTeXinterchartoks \FB@punctguilo \z@
                \ifFBAutoSpaceGuill\FBguill@spacing\
     \XeTeXinterchartoks \FB@punctguilo 255
                \ifFBAutoSpaceGuill\FBguill@spacing\ignorespaces\
     \XeTeXinterchartoks \z@ \FB@punctguilf
                \ifFBAutoSpaceGuill\FBguill@spacing\
     \XeTeXinterchartoks \FB@punctthin \FB@punctguilf
                \ifFBAutoSpaceGuill\FBguill@spacing\
     \XeTeXinterchartoks 255 \FB@punctguilf
                \ifFBAutoSpaceGuill\unskip\FBguill@spacing\
%    \end{macroc
%    This avoids spurious spaces in (!), [?],
%    \begin{macroc
     \@for\FB@char:={`\[,`\(
          {\XeTeXcharclass\FB@char=\FB@punctn
%    \end{macroc
%    These characters have their class changed by \file{xeCJK.s
%    let's reset them to 0 in Fre
%    \begin{macroc
     \@for\FB@char:={`\{,`\,,`\.,`\-,`\),`\],`\},`
                     "22,"27,"60,"2019
          {\XeTeXcharclass\FB@char=\
%    \end{macroc
%    With Xe(La)TeX, French defines no active shortha
%    \begin{macroc
      \def\shorthandoff#
        \@ifundefined{PackageWarni
         {\let\PackageWarning\undefin
         {\PackageWarning{frenchb.ldf}{\protect\shorthandoff{;:!?
          helpless with XeTeX,\MessageBreak use \protect\NoAutoSpa
          \space *inside a group* instead;\MessageBreak report


      \def\shorthandon##

%    \end{macroc
%  \end{ma

%  \begin{macro}{\FB@xetex@punct@nonfre
%    The following command will be executed when leaving French
%    restoring classes and commands modified in Fre
%    When French is not the main language,  |\noextrasfrench
%    executed `AtBeginDocument', so the tes
%    |\FB@charclassesORI| is mandat
%    \begin{macroc
   \newcommand*{\FB@xetex@punct@nonfrenc
     \ifx\FB@charclassesORI\e
     \
       \@for\FB@char:={`\:,`\;,`\!,`\?,"AB,"BB,`\(,`\[,`\{,`\,,`
                       `\-,`\),`\],`\},"22,"25,"27,"60,"2019
            {\expandafter\FB@parse\FB@charclassesORI\endp
             \XeTeXcharclass\FB@char=\FB@cla
       \def\FB@charclassesOR
       \XeTeXinterchartokenstate=\FB@interchartokenstat
       \let\shorthandon\shorthando
       \let\shorthandoff\shorthandof


   \addto\extrasfrench{\FB@xetex@punct@fre
   \addto\noextrasfrench{\FB@xetex@punct@nonfre

%    \end{macroc
%  \end{ma

%    Otherwise we need to make the four characters |;| |!| |?| and
%    `active' and provide their definiti
%    \begin{macroc
\ifFB@active@p
  \initiate@active@char
  \initiate@active@char
  \initiate@active@char
  \initiate@active@char
%    \end{macroc
%    We first tune the amount of space before \textt
%    \texttt{!}  \texttt{?} and \texttt{:}.  This should only ha
%    in horizontal mode, hence the test |\ifhmo

%    In horizontal mode, if a space has been typed before `;
%    remove it and put an unbreakable |\thinspace| instead. I
%    space has been typed, we add |\FDP@thinspace| which wil
%    defined, up to the user's wishes, as an automatic a
%    thin space, or as |\@emp
%    \begin{macroc
  \declare@shorthand{french}{
      \ifh
      \ifdim\lastskip
          \unskip\penalty\@M\Fthins
          \
            \FDP@thins


%    \end{macroc
%    Now we can insert a |;| charac
%    \begin{macroc
      \stri
%    \end{macroc
%    The next three definitions are very simi
%    \begin{macroc
  \declare@shorthand{french}{
      \ifh
        \ifdim\lastskip
          \unskip\penalty\@M\Fthins
        \
          \FDP@thins


      \stri
  \declare@shorthand{french}{
      \ifh
        \ifdim\lastskip
          \unskip\penalty\@M\Fthins
        \
          \FDP@thins


      \stri
  \declare@shorthand{french}{
      \ifh
        \ifdim\lastskip
          \unskip\penalty\@M\Fcolons
        \
          \FDP@colons


      \stri
%    \end{macroc
%    When the active characters appear in an environment where t
%    French behaviour is not wanted they should give an `expec
%    result. Therefore we define shorthands at system level as w
%    \begin{macroc
  \declare@shorthand{system}{:}{\stri
  \declare@shorthand{system}{!}{\stri
  \declare@shorthand{system}{?}{\stri
  \declare@shorthand{system}{;}{\stri

%    \end{macroc
%    We specify that the French group of shorthands should be used
%    switching to Fre
%    \begin{macroc
  \addto\extrasfren
     \languageshorthands{fren
%    \end{macroc
%    These characters are `turned on' once, later their definition
%    vary. Don't misunderstand the following code: they keep b
%    active all along the document, even when leaving Fre
%    \begin{macroc
    \bbl@activate{:}\bbl@activate
    \bbl@activate{!}\bbl@activate

  \addto\noextrasfren
    \bbl@deactivate{:}\bbl@deactivate
    \bbl@deactivate{!}\bbl@deactivate

%    \end{macroc

%    A new `if' |\FBAutoSpacePunctuation| needs to be defined
%    \begin{macroc
\newif\ifFBAutoSpacePunctuation  \FBAutoSpacePunctuation
%    \end{macroc

% \changes{v2.3a}{2008/10/10}{\cs{NoAutoSpaceBeforeFDP}
%    \cs{AutoSpaceBeforeFDP} now set the
%    \cs{ifFBAutoSpacePunctuation} accordingly (LaTeX onl

% \changes{v2.3e}{2009/10/10}{Execute \cs{AutoSpaceBefore
%    also in LaTeX to define \cs{FDP@colonspace}: needed
%    tex4ht, pointed out by M

%  \begin{macro}{\AutoSpaceBefore
%  \begin{macro}{\NoAutoSpaceBefore
%    |\FDP@thinspace| and |\FDP@colonspace| are defined as unbreak
%    spaces by |\autospace@beforeFDP| or as |\@empty
%    |\noautospace@beforeFDP| (internal commands), user comm
%    |\AutoSpaceBeforeFDP| and |\NoAutoSpaceBeforeFDP| do the same
%    take care of the flag |\ifFBAutoSpacePunctuation| in \LaTe
%    Set the default now for Plain (done later for \LaT
%    \begin{macroc
\def\autospace@beforeF
          \def\FDP@thinspace{\penalty\@M\Fthinspa
          \def\FDP@colonspace{\penalty\@M\Fcolonspa
\def\noautospace@beforeFDP{\let\FDP@thinspace\@e
                            \let\FDP@colonspace\@em
\ifLa
    \def\AutoSpaceBeforeFDP{\autospace@befor
                            \FBAutoSpacePunctuationt
    \def\NoAutoSpaceBeforeFDP{\noautospace@befor
                              \FBAutoSpacePunctuationfa
\
    \let\AutoSpaceBeforeFDP\autospace@befor
    \let\NoAutoSpaceBeforeFDP\noautospace@befor

\AutoSpaceBefor
%    \end{macroc
% \end{ma
% \end{ma

% \changes{v2.3a}{2008/10/10}{In LaTeX, frenchb no longer adds sp
%     before `double punctuation' characters in computer c
%     Suggested by Yannis Haralambo

% \changes{v2.3c}{2009/02/07}{Commands \cs{ttfamily}, \cs{rmfam
%    and \cs{sffamily} have to be robust.  Bug introduced in 2
%    pointed out by Manuel P\'egouri\'e-Gonna

%    In \LaTeXe{} |\ttfamily| (and hence |\texttt|) will be redef
%    `AtBeginDocument' as |\ttfamilyFB| so that no s
%    is added before the four |; : ! ?| characters, eve
%    |AutoSpacePunctuation| is true.  |\rmfamily| and |\sffamily|
%    to be redefined also (|\ttfamily| is not always used insi
%    group, its effect can be cancelled by |\rmfamily| or |\sffamil

%    These redefinitions can be canceled if necessary, for instanc
%    recompile older documents, see option |OriginalTypewriter| be

% \changes{v2.4c}{2010/05/23}{In \cs{ttfamilyFB}, also ca
%    automatic spaces inside French guillemets entered as charac
%    (see \cs{frenchbsetup

%    To be consistent with what is done for the |; :
%    characters, |\ttfamilyFB| also switches off insertion of sp
%    inside French guillemets \emph{when they are typed i
%    characters} with the `og'/`fg' options in |\frenchbsetup
%    This is also a workaround for the weird behaviour of t
%    characters in verbatim m

% \changes{v2.5g}{2012/01/02}{Switch flags \cs{ifFBog@addspace}
%    \cs{ifFBfg@addspace} to true, otherwise \cs{og} and \cs
%    provide no spacing in XeLaTeX with options `og' and
%    activated in \cs{frenchbsetup} (bug v.2.5a-

%    \begin{macroc
\ifLa
    \let\ttfamilyORI\ttfa
    \let\rmfamilyORI\rmfa
    \let\sffamilyORI\sffa
    \DeclareRobustCommand\ttfamily
         \FBAutoSpaceGuillf
         \FBog@addspacetrue       \FBfg@addspace
         \noautospace@beforeFDP\ttfamilyO
    \DeclareRobustCommand\rmfamily
         \FBAutoSpaceGuill
         \ifFBguillo@addspace\FBog@addspacefals
         \ifFBguillf@addspace\FBfg@addspacefals
         \ifFBAutoSpacePunctua
           \autospace@befor
         \
           \noautospace@befor

         \rmfamilyO
    \DeclareRobustCommand\sffamily
         \FBAutoSpaceGuill
         \ifFBguillo@addspace\FBog@addspacefals
         \ifFBguillf@addspace\FBfg@addspacefals
         \ifFBAutoSpacePunctua
           \autospace@befor
         \
           \noautospace@befor

         \sffamilyO

%    \end{macroc

% \changes{v2.5a}{2010/08/14}{New command \cs{NoAutoSpaci
%    suggested by M

%  \begin{macro}{\NoAutoSpac
%    The following command will switch off active punctua
%    characters (if any) and disable automatic spacing for French q
%    characters. It is engine independent (works for \TeX{}
%    Xe\TeX{} based engines) and is meant to be used inside a gr

% \changes{v2.5g}{2012/01/02}{Switch flags \cs{ifFBog@addspace}
%    \cs{ifFBfg@addspace} to true, otherwise \cs{og} and \cs
%    provide no spacing in XeLaTeX with options `og' and
%    activated in \cs{frenchbsetup} (bug v.2.5a-

%    \begin{macroc
\newcommand*{\NoAutoSpacing}{\FBAutoSpaceGuillf
   \FBog@addspacetrue        \FBfg@addspace
   \ifFB@active@punct\shorthandoff{;:!?
   \ifFB@xetex@punct\XeTeXinterchartokenstate=

%    \end{macroc
%  \end{ma

%  \subsection{Commands for French quotation ma
%  \label{sec-quo

%  \begin{macro}{
%  \begin{macro}{
%    The top macros for quotation marks will be called |
%    (``\underline{o}uvrez \underline{g}uillemets'') and |
%    (``\underline{f}ermez \underline{g}uillemets
%    Another option for typesetting quotes in multilingual t
%    is to use the package \file{csquotes.sty} and its com
%    |\enquo

%    \begin{macroc
\newcommand*{\og}{\@em
\newcommand*{\fg}{\@em
%    \end{macroc
%  \end{ma
%  \end{ma

%  \begin{macro}{\guillemotl
%  \begin{macro}{\guillemotri
%  \begin{macro}{\textquoteddbll
%  \begin{macro}{\textquoteddblri
%    \LaTeX{} users are supposed to use 8-bit output encodings
%    LY1,\dots) to typeset French, those who still stick to OT1 sh
%    call |aeguill.sty| or a similar package. In both cases
%    commands |\guillemotleft| and |\guillemotright| will print
%    French opening and closing quote characters from the output f
%    For XeLaTeX, |\guillemotleft| and |\guillemotright| are def
%    by package \file{xunicode.s
%    We will check `AtBeginDocument' that the proper output encod
%    are in use (see end of section~\ref{sec-keyva

% \changes{v2.5a}{2010/08/20}{Change \cs{guillemotleft}
%    \cs{guillemotright} definitions for Unicode anf pro
%    definitions for \cs{textquotedblleft}
%    \cs{textquotedbright}.  Insures correct printing of qu
%    by \cs{og} and \cs{fg} in French and outsi

%    We give the following definitions for non-LaTeX users only
%    fall-back, they are welcome to change them for anything bet
%    \begin{macroc
\ifLa
\
  \ifFBuni
     \def\guillemotleft{{\char"00
     \def\guillemotright{{\char"00
     \def\textquotedblleft{{\char"20
     \def\textquotedblright{{\char"20
  \
    \def\guillemotleft{\leavevmode\raise0.
                       \hbox{$\scriptscriptstyle\l
    \def\guillemotright{\raise0.
                        \hbox{$\scriptscriptstyle\g
    \def\textquotedblleft
    \def\textquotedblright

  \let\xspace\r

%    \end{macroc
%  \end{ma
%  \end{ma
%  \end{ma
%  \end{ma

%    The next step is to provide correct spacing after |\guillemotl
%    and before |\guillemotright|: a space precedes and fol
%    quotation marks but no line break is allowed neither \emph{af
%    the opening one, nor \emph{before} the closing
%    |\FBguill@spacing| which does the spacing, has been fine tune
%    Thierry Bouche.  French quotes (including spacing) are printe
%    |\FB@og| and |\FB@fg|, the expansion of the top level comm
%    |\og| and |\og| is different in and outside Fre
%    We'll try to be smart to users of David~Carlisle's \file{xsp
%    package: if this package is loaded there will be no need for
%    or |\ | to get a space after |\fg|, otherwise |\xspace| wil
%    defined as |\relax| (done at the end of this fi

% \changes{v2.5g}{2012/01/01}{\cs{FBguill@spacing} needs to be ski
%    in commands \cs{FB@og} and \cs{FB@fg} only when XeTeX's ``i
%    char'' mechanism is triggered for quotes, see \cs{frenchbsetu

%    \begin{macroc
\newcommand*{\FBguill@spacing}{\penalty\@M\hskip.8\fontdimen2\
                                            plus.3\fontdimen3\
                                           minus.8\fontdimen4\f
\DeclareRobustCommand*{\FB@og}{\leavevmode\guillemot
                               \ifFBog@addspace\FBguill@spacing
\DeclareRobustCommand*{\FB@fg}{\ifdim\lastskip>\z@\unski
                               \ifFBfg@addspace\FBguill@spacin
                               \guillemotright\xsp
%    \end{macroc

%    The top level definitions for French quotation marks are swit
%    on and off through the |\extrasfrench| |\noextrasfre
%    mechanism. Outside French, |\og| and |\fg| will typeset stan
%    English opening and closing double quo

% \changes{v2.5a}{2010/08/20}{\cs{og} and \cs{fg} do not p
%    correctly in English when using XeTeX or LuaTeX, fixed by u
%    \cs{textquotedblleft} and \cs{textquotedblright} defined abo

%    \begin{macroc
\ifLa
  \def\bbl@frenchguillemets{\renewcommand*{\og}{\FB@
                            \renewcommand*{\fg}{\FB@
  \def\bbl@nonfrenchguillemets{\renewcommand*{\og}{\textquotedblle
            \renewcommand*{\fg}{\ifdim\lastskip>\z@\unski
                                   \textquotedblrig
\
  \def\bbl@frenchguillemets{\let\og\F
                            \let\fg\FB
  \def\bbl@nonfrenchguillemets{\def\og{\textquotedblle
              \def\fg{\ifdim\lastskip>\z@\unskip\fi\textquotedblrig

\addto\extrasfrench{\bbl@frenchguillem
\addto\noextrasfrench{\bbl@nonfrenchguillem
%    \end{macroc

%  \subsection{Date in Fre

% \begin{macro}{\datefre
%    The macro |\datefrench| redefines the command |\today
%    produce French da

% \changes{v2.0}{2006/11/06}{2 '\cs{relax}' adde
%    \cs{today}'s definiti

% \changes{v2.1a}{2008/03/25}{\cs{today} changed (correction in
%    was wrong: \cs{today} was printed without spaces in to

% \changes{v2.5a}{2010/08/20}{Replaced \cs{'}e and \cs{\char
%    by c.s. to work with XeTeX and LuaT

%    \begin{macroc
\@namedef{date\CurrentOptio
  \def\today{{\number\day}\ifnum1=\day {\ier}\fi \s
    \ifcase\m
      \or janvier\or f{\FBeacute}vrier\or mars\or avril\or ma
      juin\or juillet\or ao{\FBucirconflexe}t\or septembr
      octobre\or novembre\or d{\FBeacute}cembr
    \space \number\ye
%    \end{macroc
% \end{ma

%  \subsection{Extra utilit

%    Let's provide the French user with some extra utilit

% \changes{v2.1a}{2008/03/24}{Command \cs{fup} added to pro
%    better superscripts than \cs{textsuperscrip

%  \begin{macro}{

% \changes{v2.1c}{2008/04/29}{Provide a temporary defini
%    (hyperref safe) of \cs{up} in case it has to be expande
%    the preamble (by beamer's \cs{title} command for instanc

% \changes{v2.4d}{2010/07/28}{Command \cs{up} defined
%    \cs{providecommand} instead of \cs{newcommand} as \cs{up} ma
%    defined elsewhere (catalan.l
%    Bug pointed out by Felip Many\'e i Ballest

% \changes{v2.5a}{2010/08/21}{Test \cs{@ifundefined} leaves
%    tested control sequence defined as \cs{relax} when T
%    Changed \cs{relax} to \cs{undefined} when tes
%    \cs{realsuperscrip

%  \begin{macro}{\

% \changes{v2.1b}{2008/04/02}{Command \cs{fup} changed to
%    real superscripts from fourier v. 1

% \changes{v2.2a}{2008/05/08}{\cs{newif} and \cs{newdimen} m
%    before \cs{ifLaTeXe} to avoid an error with plainT

% \changes{v2.3a}{2008/09/30}{\cs{lowercase} change
%    \cs{MakeLowercase} as the former doesn't work for non A
%    characters in encodings like applemac, utf-8,\d

%    |\up| eases the typesetting of superscripts
%    `1\textsuperscript{er}'.  Up to version 2.0 of |frenchb| |\up|
%    just a shortcut for |\textsuperscript| in \LaTeXe, but sev
%    users complained that |\textsuperscript| typesets superscr
%    too high and too big, so we now define |\fup| as an attemp
%    produce better looking superscripts.  |\up| is defined as |\
%    but |\frenchbsetup{FrenchSuperscripts=false}| redefines |
%    as |\textsuperscript| for compatibility with previous versi

%    When a font has built-in superscripts, the best thing to d
%    to just use them, otherwise |\fup| has to simulate superscr
%    by scaling and raising ordinary letters.  Scaling is done u
%    package \file{scalefnt} which will be loaded at the en
%    \babel's loading (|frenchb| being an option of babel, it ca
%    load a package while being re

%    \begin{macroc
\newif\ifFB@poo
\newdimen\FB
\ifLa
  \AtEndOfPackage{\RequirePackage{scalef
%    \end{macroc
%    |\FB@up@fake| holds the definition of fake superscri
%    The scaling ratio is 0.65, raising is computed to put the to
%    lower case letters (like `m') just under the top  of upper
%    letters (like `M'), precisely 12\% down.  The chosen sett
%    look correct for most fonts, but can be tuned by the end-
%    if necessary by changing |\FBsupR| and |\FBsupS| comma

%    |\FB@lc| is defined as |\MakeLowercase| to inhibit the upperca
%    of superscripts (this may happen in page headers with the stan
%    classes but is wrong); |\FB@lc| can be redefined to do not
%    by option |LowercaseSuperscripts=false| of |\frenchbsetup
%    \begin{macroc
  \newcommand*{\FBsupR}{-0
  \newcommand*{\FBsupS}{0
  \newcommand*{\FB@lc}[1]{\MakeLowercase{
  \DeclareRobustCommand*{\FB@up@fake}[
    \settoheight{\FB@Mht}
    \addtolength{\FB@Mht}{\FBsupR \FB@M
    \addtolength{\FB@Mht}{-\FBsupS
    \raisebox{\FB@Mht}{\scalefont{\FBsupS}{\FB@lc{#1

%    \end{macroc
%    The only packages I currently know to take advantage of
%    superscripts are a) \file{xltxtra} used in conjunction
%    XeLaTeX and OpenType fonts having the font fea
%    'VerticalPosition=Superior' (\file{xltxtra} def
%    |\realsuperscript| and |\fakesuperscript|) and b) \file{four
%    (from version 1.6) when Expert Utopia fonts are availa

%    |\FB@up| checks whether the current font is a Type1 `Exp
%    (or `Pro') font with real superscripts or not (the code w
%    currently only with \file{fourier-1.6} but could work with
%    Expert Type1 font with built-in superscripts, see below),
%    decides to use real or fake superscri
%    It works as follows: the content of |\f@family| (family nam
%    the current font) is split by |\FB@split| into two pieces,
%    first three characters (`|fut|' for Fourier, `|ppl|' for Ado
%    Palatino, \dots) stored in |\FB@firstthree| and the rest st
%    in |\FB@suffix| which is expected to be `|x|' or `|j|' for ex
%    fo
%    \begin{macroc
  \def\FB@split#1#2#3#4\@nil{\def\FB@firstthree{#1#2
                             \def\FB@suffix{
  \def\FB@
  \def\FB@
  \DeclareRobustCommand*{\FB@up}[
    \bgroup \FB@poorman
      \expandafter\FB@split\f@family\
%    \end{macroc
%    Then |\FB@up| looks for a \file{.fd} file named \file{t1fut-sup
%    (Fourier) or \file{t1ppl-sup.fd} (Palatino), etc. suppose
%    define the subfamily (|fut-sup| or |ppl-sup|, etc.) giving ac
%    to the built-in superscripts.  If the \file{.fd} file is not f
%    by |\IfFileExists|, |\FB@up| falls back on fake superscri
%    otherwise |\FB@suffix| is checked to decide whether to use fak
%    real superscri
%    \begin{macroc
      \edef\reserved@a{\lowerca
         \noexpand\IfFileExists{\f@encoding\FB@firstthree -sup.fd
      \reserv
        {\ifx\FB@suffix\FB@x \FB@poormanfals
         \ifx\FB@suffix\FB@j \FB@poormanfals
         \ifFB@poorman \FB@up@fake{
         \else         \FB@up@real{
         \
        {\FB@up@fake{#
    \egr
%    \end{macroc
%    |\FB@up@real| just picks up the superscripts from the subfa
%    (and forces lowerca
%    \begin{macroc
  \newcommand*{\FB@up@real}[1]{\bg
       \fontfamily{\FB@firstthree -sup}\selectfont \FB@lc{#1}\egr
%    \end{macroc
%    |\fup| is now defined as |\FB@up| unless |\realsuperscript
%    defined (occurs with XeLaTeX calling \file{xltxtra.st
%    \begin{macroc
  \DeclareRobustCommand*{\fup}[
    \@ifundefined{realsuperscri
      {\FB@up{#1}\let\realsuperscript\undefin
      {\bgroup\let\fakesuperscript\FB@up@
            \realsuperscript{\FB@lc{#1}}\egro
%    \end{macroc
%    Let's provide a temporary definition for |\up| (redef
%    `AtBeginDocument' as |\fup| or |\textsuperscript| accordin
%    |\frenchbsetup{}| optio
%    \begin{macroc
  \providecommand*{\up}{\re
%    \end{macroc
%    Poor man's definition of |\up| for Pl
%    \begin{macroc
\
  \providecommand*{\up}[1]{\leavevmode\raise1ex\hbox{\sevenrm

%    \end{macroc
%  \end{ma
%  \end{ma

%  \begin{macro}{\i
%  \begin{macro}{\
%  \begin{macro}{\i
%  \begin{macro}{\ie
%  \begin{macro}{\i
%  \begin{macro}{\ie
%  Some handy macros for those who don't know how to abbreviate ordin
%    \begin{macroc
\def\ieme{\up{\lowercase{e}}\xsp
\def\iemes{\up{\lowercase{es}}\xsp
\def\ier{\up{\lowercase{er}}\xsp
\def\iers{\up{\lowercase{ers}}\xsp
\def\iere{\up{\lowercase{re}}\xsp
\def\ieres{\up{\lowercase{res}}\xsp
%    \end{macroc
%  \end{ma
%  \end{ma
%  \end{ma
%  \end{ma
%  \end{ma
%  \end{ma

% \changes{v2.1c}{2008/04/29}{Added commands \cs{Nos} and \cs{no

%  \begin{macro}{
%  \begin{macro}{
%  \begin{macro}{\
%  \begin{macro}{\
%  \begin{macro}{\pr
%  \begin{macro}{\fpri
%    And some more macros relying on |\up| for number
%    first two support mac
%    \begin{macroc
\newcommand*{\FrenchEnumerate}[
                       #1\up{\lowercase{o}}\kern+.
\newcommand*{\FrenchPopularEnumerate}[
                       #1\up{\lowercase{o}})\kern+.
%    \end{macroc

%    Typing |\primo| should result in `$1^{\rm o}$\kern+.3
%    \begin{macroc
\def\primo{\FrenchEnumera
\def\secundo{\FrenchEnumera
\def\tertio{\FrenchEnumera
\def\quarto{\FrenchEnumera
%    \end{macroc
%    while typing |\fprimo)| gives `1$^{\rm o}$)\kern+.
%    \begin{macroc
\def\fprimo){\FrenchPopularEnumera
\def\fsecundo){\FrenchPopularEnumera
\def\ftertio){\FrenchPopularEnumera
\def\fquarto){\FrenchPopularEnumera
%    \end{macroc

%    Let's provide four macros for the common abbreviat
%    of ``Num\'er
%    \begin{macroc
\DeclareRobustCommand*{\No}{N\up{\lowercase{o}}\kern+.
\DeclareRobustCommand*{\no}{n\up{\lowercase{o}}\kern+.
\DeclareRobustCommand*{\Nos}{N\up{\lowercase{os}}\kern+.
\DeclareRobustCommand*{\nos}{n\up{\lowercase{os}}\kern+.
%    \end{macroc
%  \end{ma
%  \end{ma
%  \end{ma
%  \end{ma
%  \end{ma
%  \end{ma

%  \begin{macro}{\
%    As family names should be written in small capitals and neve
%    hyphenated, we provide a command (its name comes from Boxed S
%    Caps) to input them easily.  Note that this command has cha
%    with version~2 of |frenchb|: a |\kern0pt| is used instead of |\h
%    because |\hbox| would break microtype's font expansion;
%    (positive?) side effect, composed names (such as Dupont-Dur
%    can now be hyphenated on explicit hyph
%    Usage: |Jean~\bsc{Duchemi

% \changes{v2.0}{2006/11/06}{\cs{hbox} dropped, replace
%    \cs{kern0p

%    \begin{macroc
\DeclareRobustCommand*{\bsc}[1]{\leavevmode\begingroup\ker
                                           \scshape #1\endgr
\ifLaTeXe\else\let\scshape\rela
%    \end{macroc
%  \end{ma

%    Some definitions for special characters.  We won't define |\ti
%    as a Text Symbol not to conflict with the macro |\tilde| for
%    mode and use the name |\tild| instead. Note that |\boi|
%    \emph{not} be used in math mode, its name in math mod
%    |\backslash|.  |\degre|  can be accessed by the command |\
%    for ring acc

% \changes{v2.5f}{2011/07/18}{Changed definitions of \cs{
%    \cs{circonflexe}, \cs{tild}, \cs{boi} and \cs{degre}
%    Unicode based engin

%    \begin{macroc
\ifFBuni
    \newcommand*{\at}{{\char"00
    \newcommand*{\circonflexe}{{\char"00
    \newcommand*{\tild}{{\char"00
    \newcommand*{\boi}{\textbacksl
    \newcommand*{\degre}{{\char"00
\
  \ifLa
    \DeclareTextSymbol{\at}{T1}
    \DeclareTextSymbol{\circonflexe}{T1}
    \DeclareTextSymbol{\tild}{T1}{
    \DeclareTextSymbolDefault{\at}
    \DeclareTextSymbolDefault{\circonflexe}
    \DeclareTextSymbolDefault{\tild}
    \DeclareRobustCommand*{\boi}{\textbacksl
    \DeclareRobustCommand*{\degre}{\
  \
    \def\T@one
    \ifx\f@encoding\T
      \newcommand*{\degre}{{\cha
    \
      \newcommand*{\degre}{{\char

    \newcommand*{\at}{{\char
    \newcommand*{\circonflexe}{{\char
    \newcommand*{\tild}{{\char1
    \newcommand*{\boi}{$\backsla


%    \end{macroc

% \changes{v2.5a}{2010/08/20}{New definitions needed for XeTeX/Lu
%    to properly print some dates and captions: using c.s. like \cs
%    do not work with XeTeX (OK with XeLaTe

%    French dates and captions make use of four non-ascii charac
%    (\texttt{\`a}, \texttt{\`e},  \texttt{\'e} and \texttt{\^
%    This is fine except for (plain) XeTeX (|\accent| commands are
%    implemented), so we define four new commands to deal with
%    is
%    \begin{macroc
\newcommand*{\FBagrave}{
\newcommand*{\FBegrave}{
\newcommand*{\FBeacute}{
\newcommand*{\FBucirconflexe}{
\ifFBuni
  \ifLa
  \
    \def\FBagrave{{\char"00
    \def\FBegrave{{\char"00
    \def\FBeacute{{\char"00
    \def\FBucirconflexe{{\char"00


%    \end{macroc

%  \begin{macro}{\deg
%    We now define a macro |\degres| for typesetting the abbrevia
%    for `degrees' (as in `degrees Celsius'). As the bounding bo
%    the character `degree' has \emph{very} different widths in C
%    and PostScript fonts, we fix the width of the bounding bo
%    |\degres| to 0.3\,em, this lets the symbol `degree' stick to
%    preceding (e.g., |45\degres|) or following chara
%    (e.g., |20~\degres

%    If \TeX{} Companion fonts are available (\file{textcomp.st
%    we pick up |\textdegree| from them instead of using emula
%    `degrees' from the |\r{}| accent. Otherwise we overwrite
%    (poor) definition of |\textdegree| given in \file{latin1.d
%    \file{applemac.def} etc. (called by  \file{inputenc.sty}
%    our definition of |\degres|. We also advice the user (once o
%    to use TS1-encod

% \changes{v2.1c}{2008/04/29}{Provide a temporary definition (hype
%    safe) of \cs{degres} in case it has to be expanded in the prea
%    (by beamer's \cs{title} command for instanc

% \changes{v2.5h}{2012/03/21}{textcomp.sty has changed. The
%    about \cs{M@TS1} is no longer relevant, let's change

%    \begin{macroc
\ifLa
  \newcommand*{\degres}{\de
  \ifFBuni
    \DeclareRobustCommand*{\degres}{\de
  \
    \def\Warning@degree@TSo
          \PackageWarning{frenchb.ld
             Degrees would look better in TS1-encod
             \MessageBreak add \pro
             \usepackage{textcomp} to the pream
             \MessageBreak Degrees us
    \AtBeginDocument{\@ifundefined{DeclareEncodingSubs
                       {\DeclareRobustCommand*{\degre
                          \leavevmode\hbox to 0.3em{\hss\degre\h
                          \Warning@degree@T
                          \global\let\Warning@degree@TSone\rel
                        \let\textdegree\degr
                       {\DeclareRobustCommand*{\degre
                          \hbox{\UseTextSymbol{TS1}{\textdegree}


\
  \newcommand*{\degre
    \leavevmode\hbox to 0.3em{\hss\degre\h

%    \end{macroc
%  \end{ma

%  \subsection{Formatting numb
%  \label{sec-numb

%  \begin{macro}{\DecimalMathCo
%  \begin{macro}{\StandardMathCo
%    As mentioned in the \TeX{}book p.~134, the comma is of
%    |\mathpunct| in math mode: it is automatically followed
%    space. This is convenient in lists and intervals
%    unpleasant when the comma is used as a decimal separ
%    in French: it has to be entered as |{
%    |\DecimalMathComma| makes the comma be an ordinary chara
%    (of type |\mathord|) in French \emph{only} (no space add
%    |\StandardMathComma| switches back to the standard behav
%    of the co
%    \begin{macroc
\newcount\std
\newcount\dec
\std@mcc=\mathcod
\dec@mcc=\std
\@tempcnta=\std
\divide\@tempcnta by "
\multiply\@tempcnta by "
\advance\dec@mcc by -\@temp
\newcommand*{\DecimalMathComma}{\iflanguage{fren
                                 {\mathcode`\,=\dec@mcc
   \addto\extrasfrench{\mathcode`\,=\dec@m
\newcommand*{\StandardMathComma}{\mathcode`\,=\std
   \addto\extrasfrench{\mathcode`\,=\std@m
\addto\noextrasfrench{\mathcode`\,=\std@
%    \end{macroc
%  \end{ma
%  \end{ma

%  \begin{macro}{\nom

% \changes{v2.0}{2006/11/06}{\cs{nombre} requires now numprint.s

%    The command |\nombre| is now borrowed from \file{numprint.sty}
%    \LaTeXe.  There is no point to maintain the former tricky
%    when a package is dedicated to do the same job and m
%    For Plain based formats, |\nombre| no longer formats numb
%    it prints them as is and issues a warning about the cha

%    Fake command |\nombre| for Plain based formats, warning user
%    |frenchb| v.1.x. of the cha
%    \begin{macroc
\newcommand*{\nombre}[1]{{#1}\messa
     *** \noexpand\nombre no longer formats numbers\string! **
%    \end{macroc
%  \end{ma

%    The next definitions only make sense for \LaTeXe. Let's cle
%    and exit if the format in Plain ba

%    \begin{macroc
\let\FBstop@here\r
\def\FBclean@on@exit{\let\ifLaTeXe\undef
                     \let\LaTeXetrue\undef
                     \let\LaTeXefalse\undefi
\ifx\magnification\@undef
\
   \def\FBstop@here{\let\STD@makecaption\r
                    \FBclean@on@
                    \ldf@quit\CurrentOption\endin

\FBstop@
%    \end{macroc

%    What follows now is for \LaTeXe{} \emph{on
%    We redefine |\nombre| for \LaTeXe. A warning is is
%    at the first call of |\nombre| if |\numprint| is
%    defined, suggesting what to do.  The package \file{numpr
%    is \emph{not} loaded automatically by |frenchb| becaus
%    possible options confl

% \changes{v2.5a}{2010/08/21}{Test \cs{@ifundefined} leaves
%    tested control sequence defined as \cs{relax} when T
%    Changed \cs{relax} to \cs{undefined} when tes
%    \cs{numprin

%    \begin{macroc
\renewcommand*{\nombre}[1]{\Warning@nombre\numprint{
\newcommand*{\Warning@nombr
   \@ifundefined{numpri
      {\PackageWarning{frenchb.ld
         \protect\nombre\space now relies on package numprint.
         \MessageBreak add \pro
         \usepackage[autolanguage]{numprint}\MessageB
         to your preamble *after* loading babel, \MessageB
         see file numprint.pdf for more options.\MessageB
         \protect\nombre\space call
       \global\let\Warning@nombre\r
       \global\let\numprint\undef


%    \end{macroc

% \changes{v2.0c}{2007/06/25}{There is no need to define
%    numprint's command \cs{npstylefrench}, it will be redef
%    `AtBeginDocument' by \cs{FBprocess@option

% \changes{v2.0c}{2007/06/25}{\cs{ThinSpaceInFrenchNumbers} a
%     for compatibility with frenchb-1

%    \begin{macroc
\newcommand*{\ThinSpaceInFrenchNumber
   \PackageWarning{frenchb.ld
         Type \protect\frenchbsetup{ThinSpaceInFrenchNumb
         \MessageBreak Command \protect\ThinSpaceInFrenchNumbers\s
         is no longer\MessageBreak  defined in frenchb v.
%    \end{macroc

%  \subsection{Caption na

%    The next step consists of defining the French equivalents
%    the \LaTeX{} caption na

% \begin{macro}{\captionsfre
%    Let's first define  |\captionsfrench| which sets all strings
%    in the four standard document classes provided with \La

% \changes{v2.0}{2006/11/06}{`Fig.' changed to `Figure'
%     `Tab.' to `Tabl

% \changes{v2.0}{2006/12/15}{Set \cs{CaptionSeparator
%     \cs{extrasfrench} now instead of \cs{captionsfre
%     because it has to be reset when leaving Fren

% \changes{v2.5a}{2010/08/16}{\cs{emph} deleted in \cs{seen
%     and \cs{alsoname} to match what is done for the other langua
%     Suggested by Marc Baudo

% \changes{v2.5a}{2010/08/20}{Replaced \cs{'e}, \cs{`e} and \cs
%    by c.s. to work with XeT

%    \begin{macroc
\@namedef{captions\CurrentOptio
   \def\refname{R{\FBeacute}f{\FBeacute}renc
   \def\abstractname{R{\FBeacute}sum{\FBeacut
   \def\bibname{Bibliograph
   \def\prefacename{Pr{\FBeacute}fa
   \def\chaptername{Chapit
   \def\appendixname{Anne
   \def\contentsname{Table des mati{\FBegrave}r
   \def\listfigurename{Table des figur
   \def\listtablename{Liste des tablea
   \def\indexname{Ind
   \def\figurename{{\scshape Figur
   \def\tablename{{\scshape Tabl
%    \end{macroc
%   ``Premi\`ere partie'' instead of ``Part
%    \begin{macroc
   \def\partname{\protect\@Fpt part
   \def\@Fpt{{\ifcase\value{part}\or Premi{\FBegrave}r
       Deuxi{\FBegrave}me\or Troisi{\FBegrave}m
       Quatri{\FBegrave}me\or Cinqui{\FBegrave}m
       Sixi{\FBegrave}me\or Septi{\FBegrave}me\or Huiti{\FBegrave}m
       Neuvi{\FBegrave}me\or Dixi{\FBegrave}me\or Onzi{\FBegrave}m
       Douzi{\FBegrave}me\or Treizi{\FBegrave}m
       Quatorzi{\FBegrave}me\or Quinzi{\FBegrave}m
       Seizi{\FBegrave}me\or Dix-septi{\FBegrave}m
       Dix-huiti{\FBegrave}me\or Dix-neuvi{\FBegrave}m
       Vingti{\FBegrave}me\fi}\space\def\thepart
   \def\pagename{pa
   \def\seename{vo
   \def\alsoname{voir aus
   \def\enclname{P.~J
   \def\ccname{Copie {\FBagrave
   \def\headtonam
   \def\proofname{D{\FBeacute}monstrati
   \def\glossaryname{Glossai

%    \end{macroc
% \end{ma

%    As some users who choose |frenchb| or |francais| as optio
%    \babel, might customise |\captionsfrenchb| or |\captionsfranc
%    in the preamble, we merge their changes at the |\begin{docume
%    when they do
%    The other variants of French (canadien, acadian) are define
%    checking if the relevant option was used and then adding one e
%    level of expans

% \changes{v2.5b}{2010/10/30}{\cs{captionsfrench} will be exec
%    `AtBeginDocument' after \cs{FBprocess@options}, no nee
%    add it he

%    \begin{macroc
\AtBeginDocument{\let\captions@French\captionsfr
                 \@ifundefined{captionsfrenc
                    {\let\captions@Frenchb\rel
                    {\let\captions@Frenchb\captionsfrenc
                 \@ifundefined{captionsfranca
                    {\let\captions@Francais\rel
                    {\let\captions@Francais\captionsfranca
                 \def\captionsfrench{\captions@Fr
                        \captions@Francais\captions@Frenc
                 \def\captionsfrancais{\captionsfren
                 \def\captionsfrenchb{\captionsfren

\@ifpackagewith{babel}{canadie
  \def\captionscanadien{\captionsfren
  \def\datecanadien{\datefren
  \def\extrascanadien{\extrasfren
  \def\noextrascanadien{\noextrasfren

\@ifpackagewith{babel}{acadia
  \def\captionsacadian{\captionsfren
  \def\dateacadian{\datefren
  \def\extrasacadian{\extrasfren
  \def\noextrasacadian{\noextrasfren

%    \end{macroc

% \begin{macro}{\CaptionSepara
%    Let's consider now captions in figures and tab
%    In French, captions in figures and tables should be printed
%    endash (`--') instead of the standard

%    The standard definition of |\@makecaption| (e.g., the one prov
%    in article.cls, report.cls, book.cls which is frozen for \LaTe
%    according to Frank Mittelbach), has been save
%    |\STD@makecaption| before making `:' ac
%    (see section~\ref{sec-punct}). `AtBeginDocument' we compare i
%    its current definition (some classes like koma-script clas
%    AMS classes, ua-thesis.cls\dots change
%    If they are identical, |frenchb| just adds a hook ca
%    |\CaptionSeparator| to |\@makecaption|, |\CaptionSepara
%    defaults to `: ' as in the standard |\@makecaption|, and wil
%    changed to ` -- ' in Fre
%    If the definitions differ, |frenchb| doesn't overwrite the chan
%    but prints a message in the .log f

% \changes{v2.4a}{2009/11/23}{\cs{PackageWarning} change
%     \cs{FBWarning} (in case \cs{@makecaption} has been customiz
%     \cs{FBWarning} is defined as \cs{PackageWarning} by default
%     can be made silent using \cs{frenchbsetup}, (suggested by MP

%    \begin{macroc
\newcommand{\FBWarning}[2]{\PackageWarning{#1}{
\def\CaptionSeparator{\string:\sp
\long\def\FB@makecaption#1
  \vskip\abovecaption
  \sbox\@tempboxa{#1\CaptionSeparator
  \ifdim \wd\@tempboxa >\h
    #1\CaptionSeparator #2
  \
    \global \@minipagef
    \hb@xt@\hsize{\hfil\box\@tempboxa\hf

  \vskip\belowcaptions
\AtBeginDocume
  \ifx\@makecaption\STD@makecap
      \global\let\@makecaption\FB@makecap
  \
    \@ifundefined{@makecapti
       {\let\@makecaption\undefin
       {\FBWarning{frenchb.l
        {The definition of \protect\@makecaption\s
         has been changed,\MessageB
         frenchb will NOT customise it;\MessageBreak report


  \let\FB@makecaption\r
  \let\STD@makecaption\r

\addto\extrasfren
   \def\CaptionSeparator{\space\textendash\spa
\addto\noextrasfren
   \def\CaptionSeparator{\string:\spa
%    \end{macroc
% \end{ma

%  \subsection{French li
%  \label{sec-li

%  \begin{macro}{\lis
%  \begin{macro}{\list
%    Vertical spacing in general lists should be shorter in Fr
%    texts than the defaults provided by \La
%    Note that the easy way, just changing values of vertical spa
%    parameters when entering French and restoring them to t
%    defaults on exit would not work; as most lists are base
%    |\list| we will define a variant of |\list| (|\listFB|
%    be used in Fre

%    The amount of vertical space before and after a list is give
%    |\topsep| + |\parskip| (+ |\partopsep| if the list starts a
%    paragraph). IMHO, |\parskip| should be added \emph{only}
%    the list starts a new paragraph, so I subtract |\parskip|
%    |\topsep| and add it back to |\partopsep|; this will norm
%    make no difference because |\parskip|'s default value is 0pt,
%    will be noticeable when |\parskip| is \emph{not} n

%    |\endlist| is not redefined, but |\endlistORI| is provided
%    the users who prefer to define their own lists from the orig
%    command, they can code: |\begin{listORI}{}{} \end{listOR
%    \begin{macroc
\let\listORI\
\let\endlistORI\end
\def\FB@listsettin
      \setlength{\itemsep}{0.4ex plus 0.2ex minus 0.2
      \setlength{\parsep}{0.4ex plus 0.2ex minus 0.2
      \setlength{\topsep}{0.8ex plus 0.4ex minus 0.4
      \setlength{\partopsep}{0.4ex plus 0.2ex minus 0.2
%    \end{macroc
%    |\parskip| is of type `skip', its mean value only (\emph
%    the glue}) should be subtracted from |\topsep| and adde
%    |\partopsep|, so convert |\parskip| to a `dimen' u
%    |\@tempdi
%    \begin{macroc
      \@tempdima=\par
      \addtolength{\topsep}{-\@tempdi
      \addtolength{\partopsep}{\@tempdim
\def\listFB#1#2{\listORI{#1}{\FB@listsettings #
\let\endlistFB\end
%    \end{macroc
%  \end{ma
%  \end{ma

%  \begin{macro}{\itemiz
%  \begin{macro}{\itemize
%  \begin{macro}{\bbl@frenchlabelit
%  \begin{macro}{\bbl@nonfrenchlabelit
%    Let's now consider French itemize lists.  They differ from t
%    provided by the standard \LaTeXe{} clas
%    \begin{item
%      \item vertical spacing between items, before and a
%         the list, should be \emph{null} with \emph{no glue} ad
%      \item the item labels of a first level list should be vertic
%          aligned on the paragraph's first character (i.e
%          |\parindent| from the left marg
%      \item the `\textbullet' is never used in French itemize-li
%          a long dash `--' is preferred for all levels. The item l
%          used in French is stored in |\FrenchLabelItem}|, it defa
%          to `--' and can be changed using |\frenchbsetup{}|
%          section~\ref{sec-keyva
%    \end{item

%    \begin{macroc
\newcommand*{\FrenchLabelItem}{\textend
\newcommand*{\Frlabelitemi}{\FrenchLabelI
\newcommand*{\Frlabelitemii}{\FrenchLabelI
\newcommand*{\Frlabelitemiii}{\FrenchLabelI
\newcommand*{\Frlabelitemiv}{\FrenchLabelI
%    \end{macroc
%    |\bbl@frenchlabelitems| saves current itemize labels and cha
%    them to their value in French. This code should never be exec
%    twice in a row, so we need a new flag that will be set and r
%    by |\bbl@nonfrenchlabelitems| and |\bbl@frenchlabelite
%    \begin{macroc
\newif\ifFB@enterFrench  \FB@enterFrench
\def\bbl@frenchlabelite
  \ifFB@enterFr
    \let\@ltiORI\labeli
    \let\@ltiiORI\labelit
    \let\@ltiiiORI\labelite
    \let\@ltivORI\labelit
    \let\labelitemi\Frlabeli
    \let\labelitemii\Frlabelit
    \let\labelitemiii\Frlabelite
    \let\labelitemiv\Frlabelit
    \FB@enterFrenchf


\let\itemizeORI\ite
\let\enditemizeORI\endite
\let\enditemizeFB\endite
\def\itemize
    \ifnum \@itemdepth >\thr@@\@toodeep\
      \advance\@itemdepth
      \edef\@itemitem{labelitem\romannumeral\the\@itemdep
      \expanda
      \lis
      \csname\@itemitem\endcs
      {\settowidth{\labelwidth}{\csname\@itemitem\endcsna
       \setlength{\leftmargin}{\labelwid
       \addtolength{\leftmargin}{\labels
       \ifnum\@listdep
         \setlength{\itemindent}{\parinde
       \
         \addtolength{\leftmargin}{\parinde

       \setlength{\itemsep}{\
       \setlength{\parsep}{\
       \setlength{\topsep}{\
       \setlength{\partopsep}{\
%    \end{macroc
%    |\parskip| is of type `skip', its mean value only (\emph
%    the glue}) should be subtracted from |\topsep| and adde
%    |\partopsep|, so convert |\parskip| to a `dimen' u
%    |\@tempdi
%    \begin{macroc
       \@tempdima=\par
       \addtolength{\topsep}{-\@tempdi
       \addtolength{\partopsep}{\@tempdim

%    \end{macroc
%    The user's changes in labelitems are saved when leaving French
%    further use when switching back to French.  This code should n
%    be executed twice in a row (toggle with |\bbl@frenchlabelitem
%    \begin{macroc
\def\bbl@nonfrenchlabelite
  \ifFB@enterFr
  \
      \let\Frlabelitemi\labeli
      \let\Frlabelitemii\labelit
      \let\Frlabelitemiii\labelite
      \let\Frlabelitemiv\labelit
      \let\labelitemi\@lt
      \let\labelitemii\@lti
      \let\labelitemiii\@ltii
      \let\labelitemiv\@lti
      \FB@enterFrench


%    \end{macroc
%  \end{ma
%  \end{ma
%  \end{ma
%  \end{ma

%  \subsection{French indentation of secti
%  \label{sec-ind

%  \begin{macro}{\bbl@frenchind
%  \begin{macro}{\bbl@nonfrenchind
%    In French the first paragraph of each section should be inden
%    this is another difference with US-English. This is controlle
%    the flag |\if@afterinde

% \changes{v2.3d}{2009/03/16}{Bug correction: previous version
%    frenchb set the flag \cs{if@afterindent} to false out
%    French which is correct for English but wrong for some langu
%    like Spanish.  Pointed out by Juan Jos\'e Torre

%    We will need to save the value of the flag |\if@afterind
%    `AtBeginDocument' before eventually changing its va

%    \begin{macroc
\def\bbl@frenchindent{\let\@afterindentfalse\@afterindent
                      \@afterindentt
\def\bbl@nonfrenchindent{\let\@afterindentfalse\@ai
                         \@afterindentfa
%    \end{macroc
%  \end{ma
%  \end{ma

%  \subsection{Formatting footno
%  \label{sec-footno

% \changes{v2.0}{2006/11/06}{Footnotes are now pri
%     by default `\`a la fran\c caise' for the whole docume

% \changes{v2.0b}{2007/04/18}{Footnotes: Just do not
%    (except warning) when the bigfoot package is load

%    The \file{bigfoot} package deeply changes the way footnotes
%    handled. When |bigfoot| is loaded, we just warn the user
%    |frenchb| will drop the customisation of footno

%    The layout of footnotes is controlled by two f
%    |\ifFBAutoSpaceFootnotes| and |\ifFBFrenchFootnotes| which
%    set by options of |\frenchbsetup{}| (see section~\ref{sec-keyva
%    Notice that the layout of footnotes \emph{does not depend} on
%    current language (just think of two footnotes on the same
%    looking different because one was called in a French part,
%    other one in Englis

%    When |\ifFBAutoSpaceFootnotes| is true, |\@footnotemark| (w
%    definition is saved at the |\begin{document}| in order to inc
%    any customisation that packages might have done) is redefine
%    add a thin space before the number or symbol calling a foot
%    (any space typed in is removed first).  This has no effec
%    the layout of the footnote its

% \changes{v2.4a}{2009/11/23}{\cs{PackageWarning} change
%      \cs{FBWarning} (when bigfoot package in us

%    \begin{macroc
\AtBeginDocument{\@ifpackageloaded{bigfo
                   {\FBWarning{frenchb.l
                     {bigfoot package in use.\MessageB
                      frenchb will NOT customise footnotes;\MessageB
                      reporte
                   {\let\@footnotemarkORI\@footnote
                    \def\@footnotemarkFB{\leavevmode\unskip\un
                                         \,\@footnotemarkO
                    \ifFBAutoSpaceFootn
                      \let\@footnotemark\@footnotema
                    \

%    \end{macroc

%    We then define |\@makefntextFB|, a variant of |\@makefnt
%    which is responsible for the layout of footnotes, to match
%    specifications of the French `Imprimerie Nationale':  footn
%    will be indented by |\parindentFFN|, numbers (if any) typese
%    the baseline (instead of superscripts) and followed by a
%    and an half quad space. Whenever symbols are used to nu
%    footnotes (as in |\thanks| for instance), we switch back to
%    standard layout (the French layout of footnotes is meant
%    footnotes numbered by Arabic or Roman digi

% \changes{v2.0}{2006/11/06}{\cs{parindentFFN} not change
%    already defined (required by JA for cah-gut.cl

% \changes{v2.3b}{2008/12/06}{New commands \cs{dotFFN}
%    \cs{kernFFN} for more flexibility (suggested by J

%    The value of |\parindentFFN| will be redefined at
%    |\begin{document}|, as the maximum of |\parindent| and 1
%    \emph{unless} it has been set in the preamble (the weird v
%    10in is just for testing whether |\parindentFFN| has been
%    or n

%    \begin{macroc
\newcommand*{\dotFFN
\newcommand*{\kernFFN}{\kern .
\newdimen\parinden
\parindentFFN=
\def\ftnISsymbol{\@fnsymbol\c@footn
\long\def\@makefntextFB#1{\ifx\thefootnote\ftnISsy
                            \@makefntextORI{
                          \
                            \parindent=\parinden
                            \rule\z@\footnot
                            \setbox\@tempboxa\hbox{\@thefnma
                            \ifdim\wd\@tempboxa
                              \llap{\@thefnmark}\dotFFN\ker
                            \f
                          \
%    \end{macroc

%    We save the standard definition of |\@makefntext| at
%    |\begin{document}|, and then redefine |\@makefntext| accordin
%    the value of flag |\ifFBFrenchFootnotes| (true or fal

%    \begin{macroc
\AtBeginDocument{\@ifpackageloaded{bigfoot
                  {\ifdim\parindentFFN<
                   \
                      \parindentFFN=\parin
                      \ifdim\parindentFFN<1.5em\parindentFFN=1.5e

                   \let\@makefntextORI\@makefn
                   \long\def\@makefntext
                      \ifFBFrenchFootn
                         \@makefntextFB{
                      \
                         \@makefntextORI{
                      \


%    \end{macroc

%    For compatibility reasons, we provide definitions for the comm
%    dealing with the layout of footnotes in |frenchb| version~
%    |\frenchbsetup{}| (see in section \ref{sec-keyval}) shoul
%    preferred for setting these options.  |\StandardFootnotes|
%    still be used locally (in minipages for instance), that's why
%    test |\ifFBFrenchFootnotes| is done inside |\@makefnte
%    \begin{macroc
\newcommand*{\AddThinSpaceBeforeFootnotes}{\FBAutoSpaceFootnotest
\newcommand*{\FrenchFootnotes}{\FBFrenchFootnotest
\newcommand*{\StandardFootnotes}{\FBFrenchFootnotesfa
%    \end{macroc

%  \subsection{Global lay
%  \label{sec-glo

%    In multilingual documents, some typographic rules must de
%    on the current language (e.g., hyphenation, typesettin
%    numbers, spacing before double punctuation\dots), others sho
%    IMHO, be kept global to the document: especially the layou
%    lists (see~\ref{sec-lists}) and footn
%    (see~\ref{sec-footnotes}), and the indentation of the f
%    paragraph of sections (see~\ref{sec-inden

%    From version 2.2 on, if |frenchb| is \babel's ``main langua
%    (i.e. last language option at \babel's loading), |fren
%    customises the layout (i.e. lists, indentation of the f
%    paragraphs of sections and footnotes) in the whole docu
%    regardless the current language.   On the other hand, if |fren
%    is \emph{not} \babel's ``main language'', it leaves the la
%    unchanged both in French and in other langua

%  \begin{macro}{\FrenchLay
%  \begin{macro}{\StandardLay
%    The former commands |\FrenchLayout| and |\StandardLayout| are
%    for compatibility reasons but should no longer be u

% \changes{v2.0g}{2008/03/23}{Flag \cs{ifFBStandardLayout} not che
%     by \cs{FBprocess@options}, low-level flags have to be
%     one by o

%    \begin{macroc
\newcommand*{\FrenchLayou
    \FBGlobalLayoutFrench
    \PackageWarning{frenchb.l
    {\protect\FrenchLayout\space is obsolete.  Please use\MessageB
     \protect\frenchbsetup{GlobalLayoutFrench} instea

\newcommand*{\StandardLayou
  \FBReduceListSpacingf
  \FBCompactItemizef
  \FBStandardItemLabels
  \FBIndentFirstf
  \FBFrenchFootnotesf
  \FBAutoSpaceFootnotesf
  \PackageWarning{frenchb.l
    {\protect\StandardLayout\space is obsolete.  Please use\MessageB
    \protect\frenchbsetup{StandardLayout} instea

\@onlypreamble\FrenchLa
\@onlypreamble\StandardLa
%    \end{macroc
%  \end{ma
%  \end{ma

%  \subsection{Dots\d
%  \label{sec-d

%  \begin{macro}{\FBtextellip
%    \LaTeXe's standard definition of |\dots| in text-mod
%    |\textellipsis| which includes a |\kern| at the
%    this space is not wanted in some cases (before a closing b
%    for instance) and |\kern| breaks hyphenation of the next w
%    We define |\FBtextellipsis| for French (in \LaTeXe{} on

%    The |\if| construction in the \LaTeXe{} definition of |\d
%    doesn't allow the use of |xspace| (|xspace| is always foll
%    by a |\fi|), so we use the AMS-\LaTeX{} construction of |\do
%    this has to be done `AtBeginDocument' not to be overwri
%    when \file{amsmath.sty} is loaded after \ba

% \changes{v2.0}{2006/11/06}{Added special case for LY1 encod
%    see  bug report from Bruno Voisin (2004/05/1

% \changes{v2.5f}{2011/06/18}{Unicode fonts also provide a ready
%    character for \cs{textellipsis}, let's just use
%    (reported by Maxime Chupin, 2011/06/0

%    LY1 has a ready made character for |\textellipsis|, it shoul
%    used in French too. The same is true for Unicode fonts in
%    with XeTeX and Lua

%    \begin{macroc
\ifFBuni
  \let\FBtextellipsis\textelli
\
  \DeclareTextSymbol{\FBtextellipsis}{LY1}{
  \DeclareTextCommandDefault{\FBtextellipsi
    .\kern\fontdimen3\font.\kern\fontdimen3\font.\xsp

%    \end{macroc
%    |\Mdots@| and |\Tdots@ORI| hold the definitions of |\dots
%    Math and Text mode. They default to those of amsmath-2.0,
%    will revert to standard \LaTeX{} definitions `AtBeginDocume
%    if amsmath has not been loaded. |\Mdots@| doesn't change
%    switching from/to French, while |\Tdots@| is |\FBtextellip
%    in French and |\Tdots@ORI| otherw
%    \begin{macroc
\newcommand*{\Tdots@ORI}{\@xp\textellip
\newcommand*{\Tdots@}{\Tdots@
\newcommand*{\Mdots@}{\@xp\mdo
\AtBeginDocument{\DeclareRobustCommand*{\dots}{\r
                 \csname\ifmmode M\else T\fi dots@\endcsna
                 \@ifundefined{@xp}{\let\@xp\relax
                 \@ifundefined{mdots@}{\let\Tdots@ORI\textelli
                                       \let\Mdots@\mathelli
                                       \let\mdots@\undefined
\def\bbl@frenchdots{\let\Tdots@\FBtextellip
\def\bbl@nonfrenchdots{\let\Tdots@\Tdots@
\addto\extrasfrench{\bbl@frenchd
\addto\noextrasfrench{\bbl@nonfrenchd
%    \end{macroc
%  \end{ma

%  \subsection{Setup options: keyval st
%  \label{sec-key

% \changes{v2.0}{2006/11/06}{New command \cs{frenchbsetup} a
%     for global customisati

% \changes{v2.0c}{2007/06/25}{Option ThinSpaceInFrenchNumbers add

% \changes{v2.0d}{2007/07/15}{Options og and fg changed: l
%     the definition to French so that quote characters can be
%     in Germ

% \changes{v2.0e}{2007/10/05}{New option: StandardLis

% \changes{v2.0f}{2008/03/23}{Two typos correcte
%    option StandardLists: [false] $\to$ [true]
%    StandardLayout $\to$ StandardLis

% \changes{v2.0f}{2008/03/23}{StandardLayout option ha
%     effect on lists.  Test moved to \cs{FBprocess@option

% \changes{v2.0g}{2008/03/23}{Revert previous chang
%     StandardLayout. This option must set the three f
%     \cs{FBReduceListSpacingfalse}, \cs{FBCompactItemizefal
%     and \cs{FBStandardItemLabeltrue} instea
%     \cs{FBStandardListstrue}, so that later options can s
%     change their value before executing \cs{FBprocess@optio
%     Same thing for option StandardLis

% \changes{v2.1a}{2008/03/24}{New option: FrenchSuperscr
%     to define \cs{up} as \cs{fup} or as \cs{textsuperscrip

% \changes{v2.1a}{2008/03/30}{New option: LowercaseSuperscrip

% \changes{v2.2a}{2008/05/08}{The global layout of the documen
%     no longer changed when frenchb is not the last option of b
%     (\cs{bbl@main@language}). Suggested by Ulrike Fisch

% \changes{v2.2a}{2008/05/08}{Values of f
%     \cs{ifFBReduceListSpacing}, \cs{ifFBCompactItemi
%     \cs{ifFBStandardItemLabels}, \cs{ifFBIndentFir
%     \cs{ifFBFrenchFootnotes}, \cs{ifFBAutoSpaceFootnotes} chan
%     default now means `StandardLayout', it will be change
%     `FrenchLayout' AtEndOfPackage only if frenc
%     \cs{bbl@main@languag

% \changes{v2.2a}{2008/05/08}{When frenchb is babel's last opt
%     French becomes the document's main language
%     GlobalLayoutFrench appli

% \changes{v2.3a}{2008/10/10}{New option: OriginalTypewriter.
%    frenchb switches to \cs{noautospace@beforeFDP} when a tt-fon
%    in use.  When OriginalTypewriter is set to true, frenchb beh
%    as in pre-2.3 versio

% \changes{v2.4a}{2009/11/23}{New option SuppressWarni

%    We first define a collection of conditionals with their defa
%    (true or fal

%    \begin{macroc
\newif\ifFBStandardLayout           \FBStandardLayout
\newif\ifFBGlobalLayoutFrench       \FBGlobalLayoutFrenchf
\newif\ifFBReduceListSpacing        \FBReduceListSpacingf
\newif\ifFBCompactItemize           \FBCompactItemizef
\newif\ifFBStandardItemLabels       \FBStandardItemLabels
\newif\ifFBStandardLists            \FBStandardLists
\newif\ifFBIndentFirst              \FBIndentFirstf
\newif\ifFBFrenchFootnotes          \FBFrenchFootnotesf
\newif\ifFBAutoSpaceFootnotes       \FBAutoSpaceFootnotesf
\newif\ifFBOriginalTypewriter       \FBOriginalTypewriterf
\newif\ifFBThinColonSpace           \FBThinColonSpacef
\newif\ifFBThinSpaceInFrenchNumbers \FBThinSpaceInFrenchNumbersf
\newif\ifFBFrenchSuperscripts       \FBFrenchSuperscripts
\newif\ifFBLowercaseSuperscripts    \FBLowercaseSuperscripts
\newif\ifFBPartNameFull             \FBPartNameFull
\newif\ifFBSuppressWarning          \FBSuppressWarningf
\newif\ifFBShowOptions              \FBShowOptionsf
%    \end{macroc

%    The defaults values of these flags have been set so that |fren
%    does not change anything regarding the global lay
%    |\bbl@main@language| (set by the last option of babel) cont
%    the global layout of the document.  We check the current lang
%    `AtEndOfPackage' (it is |\bbl@main@language|); if it is Fre
%    the values of some flags have to be changed to ensure a Fr
%    looking layout for the whole document (even in parts writte
%    languages other than French); the end-user will then be abl
%    customise the values of all these flags with |\frenchbsetup

% \changes{v2.5b}{2010/10/30}{Do not use
%    test \cs{iflanguage}\{french\} to check whether French is
%    main language or not, as it might be be erroneously posi
%    when English is the main language and no hyphenation patt
%    are available for Fre
%    In this case \cs{l@french} and \cs{l@english} ar
%    Pointed out by G\"unter Mil

%    \begin{macroc
\def\FB@french{fre
\AtEndOfPacka
  \ifx\bbl@main@language\FB@fr
    \FBReduceListSpacing
    \FBCompactItemize
    \FBStandardItemLabelsf
    \FBIndentFirst
    \FBFrenchFootnotes
    \FBAutoSpaceFootnotes
    \FBGlobalLayoutFrench


%    \end{macroc

%  \begin{macro}{\frenchbse
%    From version 2.0 on, all setup options are handled by \emph{
%    command |\frenchbsetup| using the keyval syn
%    Let's now define this command which reads and sets the opt
%    to be processed later (at |\begin{document}|
%    |\FBprocess@options|. It  can only be called in the pream
%    \begin{macroc
\newcommand*{\frenchbsetup}[
  \setkeys{FB}{

\@onlypreamble\frenchbs
%    \end{macroc
%    |frenchb| being an option of babel, it cannot load a pac
%    (keyval) while |frenchb.ldf| is read, so we defer the loadin
%    \file{keyval} and the options setup at the end of \babel's load

%    |StandardLayout| resets the layout in French to the standard la
%    defined par the \LaTeX{} class and packages loaded. It deals
%    lists, indentation of first paragraphs of sections and footno
%    Other keys, entered \emph{after} |StandardLayout
%    |\frenchbsetup|, can overrule some of the |StandardLay
%     setti

%    When French is the main language, |GlobalLayoutFrench| forces
%    layout in French and (as far as possible) outside French to
%    the French typographic standa

% \changes{v2.3d}{2009/03/16}{Warning added to \cs{GlobalLayoutFre
%    when French is not the main langua

% \changes{v2.5b}{2010/10/30}{Do not use
%    test \cs{iflanguage}\{french\} to check whether French is
%    main language or not, as it might be be erroneously posi
%    when English is the main language and no hyphenation patt
%    are available for Fre
%    In this case \cs{l@french} and \cs{l@english} ar
%    Pointed out by G\"unter Mil

%    \begin{macroc
\AtEndOfPacka
    \RequirePackage{keyv
    \define@key{FB}{StandardLayout}[tr
                      {\csname FBStandardLayout#1\endcs
                       \ifFBStandardLa
                         \FBReduceListSpacingf
                         \FBCompactItemizef
                         \FBStandardItemLabels
                         \FBIndentFirstf
                         \FBFrenchFootnotesf
                         \FBAutoSpaceFootnotesf
                         \FBGlobalLayoutFrenchf
                       \
                         \FBReduceListSpacing
                         \FBCompactItemize
                         \FBStandardItemLabelsf
                         \FBIndentFirst
                         \FBFrenchFootnotes
                         \FBAutoSpaceFootnotes
                       \
    \define@key{FB}{GlobalLayoutFrench}[tr
                      {\csname FBGlobalLayoutFrench#1\endcs
                       \ifFBGlobalLayoutFr
                         \ifx\bbl@main@language\FB@fr
                            \FBReduceListSpacing
                            \FBCompactItemize
                            \FBStandardItemLabelsf
                            \FBIndentFirst
                            \FBFrenchFootnotes
                            \FBAutoSpaceFootnotes
                         \
                            \PackageWarning{frenchb.l
                              {Option `GlobalLayoutFrench' skip
                               \MessageBreak French is *
                               babel's last option.\MessageBre
                             \FBGlobalLayoutFrenchf

                       \
    \define@key{FB}{ReduceListSpacing}[tr
                      {\csname FBReduceListSpacing#1\endcsna
    \define@key{FB}{CompactItemize}[tr
                      {\csname FBCompactItemize#1\endcsna
    \define@key{FB}{StandardItemLabels}[tr
                      {\csname FBStandardItemLabels#1\endcsna
    \define@key{FB}{ItemLabel
        \renewcommand*{\FrenchLabelItem}{#
    \define@key{FB}{ItemLabel
        \renewcommand*{\Frlabelitemi}{#
    \define@key{FB}{ItemLabeli
        \renewcommand*{\Frlabelitemii}{#
    \define@key{FB}{ItemLabelii
        \renewcommand*{\Frlabelitemiii}{#
    \define@key{FB}{ItemLabeli
        \renewcommand*{\Frlabelitemiv}{#
    \define@key{FB}{StandardLists}[tr
                      {\csname FBStandardLists#1\endcs
                       \ifFBStandardL
                         \FBReduceListSpacingf
                         \FBCompactItemizef
                         \FBStandardItemLabels
                       \
                         \FBReduceListSpacing
                         \FBCompactItemize
                         \FBStandardItemLabelsf
                       \
    \define@key{FB}{IndentFirst}[tr
                      {\csname FBIndentFirst#1\endcsna
    \define@key{FB}{FrenchFootnotes}[tr
                      {\csname FBFrenchFootnotes#1\endcsna
    \define@key{FB}{AutoSpaceFootnotes}[tr
                      {\csname FBAutoSpaceFootnotes#1\endcsna
    \define@key{FB}{AutoSpacePunctuation}[tr
                      {\csname FBAutoSpacePunctuation#1\endcsna
    \define@key{FB}{OriginalTypewriter}[tr
                      {\csname FBOriginalTypewriter#1\endcsna
    \define@key{FB}{ThinColonSpace}[tr
                      {\csname FBThinColonSpace#1\endcsna
    \define@key{FB}{ThinSpaceInFrenchNumbers}[tr
                      {\csname FBThinSpaceInFrenchNumbers#1\endcsna
    \define@key{FB}{FrenchSuperscripts}[tr
                      {\csname FBFrenchSuperscripts#1\endcsn
    \define@key{FB}{LowercaseSuperscripts}[tr
                      {\csname FBLowercaseSuperscripts#1\endcsn
    \define@key{FB}{PartNameFull}[tr
                      {\csname FBPartNameFull#1\endcsna
    \define@key{FB}{SuppressWarning}[tr
                      {\csname FBSuppressWarning#1\endcs
                       \ifFBSuppressWar
                         \renewcommand{\FBWarning}[2]{\rel
                       \
                         \renewcommand{\FBWarning}[
                                          \PackageWarning{##1}{##

    \define@key{FB}{ShowOptions}[tr
                      {\csname FBShowOptions#1\endcsna
%    \end{macroc
%    Inputing French quotes as \emph{single characters} when they
%    available on the keyboard (through a compose key for insta
%    is more comfortable than typing |\og| and |\
%    The purpose of the following code is to map the French q
%    characters to |\og\ignorespaces| and |{\fg}| respectively
%    the current language is French, and to |\guillemotleft|
%    |\guillemotright| otherwise (think of German quotes); thus cor
%    unbreakable spaces will be added automatically to French quo
%    The quote characters typed in depend on the input encod
%    it can be single-byte (latin1, latin9, applemac,\dots
%    multi-bytes (utf-8, utf8x).  We first check whether XeTe
%    LuaTeX engines are used, if not the package \file{inputenc} ha
%    be loaded before the |\begin{document}| with the proper co
%    option, so we check if |\DeclareInputText| is defi

% \changes{v2.4c}{2010/05/23}{In \cs{ttfamilyFB}, also ca
%    automatic spaces inside French guillemets coded as charac
%    (see \cs{frenchbsetup

% \changes{v2.5a}{2010/08/21}{Test \cs{@ifundefined} leaves
%    tested control sequence defined as \cs{relax} when T
%    Changed \cs{relax} to \cs{undefined} when tes
%    \cs{XeTeXrevision}, \cs{DeclareInputText}, \cs{uc@dc
%    \cs{DeclareUnicodeCharacter}, \cs{mule@def} in \cs{og}
%    \cs{f

% \changes{v2.5c}{2011/01/16}{The code meant for XeTeX also works
%    LuaTeX, we just need to change the te

% \changes{v2.5g}{2011/12/31}{When \cs{ifFB@xetex@punct} is t
%   `og' and `fg' options now set XeTeXcharclasses of these charac
%   to \cs{FB@punctguilo} and \cs{FB@punctguilf}.  Otherwise Fr
%   quotes behave as normal characters (their XeTeXcharclass is

%    \begin{macroc
    \define@key{FB}{o
       \newcommand*{\FB@@o
          \iflanguage{fren
             {\ifFBAutoSpaceGuill\FB@og\ignoresp
              \else\guillemot
              \
             {\guillemotlef
       \ifFBuni
%    \end{macroc
%    LuaTeX or XeTeX in
%    \begin{macroc
         \ifFB@xetex@p
%    \end{macroc
%    |\XeTeXinterchartokenstate| is defined, we just need to
%    |\XeTeXcharclass| to |\FB@punctguilo| for the French ope
%    quote (see subsection~\ref{sec-punct}) and to sw
%    |\ifFBog@addspace| to false, otherwise commands |\og|
%    |\fg| would produce a double space; the |\ifFBguillo@addsp
%    flag is needed when switching from |\ttfamily| back to |\sf
%    |\
%    \begin{macroc
           \XeTeXcharclass"00AB = \FB@punctg
           \FBguillo@addspacetrue \FBog@addspacef
         \
%    \end{macroc
%    then LuaTeX or an old XeTeX in use, the following t
%    for defining the active quote character is borrowed
%    \file{inputenc.d
%    \begin{macroc
           \catcode`#1=\ac
           \bg
             \uccode`\~
             \upperca
           \eg
           \d
           }{\FB@@

       \
%    \end{macroc
%    This is for conventional TeX engi
%    \begin{macroc
           \AtBeginDocu
           {\@ifundefined{DeclareInputTe
             {\PackageWarning{frenchb.l
               {Option `og' requires package inputenc.\MessageBre
              \let\DeclareInputText\undef

             {\@ifundefined{uc@dc
%    \end{macroc
%    if |\uc@dclc| is undefined, utf8x is not loaded\
%    \begin{macroc
               {\@ifundefined{DeclareUnicodeCharact
%    \end{macroc
%    if |\DeclareUnicodeCharacter| is undefined, utf8 is not lo
%    either, we assume 8-bit character input encod
%    Package \file{MULEenc.sty} (from CJK) defines |\mule@def| to
%    characters to control sequen
%    \begin{macroc
                  {\@tempcnta`#1\r
                     \@ifundefined{mule@d
                       {\DeclareInputText{\the\@tempcnta}{\FB@@
                        \let\mule@def\undefin
                       {\mule@def{11}{{\FB@@og
                   \let\DeclareUnicodeCharacter\undef

%    \end{macroc
%    utf8 loaded, use |\DeclareUnicodeCharact
%    \begin{macroc
                  {\DeclareUnicodeCharacter{00AB}{\FB@@o
                \let\uc@dclc\undef

%    \end{macroc
%    utf8x loaded, use |\uc@dc
%    \begin{macroc
               {\uc@dclc{171}{default}{\FB@@o




%    \end{macroc
%    Same code for the closing qu
%    \begin{macroc
    \define@key{FB}{f
       \newcommand*{\FB@@f
          \iflanguage{fren
             {\ifFBAutoSpaceGuill\F
              \else\guillemotr
              \
             {\guillemotrigh
       \ifFBuni
         \ifFB@xetex@p
           \XeTeXcharclass"00BB = \FB@punctg
           \FBguillf@addspacetrue \FBfg@addspacef
         \
            \catcode`#1=\ac
            \bg
              \uccode`\~
              \upperca
            \eg
            \d
            }{{\FB@@f

       \
         \AtBeginDocu
           {\@ifundefined{DeclareInputTe
             {\PackageWarning{frenchb.l
               {Option `fg' requires package inputenc.\MessageBre
              \let\DeclareInputText\undef

             {\@ifundefined{uc@dc
               {\@ifundefined{DeclareUnicodeCharact
                  {\@tempcnta`#1\r
                     \@ifundefined{mule@d
                       {\DeclareInputText{\the\@tempcnta}{{\FB@@f
                        \let\mule@def\undef

                       {\mule@def{27}{{\FB@@fg
                   \let\DeclareUnicodeCharacter\undef

                  {\DeclareUnicodeCharacter{00BB}{{\FB@@f

                \let\uc@dclc\undef

               {\uc@dclc{187}{default}{{\FB@@fg





%    \end{macroc
%  \end{ma

% \begin{macro}{\FBprocess@opti
%    |\FBprocess@options| processes the options, it is called \emph{o
%    at |\begin{documen
%    \begin{macroc
\newcommand*{\FBprocess@option
%    \end{macroc
%    Nothing has to be done here for |StandardLayout|
%    |StandardLists| (the involved flags have already been se
%    |\frenchbsetup{}| or before (at babel's EndOfPacka

%    The next three options deal with the layout of lists in Fre

%    |ReduceListSpacing| reduces the vertical spaces between
%    items in French (done by changing |\list| to |\listF
%    When |GlobalLayoutFrench| is true (the default), the sam
%    done outside French except for languages that force a diffe
%    sett
%    \begin{macroc
  \ifFBReduceListSpa
    \addto\extrasfrench{\let\list\li
                        \let\endlist\endlist
    \addto\noextrasfrench{\ifFBGlobalLayoutFr
                            \let\list\li
                            \let\endlist\endli
                          \
                            \let\list\lis
                            \let\endlist\endlis
                          \
  \
    \addto\extrasfrench{\let\list\lis
                        \let\endlist\endlistO
    \addto\noextrasfrench{\let\list\lis
                          \let\endlist\endlistO

%    \end{macroc

%    |CompactItemize| suppresses the vertical spacing between
%    items in French (done by changing |\itemize| to |\itemizeF
%    When |GlobalLayoutFrench| is true the same is done outside Fre

% \changes{v2.4b}{2010/04/30}{Set flag CompactItemize=false when
%    enumitem package is loaded (ensures compatibilit

%    \begin{macroc
  \@ifpackageloaded{enumite
      \FBCompactItemizef
      \FBWarning{frenchb.l
         {Setting CompactItemize=false for compatibility\MessageB
                   with enumitem package

  \ifFBCompactIte
     \addto\extrasfrench{\let\itemize\itemi
                        \let\enditemize\enditemize
     \addto\noextrasfrench{\ifFBGlobalLayoutFr
                             \let\itemize\itemi
                             \let\enditemize\enditemi
                           \
                             \let\itemize\itemiz
                             \let\enditemize\enditemiz
                           \
  \
    \addto\extrasfrench{\let\itemize\itemiz
                        \let\enditemize\enditemizeO
    \addto\noextrasfrench{\let\itemize\itemiz
                          \let\enditemize\enditemizeO

%    \end{macroc

%    |StandardItemLabels| resets labelitems in French to t
%    standard values set by the \LaTeX{} class and packages loa
%    When |GlobalLayoutFrench| is true labelitems are identical in
%    and outside Fre
%    \begin{macroc
  \ifFBStandardItemLa
    \addto\extrasfrench{\bbl@nonfrenchlabelite
    \addto\noextrasfrench{\bbl@nonfrenchlabelite
  \
    \addto\extrasfrench{\bbl@frenchlabelite
    \addto\noextrasfrench{\ifFBGlobalLayoutFr
                            \bbl@frenchlabeli
                          \
                            \bbl@nonfrenchlabeli
                          \

%    \end{macroc

%    |IndentFirst| forces the first paragraphs of sections t
%    indented just like the other ones in Fre
%    When |GlobalLayoutFrench| is true, the same is done outside Fr
%    except for languages that force a different sett
%    |\bbl@nonfrenchindent| has been designed to be smart with o
%    languages (like Spanish) who also indent every first paragraph
%    sections (see section~\ref{sec-inden
%    \begin{macroc
  \ifFBIndentF
    \addto\extrasfrench{\bbl@frenchinde
    \addto\noextrasfrench{\ifFBGlobalLayoutFr
                            \bbl@frenchin
                          \
                            \bbl@nonfrenchin
                          \
  \
    \addto\extrasfrench{\bbl@nonfrenchinde
    \addto\noextrasfrench{\bbl@nonfrenchinde

%    \end{macroc

%    The layout of footnotes is handled at the |\begin{docume
%    depending on the values of flags |FrenchFootno
%    and |AutoSpaceFootnotes| (see section~\ref{sec-footnote
%    nothing has to be done here for footno

%    |AutoSpacePunctuation| adds an unbreakable space (in French o
%    before the four active characters (:;!?) even if none has
%    typed before t
%    \begin{macroc
  \ifFBAutoSpacePunctua
     \autospace@befor
  \
     \noautospace@befor

%    \end{macroc

%    When |OriginalTypewriter| is set to |false| (the defau
%    |\ttfamily|, |\rmfamily| and |\sffamily| are redefine
%    |\ttfamilyFB|, |\rmfamilyFB| and |\sffamilyFB| respecti
%    to prevent addition of automatic spaces before the four ac
%    characters in computer c
%    \begin{macroc
  \ifFBOriginalTypewr
  \
     \let\ttfamily\ttfami
     \let\rmfamily\rmfami
     \let\sffamily\sffami

%    \end{macroc

%    |ThinColonSpace| changes the normal unbreakable space typese
%     French before `:' to a thin sp
%    \begin{macroc
  \ifFBThinColonSpace\renewcommand*{\Fcolonspace}{\Fthinspace
%    \end{macroc

%    When |true|, |ThinSpaceInFrenchNumbers| redefines |numprint.st
%    command |\npstylefrench| to set |\npthousandsep| to
%    (thinspace) instead of |~| (default) . This option has no ef
%    if package \file{numprint.sty} is not loaded with `|autolanguag
%    As old versions of \file{numprint.sty} did not de
%    |\npstylefrench|, we have to provide this comm
%    \begin{macroc
  \@ifpackageloaded{numpri
  {\ifnprt@autolang
     \providecommand*{\npstylefrench
     \ifFBThinSpaceInFrenchNum
       \renewcommand*\npstylefren
          \npthousandsep{
          \npdecimalsign
          \npproductsign{\cd
          \npunitseparator{
          \npdegreeseparato
          \nppercentseparator{\nprt@units

     \
       \renewcommand*\npstylefren
          \npthousandsep
          \npdecimalsign
          \npproductsign{\cd
          \npunitseparator{
          \npdegreeseparato
          \nppercentseparator{\nprt@units


     \npaddtolanguage{french}{fren
   \fi
%    \end{macroc

%    |FrenchSuperscripts|: if |true| |\up=\fup|,
%    |\up=\textsuperscript|. Anyway |\up*=\FB@up@fake|. The star-
%    |\up*{}| is provided for fonts that lack some superior lett
%    Adobe Jenson Pro and Utopia Expert have no ``g superior''
%    insta
%    \begin{macroc
  \ifFBFrenchSuperscr
    \DeclareRobustCommand*{\up}{\@ifstar{\FB@up@fake}{\fu
  \
    \DeclareRobustCommand*{\up}{\@ifstar{\FB@up@fa
                                        {\textsuperscrip

%    \end{macroc

%    |LowercaseSuperscripts|: if |true| let |\FB@lc| be |\lowerca
%     else |\FB@lc| is redefined to do noth
%    \begin{macroc
  \ifFBLowercaseSuperscr
  \
    \renewcommand*{\FB@lc}[1]{#

%    \end{macroc

%    |PartNameFull|: if |false|, redefine |\partna
%    \begin{macroc
  \ifFBPartName
  \else\addto\captionsfrench{\def\partname{Partie}
%    \end{macroc

%    |ShowOptions|: if |true|, print the list of all options to
%    \file{.log} f
%    \begin{macroc
  \ifFBShowOpt
    \GenericWarning{*
     * **** List of possible options for frenchb ****\MessageB
     [Default values between brackets when frenchb is loaded *LAS
     \MessageB
     ShowOptions=true [false]\MessageB
     StandardLayout=true [false]\MessageB
     GlobalLayoutFrench=false [true]\MessageB
     StandardLists=true [false]\MessageB
     IndentFirst=false [true]\MessageB
     ReduceListSpacing=false [true]\MessageB
     CompactItemize=false [true]\MessageB
     StandardItemLabels=true [false]\MessageB
     ItemLabels=\textemdash, \textbul
        \protect\ding{43},... [\textendash]\MessageB
     ItemLabeli=\textemdash, \textbul
        \protect\ding{43},... [\textendash]\MessageB
     ItemLabelii=\textemdash, \textbul
        \protect\ding{43},... [\textendash]\MessageB
     ItemLabeliii=\textemdash, \textbul
        \protect\ding{43},... [\textendash]\MessageB
     ItemLabeliv=\textemdash, \textbul
        \protect\ding{43},... [\textendash]\MessageB
     FrenchFootnotes=false [true]\MessageB
     AutoSpaceFootnotes=false [true]\MessageB
     AutoSpacePunctuation=false [true]\MessageB
     OriginalTypewriter=true [false]\MessageB
     ThinColonSpace=true [false]\MessageB
     ThinSpaceInFrenchNumbers=true [false]\MessageB
     FrenchSuperscripts=false [true]\MessageB
     LowercaseSuperscripts=false [true]\MessageB
     PartNameFull=false [true]\MessageB
     SuppressWarning=true [false]\MessageB
     og= <left quote character>, fg= <right quote charac
     \MessageB
     *****************************************
     \MessageBreak\protect\frenchbsetup{ShowOptio


%    \end{macroc
%  \end{ma

% \changes{v2.0}{2006/12/15}{AtBeginDocument, save again
%    definitions of the `list' and `itemize' environments and
%    values of labelitems.  As of frenchb v.1.6, `ORI' values
%    set when reading frenchb.ldf, later changes were ignor

% \changes{v2.0}{2006/12/06}{Added warning for OT1 encodi

% \changes{v2.1b}{2008/04/07}{Disable some commands in bookmar

% \changes{v2.5a}{2010/08/21}{Test \cs{@ifundefined} leaves
%    tested control sequence defined as \cs{relax} when T
%    Changed \cs{relax} to \cs{undefined} when tes
%    \cs{pdfstringdefDisableCommands} AtBeginDocume

% \changes{v2.5e}{2011/04/03}{The redefinitio
%    \cs{pdfstringdefDisableCommands} from \cs{relax} to \cs{undefi
%    was misplaced.  Reported by S\'ebastien Gouez

% \changes{v2.5e}{2011/04/03}{\cs{pdfstringdefDisableCommands} sh
%    redefine \cs{FB@og} and \cs{FB@fg} instead of \cs{og} and \cs
%    so that it works also when quotes are entered as charact
%    Reported by S\'ebastien Gouez

%    At |\begin{document}| we save again the value of |\if@afterind
%    and definitions of the `list' and `itemize' environments and
%    values of labelitems so that all changes made in the preamble
%    taken into account in languages other than French and in Fr
%    with the StandardLayout opt
%    We also have to provide an |\xspace| command in case
%    \file{xspace.sty} package is not loa

%    \begin{macroc
\AtBeginDocume
   \ifx\@afterindentfalse\@afterindent
         \let\@aifORI\@afterindent
   \else \let\@aifORI\@afterindentf

   \let\listORI\
   \let\endlistORI\end
   \let\itemizeORI\ite
   \let\enditemizeORI\endite
   \let\@ltiORI\labeli
   \let\@ltiiORI\labelit
   \let\@ltiiiORI\labelite
   \let\@ltivORI\labelit
   \providecommand*{\xspace}{\rel
%    \end{macroc

% \changes{v2.5g}{2011/11/13}{Redefine \cs{degre}, \cs{degres} \cs
%    \cs{circonflexe} and \cs{tild} for bookmarks. Add \cs{fup} al

%    Let's redefine some commands in \file{hyperref}'s bookma
%    \begin{macroc
   \@ifundefined{pdfstringdefDisableComman
     {\let\pdfstringdefDisableCommands\undefin
     {\pdfstringdefDisableComman
        \let\up\r
        \let\fup\r
        \let\degre\textde
        \let\degres\textde
        \def\ieme{e\xspa
        \def\iemes{es\xspa
        \def\ier{er\xspa
        \def\iers{ers\xspa
        \def\iere{re\xspa
        \def\ieres{res\xspa
        \def\FrenchEnumerate#1{#1\degre\spa
        \def\FrenchPopularEnumerate#1{#1\degre)\spa
        \def\No{N\degre\spa
        \def\no{n\degre\spa
        \def\Nos{N\degre\spa
        \def\nos{n\degre\spa
        \def\FB@og{\guillemotleft\spa
        \def\FB@fg{\space\guillemotrig
        \def\at
        \def\circonflexe{\strin
        \def\tild{\strin
        \let\bsc\te


%    \end{macroc

%    It is time to process the options set with |\frenchboptions
%    Then we need to execute either |\extrasfrench|
%    |\captionsfrench| or |\noextrasfrench| according to the cur
%    language at the |\begin{document}| (these three commands have
%    updated by |\FBprocess@options|). \emph{But}, when French is
%    main language, |\extrasfrench| is executed \emph{again}
%    (French has been switched on `AtBeginDocument' some time befo
%    This is harmless, except for |\bbl@frenchspacing| which
%    redefine |\bbl@nonfrenchspacing| to |\relax|, this wil
%    wrong if the user switches to Engl
%    When French is \emph{not} the main language, |\noextrasfre
%    executes |\bbl@nonfrenchspacing| (=|\nonfrenchspacing|),
%    eventually overwriting a |\frenchspacing| command issued by
%    main language (German, Spanish, et
%    So we have to define |\bbl@nonfrenchspacing| as |\relax|
%    and restore it's meaning afterwa

% \changes{v2.5a}{2010/08/15}{Define \cs{bbl@nonfrenchspacing} loc
%    as \cs{relax}, otherwise the \cs{bbl@frenchspacing} com
%    included in germanb.ldf is overwritten her
%    \cs{noextrasfrench}.  Bug pointed out by Ulrike Fisch

% \changes{v2.5b}{2010/10/30}{Do not use
%    test \cs{iflanguage}\{french\} to check whether French is
%    main language or not, as it might be be erroneously posi
%    when English is the main language and no hyphenation patt
%    are available for Fre
%    In this case \cs{l@french} and \cs{l@english} ar
%    Pointed out by G\"unter Mil

%    \begin{macroc
   \FBprocess@opt
   \let\bbl@nonfrenchspacingORI\bbl@nonfrenchspa
   \let\bbl@nonfrenchspacing\r
   \ifx\bbl@main@language\FB@fr
     \extrasfrench\captionsfr
   \
     \noextrasfr

   \let\bbl@nonfrenchspacing\bbl@nonfrenchspacin
%    \end{macroc
%    Some warnings are issued when output font encodings are
%    properly set. With XeLaTeX, \file{fontspec.sty}
%    \file{xunicode.sty} should be loaded; with (pdf)\LaTeX, a war
%    is issued when OT1 encoding is in use at the |\begin{documen
%    Mind that |\encodingdefault| is defined as `long', defi
%    |\FBOTone| with |\newcommand*| would f
%    \begin{macroc
   \ifFBX
      \@ifundefined{DeclareUTFcharact
        {\PackageWarning{frenchb.l
          {Add \protect\usepackage{xltxtra} to the\MessageB
           preamble of your documen
         \let\DeclareUTFcharacter\undefin

    \
      \begingroup \newcommand{\FBOTone}{O
      \ifx\encodingdefault\FBO
        \PackageWarning{frenchb.l
           {OT1 encoding should not be used for Fre
            \MessageB
            Add \protect\usepackage[T1]{fontenc} to
            preamble\MessageBreak of your documen

      \endg


%    \end{macroc

%  \subsection{Clean up and e

%    Load |frenchb.cfg| (should do nothing, just for compatibili
%    \begin{macroc
\loadlocalcfg{fren
%    \end{macroc
%    Final clean
%    The macro |\ldf@quit| takes care for setting the main lang
%    to be switched on at |\begin{document}| and resetting
%    category code of \texttt{@} to its original va
%    The config file searched for has to be |frenchb.cfg|,
%    |\CurrentOption| has been set to `french'
%    |\ldf@finish\CurrentOption| cannot be used: we first
%    |frenchb.cfg|, then call |\ldf@quit\CurrentOpti
%    \begin{macroc
\FBclean@on@
\ldf@quit\CurrentOp
%    \end{macroc
% \iff
%</c
%<*
%

%% \CharacterT
%%  {Upper-case    \A\B\C\D\E\F\G\H\I\J\K\L\M\N\O\P\Q\R\S\T\U\V\W\X
%%   Lower-case    \a\b\c\d\e\f\g\h\i\j\k\l\m\n\o\p\q\r\s\t\u\v\w\x
%%   Digits        \0\1\2\3\4\5\6\7
%%   Exclamation   \!     Double quote  \"     Hash (number
%%   Dollar        \$     Percent       \%     Ampersand
%%   Acute accent  \'     Left paren    \(     Right paren
%%   Asterisk      \*     Plus          \+     Comma
%%   Minus         \-     Point         \.     Solidus
%%   Colon         \:     Semicolon     \;     Less than
%%   Equals        \=     Greater than  \>     Question mar
%%   Commercial at \@     Left bracket  \[     Backslash
%%   Right bracket \]     Circumflex    \^     Underscore
%%   Grave accent  \`     Left brace    \{     Vertical bar
%%   Right brace   \}     Tilde

% \iff
%</
%

% \Fi
\endi
