% \iffalse meta-comment
%
% Copyright 2017-2018
% The LaTeX3 Project and any individual authors listed elsewhere
% in this file.
%
% This file is part of the LaTeX base system.
% -------------------------------------------
%
% It may be distributed and/or modified under the
% conditions of the LaTeX Project Public License, either version 1.3c
% of this license or (at your option) any later version.
% The latest version of this license is in
%    http://www.latex-project.org/lppl.txt
% and version 1.3c or later is part of all distributions of LaTeX
% version 2005/12/01 or later.
%
% This file has the LPPL maintenance status "maintained".
%
% The list of all files belonging to the LaTeX base distribution is
% given in the file `manifest.txt'. See also `legal.txt' for additional
% information.
%
% The list of derived (unpacked) files belonging to the distribution
% and covered by LPPL is defined by the unpacking scripts (with
% extension .ins) which are part of the distribution.
%
% \fi
% Filename: ltnews28.tex
%
% This is issue 28 of LaTeX News.

\documentclass{ltnews}
\usepackage[T1]{fontenc}

\usepackage{lmodern,url,hologo}

\publicationmonth{March}
\publicationyear{2018}

\publicationissue{28}

\begin{document}

\maketitle
\tableofcontents

\setlength\rightskip{0pt plus 3em}

\section{New home for \LaTeXe{} sources}

In the past the development version of the \LaTeXe{} source files has
been managed in a Subversion source control system with read access
for the public. This way it was possible to download in an emergency
the latest version even before it was released to CTAN and made its
way into the various distributions.

We have recently changed this setup and now manage the sources using
Git and placed the master sources on GitHub at
\begin{quote}
\url{https://github.com/latex3/latex2e}
\end{quote}
where we already store the sources
for \pkg{expl3} and other work. As before, direct write access is restricted
to \LaTeX{} Project Team members, but everything is publically accessible
including the ability to download, clone (using Git) or checkout
(using SVN). More details are given in~\cite{Mittelbach:TB39-1}.

\section{Bug reports for core \LaTeXe{}}

For more than two decades we used GNATS, an open source bug tracking
system developed by the FSF. While that has served us well in the past
it started to show its age more and more. So as part of this move we
also decided to finally retire the old \LaTeX{} bug database and replace
it with the standard ``Issue Tracker'' available at Github.

The requirements and the workflow for reporting a bug in the core
\LaTeX{} software is documented at
\begin{quote}
\url{https://www.latex-project.org/bugs/}
\end{quote}
and with further details also discussed in~\cite{Mittelbach:TB39-1}.


\section{Integration of \pkg{remreset} and \pkg{chngcntr} packages 
         into the kernel}

With the optional argument to \cs{newcounter} \LaTeX{} offers to
automatically reset counters when some counter is stepped, e.g.,
stepping a \texttt{chapter} counter resets the \texttt{section}
counter (and recursively all other heading counters). However, what
was until now missing was a way to undo such a link between counters
or to link two counters after they have been defined.

This can be now down with \cs{counterwithin} and \cs{counterwithout},
respectively. In the past one had to load the \pkg{chngcntr} package
for this. For the programming level we also added
\cs{@removefromreset} as the counterpart o the already existing
\cs{@addtoreset} command. Up to now this was offered by the
\pkg{remreset} package.



\section{Further TU encoding improvements}

Anything here?

\section{Updates to Babel}

Anything here you want to report Javier?


\section{Changes to packages in the tools category}

\subsection{\LaTeX{} table columns with fixed widths}

Frank published a short paper in in
TUGBoat~\cite{Mittelbach:TB38-2-213} on producing tables that have
columns with fixed widths. The outlined approach using column
specifiers ``\texttt{w}'' and ``\texttt{W}'' has now been integrated
into the \pkg{array} package.

\subsection{Obscure overprinting with \pkg{multicol} fixed}

A rather peculiar bug was reported on StackExchange for
\pkg{multicol}. If the column/page breaking was fully controlled by
the user (through \cs{columnbreak}) instead of letting the environment
do its job and if then more \cs{columnbreak} commands showed up on the
last page then the balancing algorithm was thrown off track.
As a result some parts of the columns did overprint each other.

The fix required a redesign of the output routines used by
\pkg{multicol} and while it ``should'' be transparent in other cases
(and all tests in the regession test suite came out fine) there is the
off-chance that code that hooked into internals of \pkg{multicol}
needs adjustment.

\begin{thebibliography}{9}
  
\bibitem{Mittelbach:TB38-2-213} Frank Mittelbach:
  \emph{\LaTeX{} table columns with fixed widths}.  
  In: TUGBoat, 38\#2, 2017.
  \url{https://www.latex-project.org/publications/}

\bibitem{Mittelbach:TB39-1} Frank Mittelbach:
  \emph{New rules for reporting bugs in the \LaTeX{} core software}.  
  Submitted to TUGBoat.
  \url{https://www.latex-project.org/publications/}

\end{thebibliography}

\end{document}
