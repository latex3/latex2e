% \iffalse meta-comment
%
% Copyright 2015
% The LaTeX3 Project and any individual authors listed elsewhere
% in this file.
%
% This file is part of the LaTeX base system.
% -------------------------------------------
%
% It may be distributed and/or modified under the
% conditions of the LaTeX Project Public License, either version 1.3c
% of this license or (at your option) any later version.
% The latest version of this license is in
%    http://www.latex-project.org/lppl.txt
% and version 1.3c or later is part of all distributions of LaTeX
% version 2005/12/01 or later.
%
% This file has the LPPL maintenance status "maintained".
%
% The list of all files belonging to the LaTeX base distribution is
% given in the file `manifest.txt'. See also `legal.txt' for additional
% information.
%
% The list of derived (unpacked) files belonging to the distribution
% and covered by LPPL is defined by the unpacking scripts (with
% extension .ins) which are part of the distribution.
%
% \fi
% Filename: ltnews24.tex
%
% This is issue 24 of LaTeX News.

\documentclass{ltnews}
\usepackage[T1]{fontenc}

\usepackage{lmodern,url,hologo}

\makeatletter % -- provide command introduced in new release
              %    so this typesets with an old format

\DeclareTextCommandDefault\textcommaabove[1]{%
  \hmode@bgroup
  \ooalign{%
    \hidewidth
    \raise.7ex\hbox{%
      \check@mathfonts\fontsize\ssf@size\z@\math@fontsfalse\selectfont`%
    }%
   \hidewidth\crcr
   \null#1\crcr
  }%
  \egroup
}

\publicationmonth{January}
\publicationyear{2016}

\publicationissue{24}

\begin{document}

\maketitle

\section{\hologo{LuaTeX} support}

This release refines the \hologo{LuaTeX} support introduced in the
2015/10/01 release. A number of patches have been added to improve the
behavior of \package{ltluatex}, and the kernel code has been adjusted to
allow for changes in \hologo{LuaTeX} v0.85--v0.87. Most notably, newer
\hologo{LuaTeX} releases allow more than $16$ write streams and these are now
enabled for use by \verb|\newwrite|.

The changes in \hologo{LuaTeX}~v0.87 mean that the \package{color} and
\package{graphics} packages no longer share the \texttt{pdftex.def}
between \hologo{LuaTeX}and \hologo{pdfTeX}. A separate
\texttt{luatex.def} (distributed separately) has been produced, and
distributions are encouraged to modify \texttt{graphics.cfg} and
\texttt{color.cfg} configuration files to default the \texttt{luatex}
option if \hologo{LuaTeX}~v0.87 or later is being used.

\section{Unicode data}

As noted in \LaTeX{} News~22, the 2015/01/01 release of \LaTeX{} introduced
built-in support for extended \TeX{} systems. In particular, the kernel now
loads appropriate data from the Unicode Consortium to set \verb|\lccode|,
\verb|\uccode|, \verb|\catcode| and \verb|\sfcode| values in an automated
fashion for the entire Unicode range.

The initial approach taken by the team was to incorporate the existing model
used by (plain) \hologo{XeTeX} and to pre-process the ``raw'' Unicode data into
a ready-to-use form as \verb|unicode-letters.def|. However, the relationship
between Unicode Consortium and \TeX{} data structures is non-trivial and still
being explored. As such, it is preferable to directly parse the original
(\verb|.txt|) files at point of use. The team have therefore ``spun-out'' both
the data and the loading to a new generic package, \package{unicode-data}. This
package makes the original Unicode Consortium data files available in the
\verb|texmf| tree (in \verb|tex/generic/unicode-data|) and provides generic
loaders suitable for reading this data into the plain, \LaTeXe{} and other
formats.

At present, the following data files are included in this new package:
\begin{itemize}
  \item \verb|CaseFolding.txt|
  \item \verb|EastAsianWidth.txt|
  \item \verb|LineBreak.txt|
  \item \verb|MathClass.txt|
  \item \verb|SpecialCasing.txt|
  \item \verb|UnicodeData.txt|
\end{itemize}
These files are used either by \LaTeXe{} or by \package{expl3}
(\emph{i.e.}~they represent the set currently required by the team). The
Unicode Consortium provide various other data files and we are happy to add
these to the generic package, as this is intended to provide a single place
to collect this material in the \verb|texmf| tree. Such requests can be
mailed to the team as usual or logged at the package home page:
\url{https://github.com/latex3/unicode-data}.

The new apporach extends use of Unicode data in setting \TeX{} information in
two ways. First, the \verb|\sfcode| of all end-of-quotation/closing punctuation
is now set to $0$ (transparent to \TeX{}). Second, \verb|\Umathcode| values are
now set using \verb|MathClass.txt| rather than setting up only letters (which
was done using an arbitrary plane~$0$/plane~$1$ separation). There are also
minor refinements to the existing code setting, particularly splitting the
concepts of case and letter/non-letter category codes.

For \hologo{XeTeX}, users should note that \verb|\xtxHanGlue| and
\verb|\xtxHanSpace| are \emph{no longer defined}, and no assignments are made
to \verb|\XeTeXinterchartoks|. The values which were previously inherited from
the plain \hologo{XeTeX} set up files are \emph{not} suitable for properly
typestting east Asian text. There are third-party packages addressing this
area well, notably those in the \package{CTeX} bundle.

\section{More Support for East European Accents}

As noted in \LaTeX{} News~23, comma accent support was added for s and
t in the 2015/10/01 release. In this release a matching
\verb|\textcommaabove| accent has been added for U+0123 (\verb|\c{g}|,
\textcommaabove{g}) which is the lower case of U+0122 (\verb|\c{G}|,
\textcommabelow{G}).  In the OT1 and T1 encodings the combinations are
declared as composites with the \verb|\c| command, which matches the
unicode names \texttt{latin (capital|small) letter g with cedilla} and
also allows \verb|\MakeUppercase{\c{g}}| to produce \verb|\c{G}|, as
required.  In T1 encoding. the composite of \verb|\c| with \texttt{k},\texttt{l},
\texttt{n} and \texttt{r} are also
declared to use the comma below accent rather than cedilla to match the
conventional use of these letters.

The UTF-8 \texttt{inputenc} option \texttt{utf8} has been extended to
support all latin combinations that can be reasonably constructed with a
(single) accent command an a base character for the T1 encoding so
\textcommaabove{g}, \k{u} and similar characters may be directly input
using UTF-8 encoding.

\end{document}
