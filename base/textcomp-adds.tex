
% textcomp stuff to be added to the kernel (or tuenc.def)

\makeatletter


% if we don't know what font to use for substituations we default to the ``rm'' family

\def\textcompsubstdefault{\rmsubstdefault}







% in unicode engines make capital accents normal accents


% alternatively we could use TS1 accents if the subset of the current font is known and
% complete (ie 0) but perhaps that overkill

%\def\tc@check@accentii#1{\CheckEncodingSubset
%    \UseTextAccent{TS1}{\expandafter#1\@gobble}}
%
%\DeclareTextCommandDefault{\capitalcedilla}%
%    {\tc@check@accentii\c1\capitalcedilla}


% in unicode engines \oldstylenum will have the default kernel def
% which then can be overwritten by fontspec


%


% maybe that should be kernel error now ...
%\def\tc@errorwarn{\PackageError}
% or maybe not ...


% next macro only works when nfssaxes are also loaded. Otherwise it
% will always select \substdefault







%------------------ sub-enc 1 (drop things that don't work in lmr)


% 1 is TS1 without \cs{textcircled} (but in LM) 




% but \t looks ok to me, not sure why that was 1 in textcomp

% even if the captial accents are in a font they usually aren't really
% we mark only a few font 0 even if they have a full set of glyphs.
% instead we replace them explicitly with T1 normal accents





%------------------ sub-enc 2 




% none of those are arround other than in CM based fonts


% we have a choice here: in T1 there is definition
%  for \cs{textpertenthousand} making the symbol up from \%
%  and \verb=\char 24= (twice) but in many fonts that char doesn't
%  exist and the slot is reused for random ligatures. So better not
%  use it because often it is wrong.  But pointing to TS1 is also not
%  great as only a few fonts have it as a real symbol, so we get a
%  substitution to CM or LM.


%------------------ sub-enc 3 




%------------------ sub-enc 4 




%------------------ sub-enc 5 (most older PS fonts (they produce tofu if a symbol is missing)



%------------------ sub-enc 6


%------------------ sub-enc 7





%------------------ sub-enc 8 (faked euro plus others)





%------------------ sub-enc 9 (really most stuff missing in cochineal and AlgolRevived)


%------------------ always available or so we hope




%------------------ END of defaults (some get overwritten for Unicode engines)


    


% \subsection{Dealing with Unicode engines}






% take these from TS1 still (so probably from LMR)



%

\ifx \Umathcode\@undefined  \else


% this can't be set up if TU isn't loaded in the test suite config
%
% right now 20DD doesn't exist in LM, so maybe not a good idea at the moment
% maybe test for the slot and do the fallback oalign if it doesn't exist
%
  \expandafter\ifx\csname TU-cmd\endcsname\relax
  \else
    \DeclareUnicodeAccent{\textcircled}    \UnicodeEncodingName{"20DD}
  \fi

% we could make them point to dollar and cent glyphs in TU


\fi  % --- END of Unicode engines









% footnote symbols already use \normalfont hardwired (and that should
% probably stay like that).


% maybe this should also be available in Unicode engines. There a
% default is not enough since the symbols are defined in TU so one
% would need to undefine them there or perhaps better redefine them
% there.

\def\UseLegacyTextSymbols{%
  \DeclareTextSymbolDefault{\textasteriskcentered}{OMS}%
  \DeclareTextSymbolDefault{\textbardbl}{OMS}%
  \DeclareTextSymbolDefault{\textbullet}{OMS}%
  \DeclareTextSymbolDefault{\textdaggerdbl}{OMS}%
  \DeclareTextSymbolDefault{\textdagger}{OMS}%
  \DeclareTextSymbolDefault{\textparagraph}{OMS}%
  \DeclareTextSymbolDefault{\textperiodcentered}{OMS}%
  \DeclareTextSymbolDefault{\textsection}{OMS}%
  \UndeclareTextCommand{\textsection}{T1}%
  \expandafter\let\csname oldstylenums \expandafter\endcsname
                  \csname legacyoldstylenums \endcsname
}


% we go the roundabout way via separate OMS declarations so that
%   \renewcommand\textbullet{\textlegacybullet}
% doesn't produce an endless loop

\DeclareTextSymbol{\textlegacyasteriskcentered}{OMS}{3}   % "03
\DeclareTextSymbol{\textlegacybardbl}{OMS}{107}           % "6B
\DeclareTextSymbol{\textlegacybullet}{OMS}{15}            % "0F
\DeclareTextSymbol{\textlegacydaggerdbl}{OMS}{122}        % "7A
\DeclareTextSymbol{\textlegacydagger}{OMS}{121}           % "79
\DeclareTextSymbol{\textlegacyparagraph}{OMS}{123}        % "7B
\DeclareTextSymbol{\textlegacyperiodcentered}{OMS}{1}     % "01
\DeclareTextSymbol{\textlegacysection}{OMS}{120}          % "78

\DeclareTextSymbolDefault{\textlegacyasteriskcentered}{OMS}
\DeclareTextSymbolDefault{\textlegacybardbl}{OMS}
\DeclareTextSymbolDefault{\textlegacybullet}{OMS}
\DeclareTextSymbolDefault{\textlegacydaggerdbl}{OMS}
\DeclareTextSymbolDefault{\textlegacydagger}{OMS}
\DeclareTextSymbolDefault{\textlegacyparagraph}{OMS}
\DeclareTextSymbolDefault{\textlegacyperiodcentered}{OMS}
\DeclareTextSymbolDefault{\textlegacysection}{OMS}


% kernel leftover .. this should be like this these days ... not faked

\DeclareTextSymbolDefault{\textcompwordmark}{T1}



\makeatother    

\endinput

