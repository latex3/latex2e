



%    By the way, fontspec uses \cs{itscdefault} etc.\ whereas in the
%    \texttt{.fd} files it is always called \texttt{scit}. I
%    personally kind of think ``\texttt{itsc}'' reads better than
%    ``\texttt{scit}'' but with more than 700 fonts (in T1 encoding)
%    having \texttt{scit} and none the other I think the name is now
%    given.
%



% The \textsf{fontaxes} package also implements two further shapes,
% namely \cs{swshape} and \cs{sscshape}. They could now easily be
% integrated by specifying a few further table entries such as
%    \begin{macrocode}
%\DeclareFontShapeChangeRule {n}{sw}   {sw}     {n}
%\DeclareFontShapeChangeRule {it}{sw}  {sw}     {it}
%\DeclareFontShapeChangeRule {sc}{sw}  {scsw}   {sc}
%   ...
%    \end{macrocode}
% For \cs{swshape} I have already done that above (even though there are
% currently only a few fonts
% that support swash letters). For  ``spaced small capitals'' there a
% none among the free fonts. Nevertheless, perhaps that should also be integrated
% in the kernel so that the existing \textsf{fontaxes} support continues to
% work in full.
%
%    
%    
%
% \subsubsection{Packages that need checking or updating}
%
% These package use \cs{fontprimaryshape}
%\begin{verbatim}
%./baskervaldx/Baskervaldx.sty
%./baskervillef/baskervillef.sty
%./ebgaramond/ebgaramond.sty
%./fontaxes/fontaxes.sty
%\end{verbatim}
%
%
% These package use \cs{fontsecondaryshape}
%\begin{verbatim}
%./baskervaldx/Baskervaldx.sty
%./inriafonts/InriaSans.sty
%./inriafonts/InriaSerif.sty
%./baskervillef/baskervillef.sty
%./ebgaramond/ebgaramond.sty
%./fontaxes/fontaxes.sty
%\end{verbatim}




% ^^A Possible issue with tudscr.cls -- needs checking
