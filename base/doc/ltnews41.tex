% \iffalse meta-comment
%
% Copyright 2024
% The LaTeX Project and any individual authors listed elsewhere
% in this file.
%
% This file is part of the LaTeX base system.
% -——————————————
%
% It may be distributed and/or modified under the
% conditions of the LaTeX Project Public License, either version 1.3c
% of this license or (at your option) any later version.
% The latest version of this license is in
%    https://www.latex-project.org/lppl.txt
% and version 1.3c or later is part of all distributions of LaTeX
% version 2008 or later.
%
% This file has the LPPL maintenance status "maintained".
%
% The list of all files belonging to the LaTeX base distribution is
% given in the file `manifest.txt'. See also `legal.txt' for additional
% information.
%
% The list of derived (unpacked) files belonging to the distribution
% and covered by LPPL is defined by the unpacking scripts (with
% extension .ins) which are part of the distribution.
%
% \fi
% Filename: ltnews41.tex
%
% This is issue 41 of LaTeX News.

\NeedsTeXFormat{LaTeX2e}[2020-02-02]

\documentclass{ltnews}

%% Maybe needed only for Chris' inadequate system:
\providecommand\Dash {\unskip \textemdash}

%% NOTE: Chris' preferred hyphens!
%% \showhyphens{parameters}
%% \hyphenation{because}

\usepackage[T1]{fontenc}

\usepackage{lmodern,url,hologo}

\usepackage{csquotes}
\usepackage{multicol}
\usepackage{color}

\providecommand\hook[1]{\texttt{#1}}

\providecommand\meta[1]{$\langle$\textrm{\itshape#1}$\rangle$}
\providecommand\option[1]{\texttt{#1}}
\providecommand\env[1]{\texttt{#1}}
\providecommand\Arg[1]{\texttt\{\meta{#1}\texttt\}}


\providecommand\eTeX{\hologo{eTeX}}
\providecommand\XeTeX{\hologo{XeTeX}}
\providecommand\LuaTeX{\hologo{LuaTeX}}
\providecommand\pdfTeX{\hologo{pdfTeX}}
\providecommand\MiKTeX{\hologo{MiKTeX}}
\providecommand\CTAN{\textsc{ctan}}
\providecommand\TL{\TeX\,Live}


\providecommand\githubissue[2][]{\ifhmode\unskip\fi
     \quad\penalty500\strut\nobreak\hfill
     \mbox{\small\slshape(%
       \href{https://github.com/latex3/latex2e/issues/\getfirstgithubissue#2 \relax}%
          	    {github issue#1 #2}%
           )}%
     \par\smallskip}

% simple solution right now (just link to the first issue if there are more)
\def\getfirstgithubissue#1 #2\relax{#1}


\providecommand\taggingissue[2][]{\ifhmode\unskip\fi
     \quad\penalty500\strut\nobreak\hfill
     \mbox{\small\slshape(%
       \href{https://github.com/latex3/tagging-project/issues/\getfirstgithubissue#2 \relax}%
          	    {tagging-project issue#1 #2}%
           )}%
     \par\smallskip}

\providecommand\sxissue[1]{\ifhmode\unskip
     \else
       % githubissue preceding
       \vskip-\smallskipamount
       \vskip-\parskip
     \fi
     \quad\penalty500\strut\nobreak\hfill
     \mbox{\small\slshape(\url{https://tex.stackexchange.com/#1})}\par}

\providecommand\gnatsissue[2]{\ifhmode\unskip\fi
     \quad\penalty500\strut\nobreak\hfill
     \mbox{\small\slshape(%
       \href{https://www.latex-project.org/cgi-bin/ltxbugs2html?pr=#1\%2F\getfirstgithubissue#2 \relax}%
          	    {gnats issue #1/#2}%
           )}%
     \par}

\let\cls\pkg
\providecommand\env[1]{\texttt{#1}}
\providecommand\acro[1]{\textsc{#1}}

\vbadness=1400  % accept slightly empty columns


\let\finalpagebreak\pagebreak % for TUB (if they use it)
\let\finalvspace\vspace       % for document layout fixes

\makeatletter
% maybe not the greatest design but normally we wouldn't have subsubsections
\renewcommand{\subsubsection}{%
   \@startsection {subsubsection}{2}{0pt}{1.5ex \@plus 1ex \@minus .2ex}%
                  {-1em}{\@subheadingfont\colonize}%
}
\providecommand\colonize[1]{#1:}
\makeatother


% Undo ltnews's \verbatim@font with active < and >
\makeatletter
\def\verbatim@font{\normalsize\ttfamily}
\makeatother


%%%%%%%%%%%%%%%%%%%%%%%%%%%%%%%%%%%%%%%%%%%%%%%%%%%%%%%%%%%%%%%%%%%%%%%%%%%%%
\providecommand\tubcommand[1]{}
\tubcommand{\input{tubltmac}}

\publicationmonth{June}
\publicationyear{2025  --- DRAFT version for upcoming release}

\publicationissue{41}

\begin{document}

\maketitle
{\hyphenpenalty=10000 \exhyphenpenalty=10000 \spaceskip=3.33pt \hbadness=10000
\tableofcontents}

\setlength\rightskip{0pt plus 3em}

\medskip

\section{Introduction}

\emph{to write}

\section{Replacement for the legacy mark mechanism}



\LaTeX{}'s legacy mechanism only supported two classes (left and right
marks) and setting the left mark (with \cs{markboth}) always altered
the state of the right mark as well, i.e., they were far from
independent. For generating running headers with \enquote{chapter
  titles} on the left and \enquote{section titles} on the right they
work reasonably well but without much flexibility, e.g., \cs{leftmark}
always generated the first \enquote{left}-mark on the page, while
\cs{rightmark} always generated the last \enquote{right}-mark.

A few releases ago~\cite{41:ltnews35} we therefore introduced a new
mark mechanism for \LaTeX{} that supports arbitrary many truly
independent mark classes and also offers querying the state at the top
of the page, something that wasn't available in \LaTeX{} at all.

Up to now, both mechanisms coexisted with completely separate
implementations. With this release we have retired the legacy code and
instead implement its public interfaces by using the new concepts,
i.e., \cs{markboth}, \cs{markright}, \cs{leftmark}, and \cs{rightmark}
remain supported but internally use \cs{InsertMark}, etc.  Existing
document classes or documents using the interfaces will therefore
continue to work without any modifications but use a single underlying
implementation and new documents can benefit from the additional
flexibility, e.g., by displaying not only the last right-mark
(\cs{leftmark} or \verb=\LastMark{2e-right}=) but also the first
right-mark (\verb=\FirstMark{2e-right}=) or the top right-mark
(\verb=\TopMark{2e-right}=), etc.

See~\cite{41:ltmarks} for details on the extended functionality.



\section{News from Tagged PDF project}

\emph{to write}

\subsection{Fixing the spacing after display math}

When \LaTeX{} produces an accessible (tagged) PDF it has to add
structure data into the PDF to mark (i.e., tag) individual
elements. If the \pdfTeX{} engine is used this has to be done with the
help of \cs{pdfliteral}s which are whatsit nodes like
\cs{special} or \cs{write}. This means that they should be added only in
places, where these extra nodes are not affecting the spacing\Dash \TeX{}
can't, for example, look backwards past such a whatsit node so
consecutive spaces that are normally collapsed into one, suddenly
appear both, if such a node separates them.

The situation is especially complicated with math displays, because
there \TeX{} adds penalties and spaces with low-level procedures, that
are not directly accessible from the macro level, and the tagging
structures have to appear somewhere in the middle of that to ensure
that the formula and the PDF structures are not separated by page
break. Because of this it is necessary to use some fairly complex
methods (essentially disable \TeX's mechanisms and reprogram them on
the macro level) to get the structure data in the right places.

Our first attempt in doing that was slightly faulty and resulted in
some cases in an extra space (an additional \cs{parskip} space when
there shouldn't be one). This has now corrected and the gymnastics to
achieve this are rather an \enquote{interesting} study in obfuscated
\TeX{} coding.

In \LuaTeX{} the situation is much better because there the structures can
be added later when the formula processing has already finished.
%
\taggingissue{762}



\section{New or improved commands}

\section{Code improvements}

\subsection{Refinement of \cs{MakeTitlecase}}

We introduced \cs{MakeTitlecase} as a late addition to the June
2022 release, making use of the improved case code in
\pkg{expl3}. Unlike upper and lowercasing, making text
titlecased is more tricky to get right: this can apply either to
the whole text or on a word-by-word basis.

A subtle issue was reported against the \pkg{expl3} repository
(\url{https://github.com/latex3/latex3/issues/1316}) which links
to how we deal with the question of case changing
\enquote{words} but shows up if you titlecase text stored in a
command.

We have looked again at how to implement \cs{MakeTitlecase} to
be as predictable as possible, and have made a change in this
release. The command no longer tries to lowercase text before
applying titlecasing, and gives the correct result for text
stored in commands.

We have also added an additional key to the optional argument to
\cs{MakeTitlecase} which allows the user to decide if this will
apply only to the first word (the default) or to all words.

\subsection{Tab character as a special}

In \LaTeX{} News~38, we described the extension of \cs{verb}, etc., to cover
the tab character as equivalent to a space. We have now added tabs to the
standard list of characters covered by \cs{dospecials}. This allows them to
be used in for example a \texttt{v}~specification document command without
additional steps.

\subsection{Logging text command and symbol declarations}

For thirty years the documentation claimed that \cs{DeclareTextSymbol},
\cs{DeclareTextCommand}, and friends log their changes, but in
contrast to their math counterparts they never did. This has now
finally changed.
%
\githubissue{1242}

\subsection{Supporting the \texttt{ssc} and \texttt{sw} shapes}

The \texttt{ssc} shape (spaced small capitals) is supported in
\LaTeX{} through the commands \cs{sscshape} and \cs{textssc}. However,
until this release there where no font shape change rules defined for
this admittely seldom available shape, so that
\begin{verbatim}
  \sscshape\itshape
\end{verbatim}
changed unconditionally to \texttt{it} (italics) rather than to
\texttt{sscit} (spaced small italic capitals).  Thanks to Michael
Ummels the missing declarations have now been added so that shape
changes in font families that support spaced small capitals work
properly.

At the same time we took the opportunity to improve the fallbacks for
the \texttt{sw} (swash) shapes, which are accessible through the
commands \cs{swshape} or \cs{textsw}. If an \texttt{sw} combination is
not available, the rules now try to replace \texttt{sw} with
\texttt{it} rather than falling back to \texttt{n}.
%
\githubissue{1581}


\section{Bug fixes}

\subsection{Fix the use of  \texttt{localmathalpabets}}

In 2021 we introduced a method to overcome the problem that classic
\TeX{} engines (but not the Unicode engines) have only a limited
number of math alphabets available that got easily fill up simply by
loading math font packages, even if their symbols got used only
occasionally. The idea was to avoid allocating all math alphabets
globally, but instead allow a number of them (defined by counter
\texttt{localmathalpabets}) to vary from one formula to the next. This
way different formulas can make use of different alphabets and chances
are much higher that the processing a complex the document succeeds.
See~\cite{41:ltnews34} for details.

Unfortunately, the approach taken failed in some cases of nested
formulas with the result that the wrong symbol glyphs were used.
This has now been corrected.
%
\githubissue[s]{1101 1028}


%\section{Changes to packages in the \pkg{amsmath} category}

%\section{Changes to packages in the \pkg{graphics} category}

\section{Changes to packages in the \pkg{tools} category}

\subsection{\pkg{multicol}:\ Full support for extended marks}

In 2022 we introduced a new mark mechanism for
\LaTeX{}~\cite{41:ltnews35}. However, the initial implementation only
covered the standard output routine of \LaTeX{}. As a result the
extended marks were not available within columns produced with the
\pkg{multicol} package (where they would be especially useful). This
limitation has finally been lifted and the new mechanism is now fully
supported.
%
\githubissue{1421}



%\section{Changes to files in the \pkg{cyrillic} category}

\begin{thebibliography}{9}\frenchspacing

%\fontsize{9.3}{11.3}\selectfont

\bibitem{41:Lamport}
Leslie Lamport.
\newblock \emph{{\LaTeX}: {A} Document Preparation System: User's Guide and Reference
  Manual}.
\newblock \mbox{Addison}-Wesley, Reading, MA, USA, 2nd edition, 1994.
\newblock ISBN 0-201-52983-1.
\newblock Reprinted with corrections in 1996.

\bibitem{41:ltnews} \LaTeX{} Project Team.
  \emph{\LaTeXe{} news 1--41}. June, 2025.
  \url{https://latex-project.org/news/latex2e-news/ltnews.pdf}

\bibitem{41:ltnews34} \LaTeX{} Project Team.
  \emph{\LaTeXe{} news 34}. November 2021.
  \url{https://latex-project.org/news/latex2e-news/ltnews34.pdf}

\bibitem{41:ltnews35} \LaTeX{} Project Team.
  \emph{\LaTeXe{} news 35}. June 2022.
  \url{https://latex-project.org/news/latex2e-news/ltnews35.pdf}

\bibitem{41:ltnews40} \LaTeX{} Project Team.
  \emph{\LaTeXe{} news 40}. November 2024.
  \url{https://latex-project.org/news/latex2e-news/ltnews40.pdf}

\bibitem{41:ltmarks} Frank Mittelbach, \LaTeX{} Project Team.
  \emph{The \texttt{ltmarks.dtx} code}. June 2025.
  \url{https://latex-project.org/help/documentation/ltmarks-doc.pdf}

\end{thebibliography}

\end{document}
