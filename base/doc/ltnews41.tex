% \iffalse meta-comment
%
% Copyright 2024-2025
% The LaTeX Project and any individual authors listed elsewhere
% in this file.
%
% This file is part of the LaTeX base system.
% -——————————————
%
% It may be distributed and/or modified under the
% conditions of the LaTeX Project Public License, either version 1.3c
% of this license or (at your option) any later version.
% The latest version of this license is in
%    https://www.latex-project.org/lppl.txt
% and version 1.3c or later is part of all distributions of LaTeX
% version 2008 or later.
%
% This file has the LPPL maintenance status "maintained".
%
% The list of all files belonging to the LaTeX base distribution is
% given in the file `manifest.txt'. See also `legal.txt' for additional
% information.
%
% The list of derived (unpacked) files belonging to the distribution
% and covered by LPPL is defined by the unpacking scripts (with
% extension .ins) which are part of the distribution.
%
% \fi
% Filename: ltnews41.tex
%
% This is issue 41 of LaTeX News.

\NeedsTeXFormat{LaTeX2e}[2020-02-02]

\documentclass{ltnews}

%% Maybe needed only for Chris' inadequate system:
\providecommand\Dash {\unskip \textemdash}

%% NOTE: Chris' preferred hyphens!
%% \hyphenation{because}

\usepackage[T1]{fontenc}

\usepackage{lmodern,url,hologo}

\usepackage{csquotes}
\usepackage{multicol}
\usepackage{color}

\providecommand\hook[1]{\texttt{#1}}
\providecommand\socket[1]{\texttt{#1}}
\providecommand\plug[1]{\texttt{#1}}

\providecommand\meta[1]{$\langle$\textrm{\itshape#1}$\rangle$}
\providecommand\option[1]{\texttt{#1}}
\providecommand\env[1]{\texttt{#1}}
\providecommand\Arg[1]{\texttt\{\meta{#1}\texttt\}}


\providecommand\eTeX{\hologo{eTeX}}
\providecommand\XeTeX{\hologo{XeTeX}}
\providecommand\LuaTeX{\hologo{LuaTeX}}
\providecommand\pdfTeX{\hologo{pdfTeX}}
\providecommand\MiKTeX{\hologo{MiKTeX}}
\providecommand\CTAN{\textsc{ctan}}
\providecommand\TL{\TeX\,Live}


\providecommand\githubissue[2][]{\ifhmode\unskip\fi
     \quad\penalty500\strut\nobreak\hfill
     \mbox{\small\slshape(%
       \href{https://github.com/latex3/latex2e/issues/\getfirstgithubissue#2 \relax}%
          	    {github issue#1 #2}%
           )}%
     \par\smallskip}

% simple solution right now (just link to the first issue if there are more)
\def\getfirstgithubissue#1 #2\relax{#1}


\providecommand\taggingissue[2][]{\ifhmode\unskip\fi
     \quad\penalty500\strut\nobreak\hfill
     \mbox{\small\slshape(%
       \href{https://github.com/latex3/tagging-project/issues/\getfirstgithubissue#2 \relax}%
          	    {tagging-project issue#1 #2}%
           )}%
     \par\smallskip}
    

\newenvironment{old}{\par\itshape}{\par}  % to sort out rewrites


\providecommand\sxissue[1]{\ifhmode\unskip
     \else
       % githubissue preceding
       \vskip-\smallskipamount
       \vskip-\parskip
     \fi
     \quad\penalty500\strut\nobreak\hfill
     \mbox{\small\slshape(\url{https://tex.stackexchange.com/#1})}\par}

\providecommand\gnatsissue[2]{\ifhmode\unskip\fi
     \quad\penalty500\strut\nobreak\hfill
     \mbox{\small\slshape(%
       \href{https://www.latex-project.org/cgi-bin/ltxbugs2html?pr=#1\%2F\getfirstgithubissue#2 \relax}%
          	    {gnats issue #1/#2}%
           )}%
     \par}

\let\cls\pkg
\providecommand\env[1]{\texttt{#1}}
\providecommand\acro[1]{\textsc{#1}}

\vbadness=1400  % accept slightly empty columns

\let\finalpagebreak\pagebreak % for TUB (if they use it)
\let\finalvspace\vspace       % for document layout fixes

\makeatletter
% maybe not the greatest design but normally we wouldn't have subsubsections
\renewcommand{\subsubsection}{%
   \@startsection {subsubsection}{2}{0pt}{1.5ex \@plus 1ex \@minus .2ex}%
                  {-1em}{\@subheadingfont\colonize}%
}
\providecommand\colonize[1]{#1:}
\makeatother


% Undo ltnews's \verbatim@font with active < and >
\makeatletter
\def\verbatim@font{\normalsize\ttfamily}
\makeatother


%%%%%%%%%%%%%%%%%%%%%%%%%%%%%%%%%%%%%%%%%%%%%%%%%%%%%%%%%%%%%%%%%%%%%%%%%%%%%
\providecommand\tubcommand[1]{}
\tubcommand{\input{tubltmac}}

\publicationmonth{June}
\publicationyear{2025  --- DRAFT version for upcoming release}

\publicationissue{41}

\begin{document}

\maketitle
{\hyphenpenalty=10000 \exhyphenpenalty=10000 \spaceskip=3.33pt \hbadness=10000
\tableofcontents}

\setlength\rightskip{0pt plus 3em}

\medskip

\section{Introduction}

\begin{old}
Work continues both directly in supporting tagged output and in the wider
kernel development needed to make this happen. Probably the most notable
changes in the latest release of the kernel are those in the output routine
and mark mechanism: these are described in the following two sections.

Work continues apace on the tagging project, with new sockets, better
graphics tagging, improved math mode support and promoting PDF~2.0
all highlights.

Beyond tagging-focussed work, we have a new argument type for
\pkg{ltcmd} (\pkg{xparse}) which will make working with verbatim
content a \emph{lot} easier, work on case changing, and of
course bug fixes.

Finally, we have highlighted a few changes in the L3 programming layer:
development work there continues in parallel with improvements to the \LaTeX{}
kernel, and we are integrating more previously-\enquote{experimental} ideas
into the core programming system.
\end{old}


We are continuing to work on the direct
%% direct??
support of tagged PDF output and on the wider
kernel development, much of which is needed to make this happen. 
%% ??
Probably the most notable
changes in this latest release of the kernel are those in the output routine
and the mark mechanism: these are described in the first two sections.

Work also continues apace on further aspects of the tagging project, 
where some highlights are: new sockets, better
graphics tagging, improved math mode support and the promotion of PDF~2.0.

In addition to this tagging-focussed work, we have also made advances in these areas:
 a new argument type for use in \cs{NewDocumentEnvironment} and friends
which will make working with verbatim
content a \emph{lot} easier; further improvements to case changing; and, of
course,
%many
several
bug fixes.

Finally, 
%% where exactly is this: why "finally"?
we have highlighted a few of the recent changes in the L3 programming layer, where 
development work continues in parallel with other improvements to the
\LaTeX{} kernel and we are integrating more \enquote{experimental} ideas
into this core programming system.


%% Order of these ne two sections?
%% Better to have the very long one first! 

\section{A configurable output routine}

For nearly 40 years \LaTeX's output routine (the mechanism to paginate
the document and attach footnotes, floats and headers \& footers) was
a largely hardwired algorithm with a limited number of configuration
possibilities.  Packages that attempted to alter any 
aspect of this process had to overwrite the internals, which led to the usual
problems: incompatibilities and out of date code, whenever something
was changed in \LaTeX{}.

To improve this, and to support the production of accessible
%% Need to explain what is meant by "accessible" in this context! 
%%  Maybe relate it to the PDF UA standards? 
PDF documents, we have started to refactor the output routine by adding a
number of hooks and sockets.  This means that  packages needing to adjust the
output routine can do so safely, avoiding the dangers previously associated 
with such activities. 


For packages that need to hook into the output routine we implemented the following hooks:
\begin{description}

\item[\hook{build/page/before}, \hook{build/page/after}]

   These two hooks enable packages to prepend or append code to the
   processing of each page in the output routine. They are implemented
   as mirrored hooks.  Technically, they are executed at the start and
   the end, respectively, of the internal \LaTeXe{} \cs{@outputpage}
   command.  Currently, a number of packages change this command by
   adding code in exactly these two places\Dash so they can now
   instead simply add code to these hooks.

 \item[\hook{build/page/reset}]

   Packages that set up special conventions for the main text (such as
   catcode changes, etc.)\ can use this hook to undo these changes
   within the output routine, so that they aren't applied to unrelated
   material such as the text for running headers or footers.

\item[\hook{build/column/before}, \hook{build/column/after}]

  These two hooks enable packages to prepend or append code to the
  column processing in the output routine. They are implemented as
  mirrored hooks.  Technically, they are executed at the start and the
  end, respectively, of the internal \LaTeXe{} \cs{@makecol} command.
  A number of packages alter \cs{@makecol} to place code in exactly
  these two places\Dash they can now instead simply add their code to
  these hooks.
  
\end{description}

We have also added a number of sockets: for configuring the algorithm and also to
support tagging. 
Two of these sockets are of interest for use in
class files and also in the document preamble. 

The first and most complex of these is the socket \socket{build/column/outputbox}, 
which controls how the
column text, the column floats (top and bottom) and the footnotes are
combined in a column: i.e. their order and spacing. 

Thus in order to change the layout, all one now 
has to do is to assign a suitable
plug to this socket, like this:
\begin{flushleft}
  \verb= \AssignSocketPlug{build/column/outputbox}=
  \verb=                  {=\meta{plug-name}\verb=}=
\end{flushleft}
%
For this socket we have provided the following plugs:
\begin{description}
\item[\plug{space-footnotes-floats}]

   After the galley text there is a vertical \cs{vfil}
   followed by the footnotes, followed by the bottom floats, if any.

\item[\plug{footnotes-space-floats}]

   As before but the \cs{vfil} is between the footnotes and the floats.

\item[\plug{floats-space-footnotes}]

   The floats come directly after the text, followed by a \cs{vfil} and then the footnotes
   at the bottom.

\item[\plug{space-floats-footnotes}]

   Both floats and footnotes are pushed to the bottom, the footnotes
   coming last.\footnote{There are two more permutations but neither
   of them has ever been requested; so they aren't set up by default
   --- doing that in a class would be trivial though.}

\item[\plug{floats-footnotes}]

   All excess space is distributed across the existing
   glue on the page, e.g., within the text galley, the
   separation between blocks, etc.
   The order is text, floats, footnotes.

\item[\plug{footnotes-floats}]

   Like the previous plug, but floats and footnotes are swapped. This is
   the \LaTeX{} default for newer documents, i.e., this plug is
   assigned to the socket when \cs{DocumentMetadata} is used.

\item[\plug{footnotes-floats-legacy}]

   Like previous plug, but \LaTeX{}'s bottom skip bug is not
   corrected, i.e., in ragged bottom designs where footnotes
   are supposed to be directly attached to the text, they suddenly
   appear at the bottom of the page when the page is ended with
   \cs{newpage} or \cs{clearpage}.
   While this is clearly a bug, it has been the case since the days
   of \LaTeX~2.09; thus, for
   compatibility, we continue to support this behavior.
\end{description}

By default the separation between the last line of text and the
footnotes (\cs{skip}\cs{footins}) is not measured from the baseline of
the last text line, but from its bottom. This goes back to plain \TeX{}
where it is done in this way.  Similarly, \cs{textfloatsep} is
added between text and bottom floats not starting from the baseline of
the last text line. Typographically speaking this is suboptimal
because it means that with \cs{flushbottom} in effect, the position of
the last text line, when it is followed by footnotes or floats, depends on
whether or not that line contains characters with descenders.

For this reason there is now also a socket
\socket{build/column/baselineattach} with a plug
\plug{on}: this causes the attachment of footnotes/floats to be measured
from the baseline of the last text line. To mimic the behavior of old
documents, this socket is, by default, assigned the plug \plug{off}. 
For documents using
\cs{DocumentMetadata}, the plug \plug{on} will probably become the default here. 

There are more configuration possibilities, mainly for class
developers to use: documentation of these can be found in
\cite[\S54 ltoutput.dtx]{41:source2e}.


\section{Replacement for the legacy mark mechanism}

\LaTeX{}'s legacy mechanism only supported two classes of marks, left and right
marks, and setting the left mark (with \cs{markboth}) always altered
the state of the right mark as well, i.e., they were far from
independent. For generating running headers with \enquote{chapter
  titles} on the left and \enquote{section titles} on the right, they
work reasonably well but without much flexibility: e.g., \cs{leftmark}
always generated the first \enquote{left}-mark on the page, while
\cs{rightmark} always generated the last \enquote{right}-mark.

A few releases ago~\cite{41:ltnews35} we therefore introduced a new
mark mechanism for \LaTeX{}, one that supports any number of truly
independent mark classes. This mechanism also offers the ability to
query the mark status at the top of the page, something that wasn't
previously available in at all.

Up to now these two mechanisms coexisted, with completely separate
implementations; but we have now retired the legacy code and
implemented the same public interfaces using the new concepts.  Thus
the old commands (\cs{markboth}, \cs{markright}, \cs{leftmark} and
\cs{rightmark}) remain supported, but internally these commands all
use \cs{InsertMark}, etc.

Existing document classes, and documents using the legacy interfaces,
will therefore continue to work without any modifications; but they
now use a single underlying implementation. Also, new documents can
benefit from the additional flexibility, e.g., by being able to
display not only the last right-mark (\cs{leftmark} or
\verb=\LastMark{2e-right}=) but also or alternatively the first
right-mark (\verb=\FirstMark{2e-right}=) or the top right-mark
(\verb=\TopMark{2e-right}=), etc.

More information concerning all of this extended functionality can be
found in~\cite{41:ltmarks}.


\section{News from the Tagged PDF project}

In the Tagged PDF project we have now reached a state where, within
certain limits, it is possible to generate accessible PDF output that
conforms to PDF/UA for arbitrarily complex documents as long as they
only use (a growing number of) compatible packages and classes.

The focus for this release was on improving the tagging of math
formulas and on extending the tagging support for various kinds of
graphics.

\subsection{Sockets for tagging support}

A lot of the tagging support in packages is handled through the socket-and-plug 
mechanism that was introduced in \LaTeX{}
2023-11-01~\cite{41:ltnews38}.
%
Sockets offer an easy to use interface
for package developers to invoke variable code at pre-specified places that
can be changed from the outside by assigning a different plug, for example, to alter the processing.

%FMi: think this is a wrong rewrite 
%Sockets and plugs offer an interface
%(that is very easy to use)
%for package developers to invoke variable code at pre-specified places --- 
%by simply \enquote{changing the plug}. 

For the tagging support, a specialised set of sockets is available: 
their plugs are executed
by using the  \cs{UseTaggingSocket} command, instead of the normal
\cs{UseSocket} command. This allows tagging to be turned off or on at high
speed by the commands \cs{SuspendTagging} and \cs{ResumeTagging}, without the need
to individually reassign plugs to each of the many tagging
sockets~\cite{41:ltnews39}. This is very useful 
when there is a need to typeset material several times during trials. 

In the current release we now also offer three dedicated declaration
commands for these \enquote{tagging sockets}: this works better than 
directly using the underlying general socket interface. 
These new commands also better support the
special conventions used for \enquote{tagging sockets}. They are:  
\cs{NewTaggingSocket}, \cs{NewTaggingSocketPlug} and
\cs{AssignTaggingSocketPlug}.


\subsection{Setting the version to PDF 2.0}

Creating a PDF 2.0 version is considered essential for any document that has 
substantial mathematical content.  This is because only this PDF version supports 
the straightforward use of tags from the 
MathML namespace.
%
In earlier PDF versions, formulas can only be made accessible by
describing their content in an \texttt{alt} attribute.\marginpar{added}
%

When \cs{DocumentMetadata} is used, \LaTeX{}
will therefore, by default, set PDF~2.0 as the PDF version.
A different PDF version can, if required, be set by explicit use of 
the \texttt{pdfversion} key.


\subsection{Extended support for pictures}

\emph{todo: write more about the user options}\marginpar{Ulrike? still something to write?}

The tagging of graphics has been reimplemented and now uses tagging
sockets (see above).  The options have been extended to allow document
authors to choose between four tagging flavors on a a per-graphic
basis: as illustrative figures, as artifacts (decorations), as
replacement for symbols, as normal text (for example todo notes).
%% 
The code also supports graphics produced using the \pkg{tikz} packages
and \enquote{todo notes} from the \pkg{todonotes} package.  The
extended documentation in \texttt{latex-lab-graphics.pdf} describes
what authors of other graphic packages can do to to make their
packages tagging aware.


\subsection{New Metadata keys, to activate tagging}

Up to now users had to activate tagging by loading modules from
\texttt{latex-lab} with the help of the \texttt{testphase}
key. Further configuration of the tagging then had to be done with the
\cs{tagpdfsetup} command.  We now offer Metadata keys for this 
that do not use
\enquote{test} in their names, reflecting the fact that producing
tagged PDF documents has become \enquote{production ready}.%
%%  what does "production ready" mean ??  FMi: ready for production usage
\footnote{To be fully precise, this is true provided
only compatible\\
packages and classes are used: and these are listed at\\
\url{https://latex3.github.io/tagging-project/tagging-status/}.}

The \texttt{tagging} key allows the activation and deactivation of the tagging
support.  It accepts the three values \texttt{on}, \texttt{off} and
\texttt{draft}.  When this key is used it loads the
\pkg{tagpdf} package and all the modules from the
\texttt{testphase=latest} set.  Setting \texttt{tagging=off} deactivates the
tagging commands in the 
\texttt{class/before} hook; and \texttt{tagging=draft} leaves the 
tagging commands active, to 
preserve warnings and errors in the tagging, but it deactivates
the writing of the structure tree at the end of the compilation. This
can save time when drafting a long document.

The \texttt{tagging-setup} key allows configuration of the tagging. It
accepts as values all the keys that can be used in \cs{tagpdfsetup},
such as the \texttt{math/setup} key described below. It knows about both the
key \texttt{modules}, which allows overwriting of the set of loaded
modules, and the key \texttt{extra-modules}, which allows loading of
experimental modules that are not yet in \texttt{latest}. 
%
The \texttt{tagging-setup} key implies \texttt{tagging=on} so that, if
this key is used then it is not necessary to also set the \texttt{tagging}
key unless you want to turn tagging \texttt{off}, or to set it to
\texttt{draft}.

With these new Metadata keys a standard setup might look like this:
\begin{verbatim}
\DocumentMetadata{
   pdfstandard={UA-2,A-4f},
   tagging=on,
   tagging-setup=
      {math/setup=mathml-SE,
       extra-modules=verbatim-alt}
 }      
\end{verbatim}


\subsection{New value \texttt{latest} for \texttt{testphase} key}

With the new keys for enabling tagging the use of the
\texttt{testphase} key is now of minor importance and mainly of
interest for developers and for backwards compatibility.

With this release it also supports the value \texttt{latest}.  This
will load all modules that we current recommend to load so that it is not necessary to
specify a long list of individual modules. The list of loaded modules
will be adjusted as needed when the project progresses. For reference,
it is also written to the log.



\subsection{Setting up math tagging}

With the \LuaTeX{} engine there are now various options for the production of
accessible math 
%% This definitely needs some defining explanations !! 
%% Here!! It probably means something like this:  
%%  "containing information for use by screen readers"
%%FMi I don't hink this should have changes
which are described in full detail in \texttt{latex-lab-math.pdf}. To
simplify the setup 
%% setup of what??
a new key \texttt{math/setup} can be used in
%the document preamble
\cs{tagpdfsetup} (or in \texttt{tagging-setup} as shown above)
that accepts a comma list with the values
\texttt{mathml-SE} (add MathML structure element), \texttt{mathml-AF}
(attach MathML associated file) or \texttt{tex-AF} (attach the \TeX{}
source).
%% I got lost here! What is "a key in the preamble"
%% Should this be about Metadata keys? 
%%  And what is the effect of specifying these key values?
%FMi slightly change above


\subsection{The use of \texttt{\$\$...\$\$} for math displays}

Use of the plain \TeX{} method \verb=$$...$$= (in \LaTeX{}), to mark up a
display math formula, is not officially supported because it produces a
%% officially supported meaning what?  FMi: meaning document in Lamport or TLC or usrguide ...
fixed visual result that it not receptive to style changes such as the
\texttt{fleqn} option. Instead, the recommended way is to use
\verb=\[ ... \]= or the \env{displaymath} environment. However, since many
authors % FMi added
have used this input method in their documents, we are doing our best to
support it for the production of accessible PDFs; but users should be
aware that it has some limitations.
%% It is not made very clear here how this is related to the need 
%% to avoid totally double-dollars. 

However, it is important to understand that these accommodations do not apply 
to the use of \verb=$$= in the environment definitions for special
math environments (such as those defined in \pkg{amsmath}). 
This is because such usage makes it
technically next to % added
impossible to use the environments in documents that are tagged.
%% Above Needs more explanation?? %FMi tried a bit
%
The kernel therefore now contains these two commands:
\cs{dollardollar@begin} and \cs{dollardollar@end}.  These new commands
must be used by packages and classes to specify where
%FMi: added
inside an environment
a display math
formula starts and ends: their use is essential in order to make the
package or a class compatible with tagging and allowing it to be used
when producing accessible documents: no more \texttt{\$\$}s
%FMi: added
in code, please! 

Package and class developers can very easily prepare code that 
meets this new requirement by adding these two commands:
\begin{verbatim}
\providecommand\dollardollar@begin{$$}
\providecommand\dollardollar@end{$$}
\end{verbatim}
and replacing every occurrence of \verb=$$= with the appropriate start or
end command.

Adding these \cs{providecommand} lines to
classes and packages doesn't hurt but ensures that they will work with older \LaTeX{}
kernels.

%FMi rewrite is wrong
%The consistent use of these two commands in
%classes and packages will also ensure that they will work with 
%older \LaTeX{} kernels.



\subsection{Fixing the spacing after display math}

When \LaTeX{} produces accessible (tagged) PDF 
%% Is accessible needed here? 
it has to add
structure data in the PDF to mark (i.e., tag) individual
elements. If the \pdfTeX{} engine is used this has to be done with the
help of \cs{pdfliteral}s which are whatsit nodes like \cs{special} or
\cs{write}. This means that they should be added only in places, where
these extra nodes are not affecting the spacing\Dash \TeX{} can't, for
example, look backwards past such a whatsit node so consecutive spaces
that are normally collapsed into one, suddenly appear both, if such a
node separates them.

The situation is especially complicated in displayed math 
because \TeX{} there adds penalties and spaces using low-level
procedures that are not directly accessible from the macro level.
Moreover, 
the tagging structures 
%%  what are "tagging structures" ?  A PDF concept, or something else?
%%  Maybe it means "the tagging for these structures" 
have to appear somewhere in the middle of that
%%  the middle of what, exactly? This whole sentence probably needs rewriting! 
%% Is so much detail actually needed here?
to ensure that the formula and the PDF structures cannot get separated by
a page break. Because of this it is necessary to use some fairly complex
methods (essentially disable \TeX's mechanisms and reprogram them on
the macro level) to get the structure data in the right places.
%% This last sentence certainly suggests that there is far too much 
%%. confusing detail here. 

Our first attempt in doing that was slightly faulty and, in
some cases, resulted in the addition of an incorrect \cs{parskip} space. 
This has now corrected and the gymnastics to
achieve this are rather an \enquote{interesting} study in obfuscated
\TeX{} coding.
%% Is this much detail, about our mess-ups, needed here? 
%%  at least add a reference, for those who want to,laugh at the details! 

When using \LuaTeX{} the situation is much improved because the necessary 
extra structures can be added later, when
%FMi this rewrite seems odd:
%the formula processing of the formula itself has been completed.
\TeX's formula processing has already been completed.
%
\taggingissue{762}


\subsection{Local changes to spacing around math displays}

Due to \TeX{}'s low-level handling of display math, it is very
difficult to add the code needed for tagging within such display math
formulas whilst ensuring that such code always stays on the same page
as the formula.  This is because such code must be placed after 
the end of the display, but before the low-level \TeX{} adds 
a \cs{postdisplaypenalty} to the
page. However, there is no way to add code in the middle of this low-level 
\TeX{} processing, which is why we have to resort to complex gymnastics:
we set \cs{postdisplaypenalty} locally to 10000 and also make sure
that \cs{belowdisplayskip} when used by \TeX{} is negative. Then we
let \TeX{} do its job and afterwards regain control via
\cs{aftergroup} and insert the tagging code.  Finally, we add the real
\cs{postdisplaypenalty} and make a space correction.
%% Are all these details needed/useful here? 

With our first implementation of this approach it was not possible for
a user to add an explicit \cs{postdisplaypenalty} or 
\cs{belowdisplayskip} setting inside the formula. 
In this release We have slightly
altered our algorithm to make such user adjustments possible again.
%
\taggingissue{809}



\section{New or improved commands}

\subsection{Socket-and-plug conditionals}

It is sometimes necessary, or helpful, to know whether a particular 
socket or plug exists (or a plug is assigned to a certain socket) 
and, based on such information,
to take different actions. With the current release we added
conditionals, such as \cs{IfSocketExistsTF}, to support such
scenarios. Corresponding L3 programming layer conditionals are also
provided.
%% List them all, or give a reference! 
%
\githubissue{1577}


\subsection{Accessing the current counter}

Counter commands such as \cs{alph}, \cs{stepcounter}, can now use the
argument \texttt{*} to denote the \emph{current counter} (in the sense used by
\cs{label}).\marginpar{not sure: David?}
This is compatible with the use by the \pkg{enumitem} package 
of \verb|\alph*| in item labels; and it is now generally available.  Not all
commands accept \verb|*|: for example, \verb|\counterwithin| and
\verb|\counterwithout| still require counter names as before.
%
\githubissue{1632}

\subsection{Collecting environment bodies verbatim}

The mechanisms provided with \cs{NewDocumentCommand}, etc.\ (formerly \enquote{\pkg{xparse}}) offer a
powerful way to specify a range of types of document command and
environment syntax. This includes the ability to collect the entire
body of an environment, for cases where treating it as a standard
argument is useful. It is also possible using the mechanism
to define
arguments which grab their content verbatim.
To date, however, it was not possible to combine these
two ideas.

In this release a new specifier,~\texttt{c},
for use
in \cs{NewDocumentEnvironment} and friends
has been introduced: 
this collects the body of an environment in a verbatim-like way. 
As for the
existing \texttt{+v}~specification, each separate line is marked by
the special \cs{obeyedline} marker, which as standard issues a normal
paragraph. Thus, this new specifier is usable both for typesetting and
for collecting file contents (the letter~\texttt{c} indicates
\enquote{collect code}). 
Thus, we may use
\begin{verbatim}
  \NewDocumentEnvironment
    {MyVerbatim}{!O{\ttfamily} c}
    {\begin{flushright}#1 #2\end{flushright}}
    {}
    
  \begin{MyVerbatim}[\ttfamily\itshape]
    % Some code is shown here
    $y = mx + c$
  \end{MyVerbatim}
\end{verbatim}
to obtain
\begin{quote}
\makeatletter
\def\@verbatim{%
  \trivlist
  \raggedleft
   %% why centred ? 
  \let \do \@makeother
  \dospecials
  \obeylines
  \normalfont \ttfamily \itshape
  \@noligs
}
\begin{verbatim}
  % Some code is shown here
  $y = mx + c$
\end{verbatim}
\end{quote}



\section{Code improvements}

\subsection{Refinement of \cs{MakeTitlecase}}

We introduced \cs{MakeTitlecase} as a late addition to the June 2022
release, making use of the improved case code in \pkg{expl3}. Unlike
upper and lowercasing, making text titlecased is more tricky to get
right: this can be applied either
to the whole text, or on a word-by-word
basis.

A subtle issue was reported concerning the \pkg{expl3} repository
(\url{https://github.com/latex3/latex3/issues/1316}); this
is related to 
how we deal with the case changing of \enquote{words} but it also
shows up when you titlecase some text stored in a command.

We have looked again at how to implement \cs{MakeTitlecase} in order to make it as
predictable as possible, and we have made a change in this release. The
command no longer tries to lowercase text before applying titlecasing,
and it therefore gives correct results for text stored in commands.

\begin{old}
We have also added an additional key to the optional argument to
\cs{MakeTitlecase} which allows the user to decide if this will apply
only to the first word (the default) or to all words.
\end{old}

\marginpar{Joseph: rewrite right or wrong?}

We have also added an additional key to the optional argument to
\cs{MakeTitlecase} which allows the user to decide if the case change gets applied 
only to the first word (the default) or to all the words.


\subsection{Tab character as a special character}

In \LaTeX{} News~38, we described a change to \cs{verb}, etc., that
makes the tab character equivalent to a space; we have now completed
this work by adding the tab character to the list of characters
covered by \cs{dospecials}.  This allows tab to be used, for example,
in a \texttt{v}~specification document command without the need for
additional steps.


\subsection{Refinement of \texttt{v}~specification category codes}

Work on verbatim argument handling has highlighted that 
it is problematic to store all
characters as \enquote{other} (category code~12) when using a
\texttt{v}~specification in \pkg{ltcmd}. We have therefore now
revised this
so that characters of category code letter retain their original category code.


\subsection{Logging declarations of commands and symbols}

For thirty years the documentation claimed that
\cs{DeclareTextSymbol}, \cs{DeclareTextCommand} and friends all log their
changes: but, in contrast to their math counterparts, they never in fact
did so. This behavior has now finally been corrected. 
%
\githubissue{1242}

\subsection{Improved management of the NFSS font series} 

\LaTeX's font selection mechanism (NFSS) supports 9~weight levels,
from ultra-light~(\texttt{ul}) to ultra-bold~(\texttt{ub}), and also
9~width levels, from ultra-condensed~(\texttt{uc}) to
ultra-expanded~(\texttt{ux}). In the February~2020 release this
mechanism was extended, so that requests to set the weight or the width
attributes of a font series are combined in a sensible
way~\cite{41:ltnews31}: for example, if you typeset a paragraph in a
condensed face using \verb+\fontseries{c}\selectfont+ and then you use
\cs{textbf} inside the paragraph, a bold condensed face is
selected. The combination of such values is done by consulting a
simple lookup table whose entries are defined by using the command
\cs{DeclareFontSeriesChangeRule}.

Until now, this lookup table was missing some entries, especially with
regard to rarely used width values. In such cases, the series values
were not combined as expected. This has been fixed (thanks to Maurice
Hansen) by adding numerous \cs{DeclareFontSeriesChangeRule} entries so
that, when combining these font series values,
the full range of weights (from \texttt{ul} to \texttt{ub}) and
widths (from \texttt{uc} to \texttt{ux}) is now supported. 
%
\githubissue{1396}


\subsection{Supporting the \texttt{ssc} and \texttt{sw} font shapes}

The \texttt{ssc} font shape (spaced small capitals) is supported in
\LaTeX{} through the commands \cs{sscshape} and \cs{textssc}. However,
until this release there where no font shape change rules defined for
this, admittely seldom available, shape; so 
\begin{verbatim}
  \sscshape\itshape
\end{verbatim}
changed unconditionally to \texttt{it} (italics) rather than to
\texttt{sscit} (spaced small italic capitals).  Thanks to Michael
Ummels, the missing declarations have now been added, so shape
changes in font families that support spaced small capitals work
properly.

At the same time we took the opportunity to improve the fallbacks for
the \texttt{sw} (swash) shapes, which are accessible through the
commands \cs{swshape} or \cs{textsw}. If an \texttt{sw} combination is
not available, the rules now try to replace \texttt{sw} with
\texttt{it} rather than falling back to \texttt{n}.
%
\githubissue{1581}


\subsection{Improving the handling of \cs{label}, \cs{index}, and \cs{glossary}}
%%\subsection{Improving \cs{label}, \cs{index}, and \cs{glossary}} - this rewrite seems wrong

In standard \LaTeX{}, the three commands \cs{label}, \cs{index}, and
\cs{glossary} take exactly one mandatory argument, e.g.,
\verb=\index{=\meta{entry}\verb=}=. In some extension packages, for
example \pkg{index} or \pkg{cleveref}, these are all augmented to
accept an optional argument and, in the case of \cs{index}, also a star
form. These extensions conflicted with \LaTeX's way of disabling these
commands within the table of contents and within running headers 
because they were, in these places, redefined to expect just a mandatory argument
and then do nothing. We have now changed this behavior, so that the
redefinitions in these places now accept this extended syntax.
%
\githubissue{311}

\subsection{Tracing lost characters}

In LaTeX News 33~\cite{41:ltnews33} we announced that \cs{tracingall}
changes \cs{tracinglostchars} to an error condition. This change has
been reverted and \cs{tracingall} and \cs{tracingnone} no longer alter
\cs{tracinglostchars} but retain its current setting.

The default value used in \LaTeX{} is set so that lost character
information is written to both the log and the terminal. 
Users may wish to make this 
into an error, in which case \cs{tracinglostchars} should be set 
to~5 (not~3) as this works in all engines.
\githubissue{1687}


\subsection{Always use the extended pool of registers} 

As the kernel has grown, the use of registers has expanded to the point where
rolling back to the classical register allocation approach (using only 256
registers) is no longer viable. We have therefore adjusted the rollback code
so that even when requesting a pre-2015 \LaTeX{}, the extended pool remains in
use.

\subsection{A version of \cs{input} for expansion contexts}

The \LaTeX{} definition of \cs{input} cannot be used in places where \TeX{} is
performing expansion: the classic example is at the start of a tabular cell.
There are a number of reasons for this: the key ones are that \cs{input}
records which files are read and provides pre- and post-file hooks.
%
To support the need to carry out file input in expansion contexts, we have now
added \cs{expandableinput}: this skips recording the file name and does not
apply any file hooks, but otherwise behaves like \cs{input}. In particular, it
still uses \cs{input@path} when doing file lookup (contrasting with the behavior of the  \TeX{}
primitive, which is internally available for programmers as \cs{@@input}).
%
\githubissue{514}


\section{Bug fixes}


\subsection{Avoid problems with page breaks in the middle of a \env{verbatim}-like environments}
%%\subsection{Avoid problems with page breaks in a \env{verbatim}-like environment} %FMi longer one breaks better

If a page break occurs in the middle of an environment that sets up
special \cs{catcode} settings, such as a \env{verbatim} environment,
then these settings will remain active when the output routine is building
the page. This is normally harmless, because the material contained in the
page had been previously tokenized, so that the \cs{catcode}
changes do not matter. However, in certain circumstances tokenization
can happen during this page processing: for example, if processing 
the header header involves reading 
in a file; or if there is a command that uses \cs{scantokens} 
so that it retokenizes some material using the verbatim settings.

This has now been fixed and \LaTeX{} explicitly resets the
\cs{catcode} values to their default settings when entering the output
routine. Furthermore, packages that make changes to the tokenization
beyond what is done by \env{verbatim} can use the newly introduced
hook \hook{build/page/reset} to add their own resets to the output
routine processing. This hook is evaluated after \LaTeX{} has done its
reset, so it is also possible, if necessary,  to overwrite \LaTeX{}'s 
default behavior. 
%
\githubissue{600}



\subsection{Fix the use of  \texttt{localmathalpabets}}

In 2021 we introduced a method to overcome the problem that classic
\TeX{} engines (but not the Unicode engines) have only a limited
number of math alphabets available (so they easily got used up by
loading math font packages, even if their symbols got used only
occasionally). The idea was to avoid allocating all math alphabets
globally, but instead to allow a number of them (defined by counter
\texttt{localmathalpabets}) to vary from one formula to the next. This
means that different formulas can make use of different alphabets, and so the chances
are much higher that the processing of a complex the document succeeds.
See~\cite{41:ltnews34} for details.

Unfortunately, the approach we took
back
then failed in some cases of nested
formulas, with the result that the wrong glyphs were used.
This has now been corrected.
%
\githubissue[s]{1101 1028}


\subsection{\pkg{docstrip}:\ Error if \texttt{.ins} file is problematical}

If the file to be generated had the same name as a preamble declared with
\cs{declarepreamble} then the preamble definition was overwritten, because
the macro used to store it got reused to denote the output
stream. The same problem happened with postambles declared with
\cs{declarepostamble}. This situation is now detected and an error message is
issued. To circumvent the issue in that case, simply use a different
macro name for the preamble or postamble.
%
\githubissue{1150}


\subsection{Prevent a \texttt{cmd} hook from defining an undefined command}

Using \verb=\AddToHook{cmd/FOO/...}= when the command \cs{FOO} was
undefined resulted in this command becoming \cs{relax}. Thus, if used,
it no longer raised an \enquote{Undefined control sequence} error, but
silently did nothing. This behavior has been corrected, and, if the
command \cs{FOO} does not get defined later, e.g., in a package, it now
raises an error when it is used in the document.
%
\githubissue{1591}


\subsection{Process global options just once per package}

In 2022, we introduced key--value (keyval) option processing in the
kernel~\cite{41:ltnews35}. This also added the idea that keys could
have scope: load-only, preamble-only and general use. However, we
overlooked that an option given globally (in the optional argument to
\cs{documentclass}) would be repeatedly processed and could therefore
lead to spurious warnings. This has now been corrected so that now 
each global option is seen, by the keyval-based option handling
system, exactly once per package.
%
\githubissue{1619}


\subsection{Make \cs{label}, \cs{index}, and \cs{glossary} truly invisible in running headers}

\LaTeX{} has had this bug since its initial implementation: whilst it
correctly ignored any \cs{label}, \cs{index}, or \cs{glossary} command
that appears in a mark, it neglected correct handling of the spaces around the
command. As a result, one could end up with two spaces in the running
header where only one should be present. This was detected as part of
working on issue~311 and has now been corrected.
%
\githubissue{1638}



\subsection [Fully expand the arguments of \cs{counterwithin} and \cs{counterwithout}]%
{Fully expand the arguments of the declarations \cs{counterwithin} and \cs{counterwithout}}

The arguments of the commands \cs{counterwithin} and \cs{counterwithout} are
two counter names that are used to reset (or not reset) one counter when the other is
stepped. 
They also redefine the representation of that counter, e.g., \verb=\counterwithin{section}{chapter}= would lead to:  
\begin{verbatim}
  \renewcommand\thesection
     {\thechapter.\arabic{section}}
\end{verbatim}
However, if one of these counters was not named explicitly, as in this example:
\begin{verbatim}
  \newcommand\sectioncounter{section}
  \counterwithin{\sectioncounter}{chapter}
\end{verbatim}
then we ended up with
\begin{verbatim}
  \renewcommand\thesection
     {\thechapter.\arabic{\sectioncounter}}
\end{verbatim}
which could lead to strange results if \cs{sectioncounter} got
changed later on. 
%
This has been corrected: these arguments now get fully
expanded when the declaration 
is made.
%
\githubissue{1675}



\subsection{Correction in the float placement algorithm}

When floats are added to the current or next page, \LaTeX{} makes
several tests in order to find an area that can receive the float. One of these
tests calculates how much space is already used on the page and how
much additional space is needed to place the float in a particular
area. This means that it looks not only at the height of the float but
also at the values from \cs{intextsep} (for \texttt{h} floats) or
\cs{textfloatsep} and \cs{floatsep} (for \texttt{t} and \texttt{b}
floats). The resulting space requirement is then stored in an internal
variable and compared to the space still available on the page.
If the test fails, the algorithm tries the next area. 

Unfortunately, the code 
was reusing the value in that internal variable as the starting
point for the next test, without removing the added space for the float
separation (\cs{intextsep}, \cs{floatsep}, or \cs{textfloatsep}). Thus
the comparison was being made with the wrong value (i.e., too high);
therefore the test may have incorrectly concluded that a float doesn't
fit, even when, in fact, it did fit. 
This has now been corrected.
%
\githubissue{1645}


\subsection{Correct \cs{CheckEncodingSubset}}

In \cite{41:ltnews36}, and again in \cite{41:ltnews39}, we suggested
that font maintainers should place an appropriate
\cs{DeclareEncodingSubset} declaration in each
\texttt{ts1\meta{family}.fd} file, so that this is tied to the font
definition and so will be available whenever a font family is explicitly selected
by \cs{fontfamily}\texttt{\{\meta{name}\}}
%%FMi sorry but this is not correct: we should not claim that the work of font package maintainers is useless! It is not correct
%. This is better than using a
instead of using a
font support package.
Unfortunately, however, it could result in
incorrect selection of glyphs if  the font encoding subset setting was
evaluated before the \texttt{.fd} file was loaded (as subset
9 would then be assumed).  
This has been corrected: 
\cs{CheckEncodingSubset} now first loads the \texttt{.fd} file when this is 
necessary.
%
\githubissue{1669}


\subsection{Ensuring late \cs{write} commands aren't lost}

If a non-\cs{immediate} \cs{write} command is used after the final
page has been shipped out then no write will happen because the system waits for
a \cs{shipout} that will never happen.  After the last page has been shipped
out, we therefore force all further \cs{write} calls to be
\cs{immediate}: this ensures that they get written even though we are
not going to ship out any more pages. This change of behavior is
implemented just before the \texttt{enddocument/afterlastpage} hook
because this hook may contain such \cs{write} commands.
%
\githubissue{1689}


\section{Documentation}

\subsection{Clarifying the handling of spaces by \cs{textcolor}}

In contrast to other \cs{text}-commands such as \cs{textbf} or
\cs{textrm}, the command \cs{textcolor} gobbles spaces at the start of
its argument. Thus, for example, \verb*=Hello\textcolor{red}{ World}=
will produce the output Hello\textcolor{red}{ World}.  
There are technical as well
as compatibility reasons for this, so the behavior will not
change. This is now correctly documented. 
%
\githubissue{1474}


\section{Changes to packages in the \pkg{amsmath} category}

\subsection{\cs{numberwithin} now aliased to \cs{counterwithin}}

The \pkg{amsmath} package offers a \cs{numberwithin} declaration to
specify that a counter should be reset whenever some other counter is
stepped. This is a restricted version of the more general kernel
command \cs{counterwithin} which was introduced in the \LaTeX{} kernel
in 2018 and extended in 2021~\cite{41:ltnews34}. With the current
release we have made \cs{numberwithin} an alias for the more powerful
\cs{counterwithin} and we suggest that the latter command is used in new
documents.
%
\githubissue{1673}


\subsection{\pkg{amsmath}:\ Correct equation tag placement}

If there is not enough space to place an equation tag on the same line
as the equation then \pkg{amsmath} calculates a suitable offset placement 
for the tag, above (or below) the equation. In the case of the
\env{gather} environment this offset was not reset correctly so that 
it also got applied to these tags in any following environment,
which gave incorrect placement in certain situations. The fix for
this, implemented in 2024/06, was not entirely correct: so this has been changed, 
to do such resetting at the start of every displayed math environments.
%
\githubissue{1289}


\section{Changes to packages in the \pkg{graphics} category}
%% are these  "categories" ? Not "collections" ??  %% yes that's what we call them in all ltnews since day one

\subsection{More accessibility keys in \pkg{graphicx}}

The \cs{includegraphics} command now accepts \verb|actualtext| and
\verb|artifact| keys, which by default do nothing but are used by the
tagging code to provide an ActualText string or a boolean flag to indicate that
the graphic is an artifact.
%
\githubissue{1552}


\section{Changes to packages in the \pkg{tools} category}

\subsection{\pkg{multicol}:\ Full support for extended marks}

In 2022 we introduced a new mark mechanism for
\LaTeX{}~\cite{41:ltnews35}. However, the initial implementation 
covered only the standard output routine of \LaTeX{}. As a result the
extended marks were not available within columns produced with the
\pkg{multicol} package (where they would be especially useful). This
limitation has finally been lifted so that the new mechanism is now fully
supported by all of our packages.
%
\githubissue{1421}


\subsection{\pkg{array}:\ Improve preamble code for \texttt{p}, \texttt{m} and \texttt{b}}

When the preamble of a \env{tabular} or \env{array} is being built, 
the arguments to \texttt{p}, \texttt{m}, or \texttt{b} columns all get
expanded several times. This is normally harmless because that
argument usually contains just an explicit dimension. However, 
in a case such as 
\verb=p{\fpeval{15}pt}= these expansions resulted in an error: 
this happened because \cs{fpeval}
was expanded a few times, but not often enough to result in a single
number. This has now been corrected: these arguments are not expanded
at all. This allows for such edge cases 
% what edge cases %FMi: in other place you think there is too much detail :-)
and also for the extensions available
with the \pkg{calc} package, such as \verb=p{\widthof{AAAAAA}}=.
%
\githubissue{1585}


\subsection{\pkg{array}:\ Fix handling of empty p-cells}

If an \cs{arraystretch} greater than one is used, table rows are spread
apart by placing suitable struts (invisible rules) into each row, or
in case of p-cells into each cell. If such a cell was empty the
placement of the strut was not correct so that the cell appeared to be
larger than should have been. This has now been corrected.
%
\githubissue{1730}




\subsection[\pkg{varioref}:\ How to make \cs{reftex...}\ empty]
           {\pkg{varioref}:\ How to make \cs{reftexfaceafter}, etc.\ empty}

%% This does not seem to make much sense! FMi: better?
In the case that one wants to make a command such as
\cs{reftextfaceafter} produce truly nothing, one has to get rid of the space
that is automatically placed in front of the command by \cs{vref}. This can be done
by simply defining the command to remove it, e.g.,
\begin{verbatim}
  \renewcommand\reftextfaceafter{\unskip}
\end{verbatim}
The \pkg{varioref} package does not test if such strings are empty,
because that would require a lot of tests each time \cs{vref} is used,
and it would nearly always find that the text is not empty. However,
as shown above, the solution for this uncommon case is simple, and it
is now explicitly documented in the package documentation.
%
\githubissue{1622}


\section{Changes to files in the L3 programming layer}


Work on the L3 programming layer continues 
in parallel with development of the
rest of the
\LaTeX{} kernel. 
Of note for developers is that we have integrated more code
into the main \pkg{l3kernel} bundle, and therefore into the functionality
available automatically in \LaTeX{}. Most notably, \pkg{l3benchmark}, which
provides tools for checking code performance, is now part of \pkg{l3kernel}.

We have also extended the \pkg{color} module to recognise the Oklab
and Oklch color models; thanks to Markus Kurtz for contributing this
code. The Oklab color space
(\url{https://bottosson.github.io/posts/oklab}) is a perceptual color
space which is supported by CSS and so also by modern web browsers.

The final \enquote{highlight} here is for developers who want to create
packages that work not only with \LaTeX{} but also with other formats: 
generic mode \pkg{expl3}. 
Supporting Con\TeX{}t, in particular when used with
LuaMeta\TeX{}, requires a lot of work in order to keep up with the frequent 
changes to this engine.  After a
period in which the L3 programming layer was not working there due to LuaMeta\TeX{} engine extensions 
to (mainly) \cs{numexpr}, work has now been completed to get this
support working again. As well as adapting to the extended \cs{numexpr} syntax,
we have also tracked changes in the format of \cs{meaning} and the subtle
differences between \cs{scantokens} and \cs{scantextokens}. Not \emph{everything}
works in generic mode, 
but we do aim to have a broadly working, format-neutral codebase.


%\section{Changes to files in the \pkg{cyrillic} category}

\begin{thebibliography}{9}\frenchspacing

%\fontsize{9.3}{11.3}\selectfont

\bibitem{41:Lamport}
Leslie Lamport.
\newblock \emph{{\LaTeX}: {A} Document Preparation System: User's Guide and Reference
  Manual}.
\newblock \mbox{Addison}-Wesley, Reading, MA, USA, 2nd edition, 1994.
\newblock ISBN 0-201-52983-1.
\newblock Reprinted with corrections in 1996.

\bibitem{41:ltnews} \LaTeX{} Project Team.
  \emph{\LaTeXe{} news 1--41}. June, 2025.
  \url{https://latex-project.org/news/latex2e-news/ltnews.pdf}

\bibitem{41:ltnews31} \LaTeX{} Project Team.
  \emph{\LaTeXe{} news 31}. February 2020.
  \url{https://latex-project.org/news/latex2e-news/ltnews31.pdf}

\bibitem{41:ltnews33} \LaTeX{} Project Team.
  \emph{\LaTeXe{} news 33}. June 2021.
  \url{https://latex-project.org/news/latex2e-news/ltnews33.pdf}

\bibitem{41:ltnews34} \LaTeX{} Project Team.
  \emph{\LaTeXe{} news 34}. November 2021.
  \url{https://latex-project.org/news/latex2e-news/ltnews34.pdf}

\bibitem{41:ltnews35} \LaTeX{} Project Team.
  \emph{\LaTeXe{} news 35}. June 2022.
  \url{https://latex-project.org/news/latex2e-news/ltnews35.pdf}

\bibitem{41:ltnews36} \LaTeX{} Project Team.
  \emph{\LaTeXe{} news 36}. November 2022.
  \url{https://latex-project.org/news/latex2e-news/ltnews36.pdf}

\bibitem{41:ltnews38} \LaTeX{} Project Team.
  \emph{\LaTeXe{} news 38}. November 2023.
  \url{https://latex-project.org/news/latex2e-news/ltnews38.pdf}

\bibitem{41:ltnews39} \LaTeX{} Project Team.
  \emph{\LaTeXe{} news 39}. June 2024.
  \url{https://latex-project.org/news/latex2e-news/ltnews39.pdf}

\bibitem{41:ltnews40} \LaTeX{} Project Team.
  \emph{\LaTeXe{} news 40}. November 2024.
  \url{https://latex-project.org/news/latex2e-news/ltnews40.pdf}

\bibitem{41:ltmarks} Frank Mittelbach, \LaTeX{} Project Team.
  \emph{The \texttt{ltmarks.dtx} code}. June 2025.
  \url{https://latex-project.org/help/documentation/ltmarks-doc.pdf}

\bibitem{41:source2e} \LaTeX{} Project Team.
  \emph{The \LaTeXe{} Sources}. June 2025.
  \url{https://latex-project.org/help/documentation/source2e.pdf}

\end{thebibliography}

\end{document}
