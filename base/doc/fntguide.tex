% \iffalse meta-comment
%
% Copyright (C) 1993-2025
%
% The LaTeX Project and any individual authors listed elsewhere
% in this file.
%
% This file is part of the LaTeX base system.
% -------------------------------------------
%
% It may be distributed and/or modified under the
% conditions of the LaTeX Project Public License, either version 1.3c
% of this license or (at your option) any later version.
% The latest version of this license is in
%    https://www.latex-project.org/lppl.txt
% and version 1.3c or later is part of all distributions of LaTeX
% version 2008 or later.
%
% This file has the LPPL maintenance status "maintained".
%
% The list of all files belonging to the LaTeX base distribution is
% given in the file `manifest.txt'. See also `legal.txt' for additional
% information.
%
% The list of derived (unpacked) files belonging to the distribution
% and covered by LPPL is defined by the unpacking scripts (with
% extension .ins) which are part of the distribution.
%
% \fi
% Filename: fntguide.tex

\NeedsTeXFormat{LaTeX2e}[1995/12/01]

\documentclass{ltxguide}[1995/11/28]

\usepackage[T1]{fontenc}
  
\newcommand\cs[1]{\texttt{\textbackslash #1}}
\newcommand\meta[1]{\textnormal{\textlangle\textit{#1}\textrangle}}




\setcounter{totalnumber}{8}
\setcounter{topnumber}{8}

\usepackage[trace]{fewerfloatpages}
\usepackage{booktabs,varioref}

\title{\LaTeXe{} font selection}

\author{\copyright~Copyright 1995--2025, \LaTeX\ Project
  Team.\thanks{Thanks to Arash Esbati for documenting the
    newer NFSS features of 2020}\\
  All rights reserved.%
  \footnote{This file may be distributed and/or modified under the
    conditions of the \LaTeX{} Project Public License, either version 1.3c
    of this license or (at your option) any later version. See the source
   \texttt{fntguide.tex} for full details.}%
}

\date{Sep 2025}

\begin{document}

\maketitle

\tableofcontents

\section{Introduction}

This document describes the new font selection features of the \LaTeX{}
Document Preparation System.  It is intended for package writers who
want to write font-loading packages similar to |times| or |latexsym|.

This document is only a brief introduction to the new facilities and is
intended for package writers who are familiar with \TeX{} fonts and
\LaTeX{} packages.  It is \emph{neither} a user-guide \emph{nor} a
reference manual for fonts in \LaTeXe.

\subsection{\LaTeXe~fonts}

The most important difference between \LaTeX~2.09 and \LaTeXe{} is the
way that fonts are selected.  In \LaTeX~2.09, the Computer Modern fonts
were built into the \LaTeX~format, and so customizing \LaTeX{} to use
other fonts was a major effort.

In \LaTeXe, very few fonts are built into the format, and there are
commands to load new text and math fonts.  Packages such as |times| or
|latexsym| allow authors to access these fonts.  This document describes
how to write similar font-loading packages.

The \LaTeXe{} font selection system was first released as the `New Font
Selection Scheme' (NFSS) in 1989, and then in release~2 in 1993.
\LaTeXe{} includes NFSS release~2 as standard.

\subsection{Overview}

This document contains an overview of the new font commands of \LaTeX.

\begin{description}

\item[Section~\ref{sec:text}] describes the commands for selecting fonts
  in classes and packages.  It lists the five \LaTeX{} font attributes,
  and lists the commands for selecting fonts.  It also describes how to
  customize the author commands such as |\textrm| and |\textit| to suit
  your document design.

\item[Section~\ref{sec:math}] explains the commands for controlling
  \LaTeX{} math fonts.  It describes how to specify new math fonts and
  new math symbols.

\item[Section~\ref{sec:install}] explains how to install new fonts into
  \LaTeX.  It shows how \LaTeX{} font attributes are turned into \TeX{}
  font names, and how to specify your own fonts using font definition
  files.

\item[Section~\ref{sec:encode}] discusses text font encodings.  It
  describes how to declare a new encoding and how to define commands,
  such as |\AE| or |\"|, which have different definitions in different
  encodings, depending on whether ligatures, etc.\ are available in the
  encoding.

\item[Section~\ref{sec:misc}] covers font miscellanea.  It describes how
  \LaTeX{} performs font substitution, how to customize fonts that are
  preloaded in the \LaTeX{} format, and the naming conventions used in
  \LaTeX{} font selection.

\end{description}

\subsection{Further information}

For a general introduction to \LaTeX, including the new features of
\LaTeXe, you should read \emph{\LaTeXbook}, Leslie Lamport, Addison
Wesley, 2nd~ed, 1994.

A more detailed description of the \LaTeX{} font selection scheme is to
be found in \emph{\LaTeXcomp}, 2nd~ed, by Mittelbach and Goossens,
Addison Wesley, 2004.

The \LaTeX{} font selection scheme is based on \TeX, which is described
by its developer in \emph{The \TeX book}, Donald E.~Knuth, Addison
Wesley, 1986, revised in 1991 to include the features of \TeX~3.

Sebastian Rahtz's |psnfss| software contains the software for using a
large number of Type~1 fonts (including the Adobe Laser Writer 35 and
the Monotype CD-ROM fonts) in \LaTeX.  It should be available from the
same source as your copy of \LaTeX.

The |psnfss| software uses fonts generated by Alan Jeffrey's |fontinst|
software.  This can convert fonts from Adobe Font Metric format into a
format readable by \LaTeX, including the generation of the font
definition files described in Section~\ref{sec:install}.  The |fontinst|
software should be available from the same source as your copy of
\LaTeX.

Whenever practical, \LaTeX{} uses the font naming scheme called
`fontname'; this was described in \emph{Filenames for fonts},%
\footnote{An up-to-date electronic version of this document can be found
  on any CTAN server, in the directory \texttt{info/fontname}.}
\emph{TUGboat}~11(4),~1990.

The class-writer's guide \emph{\clsguide} describes the new \LaTeX{}
features for writers of document classes and packages and is kept in
|clsguide.tex|. Configuring \LaTeX{} is covered by the guide
\emph{\cfgguide} in \texttt{cfgguide.tex} whilst the philosophy behind
our policy on modifying \LaTeX{} is described in \emph{\modguide} in
\texttt{modguide.tex}.

The documented source code (from the files used to produce the kernel
format via |latex.ltx|) is now available as \emph{The \LaTeXe\ Sources}.
This very large document also includes an index of \LaTeX{} commands.
It can be typeset from the \LaTeX{} file |source2e.tex| in the |base|
directory; this uses the class file |ltxdoc.cls|.

For more information about \TeX{} and \LaTeX{}, please contact your
local \TeX{} Users Group, or the international \TeX{} Users Group.
Addresses and other details can be found at:
\begin{quote}\small\label{addrs}
  \texttt{https://www.tug.org/lugs.html}
\end{quote}


\section{Text fonts}
\label{sec:text}

This section describes the commands available to class and package
writers for specifying and selecting fonts.

\subsection{Text font attributes}
\label{sec:textfontattributes}

Every text font in \LaTeX{} has five \emph{attributes}:
\begin{description}

\item[encoding] This specifies the order that characters appear in the
  font.  The two most common text encodings used in \LaTeX{} are Knuth's
  `\TeX{} text' encoding, and the `\TeX{} text extended' encoding
  developed by the \TeX{} Users Group members during a \TeX{} Conference
  at Cork in~1990 (hence its informal name `Cork encoding').

\item[family] The name for a collection of fonts, usually grouped under
  a common name by the font foundry.  For example, `Adobe Times', `ITC
  Garamond', and Knuth's `Computer Modern Roman' are all font families.

\item[series] How heavy and/or expanded a font is.  For example, `medium
  weight', `narrow' and `bold extended' are all series.

\item[shape] The form of the letters within a font family.  For example,
  `italic', `oblique' and `upright' (sometimes called `roman') are all
  font shapes.

\item[size] The design size of the font, for example `10pt'. If no
  dimension is specified, `pt' is assumed.

\end{description}
The possible values for these attributes are given short acronyms by
\LaTeX.  The most common values for the font encoding are:
\begin{center}
  \begin{minipage}{.7\linewidth}
    \begin{tabular}{rl}
      |OT1|   & \TeX{} text  \\
      |T1|    & \TeX{} extended text \\
      |OML|   & \TeX{} math italic \\
      |OMS|   & \TeX{} math symbols \\
      |OMX|   & \TeX{} math large symbols \\
      |U|     & Unknown \\
      |L<xx>| & A local encoding
    \end{tabular}
  \end{minipage}
\end{center}
The `local' encodings are intended for font encodings which are only
locally available, for example a font containing an organization's logo
in various sizes.

There are far too many font families to list them all, but some common
ones are:
\begin{center}
  \begin{minipage}{.7\linewidth}
    \begin{tabular}{rl}
      |cmr|  & Computer Modern Roman \\
      |cmss| & Computer Modern Sans \\
      |cmtt| & Computer Modern Typewriter \\
      |cmm|  & Computer Modern Math Italic \\
      |cmsy| & Computer Modern Math Symbols \\
      |cmex| & Computer Modern Math Extensions \\
      |ptm|  & Adobe Times \\
      |phv|  & Adobe Helvetica \\
      |pcr|  & Adobe Courier
    \end{tabular}
  \end{minipage}
\end{center}
\NEWdescription{2019/07/10}\label{page:seriesvalues}
The font series is denoting a combination of the weight (boldness) and
the width (amount of expansion).  The standard supported for weights and
widths are:
\begin{center}
%\begin{minipage}{.7\linewidth}
  \begin{tabular}{rl}
    |ul| & Ultra Light     \\
    |el| & Extra Light     \\
    |l|  & Light          \\
    |sl| & Semi Light      \\
    |m|  & Medium (normal)\\
    |sb| & Semi Bold       \\
    |b|  & Bold           \\
    |eb| & Extra Bold      \\
    |ub| & Ultra Bold      \\
  \end{tabular}
  \qquad
  \begin{tabular}{rlr}
    |uc| & Ultra Condensed &  50\%    \\
    |ec| & Extra Condensed &  62.5\%  \\
    |c|  & Condensed       &  75\%    \\
    |sc| & Semi Condensed  &  87.5\%  \\
    |m|  & Medium          &  100\%   \\
    |sx| & Semi Expanded   &  112.5\% \\
    |x|  & Expanded        &  125\%   \\
    |ex| & Extra Expanded  &  150\%   \\
    |ux| & Ultra Expanded  &  200\%   \\
  \end{tabular}
\end{center}
These are concatenated to a single series value except that |m| is
dropped unless both weight and width are medium in which case a single
|m| is used.

Examples for series values are then:
\begin{center}
  \begin{minipage}{.7\linewidth}
    \begin{tabular}{rl}
      |m|   & Medium weight and width  \\
      |b|   & Bold weight, medium width  \\
      |bx|  & Bold extended \\
      |sb|  & Semi-bold, medium width\\
      |sbx| & Semi-bold extended\\
      |c|   & Medium weight, condensed width
    \end{tabular}
  \end{minipage}
\end{center}
\NEWdescription{2019/07/10}
Note, that there are a large variety of names floating around like
``regular'', ``black'', ``demi-bold'', ``thin'', ``heavy'' and many
more.  If at all possible they should be matched into the standard
naming scheme to allow for sensible default substitutions if necessary,
e.g., ``demi-bold'' is normally just another name for ``semi-bold'', so
should get |sb| assigned, etc.


\NEWdescription{2020/02/02}
The most common values for the font shape are:
\begin{center}
  \begin{minipage}{.7\linewidth}
    \begin{tabular}{rl}
      |n|    & Normal (that is `upright' or `roman') \\
      |it|   & Italic                                \\
      |sl|   & Slanted (or `oblique')                \\
      |sc|   & Caps and small caps                   \\
      |scit| & Caps and small caps italic            \\
      |scsl| & Caps and small caps slanted           \\
      |sw|   & Swash
    \end{tabular}
  \end{minipage}
\end{center}
A less common value for font shape is:
\begin{center}
  \begin{minipage}{.7\linewidth}
    \begin{tabular}{rl}
      |ssc|  & Spaced caps and small caps
    \end{tabular}
  \end{minipage}
\end{center}
and there is also |ui| for upright italic, i.e., an italic shape but
artificially turned upright. This is sometimes useful and available in
some fonts.


The font size is specified as a dimension, for example |10pt| or |1.5in|
or |3mm|; if no unit is specified, |pt| is assumed.  These five
parameters specify every \LaTeX{} font, for example:
\begin{center}
  \begin{tabular}{@{}r@{\,}l@{\,}c@{\,}c@{\,}cc@{}r@{}}
    \multicolumn{5}{@{}c}{\emph{\LaTeX{} specification}}
    & \emph{Font}  & \emph{\TeX{} font name} \\
    |OT1| & |cmr|  & |m| & |n| & |10|
                         & Computer Modern Roman 10 point
                               &  |cmr10| \\
    |OT1| & |cmss| & |m| & |sl| & |1pc|
                         & Computer Modern Sans Oblique 1 pica
                               & |cmssi12| \\
    |OML| & |cmm|  & |m| & |it| & |10pt|
                         & Computer Modern Math Italic 10 point
                               & |cmmi10| \\
    |T1| & |ptm|   & |b| & |it| & |1in|
                         & Adobe Times Bold Italic 1 inch
                               & |ptmb8t at 1in|
   \end{tabular}
\end{center}
These five parameters are displayed whenever \LaTeX{} gives an overfull
box warning, for example:
\begin{verbatim}
   Overfull \hbox (3.80855pt too wide) in paragraph at lines 314--318
   []\OT1/cmr/m/n/10 Normally [] and [] will be iden-ti-cal,
\end{verbatim}
The author commands for fonts set the five attributes as shown in table~\vref{tab:attributes}.
\begin{table}[!tbp]
  \centering
  \begin{tabular}{@{}r@{\hspace{3pt}}c@{\hspace{1pt}}c@{}}
    \toprule
    \multicolumn{1}{c}{\emph{Author command}} & \emph{Attribute}
                          & \emph{Value in} |article| \emph{class} \\
    \midrule
    |\textnormal{..}| or |\normalfont| & family & |cmr|     \\
                                       & series & |m|       \\
                                       & shape  & |n|       \\ \midrule%[2pt]
    |\textrm{..}| or |\rmfamily|       & family & |cmr|     \\
    |\textsf{..}| or |\sffamily|       & family & |cmss|    \\
    |\texttt{..}| or |\ttfamily|       & family & |cmtt|    \\
    |\textmd{..}| or |\mdseries|       & series & |m|       \\
    |\textbf{..}| or |\bfseries|       & series & |bx|      \\ \midrule%[2pt]
    |\textit{..}| or |\itshape|        & shape  & |it|      \\
    |\textsl{..}| or |\slshape|        & shape  & |sl|      \\
    |\textsc{..}| or |\scshape|        & shape  & |sc|      \\
    |\textssc{..}| or |\sscshape|      & shape  & |ssc|     \\
    |\textsw{..}| or |\swshape|        & shape  & |sw|      \\
    |\textulc{..}| or |\ulcshape|      & shape  & |ulc| (virtual) $\to$ |n|, |it|, |sl| or |ssc|     \\
    |\textup{..}| or |\upshape|        & shape  & | up| (virtual) $\to$ |n| or |sc     |\phantom{, , } \\\midrule%[2pt]
    |\tiny|                            & size   & |5pt|     \\
    |\scriptsize|                      & size   & |7pt|     \\
    |\footnotesize|                    & size   & |8pt|     \\
    |\small|                           & size   & |9pt|     \\
    |\normalsize|                      & size   & |10pt|    \\
    |\large|                           & size   & |12pt|    \\
    |\Large|                           & size   & |14.4pt|  \\
    |\LARGE|                           & size   & |17.28pt| \\
    |\huge|                            & size   & |20.74pt| \\
    |\Huge|                            & size   & |24.88pt| \\
    \bottomrule
  \end{tabular}
  \caption{Author font commands and their effects (article class)}\label{tab:attributes}
\end{table}
The values used by these commands are determined by the document class,
using the parameters defined in Section~\ref{sec:text.param}.

Note that there are no author commands for selecting new encodings.
These should be provided by packages, such as the |fontenc| package.

This section does not explain how \LaTeX{} font specifications are
turned into \TeX{} font names.  This is described in
Section~\ref{sec:install}.



\subsection{Selection commands}

The low-level commands used to select a text font are as follows.

\begin{decl}
  |\fontencoding| \arg{encoding} \\
  |\fontfamily|  \arg{family} \qquad
  |\fontseries| \arg{series} \qquad
  |\fontshape| \arg{shape} \\
  |\fontsize| \arg{size} \arg{baselineskip} \qquad
  |\linespread| \arg{factor}
\end{decl}

\NEWdescription{1998/12/01}
Each of the commands starting with |\font...| sets one of the font
attributes; |\fontsize| also sets |\baselineskip|. The |\linespread|
command prepares to multiply the current (or newly defined)
|\baselineskip| with \m{factor} (e.g., spreads the lines apart for
values greater one).

The actual font in use is not altered by these commands, but the
current attributes are used to determine which font and baseline skip
to use after the next |\selectfont| command.

\begin{decl}
  |\selectfont|
\end{decl}
Selects a text font, based on the current values of the font attributes.

\emph{Warning}: There \emph{must} be a |\selectfont| command immediately
after any settings of the font parameters by (some of) the six commands
above, before any following text.  For example, it is legal to say:
\begin{verbatim}
   \fontfamily{ptm}\fontseries{b}\selectfont Some text.
\end{verbatim}
but it is \emph{not} legal to say:
\begin{verbatim}
   \fontfamily{ptm} Some \fontseries{b}\selectfont text.
\end{verbatim}
You may get unexpected results if you put text between a
|\font<parameter>| command (or |\linespread|) and a |\selectfont|.

\begin{decl}
  |\usefont| \arg{encoding} \arg{family} \arg{series} \arg{shape}
\end{decl}
A short hand for the equivalent |\font...| commands followed by a call
to |\selectfont|.

\subsection{Internals}

The current values of the font attributes are held in internal macros.

\begin{decl}
  |\f@encoding| \quad
  |\f@family| \quad
  |\f@series| \quad
  |\f@shape|  \quad
  |\f@size| \quad
  |\f@baselineskip| \\
  |\tf@size| \quad
  |\sf@size| \quad
  |\ssf@size|
\end{decl}

These hold the current values of the encoding, the family, the series,
the shape, the size, the baseline skip, the main math size, the `script'
math size and the `scriptscript' math size. The last three are
accessible only within a formula; outside of math they may contain
arbitrary values.

For example, to set the size to 12 without changing the baseline skip:
\begin{verbatim}
   \fontsize{12}{\f@baselineskip}
\end{verbatim}
However, you should \emph{never} alter the values of the internal
commands directly; they must only be modified using the low-level
commands like |\fontfamily|, |\fontseries|, etc. If you disobey this
warning you might produce code that loops.

\subsection{Parameters for author commands}
\label{sec:text.param}

The parameter values set by author commands such as |\textrm| and
|\rmfamily|, etc.\ are not hard-wired into \LaTeX; instead these
commands use the values of a number of parameters set by the document
class and packages.  For example, |\rmdefault| is the name of the
default family selected by |\textrm| and |\rmfamily|. Thus to set a
document in Adobe Times, Helvetica and Courier, the document designer
specifies:
\begin{verbatim}
   \renewcommand{\rmdefault}{ptm}
   \renewcommand{\sfdefault}{phv}
   \renewcommand{\ttdefault}{pcr}
\end{verbatim}

\begin{decl}
  |\encodingdefault| \qquad
  |\familydefault|   \qquad
  |\seriesdefault|   \qquad
  |\shapedefault|
\end{decl}
The encoding, family, series and shape of the main body font.  By
default these are |OT1|, |\rmdefault|, |m| and |n|.  Note that since the
default family is |\rmdefault|, this means that changing |\rmdefault|
will change the main body font of the document.

\begin{decl}
  |\rmdefault| \qquad
  |\sfdefault| \qquad
  |\ttdefault|
\end{decl}
The families selected by |\textrm|, |\rmfamily|, |\textsf|, |\sffamily|,
|\texttt| and |\ttfamily|.  By default these are |cmr|, |cmss| and
|cmtt|.

\begin{decl}
  |\bfdefault| \qquad
  |\mddefault|
\end{decl}
The series selected by |\textbf|, |\bfseries|, |\textmd| and
|\mdseries|.  By default these are |bx| and |m|.  These values are
suitable for the default families used. If other fonts are used as
standard document fonts (for example, certain PostScript fonts) it might
be necessary to adjust the value of |\bfdefault| to |b| since only a few
such families have a `bold extended' series.  An alternative (taken for
the fonts provided by |psnfss|) is to define silent substitutions from
|bx| series to |b| series with special |\DeclareFontShape| declarations
and the |ssub| size function, see Section~\ref{sec:sizefunct}.

\begin{decl}
  |\itdefault|  \qquad
  |\sldefault|  \qquad
  |\scdefault|  \qquad
  |\sscdefault| \qquad
  |\swdefault|  \\
  |\ulcdefault| \qquad
  |\updefault|
\end{decl}
\NEWfeature{2020/02/02}
The shapes selected by |\textit|, |\itshape|, |\textsl|, |\slshape|,
|\textsc|, |\scshape|, |\textssc|, |\sscshape|, |\textsw|, |\swshape|,
|\textulc|, |\ulcshape|, |\textup| and |\upshape|.  By default these are
|it|, |sl|, |sc|, |ssc|, |sw|, |ulc| and |up|.  Note that |ulc| and |up| are special here
because they are virtual shapes; they don't exist as real shape values. Instead they alter
the existing shape value based on rules, i.e., the result depends on context. The
respective macros |\textulc| or |\ulcshape| change small capitals back
to upper/lower case but will not change the font with respect to
italics, slanted or swash.  |\upshape| or |\textup| in contrast will
switch italics or slanted back to upright but not alter the state of
upper/lower case, e.g., keep small capitals if present.  Finally, the
command |\normalshape| is provided to reset the shape back to normal
which is a shorthand for |\upshape\ulcshape|.

Note that there are no parameters for the size commands.  These should
be defined directly in class files, for example:
\begin{verbatim}
   \renewcommand{\normalsize}{\fontsize{10}{12}\selectfont}
\end{verbatim}
More elaborate examples (setting additional parameters when the text
size is changed) can be found in |classes.dtx| the source documentation
for the classes |article|, |report|, and |book|.

\subsection{Special font declaration commands}

\begin{decl}
  |\DeclareFixedFont| \arg{cmd} \arg{encoding} \arg{family} \arg{series}
                      \arg{shape} \arg{size}
\end{decl}

Declares command \m{cmd} to be a font switch which selects the font that
is specified by the attributes \m{encoding}, \m{family}, \m{series},
\m{shape}, and \m{size}.

The font is selected without any adjustments to baselineskip and other
surrounding conditions.

This example makes |{\picturechar .}| select a small dot very quickly:
\begin{verbatim}
   \DeclareFixedFont{\picturechar}{OT1}{cmr}{m}{n}{5}
\end{verbatim}

\begin{decl}
|\DeclareTextFontCommand| \arg{cmd} \arg{font-switches}
\end{decl}

Declares command \m{cmd} to be a font command with one argument.  The
current font attributes are locally modified by \m{font-switches} and
then the argument of \m{cmd} is typeset in the resulting new font.

Commands defined by |\DeclareTextFontCommand| automatically take care of
any necessary italic correction (on either side).

The following example shows how |\textrm| is defined by the kernel.
\begin{verbatim}
   \DeclareTextFontCommand{\textrm}{\rmfamily}
\end{verbatim}

To define a command that always typeset its argument in the italic shape
of the main document font you could declare:
\begin{verbatim}
   \DeclareTextFontCommand{\normalit}{\normalfont\itshape}
\end{verbatim}

This declaration can be used to change the meaning of a command; if
\m{cmd} is already defined, a log that it has been redefined is put in
the transcript file.

\begin{decl}
  |\DeclareOldFontCommand| \arg{cmd} \arg{text-switch}
                                     \arg{math-switch}
\end{decl}

Declares command \m{cmd} to be a font switch (i.e.~used with the
syntax |{<cmd>...}|) having the definition \m{text-switch}
when used in text and the definition \m{math-switch} when used in a
formula.
Math alphabet commands, like |\mathit|, when used within \m{math-switch}
should not have an argument.  Their use in this argument causes their
semantics to change so that they here act as a font switch, as
required by the usage of the \m{cmd}.

This declaration is useful for setting up commands like |\rm| to behave
as they did in \LaTeX~2.09. We strongly urge you \emph{not} to misuse
this declaration to invent new font commands.

\newpage

The following example defines |\it| to produce the italic shape of the
main document font if used in text and to switch to the font that would
normally be produced by the math alphabet |\mathit| if used in a
formula.
\begin{verbatim}
   \DeclareOldFontCommand{\it}{\normalfont\itshape}{\mathit}
\end{verbatim}

This declaration can be used to change the meaning of a command; if
\m{cmd} is already defined, a log that it has been redefined is put in
the transcript file.


\section{Math fonts}
\label{sec:math}

This section describes the commands available to class and package
writers for specifying math fonts and math commands.

\subsection{Math font attributes}

The selection of fonts within math mode is quite different to that of
text fonts.

Some math fonts are selected explicitly by one-argument commands such as
|\mathsf{max}| or |\mathbf{vec}|; such fonts are called \emph{math
  alphabets}.  These math alphabet commands affect only the font used
for letters and symbols of type |\mathalpha| (see
Section~\ref{sec:math.commands}); other symbols within the argument will
be left unchanged.  The predefined math alphabets are:
\begin{center}
  \begin{tabular}{ccc}
    \emph{Alphabet} & \emph{Description} & \emph{Example} \\
    |\mathnormal|   & default            & $abcXYZ$ \\
    |\mathrm|       & roman              & $\mathrm{abcXYZ}$ \\
    |\mathbf|       & bold roman         & $\mathbf{abcXYZ}$ \\
    |\mathsf|       & sans serif         & $\mathsf{abcXYZ}$ \\
    |\mathit|       & text italic        & $\mathit{abcXYZ}$ \\
    |\mathtt|       & typewriter         & $\mathtt{abcXYZ}$ \\
    |\mathcal|      & calligraphic       & $\mathcal{XYZ}$
  \end{tabular}
\end{center}
Other math fonts are selected implicitly by \TeX{} for symbols, with
commands such as |\oplus| (producing $\oplus$) or with straight
characters like |>>| or |+|.  Fonts containing such math symbols are
called \emph{math symbol fonts}.  The predefined math symbol fonts are:
\begin{center}
  \begin{tabular}{ccc}
    \emph{Symbol font} & \emph{Description}         & \emph{Example} \\
    |operators|        & symbols from |\mathrm|     & $[\;+\;]$ \\
    |letters|          & symbols from |\mathnormal| & $<<\star>>$ \\
    |symbols|          & most \LaTeX{} symbols      & $\leq*\geq$ \\
    |largesymbols|     & large symbols              & $\sum\prod\int$
   \end{tabular}
\end{center}
Some math fonts are both \emph{math alphabets} and \emph{math symbol
  fonts}, for example |\mathrm| and |operators| are the same font, and
|\mathnormal| and |letters| are the same font.

Math fonts in \LaTeX{} have the same five attributes as text fonts:
encoding, family, series, shape and size.  However, there are no
commands that allow the attributes to be individually changed.  Instead,
the conversion from math fonts to these five attributes is controlled by
the \emph{math version}.  For example, the |normal| math version maps:
\begin{center}
  \begin{tabular}{rlc@{ }c@{ }c@{ }c}
    \multicolumn{2}{c}{\emph{Math font}} &
    \multicolumn{4}{c}{\emph{External font}} \\
    \emph{Alphabets} & \emph{Symbol fonts} &
    \multicolumn{4}{c}{\emph{Attributes}} \\
    |\mathnormal| & |letters|      & |OML| & |cmm|  & |m|  & |it| \\
    |\mathrm|     & |operators|    & |OT1| & |cmr|  & |m|  & |n|  \\
    |\mathcal|    & |symbols|      & |OMS| & |cmsy| & |m|  & |n|  \\
                  & |largesymbols| & |OMX| & |cmex| & |m|  & |n|  \\
    |\mathbf|     &                & |OT1| & |cmr|  & |bx| & |n|  \\
    |\mathsf|     &                & |OT1| & |cmss| & |m|  & |n|  \\
    |\mathit|     &                & |OT1| & |cmr|  & |m|  & |it| \\
    |\mathtt|     &                & |OT1| & |cmtt| & |m|  & |n|
  \end{tabular}
\end{center}
The |bold| math version is similar except that it contains bold fonts.
The command |\boldmath| selects the |bold| math version.

Math versions can only be changed outside of math mode.

The two predefined math versions are:
\begin{center}
  \begin{tabular}{rl}
    |normal| & the default math version \\
    |bold|   & the bold math version
  \end{tabular}
\end{center}
Packages may define new math alphabets, math symbol fonts, and math
versions.  This section describes the commands for writing such
packages.

\subsection{Selection commands}

There are no commands for selecting symbol fonts.  Instead, these are
selected indirectly through symbol commands like |\oplus|.
Section~\ref{sec:math.commands} explains how to define symbol commands.

\begin{decl}
  |\mathnormal{<math>}| \quad
  |\mathcal{<math>}| \quad
  |\mathbf{<math>}| \quad
  |\mathit{<math>}| \\
  |\mathrm{<math>}| \qquad
  |\mathsf{<math>}| \qquad
  |\mathtt{<math>}| 
\end{decl}
Each math alphabet is a command which can only be used inside math mode.
For example, |$x + \mathsf{y} + \mathcal{Z}$| produces
$x + \mathsf{y} + \mathcal{Z}$.

\begin{decl}
  |\mathversion{<version>}|
\end{decl}
This command selects a math version; it can only be used outside math
mode.  For example, |\boldmath| is defined to be |\mathversion{bold}|.

\subsection{Declaring math versions}

\begin{decl}
  |\DeclareMathVersion| \arg{version}
\end{decl}

Defines \m{version} to be a math version.

The newly declared version is initialized with the defaults for all
symbol fonts and math alphabets declared so far (see the commands
|\DeclareSymbolFont| and |\DeclareMathAlphabet|).

If used on an already existing version, an information message is
written to the transcript file and all previous |\SetSymbolFont| or
|\SetMathAlphabet| declarations for this version are overwritten by the
math alphabet and symbol font defaults, i.e.~one ends up with a virgin
math version.

Example:
\begin{verbatim}
   \DeclareMathVersion{normal}
\end{verbatim}

\subsection{Declaring math alphabets}

\begin{decl}
  |\DeclareMathAlphabet| \arg{math-alph} \arg{encoding} \arg{family}
                         \arg{series}    \arg{shape}
\end{decl}

\NEWdescription{1997/12/01}
If this is the first declaration for \m{math-alph} then a new math
alphabet with this as its command name is created.

The arguments \m{encoding} \m{family} \m{series} \m{shape} are used to
set, or reset, the default values for this math alphabet in all math
versions; if required, these must be further reset later for a
particular math version by a |\SetMathAlphabet| command.

If \m{shape} is empty then this \m{math-alph} is declared to be invalid
in all versions, unless it is set by a later |\SetMathAlphabet| command
for a particular math version.

Checks that the command \m{math-alph} is either already a math alphabet
command or is undefined; and that \m{encoding} is a known encoding
scheme, i.e., has been previously declared.

In these examples, |\foo| is defined for all math versions but |\baz|,
by default, is defined nowhere.
\begin{verbatim}
   \DeclareMathAlphabet{\foo}{OT1}{cmtt}{m}{n}
   \DeclareMathAlphabet{\baz}{OT1}{}{}{}
\end{verbatim}


\begin{decl}
  |\SetMathAlphabet| \arg{math-alph} \arg{version}\\
         \null\hfill \arg{encoding} \arg{family} \arg{series} \arg{shape}
\end{decl}

Changes, or sets, the font for the math alphabet \m{math-alph} in math
version \m{version} to \m{encoding}\m{family}\m{series}\m{shape}.

Checks that \m{math-alph} has been declared as a math alphabet,
\m{version} is a known math version and \m{encoding} is a known encoding
scheme.

\newpage

This example defines |\baz| for the `normal' math version only:
\begin{verbatim}
   \SetMathAlphabet{\baz}{normal}{OT1}{cmss}{m}{n}
\end{verbatim}

Note that this declaration is not used for all math alphabets:
Section~\ref{sec:symalph} describes |\DeclareSymbolFontAlphabet|, which
is used to set up math alphabets contained in fonts which have been
declared as symbol fonts.

\subsection{Declaring symbol fonts}
\label{sec:symalph}

\begin{decl}
  |\DeclareSymbolFont| \arg{sym-font} \arg{encoding} \arg{family}
                       \arg{series} \arg{shape}
\end{decl}

\NEWdescription{1997/12/01}
If this is the first declaration for \m{sym-font} then a new symbol font
with this name is created (i.e.~this identifier is assigned to a new
\TeX{} math group).

The arguments \m{encoding} \m{family} \m{series} \m{shape} are used to
set, or reset. the default values for this symbol font in \emph{all}
math versions; if required, these must be further reset later for a
particular math version by a |\SetSymbolFont| command.

Checks that \m{encoding} is a declared encoding scheme.

For example, the following sets up the first four standard math symbol
fonts:
\begin{verbatim}
   \DeclareSymbolFont{operators}{OT1}{cmr}{m}{n}
   \DeclareSymbolFont{letters}{OML}{cmm}{m}{it}
   \DeclareSymbolFont{symbols}{OMS}{cmsy}{m}{n}
   \DeclareSymbolFont{largesymbols}{OMX}{cmex}{m}{n}
\end{verbatim}

\begin{decl}
  |\SetSymbolFont| \arg{sym-font} \arg{version}\\
       \null\hfill \arg{encoding} \arg{family} \arg{series} \arg{shape}
\end{decl}

Changes the symbol font \m{sym-font} for math version \m{version} to
\m{encoding} \m{family} \m{series} \m{shape}.

Checks that \m{sym-font} has been declared as a symbol font, \m{version}
is a known math version and \m{encoding} is a declared encoding scheme.

For example, the following come from the set up of the `bold' math
version:
\begin{verbatim}
   \SetSymbolFont{operators}{bold}{OT1}{cmr}{bx}{n}
   \SetSymbolFont{letters}{bold}{OML}{cmm}{b}{it}
\end{verbatim}


\begin{decl}
  |\DeclareSymbolFontAlphabet| \arg{math-alph} \arg{sym-font}
\end{decl}

\NEWdescription{1997/12/01}
Allows the previously declared symbol font \m{sym-font} to be the math
alphabet with command \m{math-alph} in \emph{all} math versions.

Checks that the command \m{math-alph} is either already a math alphabet
command or is undefined; and that \m{sym-font} is a symbol font.

\newpage

Example:
\begin{verbatim}
   \DeclareSymbolFontAlphabet{\mathrm}{operators}
   \DeclareSymbolFontAlphabet{\mathcal}{symbols}
\end{verbatim}

This declaration should be used in preference to |\DeclareMathAlphabet|
and |\SetMathAlphabet| when a math alphabet is the same as a symbol
font; this is because it makes better use of the limited number (only
16) of \TeX's math groups.

\NEWdescription{1997/12/01}
Note that, whereas a \TeX{} math group is allocated to each symbol font
when it is first declared, a math alphabet uses a \TeX{} math group only
when its command is used within a math formula.

\subsection{Declaring math symbols}
\label{sec:math.commands}

\begin{decl}
  |\DeclareMathSymbol| \arg{symbol} \arg{type} \arg{sym-font}
                       \arg{slot}
\end{decl}

The \m{symbol} can be either a single character such as `|>>|', or a
macro name, such as |\sum|.

Defines the \m{symbol} to be a math symbol of type \m{type} in slot
\m{slot} of symbol font \m{sym-font}. The \m{type} can be given as a
number or as a command:
\begin{center}
  \begin{tabular}{ccc}
    \emph{Type}         & \emph{Meaning}     & \emph{Example} \\
    |0| or |\mathord  | & Ordinary           & $\alpha$ \\
    |1| or |\mathop   | & Large operator     & $\sum$ \\
    |2| or |\mathbin  | & Binary operation   & $\times$ \\
    |3| or |\mathrel  | & Relation           & $\leq$ \\
    |4| or |\mathopen | & Opening            & $\langle$ \\
    |5| or |\mathclose| & Closing            & $\rangle$ \\
    |6| or |\mathpunct| & Punctuation        & $;$ \\
    |7| or |\mathalpha| & Alphabet character & $A$
  \end{tabular}
\end{center}
Only symbols of type |\mathalpha| will be affected by math alphabet
commands: within the argument of a math alphabet command they will
produce the character in slot \m{slot} of that math alphabet's font.
Symbols of other types will always produce the same symbol (within one
math version).

|\DeclareMathSymbol| allows a macro \m{symbol} to be redefined only if
it was previously defined to be a math symbol.  It also checks that the
\m{sym-font} is a declared symbol font.

Example:
\begin{verbatim}
   \DeclareMathSymbol{\alpha}{0}{letters}{"0B}
   \DeclareMathSymbol{\lessdot}{\mathbin}{AMSb}{"0C}
   \DeclareMathSymbol{\alphld}{\mathalpha}{AMSb}{"0C}
\end{verbatim}


\begin{decl}
  |\DeclareMathDelimiter| \arg{cmd} \arg{type}
                          \arg{sym-font-1} \arg{slot-1}\\
              \null\hfill \arg{sym-font-2} \arg{slot-2}
\end{decl}
Defines \m{cmd} to be a math delimiter where the small variant is in
slot \m{slot-1} of symbol font \m{sym-font-1} and the large variant is
in slot \m{slot-2} of symbol font \m{sym-font-2}.  Both symbol fonts
must have been declared previously.

Checks that \m{sym-font-i} are both declared symbol fonts.

If \TeX{} is not looking for a delimiter, \m{cmd} is treated just as if
it had been defined with |\DeclareMathSymbol| using \m{type},
\m{sym-font-1} and \m{slot-1}.  In other words, if a command is defined
as a delimiter then this automatically defines it as a math symbol.

\NEWdescription{1998/06/01}
In case \m{cmd} is a single character such as `|[|', the same syntax is
used.  Previously the \arg{type} argument was not present (and thus the
corresponding math symbol declaration had to be provided separately).

Example:
\begin{verbatim}
   \DeclareMathDelimiter{\langle}{\mathopen}{symbols}{"68}
                                            {largesymbols}{"0A}
   \DeclareMathDelimiter{(}      {\mathopen}{operators}{"28}
                                            {largesymbols}{"00}
\end{verbatim}


\begin{decl}
  |\DeclareMathAccent| \arg{cmd} \arg{type} \arg{sym-font} \arg{slot}
\end{decl}

Defines \m{cmd} to act as a math accent.

The accent character comes from slot \m{slot} in \m{sym-font}. The
\m{type} can be either |\mathord| or |\mathalpha|; in the latter case
the accent character changes font when used in a math alphabet.

Example:
\begin{verbatim}
   \DeclareMathAccent{\acute}{\mathalpha}{operators}{"13}
   \DeclareMathAccent{\vec}{\mathord}{letters}{"7E}
\end{verbatim}


\begin{decl}
  |\DeclareMathRadical| \arg{cmd}
                        \arg{sym-font-1} \arg{slot-1}\\
            \null\hfill \arg{sym-font-2} \arg{slot-2}
\end{decl}

Defines \m{cmd} to be a radical where the small variant is in slot
\m{slot-1} of symbol font \m{sym-font-1} and the large variant is in
slot \m{slot-2} of symbol font \m{sym-font-2}.  Both symbol fonts must
have been declared previously.

Example (probably the only use for it!):
\begin{verbatim}
   \DeclareMathRadical{\sqrt}{symbols}{"70}{largesymbols}{"70}
\end{verbatim}

\subsection{Declaring math sizes}

\begin{decl}
  |\DeclareMathSizes| \arg{t-size} \arg{mt-size} \arg{s-size}
                      \arg{ss-size}
\end{decl}

Declares that \m{mt-size} is the (main) math text size, \m{s-size} is
the `script' size and \m{ss-size} the `scriptscript' size to be used in
math, when \m{t-size} is the current text size. For text sizes for which
no such declaration is given the `script' and `scriptscript' size will
be calculated and then fonts are loaded for the calculated sizes or the
best approximation (this may result in a warning message).

Normally, \m{t-size} and \m{mt-size} will be identical; however, if, for
example, PostScript text fonts are mixed with bit-map math fonts then
you may not have available a \m{mt-size} for every \m{t-size}.

Example:
\begin{verbatim}
   \DeclareMathSizes{13.82}{14.4}{10}{7}
\end{verbatim}

\subsection{Declaring math script fonts}

\begin{decl}
  |\DeclareMathScriptfontMapping| \arg{t-enc} \arg{t-family}
                                  \arg{s-enc} \arg{s-family}
                                  \arg{ss-enc} \arg{ss-family}
\end{decl}

Declares that the font family \m{s-family} in encoding \m{s-enc} should
be used in `script' position and \m{ss-family} in encoding \m{ss-enc}
in `scriptscript' encoding should be used in `scriptscript' positions
whenever \arg{t-family} is selected in encoding \m{t-enc} as a math font.

If no such declaration is given for a \arg{t-enc}/\arg{t-family} pair then
different sizes of the same font are used in all `text', `script', and
`scriptscript' positions.
In either case, the sizes are specified through a |\DeclareMathSizes|
declaration.

Example:
\begin{verbatim}
  \DeclareMathScriptfontMapping{OML}{cmm}{OML}{cmm-script}{OML}{cmm-scriptscript}
\end{verbatim}

\section{Font installation}
\label{sec:install}

This section explains how \LaTeX's font attributes are turned into
\TeX{} font specifications.


\subsection{Font definition files}

\NEWdescription{1997/12/01}
The description of how \LaTeX{} font attributes are turned into \TeX{}
fonts is usually kept in a \emph{font definition file} (|.fd|).  The
file for family \m{family} in encoding \m{ENC} must be called
|<enc><family>.fd|: for example, |ot1cmr.fd| for Computer Modern Roman
with encoding |OT1| or |t1ptm.fd| for Adobe Times with encoding |T1|.
Note that encoding names are converted to lowercase when used as part of
file names.

Whenever \LaTeX{} encounters an encoding/family combination that it does
not know (e.g.~if the document designer says
|\fontfamily{ptm}\selectfont|) then \LaTeX{} attempts to load the
appropriate |.fd| file.  ``Not known'' means: there was no
|\DeclareFontFamily| declaration issued for this encoding/family
combination.  If the |.fd| file could not be found, a warning is issued
and font substitutions are made.

The declarations in the font definition file are responsible for telling
\LaTeX{} how to load fonts for that encoding/family combination.



\subsection{Font definition file commands}

\emph{Note}: A font definition file should contain only commands from
this subsection.

Note that these commands can also be used outside a font definition
file: they can be put in package or class files, or even in the preamble
of a document.

\begin{decl}
  |\ProvidesFile{<file-name>}[<release-info>]|
\end{decl}
The file should announce itself with a |\ProvidesFile| command, as
described in \emph{\clsguide}.

\newpage

For example:
\begin{verbatim}
   \ProvidesFile{t1ptm.fd}[1994/06/01 Adobe Times font definitions]
\end{verbatim}

Spaces within the arguments specific to font definition files are
ignored to avoid surplus spaces in the document. If a real space is
necessary use |\space|.
\NEWdescription{2004/02/10}
However, note that this is only true if the declaration is made at top
level! If used within the definition of another command, within
|\AtBeginDocument|, option code or in similar places, then spaces within
the argument will remain and may result in incorrect table entries.

\begin{decl}
  |\DeclareFontFamily| \arg{encoding} \arg{family} \arg{loading-settings}
\end{decl}

Declares a font family \m{family} to be available in encoding scheme
\m{encoding}.

The \m{loading-settings} are executed immediately after loading any font
with this encoding and family.

Checks that \m{encoding} was previously declared.

This example refers to the Computer Modern Typewriter font family in the
Cork encoding:
\begin{verbatim}
   \DeclareFontFamily{T1}{cmtt}{\hyphenchar\font=-1}
\end{verbatim}

Each |.fd| file should contain exactly one |\DeclareFontFamily| command,
and it should be for the appropriate encoding/family combination.

\begin{decl}
|\DeclareFontShape| \arg{encoding} \arg{family} \arg{series}
                    \arg{shape}\\
        \null\hfill \arg{loading-info} \arg{loading-settings}
\end{decl}

Declares a font shape combination; here \m{loading-info} contains the
information that combines sizes with external fonts. The syntax is
complex and is described in Section~\ref{sec:loadinfo} below.

The \m{loading-settings} are executed after loading any font with this
font shape.  They are executed immediately after the `loading-settings'
which were declared by |\DeclareFontFamily| and so they can be used to
overwrite the settings made at the family level.

Checks that the combination \m{encoding}\m{family} was previously
declared via |\DeclareFontFamily|.

Example:
\begin{verbatim}
   \DeclareFontShape{OT1}{cmr}{m}{sl}{%
             <<5-8>> sub * cmr/m/n
             <<8>> cmsl8
             <<9>> cmsl9
             <<10>> <<10.95>> cmsl10
             <<12>> <<14.4>> <<17.28>> <<20.74>> <<24.88>> cmsl12
             }{}
\end{verbatim}
The file can contain any number of |\DeclareFontShape| commands, which
should be for the appropriate \m{encoding} and \m{family}.

\NEWfeature{1996/06/01}
The font family declarations for the |OT1|-encoded fonts now all
contain:
\begin{verbatim}
   \hyphenchar\font=`\-
\end{verbatim}
This enables the use of an alternative |\hyphenchar| in other encodings
whilst maintaining the correct value for all fonts.

\NEWfeature{2020/02/02} According to NFSS conventions the series value
should be a combination of weight and width abbreviated each with one or
two letters as described on page~\pageref{page:seriesvalues}. In
particular it should not contain an ``\texttt{m}'' unless it
consists of just one character. In the past incorrect values such as
``\texttt{cm}'' were simply accepted, but since this now leads to
problems with the extended mechanism, the correct syntax is now
enforced.

More exactly, if the series values is a member of a specific set of
values (\texttt{ulm}, \texttt{elm}, \texttt{lm}, \texttt{slm},
\texttt{mm}, \texttt{sbm}, \texttt{bm}, \texttt{ebm}, \texttt{ubm},
\texttt{muc}, \texttt{mec}, \texttt{mc}, \texttt{msc}, \texttt{msx},
\texttt{mx}, \texttt{mex} or \texttt{mux}) it is assumed to be in
incorrect NFSS notation and so a warning is given and a surplus
``\texttt{m}'' is dropped.  Other values are not touched to allow for
the usage of values like ``\texttt{semibold}'' or ``\texttt{medium}'' as
used by the \texttt{autoinst} program.


\subsection{Font file loading information}
\label{sec:loadinfo}

The information which tells \LaTeX{} exactly which font (\texttt{.tfm})
files to load is contained in the \m{loading-info} part of a
|\DeclareFontShape| declaration. This part consists of one or more
\m{fontshape-decl}s, each of which has the following form:

\begin{center}
  \begin{tabular}{r@{ $::=$ }l}
    \m{fontshape-decl} &  \m{size-infos} \m{font-info} \\
    \m{size-infos}     &  \m{size-infos} \m{size-info} $\mid$
                          \m{size-info} \\
    \m{size-info}      & ``|<<|''  \m{number-or-range} ``|>>|'' \\
    \m{font-info}      & $[$ \m{size-function} ``|*|''  $]$
                         $[$ ``|[|'' \m{optarg} ``|]|'' $]$ \m{fontarg} \\
  \end{tabular}
\end{center}
The \m{number-or-range} denotes the size or size-range for which this
entry applies.

If it contains a hyphen it is a range: lower bound on the left (if
missing, zero implied), upper bound on the right (if missing, $\infty$
implied). For ranges, the upper bound is \emph{not} included in the
range and the lower bound is.

Examples:
\begin{center}
  \begin{tabular}{lll}
    |<<10>>|     & simple size & 10pt only \\
    |<<-8>>|     & range       & all sizes less than 8pt \\
    |<<8-14.4>>| & range       & all sizes greater than or equal to 8pt \\
                 &             & \ but less than 14.4pt \\
    |<<14.4->>|  & range       & all sizes greater than or equal 14.4pt
\end{tabular}
\end{center}
If more than one \m{size-info} entry follows without any intervening
\m{font-info}, they all share the next \m{font-info}.

The \m{size-function}, if present, handles the use of \m{font-info}.  If
not present, the `empty' \m{size-function} is assumed.

All the \m{size-info}s are inspected in the order in which they appear
in the font shape declaration. If a \m{size-info} matches the requested
size, its \m{size-function} is executed. If |\external@font| is
non-empty afterwards this process stops, otherwise the next
\m{size-info} is inspected. (See also |\DeclareSizeFunction|.)

If this process does not lead to a non-empty |\external@font|, \LaTeX{}
tries the nearest simple size. If the entry contains only ranges an
error is returned.

\subsection{Size functions}
\label{sec:sizefunct}

\LaTeX{} provides the following size functions, whose `inputs' are
\m{fontarg} and \m{optarg} (when present).

\begin{description}
\item[`' (empty)]
  Load the external font \m{fontarg} at the user-requested size. If
  \m{optarg} is present, it is used as the scale-factor.

\item[s]
  Like the empty function but without terminal warnings, only
  loggings.

\item[gen]
  Generates the external font from \m{fontarg} followed by the
  user-requested size, e.g.~|<<8>> <<9>> <<10>> gen * cmtt|

\item[sgen]
  Like the `gen' function but without terminal warnings, only loggings.

\item[genb]
  \NEWfeature{1995/12/01}
  Generates the external font from \m{fontarg} followed by the
  user-requested size, using the conventions of the `ec' fonts.
  e.g.~|<<10.98>> genb * dctt| produces |dctt1098|.

\item[sgenb]
  \NEWfeature{1995/12/01}
  Like the `genb' function but without terminal warnings, only loggings.

\item[sub]
  Tries to load a font from a different font shape declaration given by
  \m{fontarg} in the form \m{family}|/|\m{series}|/|\m{shape}.

\item[ssub]
  Silent variant of `sub', only loggings.

\item[alias]
  \NEWfeature{2019/10/15}
  Same as `ssub' but with a different logging message. Intended for
  cases where the substitution is only done to change the name, e.g.,
  going from \texttt{regular} series to the official name \texttt{m}. In
  that case given a warning that some shape is not found is not correct.

\item[subf]
  Like the empty function but issues a warning that it has to substitute
  the external font \m{fontarg} because the desired font shape was not
  available in the requested size.

\item[ssubf]
  Silent variant of `subf', only loggings.

\item[fixed]
  Load font \m{fontarg} as is, disregarding the user-requested size.  If
  present, \m{optarg} gives the ``at \ldots pt'' size to be used.

\item[sfixed]
  Silent variant of `fixed', only loggings.
\end{description}

Examples for the use of most of the above size functions can be found in
the file |cmfonts.fdd|---the source for the standard |.fd| files
describing the Computer Modern fonts by Donald Knuth.


\begin{decl}
  |\DeclareSizeFunction| \arg{name} \arg{code}
\end{decl}

Declares a size-function \m{name} for use in |\DeclareFontShape|
commands.  The interface is still under development but there should be
no real need to a define new size functions.

The \m{code} is executed when the size or size-range in
|\DeclareFontShape| matches the user-requested size.

The arguments of the size-function are automatically parsed and placed
into |\mandatory@arg| and |\optional@arg| for use in \m{code}. Also
available, of course, is |\f@size|, which is the user-requested size.

To signal success \m{code} must define the command |\external@font| to
contain the external name and any scaling options (if present) for the
font to be loaded.

This example sets up the `empty' size function (simplified):
\begin{verbatim}
   \DeclareSizeFunction{}
           {\edef\external@font{\mandatory@arg\space at\f@size}
\end{verbatim}


\section{Encodings}
\label{sec:encode}

This section explains how to declare and use new font encodings and how
to declare commands for use with particular encodings.

\subsection{The \textsf{fontenc} package}

Users can select new font encodings using the |fontenc| package.  The
|fontenc| package has options for encodings; the last option becomes the
default encoding.  For example, to use the |OT2| (Washington University
Cyrillic encoding) and |T1| encodings, with |T1| as the default, an
author types:
\begin{verbatim}
   \usepackage[OT2,T1]{fontenc}
\end{verbatim}

\NEWdescription{1997/12/01}
For each font encoding \m{ENC} given as an option, this package loads
the \emph{encoding definition} (|<enc>enc.def|, with an all lower-case
name) file; it also sets |\encodingdefault| to be the last encoding in
the option list.

The declarations in the encoding definition file |<enc>enc.def| for
encoding \m{ENC} are responsible for declaring this encoding and telling
\LaTeX{} how to produce characters in this encoding; this file should
contain nothing else (see Section~\ref{sec:encode.def}.

The standard \LaTeX{} format declares the |OT1| and |T1| text encodings
by inputting the files |ot1enc.def| and |t1enc.def|; it also sets up
various defaults which require that |OT1|-encoded fonts are available.
Other encoding set-ups might be added to the distribution at a later
stage.

Thus the example above loads the files |ot2enc.def| and |t1enc.def| and
sets |\encodingdefault| to |T1|.

\emph{Warning}: If you wish to use |T1|-encoded fonts other than the
`cmr' family then you may need to load the package (e.g.~\texttt{times})
that selects the fonts \emph{before} loading \texttt{fontenc} (this
prevents the system from attempting to load any |T1|-encoded fonts from
the `cmr' family).

\subsection{Encoding definition file commands}
\label{sec:encode.def}

\emph{Note}: An encoding definition file should contain only commands
from this subsection.

\NEWdescription{2019/07/10}
As an exception it may also contain a |\DeclareFontSubstitution|
declaration (described in \ref{sec:encoding-defaults}) to specify how
font substitution for this encoding should be handled.  In that case it
is important that the values used point to a font that is guaranteed to
be available on all \LaTeX{} installations.\footnote{If the font
  encoding file is made available as part of a CTAN bundle, that could
  be a font that is provided together with that bundle, but it should
  not point to font which requires further installation steps and
  therefore may or may not be installed.}

\NEWdescription{1997/12/01}
As with the font definition file commands, it is also possible (although
normally not necessary) to use these declarations directly within a
class or package file.

\emph{Warning}: Some aspects of the contents of font definition files
are still under development.  Therefore, the current versions of the
files |ot1enc.def| and |t1enc.def| are temporary versions and should not
be used as models for producing further such files.  For further
information you should read the documentation in |ltoutenc.dtx|.

\begin{decl}
  |\ProvidesFile{<file-name>}[<release-info>]|
\end{decl}
The file should announce itself with a |\ProvidesFile| command,
described in \emph{\clsguide}.  For example:
\begin{verbatim}
   \ProvidesFile{ot2enc.def}
                [1994/06/01 Washington University Cyrillic encoding]
\end{verbatim}


\begin{decl}
  |\DeclareFontEncoding| \arg{encoding} \arg{text-settings}
                         \arg{math-settings}
\end{decl}

Declares a new encoding scheme \m{encoding}.

The \m{text-settings} are declarations which are executed every time
|\selectfont| changes the encoding to be \m{encoding}.

The \m{math-settings} are similar but are for math alphabets. They are
executed whenever a math alphabet with this encoding is called.

\NEWfeature{1998/12/01}
It also saves the value of \m{encoding} in the macro
|\LastDeclaredEncoding|.

\newpage

Example:
\begin{verbatim}
   \DeclareFontEncoding{OT1}{}{}
\end{verbatim}

\NEWfeature{2021/06/01}
Fonts in encoding \texttt{TS1} are usually not implementing the full encoding but only a subset.
This subset should be declared  with a |\DeclareEncodingSubset|
declaration:

\begin{decl}
  |\DeclareEncodingSubset| \arg{encoding}
                           \arg{font family}
                           \arg{subset number}
\end{decl}
This should even be done if the font is implementing the full
\texttt{TS1} encoding; see page~\pageref{page:declareencodingsubset} for
further details.

Some author commands need to change their definition depending on which
encoding is currently in use.  For example, in the |OT1| encoding, the
letter `\AE' is in slot |"1D|, whereas in the |T1| encoding it is in
slot |"C6|.  So the definition of |\AE| has to change depending on
whether the current encoding is |OT1| or |T1|.  The following commands
allow this to happen.

\begin{decl}
  |\DeclareTextCommand| \arg{cmd} \arg{encoding}
                        \oarg{num} \oarg{default} \arg{definition}
\end{decl}
This command is like |\newcommand|, except that it defines a command
which is specific to one encoding.  For example, the definition of |\k|
in the |T1| encoding is:
\begin{verbatim}
   \DeclareTextCommand{\k}{T1}[1]
      {\hmode@bgroup\ooalign{\null#1\crcr\hidewidth\char12}\egroup}
\end{verbatim}
|\DeclareTextCommand| takes the same optional arguments as
|\newcommand|.

The resulting command is robust, even if the code in \m{definition} is
fragile.

It does not produce an error if the command has already been defined but
logs the redefinition in the transcript file.

\begin{decl}[1994/12/01]
  |\ProvideTextCommand| \arg{cmd} \arg{encoding}
                        \oarg{num} \oarg{default} \arg{definition}
\end{decl}
This command is the same as |\DeclareTextCommand|, except that if
\m{cmd} is already defined in encoding \m{encoding}, then the definition
is ignored.

\begin{decl}
  |\DeclareTextSymbol| \arg{cmd} \arg{encoding} \arg{slot}
\end{decl}
This command defines a text symbol with slot \m{slot} in the encoding.
For example, the definition of |\ss| in the |OT1| encoding is:
\begin{verbatim}
   \DeclareTextSymbol{\ss}{OT1}{25}
\end{verbatim}
It does not produce an error if the command has already
been defined but logs the redefinition in the transcript file.

\newpage

\begin{decl}
  |\DeclareTextAccent| \arg{cmd} \arg{encoding} \arg{slot}
\end{decl}
This command declares a text accent, with the accent taken from slot
\m{slot} in the encoding.  For example, the definition of |\"| in the
|OT1| encoding is:
\begin{verbatim}
   \DeclareTextAccent{\"}{OT1}{127}
\end{verbatim}
It does not produce an error if the command has already been defined but
logs the redefinition in the transcript file.

\begin{decl}
  |\DeclareTextComposite| \arg{cmd} \arg{encoding} \arg{letter}
                          \arg{slot}
\end{decl}
This command declares that the composite letter formed from applying
\m{cmd} to \m{letter} is defined to be simply slot \m{slot} in the
encoding.  The \m{letter} should be a single letter (such as |a|) or a
single command (such as |\i|).

For example, the definition of |\'{a}| in the |T1| encoding could be
declared like this:
\begin{verbatim}
   \DeclareTextComposite{\'}{T1}{a}{225}
\end{verbatim}

The \m{cmd} will normally have been previously declared for this
encoding, either by using |\DeclareTextAccent|, or as a one-argument
|\DeclareTextCommand|.

\begin{decl}[1994/12/01]
  |\DeclareTextCompositeCommand| \arg{cmd} \arg{encoding} \arg{letter}
                                 \arg{definition}
\end{decl}
This is a more general form of |\DeclareTextComposite|, which allows for
an arbitrary \m{definition}, not just a \m{slot}.  The main use for this
is to allow accents on |i| to act like accents on |\i|, for example:
\begin{verbatim}
   \DeclareTextCompositeCommand{\'}{OT1}{i}{\'\i}
\end{verbatim}
It has the same restrictions as |\DeclareTextComposite|.


\begin{decl}[1998/12/01]
  |\LastDeclaredEncoding|
\end{decl}
This holds the name of the last encoding declared via
|\DeclareFontEncoding| (this should also be the currently most efficient
encoding).  It can be used in the \m{encoding} argument of the above
declarations in place of explicitly mentioning the encoding, e.g.
\begin{verbatim}
   \DeclareFontEncoding{T1}{}{}
   \DeclareTextAccent{\`}{\LastDeclaredEncoding}{0}
   \DeclareTextAccent{\'}{\LastDeclaredEncoding}{1}
\end{verbatim}
This can be useful in cases where encoding files sharing common code
are generated from one source.

\subsection{Default definitions}

\NEWdescription{1997/12/01}
The declarations used in encoding definition files define
encoding-specific commands but they do not allow those commands to be
used without explicitly changing the encoding.  For some commands, such
as symbols, this is not enough.  For example, the~|OMS| encoding
contains the symbol~`\S', but we need to be able to use the command~|\S|
whatever the current encoding may be, without explicitly selecting the
encoding~|OMS|.

\NEWdescription{1997/12/01}
To allow this, \LaTeX{} has commands that declare default definitions
for commands; these defaults are used when the command is not defined in
the current encoding.  For example, the default encoding for~|\S|
is~|OMS|, and so in an encoding (such as |OT1|) which does not
contain~|\S|, the~|OMS| encoding is selected in order to access this
glyph.  But in an encoding (such as~|T1|) which does contain~|\S|, the
glyph in that encoding is used.  The standard \LaTeXe{} format sets up
several such defaults using the following encodings: |OT1|,~|OMS|
and~|OML|.

\emph{Warning}: These commands should \emph{not} occur in encoding
definition files, since those files should declare only commands for use
when that encoding has been selected.  They should instead be placed in
packages; they must, of course, always refer to encodings that are known
to be available.

\begin{decl}[1994/12/01]
  |\DeclareTextCommandDefault| \arg{cmd} \arg{definition}
\end{decl}
This command allows an encoding-specific command to be given a default
definition.  For example, the default definition for |\copyright| is
defined be a circled `c' with:
\begin{verbatim}
   \DeclareTextCommandDefault{\copyright}{\textcircled{c}}
\end{verbatim}
\begin{decl}[1994/12/01]
  |\DeclareTextAccentDefault| \arg{cmd} \arg{encoding} \\
  |\DeclareTextSymbolDefault| \arg{cmd} \arg{encoding}
\end{decl}
These commands allow an encoding-specific command to be given a default
encoding.  For example, the default encoding for |\"| and |\ae| is set
to be |OT1| by:
\begin{verbatim}
   \DeclareTextAccentDefault{\"}{OT1}
   \DeclareTextSymbolDefault{\ae}{OT1}
\end{verbatim}
Note that |\DeclareTextAccentDefault| can be used on any one-argument
encoding-specific command, not just those defined with
|\DeclareTextAccent|.  Similarly, |\DeclareTextSymbolDefault| can be
used on any encoding-specific command with no arguments, not just those
defined with |\DeclareTextSymbol|.

For more examples of these definitions, see |ltoutenc.dtx|.

\newpage

\begin{decl}[1994/12/01]
  |\ProvideTextCommandDefault| \arg{cmd} \arg{definition}
\end{decl}
This command is the same as |\DeclareTextCommandDefault|, except that if
the command already has a default definition, then the definition is
ignored.  This is useful to give `faked' definitions of symbols which
may be given `real' definitions by other packages.  For example, a
package might give a fake definition of |\textonequarter| by saying:
\begin{verbatim}
   \ProvideTextCommandDefault{\textonequarter}{$\m@th\frac14$}
\end{verbatim}

\subsection{Encoding defaults} \label{sec:encoding-defaults}

\begin{decl}
  |\DeclareFontEncodingDefaults| \arg{text-settings} \arg{math-settings}
\end{decl}

Declares \m{text-settings} and \m{math-settings} for all encoding
schemes.  These are executed before the encoding scheme dependent ones
are executed so that one can use the defaults for the major cases and
overwrite them if necessary using |\DeclareFontEncoding|.

If |\relax| is used as an argument, the current setting of this default
is left unchanged.

This example is used by amsfonts.sty for accent positioning; it changes
only the math settings:
\begin{verbatim}
   \DeclareFontEncodingDefaults{\relax}{\def\accentclass@{7}}
\end{verbatim}


\begin{decl}
  |\DeclareFontSubstitution|  \arg{encoding} \arg{family} \arg{series}
                              \arg{shape}
\end{decl}

Declares the default values for font substitution which will be used
when a font with encoding \m{encoding} should be loaded but no font can
be found with the current attributes.

These substitutions are local to the encoding scheme because the
encoding scheme is never substituted!  They are tried in the order
\m{shape} then \m{series} and finally \m{family}.

\NEWdescription{2019/07/10}
This declaration is normally done in an encoding definition file
(see~\ref{sec:encode.def}), but can also be used in a class file or the
document preamble to alter the default for a specific encoding.

If no defaults are set up for an encoding, the values given by
|\DeclareErrorFont| are used.

The font specification for \m{encoding}\m{family}\m{series}\m{shape}
must have been defined by |\DeclareFontShape| before the
|\begin{document}| is reached.

Example:
\begin{verbatim}
   \DeclareFontSubstitution{T1}{cmr}{m}{n}
\end{verbatim}

\subsection{Case changing}
\label{sec:case}

\begin{decl}
  |\MakeUppercase| \arg{text} \\
  |\MakeLowercase| \arg{text}
\end{decl}

\NEWfeature{1995/06/01}
\TeX{} provides the two primitives |\uppercase| and |\lowercase| for
changing the case of text.  Unfortunately, these \TeX{} primitives do
not change the case of characters accessed by commands like |\ae| or
|\aa|.  To overcome this problem, \LaTeX{} provides these two commands.

In the long run, we would like to use all-caps fonts rather than any
command like |\MakeUppercase| but this is not possible at the moment
because such fonts do not exist.

For further details, see \texttt{clsguide.tex}.

\NEWdescription{1999/04/23}
In order that upper/lower-casing will work reasonably well, and in order
to provide any correct hyphenation, \LaTeXe{} \emph{must} use,
throughout a document, the same fixed table for changing case.  The
table used is designed for the font encoding |T1|; this works well with
the standard \TeX{} fonts for all Latin alphabets but will cause
problems when using other alphabets.  As an experiment, it has now been
extended for use with some Cyrillic encodings.

\section{Miscellanea}
\label{sec:misc}

This section covers the remaining font commands in \LaTeX{} and some
other issues.

\subsection{Font substitution}

\begin{decl}
  |\DeclareErrorFont| \arg{encoding} \arg{family} \arg{series}
                      \arg{shape} \arg{size}
\end{decl}

Declares \m{encoding}\m{family}\m{series}\m{shape} to be the font shape
used in cases where the standard substitution mechanism fails
(i.e.~would loop). For the standard mechanism see the command
|\DeclareFontSubstitution| above.

The font specification for \m{encoding}\m{family}\m{series}\m{shape}
must have been defined by |\DeclareFontShape| before the
|\begin{document}| is reached.

Example:
\begin{verbatim}
   \DeclareErrorFont{OT1}{cmr}{m}{n}{10}
\end{verbatim}

\NEWdescription{2019/07/10}
This declaration is a system wide fallback and it should normally not be
changed, in particular it does not belong into font encoding definition
files but rather into the \LaTeX{} format. It is normally set up in
\texttt{fonttext.cfg}. Adjustments on a per encoding base should be made
through |\DeclareFontSubstitution| instead!

\begin{decl}
  |\fontsubfuzz|
\end{decl}

This parameter is used to decide whether or not to produce a terminal
warning if a font size substitution takes place. If the difference
between the requested and the chosen size is less than |\fontsubfuzz|
the warning is only written to the transcript file. The default value is
|0.4pt|.  This can be redefined with |\renewcommand|, for example:
\begin{verbatim}
   \renewcommand{\fontsubfuzz}{0pt}   % always warn
\end{verbatim}

\subsection{Preloading}

\begin{decl}
  |\DeclarePreloadSizes| \arg{encoding} \arg{family} \arg{series}
                         \arg{shape}
\arg{size-list}
\end{decl}

Specifies the fonts that should be preloaded by the format.  These
commands should be put in a |preload.cfg| file, which is read in when
the \LaTeX{} format is being built.  Read |preload.dtx| for more
information on how to built such a configuration file.

Example:
\begin{verbatim}
   \DeclarePreloadSizes{OT1}{cmr}{m}{sl}{10,10.95,12}
\end{verbatim}

\NEWdescription{2019/07/10}
Preloading is really an artifact of the days when loading fonts while
processing a document contributed substantially to the processing
time. These days it is usually best not to use this mechanism any more.

\subsection{Accented characters}

\NEWdescription{1996/06/01}
Accented characters in \LaTeX{} can be produced using commands such as
|\"a| etc.  The precise effect of such commands depends on the font
encoding being used.  When using a font encoding that contains the
accented characters as individual glyphs (such as the |T1| encoding, in
the case of |\"a|) words that contain such accented characters can be
automatically hyphenated.  For font encodings that do not contain the
requested individual glyph (such as the |OT1| encoding) such a command
invokes typesetting instructions that produce the accented character as
a combination of character glyphs and diacritical marks in the font.  In
most cases this involves a call to the \TeX{} primitive |\accent|.
Glyphs constructed as composites in this way inhibit hyphenation of the
current word; this is one reason why the |T1| encoding is preferable to
the original \TeX{} font encoding |OT1|.

It is important to understand that commands like |\"a| in \LaTeXe{}
represent just a name for a single glyph (in this case `umlaut a') and
contain no information about how to typeset that glyph---thus it does
\emph{not} mean `put two dots on top of the character a'.  The decision
as to what typesetting routine to use will depend on the encoding of the
current font and so this decision is taken at the last minute.  Indeed,
it is possible that the same input will be typeset in more than one way
in the same document; for example, text in section headings may also
appear in table of contents and in running heads; and each of these may
use a font with a different encoding.

For this reason the notation |\"a| is \emph{not} equivalent to:
\begin{verbatim}
   \newcommand \chara {a}     \"\chara
\end{verbatim}
In the latter case, \LaTeX{} does not expand the macro |\chara| but
simply compares the notation (the string |\"\chara|) to its list of
known composite notations in the current encoding; when it fails to find
|\"\chara| it does the best it can and invokes the typesetting
instructions that put the umlaut accent on top of the expansion of
|\chara|. Thus, even if the font actually contains `\"a' as an
individual glyph, it will not be used.

The low-level accent commands in \LaTeX{} are defined in such a way that
it is possible to combine a diacritical mark from one font with a glyph
from another font; for example, |\"\textparagraph| will produce
\"\textparagraph.  The umlaut here is taken from the |OT1| encoded font
|cmr10| whilst the paragraph sign is from the |OMS| encoded font
|cmsy10|. (This example may be typographically silly but better ones
would involve font encodings like |OT2| (Cyrillic) that might not be
available at every site.)

There are, however, restrictions on the font-changing commands that will
work within the argument to such an accent command.  These are
\TeX{}nical in the sense that they follow from the way that \TeX{}'s
|\accent| primitive works, allowing only a special class of commands
between the accent and the accented character.

The following are examples of commands that will not work correctly as
the accent will appear above a space: the font commands with text
arguments (|\textbf{...}| and friends); all the font size declarations
(|\fontsize| and |\Large|, etc.); |\usefont| and declarations that
depend on it, such as |\normalfont|; box commands (e.g.~|\mbox{...}|).

The lower-level font declarations that set the attributes family, series
and shape (such as |\fontshape{sl}\selectfont|) will produce correct
typesetting, as will the default declarations such as |\bfseries|.

\subsection{Naming conventions}

\begin{itemize}
\item
  Math alphabet commands all start with |\math...|: examples are
  |\mathbf|, |\mathcal|, etc.

\item
  The text font changing commands with arguments all start with
  |\text...|: e.g.~|\textbf| and |\textrm|.  The exception to this is
  |\emph|, since it occurs very commonly in author documents and so
  deserves a shorter name.

\item
  Names for encoding schemes are strings of up to three letters (all
  upper case) plus digits.

  The \LaTeX\ Project reserves the use of encodings starting with the
  following letters: |T| (standard 256-long text encodings), |TS|
  (symbols that are designed to extend a corresponding |T| encoding),
  |X| (text encodings that do not conform to the strict requirements for
  |T| encodings), |M| (standard 256-long math encodings), |S| (other
  symbol encodings), |A| (other special applications), |OT| (standard
  128-long text encodings) and |OM| (standard 128-long math encodings).

  Please do not use the above starting letters for non-portable
  encodings.  If new standard encodings emerge then we shall add them in
  a later release of \LaTeX.

  Encoding schemes which are local to a site or a system should start
  with |L|, experimental encodings intended for wide distribution will
  start with |E|, whilst |U| is for Unknown or Unclassified encodings.

\item
  \NEWdescription{2019/10/15}
  Font family names should contain only upper and lower case letters and
  hyphen characters.  Where possible, these should conform to the
  \emph{Filenames for fonts} font naming scheme of the scheme
  implemented by \texttt{autoinst} with suffixes such as \texttt{-LF},
  \texttt{-OsF}, etc.\ to indicate different figure styles.

\item
  \NEWdescription{2019/10/15}
  Font series names should contain up to four lower case letters. If at
  all possible standard names as suggested in
  Section~\ref{sec:textfontattributes} should be used. Font specific
  names such as \texttt{regular} or \texttt{black}, etc.\ should be at
  least aliased to a corresponding standard name.

\item
  \NEWdescription{2019/10/15}
  Font shapes should contain up to four letters lower case. Use the
  names suggested in Section~\ref{sec:textfontattributes}.

\item
  Names for symbol fonts are built from lower and upper case letters
  with no restriction.
\end{itemize}

Whenever possible, you should use the series and shape names suggested
in \emph{\LaTeXcomp} since this will make it easier to combine new fonts
with existing fonts.

\NEWdescription{1994/12/01}
Where possible, text symbols should be named as |\text| followed by the
Adobe glyph name: for example |\textonequarter| or |\textsterling|.
Similarly, math symbols should be named as |\math| followed by the glyph
name, for example |\mathonequarter| or |\mathsterling|.  Commands which
can be used in text or math can then be defined using |\ifmmode|, for
example:
\begin{verbatim}
   \DeclareRobustCommand{\pounds}{%
      \ifmmode \mathsterling \else \textsterling \fi
   }
\end{verbatim}
   Note that commands defined in this way must be robust, in case they
   get put into a section title or other moving argument.

\subsection{The order of declaration}

\NEWdescription{2019/10/15}
\NFSS{} forces you to give all declarations in a specific order so that
it can check whether you have specified all necessary information.  If
you declare objects in the wrong order, it will complain.  Here are the
dependencies that you have to obey:
\begin{itemize}
\item
  |\DeclareFontFamily| checks that the encoding scheme was previously
  declared with |\DeclareFontEncoding|.

\item
  |\DeclareFontShape| checks that the font family was declared to be
  available in the requested encoding (|\DeclareFontFamily|).

\item
  |\DeclareSymbolFont| checks that the encoding scheme is valid.

\item
  |\SetSymbolFont| additionally ensures that the requested math version
  was declared (|\DeclareMathVersion|) and that the requested symbol
  font was declared (|\DeclareSymbolFont|).

\item
  |\DeclareSymbolFontAlphabet| checks that the command name for the
  alphabet identifier can be used and that the symbol font was declared.

\item
  |\DeclareMathAlphabet| checks that the chosen command name can be used
  and that the encoding scheme was declared.

\item
  |\SetMathAlphabet| checks that the alphabet identifier was previously
  declared with |\DeclareMathAlphabet| or |\DeclareSymbolFontAlphabet|
  and that the math version and the encoding scheme are known.

\item
  |\DeclareMathSymbol| makes sure that the command name can be used
  (i.e., is undefined or was previously declared to be a math symbol)
  and that the symbol font was previously declared.

\item
  When the |\begin{document}| command is reached, \NFSS{} makes some
    additional checks---for example, verifying that substitution
    defaults for every encoding scheme point to known font shape group
    declarations.
\end{itemize}



\subsection{Document font metafamilies}

\LaTeX{} knows three font families accessible through \cs{rmfamily} (the
serifed font family of the document), \cs{sffamily} (the sans serif
family), and \cs{ttfamily} (the documents mono-spaced/typewriter family).
These abstract families, i.e., \texttt{rm}, \texttt{sf}, and \texttt{tt},
are called ``metafamilies''.
The concrete font families selected for them depend on the settings of
\cs{rmdefault}, \cs{sfdefault}, and \cs{ttdefault}.

Besides these metafamilies it is, of course, possible to access further
font families by selecting them through
\cs{fontfamily}\arg{name}\cs{selectfont}.

\NEWfeature{2025/11/01}
In some cases it is helpful to know which of the three metafamilies
(if any) is currently used for typesetting, and this information is
available for programmers in \cs{@currentmetafamily}. The command
returns either \texttt{rm}, \texttt{sf}, \texttt{tt}, or \texttt{??} (in case
none of the metafamilies but some other font family is currently used).

As one application of this there is \cs{@restoremetafamily}. If the
current metafamily is \meta{name} it executes \cs{\meta{name}family},
e.g., \cs{sffamily}, and that then executes the hook
\texttt{\meta{name}family} besides other re-initializations. This can be
useful if that hook contains conditional code and the condition has
changed and therefore requires re-initialization.


\subsection{Font series defaults per document metafamily}

\NEWfeature{2020/02/02}
With additional weights and widths being available in many font families
nowadays, it is more likely that somebody will want to match, say, a
medium weight serif family with a semi-light sans serif family, or that
with one family one wants to use the bold-extended face when |\textbf|
is used, while with another it should be bold (not extended) or
semi-bold, etc.  The default values can be altered using the
|\DeclareFontSeriesDefault| declaration in packages or document
preambles:
\begin{decl}
  |\DeclareFontSeriesDefault| \oarg{metafamily}
  \arg{metaseries} \arg{series value}
\end{decl}
This declaration takes three arguments:
\begin{description}
\item[Metafamily interface:] Can be either |rm|, |sf| or |tt|.  This is
  optional and if not present the next two arguments apply to the
  overall default.
\item[Metaseries interface:] Can be |md| or |bf|.
\item[Series value:] This is the value that is going to be used when the
  combination of \m{metafamily} and \m{metaseries} is requested.
\end{description}
For example,
\begin{verbatim}
   \DeclareFontSeriesDefault[rm]{bf}{sb}
\end{verbatim}
would use |sb| (semi-bold) when |\rmfamily\bfseries| is requested in the
document.

\subsection{Handling of current and requested font series and shape}

In the original NFSS implementation, the series was a single attribute
stored in |\f@series| and so one always had to specify both weight and
width together.  Hence, it was impossible to typeset a paragraph in a
condensed font and inside have a few words in bold weight (but still
condensed) without doing this manually by requesting
|\fontseries{bc}\selectfont|.

\NEWfeature{2020/02/02}
The new implementation now works differently by looking both at the
current value of |\f@series| and the requested new series and out of
that combination selects a resulting series value.  Thus, if the current
series is |c| and we ask for |b|, we now get |bc|.  This is done by
consulting a simple lookup table where entries can be added or changed
with |\DeclareFontSeriesChangeRule|:
\begin{decl}
  |\DeclareFontSeriesChangeRule| %
  \arg{current series} \arg{requested series} \\
  \leavevmode\hfill    \arg{result} \arg{alternative result}
\end{decl}

The \m{current series} is the value currently stored in |\f@series|,
\m{requested series} is the new series requested, \m{result} is the
combined value if it exists for the given font family, and
\m{alternative result} is a fallback in case \m{result} doesn't exist.
The example above now looks like this:
\begin{verbatim}
   \DeclareFontSeriesChangeRule{c}{b}{bc}{}
\end{verbatim}
which means: switch to the |bc| series if |c| is current and |b| is
(additionally) requested, and if the current font doesn't have the
combination, start the normal font substitution, i.e., switch back shape
to |n| and if this combination doesn't succeed, switch back series to
|m| as well, ending up with |m/n|.

Another example is:
\begin{verbatim}
   \DeclareFontSeriesChangeRule{bc}{sc}{bsc}{bc}
\end{verbatim}
which means: if the current series is bold condensed (|bc|) and
semi-condensed (|sc|) is requested (additionally), try bold
semi-condensed (|bsc|) if available but stay with bold condensed if not.

A special value is |m| which is used to reset both weight and width.  In
order to reset only one of them, the special values |?m| (reset width)
and |m?| (reset weight) are provided, e.g.:
\begin{verbatim}
   \DeclareFontSeriesChangeRule{bc}{m?}{c}{}
\end{verbatim}

The corresponding macro |\DeclareFontShapeChangeRule| is also provided
for setting database entries for font shapes:
\begin{decl}
  |\DeclareFontShapeChangeRule| %
  \arg{current shape} \arg{requested shape} \\
  \leavevmode\hfill   \arg{result} \arg{alternative result}
\end{decl}

An example would be:
\begin{verbatim}
   \DeclareFontShapeChangeRule{it}{sc}{scit}{scsl}
\end{verbatim}
If italics is the current shape and small caps is requested, switch to
|scit| (small caps italics) and if that doesn't exist, try |scsl| (small
caps slanted).

Finally, it is also possible to overrule the entries in the lookup
tables and forcibly select a series or shape with:
\begin{decl}
  |\fontseriesforce| \arg{series} \qquad |\fontshapeforce| \arg{shape}
\end{decl}

With the example above for the |c| series, issuing |\fontseriesforce{b}|
means that the series switches to |b| and not to |bc|.  Same applies to
|\fontshapeforce|.

\subsection{Handling of nested emphasis}

\begin{decl}
  |\DeclareEmphSequence| \arg{list of font declarations}
\end{decl}

\NEWfeature{2020/02/02}
This declaration takes a comma separated list of font declarations each
specifying how increasing levels of emphasis should be handled.  For
example:
\begin{verbatim}
   \DeclareEmphSequence{\itshape,%
                        \upshape\scshape,%
                        \itshape}
\end{verbatim}
uses italics for the first, small capitals for the second, and italic
small capitals for the third level.  If there are more nesting levels
than provided, declarations stored in |\emreset| (by default
|\ulcshape\upshape|) are used for the next level and then the list
restarts.

\subsection{Providing font family substitutions}

\begin{decl}
  |\DeclareFontFamilySubstitution| \arg{encoding}
                                   \arg{family}
                                   \arg{new-family}
\end{decl}

\NEWfeature{2020/02/02}
This declaration selects the font family \m{new-family} as replacement
for \m{family} in the font encoding \m{encoding}.  For example,
\begin{verbatim}
   \DeclareFontFamilySubstitution{LGR}
           {Montserrat-LF}{IBMPlexSans-TLF}
\end{verbatim}
tells \LaTeX{} to substitute the sans serif font |Montserrat-LF| in the
Greek encoding |LGR| with |IBMPlexSans-TLF| once requested in a
document.

\section{Additional text symbols -- \textsf{textcomp}}

\NEWfeature{2020/02/02}
For a long time the interface to additional text symbols and the text
companion encoding |TS1| in general was the \textsf{textcomp} package.
All the symbols provided by the \textsf{textcomp} package are now
available in \LaTeX{} kernel.  Furthermore, an intelligent substitution
mechanism has been implemented so that glyphs missing in some fonts are
automatically substituted with default glyphs that are sans serif if you
typeset in |\textsf| and monospaced if you typeset using |\texttt|.  In
the past they were always taken from Computer Modern Roman if
substitution was necessary.

{\sffamily This is most noticeable with |\oldstylenums| which are now
  taken from |TS1| so that you no longer get \legacyoldstylenums{1234}
  but \oldstylenums{1234} when typesetting in sans serif fonts \ttfamily
  and \oldstylenums{1234} when using typewriter fonts.}

\begin{decl}
  |\legacyoldstylenums| \arg{nums}\\
  |\UseLegacyTextSymbols|
\end{decl}
If there ever is a need to use the original (inferior) definition, then
that remains available as |\legacyoldstylenums|; and to fully revert to
the old behavior there is also |\UseLegacyTextSymbols|.  The latter
declaration reverts |\oldstylenums| and also changes the footnote
symbols, such as |\textdagger|, |\textparagraph|, etc., to pick up their
glyphs from the math fonts instead of the current text font (this means
they always keep the same shape and do not nicely blend in with the text
font).

%\newpage

The following tables show the macros available.  The next commands are
`constructed' accents and are built via \TeX{} macros:
\begin{center}
  \begin{tabular}[t]{@{}ll}
    \verb*|\capitalcedilla A| & \capitalcedilla A \\
    \verb*|\capitalogonek A|  & \capitalogonek A  \\
  \end{tabular}
  \quad
  \begin{tabular}[t]{@{}ll}
    \verb*|\textcircled a|    & \textcircled a
  \end{tabular}
\end{center}

These accents are available via font encoding.  The numbers in third row
show the slot number:
\begin{center}
  \begin{tabular}[t]{@{}p{0.32\linewidth}p{1em}p{2em}@{}}
    \verb|\capitalgrave|        & \capitalgrave{}        & 0 \\
    \verb|\capitalacute|        & \capitalacute{}        & 1 \\
    \verb|\capitalcircumflex|   & \capitalcircumflex{}   & 2 \\
    \verb|\capitaltilde|        & \capitaltilde{}        & 3 \\
    \verb|\capitaldieresis|     & \capitaldieresis{}     & 4 \\
    \verb|\capitalhungarumlaut| & \capitalhungarumlaut{} & 5 \\
    \verb|\capitalring|         & \capitalring{}         & 6 \\
    \verb|\capitalcaron|        & \capitalcaron{}        & 7
  \end{tabular}
  \quad
  \begin{tabular}[t]{@{}p{0.32\linewidth}p{1em}p{2em}@{}}
    \verb|\capitalbreve|        & \capitalbreve{}        & 8  \\
    \verb|\capitalmacron|       & \capitalmacron{}       & 9  \\
    \verb|\capitaldotaccent|    & \capitaldotaccent{}    & 10 \\
    \verb|\t|                   & \t{}                   & 26 \\
    \verb|\capitaltie|          & \capitaltie{}          & 27 \\
    \verb|\newtie|              & \newtie{}              & 28 \\
    \verb|\capitalnewtie|       & \capitalnewtie{}       & 29
  \end{tabular}
\end{center}

Table~\vref{tab:textcomp}
contains the full list of commands to access the text symbols.
%
% Tables~\vrefrange{tab:ts1-subset-always}{tab:ts1-subset-9-disable}
% contain the commands to access the text symbols.
%
Again,
the numbers are the slots in the encoding.

\iftrue
\begin{table}[!tp]
  \centering\footnotesize
  \renewcommand\arraystretch{1.07}
  \begin{tabular}[t]{@{}lp{1.5em}l@{}}
    \toprule
    \verb|\textcapitalcompwordmark|  & \textcapitalcompwordmark  & 23 \\
    \verb|\textascendercompwordmark| & \textascendercompwordmark & 31 \\
    \verb|\textquotestraightbase|    & \textquotestraightbase    & 13 \\
    \verb|\textquotestraightdblbase| & \textquotestraightdblbase & 18 \\
    \verb|\texttwelveudash|          & \texttwelveudash          & 21 \\
    \verb|\textthreequartersemdash|  & \textthreequartersemdash  & 22 \\
    \verb|\textleftarrow|            & \textleftarrow            & 24 \\
    \verb|\textrightarrow|           & \textrightarrow           & 25 \\
    \verb|\textblank|                & \textblank                & 32 \\
    \verb|\textdollar|               & \textdollar               & 36 \\
    \verb|\textquotesingle|          & \textquotesingle          & 39 \\
    \verb|\textasteriskcentered|     & \textasteriskcentered     & 42 \\
    \verb|\textdblhyphen|            & \textdblhyphen            & 45 \\
    \verb|\textfractionsolidus|      & \textfractionsolidus      & 47 \\
    \verb|\textzerooldstyle|         & \textzerooldstyle         & 48 \\
    \verb|\textoneoldstyle|          & \textoneoldstyle          & 49 \\
    \verb|\texttwooldstyle|          & \texttwooldstyle          & 50 \\
    \verb|\textthreeoldstyle|        & \textthreeoldstyle        & 51 \\
    \verb|\textfouroldstyle|         & \textfouroldstyle         & 52 \\
    \verb|\textfiveoldstyle|         & \textfiveoldstyle         & 53 \\
    \verb|\textsixoldstyle|          & \textsixoldstyle          & 54 \\
    \verb|\textsevenoldstyle|        & \textsevenoldstyle        & 55 \\
    \verb|\texteightoldstyle|        & \texteightoldstyle        & 56 \\
    \verb|\textnineoldstyle|         & \textnineoldstyle         & 57 \\
    \verb|\textlangle|               & \textlangle               & 60 \\
    \verb|\textminus|                & \textminus                & 61 \\
    \verb|\textrangle|               & \textrangle               & 62 \\
    \verb|\textmho|                  & \textmho                  & 77 \\
    \verb|\textbigcircle|            & \textbigcircle            & 79 \\
    \verb|\textohm|                  & \textohm                  & 87 \\
    \verb|\textlbrackdbl|            & \textlbrackdbl            & 91 \\
    \verb|\textrbrackdbl|            & \textrbrackdbl            & 93 \\
    \verb|\textuparrow|              & \textuparrow              & 94 \\
    \verb|\textdownarrow|            & \textdownarrow            & 95 \\
    \verb|\textasciigrave|           & \textasciigrave           & 96 \\
    \verb|\textborn|                 & \textborn                 & 98 \\
    \verb|\textdivorced|             & \textdivorced             & 99 \\
    \verb|\textdied|                 & \textdied                 & 100 \\
    \verb|\textleaf|                 & \textleaf                 & 108 \\
    \verb|\textmarried|              & \textmarried              & 109 \\
    \verb|\textmusicalnote|          & \textmusicalnote          & 110 \\
    \verb|\texttildelow|             & \texttildelow             & 126 \\
    \verb|\textdblhyphenchar|        & \textdblhyphenchar        & 127 \\
    \verb|\textasciibreve|           & \textasciibreve           & 128 \\
    \verb|\textasciicaron|           & \textasciicaron           & 129 \\
    \verb|\textacutedbl|             & \textacutedbl             & 130 \\
    \verb|\textgravedbl|             & \textgravedbl             & 131 \\
    \verb|\textdagger|               & \textdagger               & 132 \\
    \verb|\textdaggerdbl|            & \textdaggerdbl            & 133 \\
    \verb|\textbardbl|               & \textbardbl               & 134 \\
    \verb|\textperthousand|          & \textperthousand          & 135 \\
    \verb|\textbullet|               & \textbullet               & 136 \\
    \verb|\textcelsius|              & \textcelsius              & 137 \\
    \verb|\textdollaroldstyle|       & \textdollaroldstyle       & 138 \\
    \verb|\textcentoldstyle|         & \textcentoldstyle         & 139 \\
    \bottomrule
  \end{tabular}\qquad
  \begin{tabular}[t]{lp{1.5em}l}
    \toprule
    \verb|\textflorin|               & \textflorin               & 140 \\
    \verb|\textcolonmonetary|        & \textcolonmonetary        & 141 \\
    \verb|\textwon|                  & \textwon                  & 142 \\
    \verb|\textnaira|                & \textnaira                & 143 \\
    \verb|\textguarani|              & \textguarani              & 144 \\
    \verb|\textpeso|                 & \textpeso                 & 145 \\
    \verb|\textlira|                 & \textlira                 & 146 \\
    \verb|\textrecipe|               & \textrecipe               & 147 \\
    \verb|\textinterrobang|          & \textinterrobang          & 148 \\
    \verb|\textinterrobangdown|      & \textinterrobangdown      & 149 \\
    \verb|\textdong|                 & \textdong                 & 150 \\
    \verb|\texttrademark|            & \texttrademark            & 151 \\
    \verb|\textpertenthousand|       & \textpertenthousand       & 152 \\
    \verb|\textpilcrow|              & \textpilcrow              & 153 \\
    \verb|\textbaht|                 & \textbaht                 & 154 \\
    \verb|\textnumero|               & \textnumero               & 155 \\
    \verb|\textdiscount|             & \textdiscount             & 156 \\
    \verb|\textestimated|            & \textestimated            & 157 \\
    \verb|\textopenbullet|           & \textopenbullet           & 158 \\
    \verb|\textservicemark|          & \textservicemark          & 159 \\
    \verb|\textlquill|               & \textlquill               & 160 \\
    \verb|\textrquill|               & \textrquill               & 161 \\
    \verb|\textcent|                 & \textcent                 & 162 \\
    \verb|\textsterling|             & \textsterling             & 163 \\
    \verb|\textcurrency|             & \textcurrency             & 164 \\
    \verb|\textyen|                  & \textyen                  & 165 \\
    \verb|\textbrokenbar|            & \textbrokenbar            & 166 \\
    \verb|\textsection|              & \textsection              & 167 \\
    \verb|\textasciidieresis|        & \textasciidieresis        & 168 \\
    \verb|\textcopyright|            & \textcopyright            & 169 \\
    \verb|\textordfeminine|          & \textordfeminine          & 170 \\
    \verb|\textcopyleft|             & \textcopyleft             & 171 \\
    \verb|\textlnot|                 & \textlnot                 & 172 \\
    \verb|\textcircledP|             & \textcircledP             & 173 \\
    \verb|\textregistered|           & \textregistered           & 174 \\
    \verb|\textasciimacron|          & \textasciimacron          & 175 \\
    \verb|\textdegree|               & \textdegree               & 176 \\
    \verb|\textpm|                   & \textpm                   & 177 \\
    \verb|\texttwosuperior|          & \texttwosuperior          & 178 \\
    \verb|\textthreesuperior|        & \textthreesuperior        & 179 \\
    \verb|\textasciiacute|           & \textasciiacute           & 180 \\
    \verb|\textmu|                   & \textmu                   & 181 \\
    \verb|\textparagraph|            & \textparagraph            & 182 \\
    \verb|\textperiodcentered|       & \textperiodcentered       & 183 \\
    \verb|\textreferencemark|        & \textreferencemark        & 184 \\
    \verb|\textonesuperior|          & \textonesuperior          & 185 \\
    \verb|\textordmasculine|         & \textordmasculine         & 186 \\
    \verb|\textsurd|                 & \textsurd                 & 187 \\
    \verb|\textonequarter|           & \textonequarter           & 188 \\
    \verb|\textonehalf|              & \textonehalf              & 189 \\
    \verb|\textthreequarters|        & \textthreequarters        & 190 \\
    \verb|\texteuro|                 & \texteuro                 & 191 \\
    \verb|\texttimes|                & \texttimes                & 214 \\
    \verb|\textdiv|                  & \textdiv                  & 246 \\
    \null \\
    \bottomrule
  \end{tabular}
  \caption{Text symbols formerly from the \textsf{textcomp} package}
  \label{tab:textcomp}
\end{table}

\fi

The |TS1| encoding contains a rich set of symbols which means that
several symbols are only available in a few \TeX{} fonts and some, such
as the capital accents, not available at all but developed as part of
the reference font implementation.  In reality, many existing fonts
don't provide a full set of glyphs defined in |TS1| encoding and the
question arises: ``Which glyphs of the |TS1| encoding are implemented by
which font?''

\NEWfeature{2021/06/01}
Fonts\label{page:declareencodingsubset}
can be ordered in sub-encodings with the |\DeclareEncodingSubset|
macro:
\begin{decl}
  |\DeclareEncodingSubset| \arg{encoding}
                           \arg{font family}
                           \arg{subset number}
\end{decl}
The macro takes 3 mandatory arguments: An \m{encoding} for which a
subsetting is wanted (currently only |TS1|), the \m{font family} for
which we declare the subset and finally the \m{subset number} between
|0| (all of the encoding is supported) and |9| (many glyphs are
missing).  Hence, it is assumed that some symbols are always available
by all fonts and each sub-encoding defines macros which become
unavailable (i.e., they are not provided in the sub-encoding with that
number and all sub-encodings with higher numbers.)

\begin{table}[tbp]
  \centering\footnotesize
  \begin{tabular}[t]{@{}lp{1.5em}l@{}}
    \toprule
    \verb|\textquotestraightbase|     & \textquotestraightbase     & 13  \\
    \verb|\textquotestraightdblbase|  & \textquotestraightdblbase  & 18  \\
    \verb|\textcapitalcompwordmark|   & \textcapitalcompwordmark   & 23  \\
    \verb|\textascendercompwordmark|  & \textascendercompwordmark  & 31  \\
    \verb|\textdollar|                & \textdollar                & 36  \\
    \verb|\textquotesingle|           & \textquotesingle           & 39  \\
    \verb|\textasteriskcentered|      & \textasteriskcentered      & 42  \\
    \verb|\textdagger|                & \textdagger                & 132 \\
    \verb|\textdaggerdbl|             & \textdaggerdbl             & 133 \\
    \verb|\textperthousand|           & \textperthousand           & 135 \\
    \verb|\textbullet|                & \textbullet                & 136 \\
    \verb|\texttrademark|             & \texttrademark             & 151 \\
    \verb|\textcent|                  & \textcent                  & 162 \\
    \verb|\textsterling|              & \textsterling              & 163 \\
    \verb|\textyen|                   & \textyen                   & 165 \\
    \verb|\textbrokenbar|             & \textbrokenbar             & 166 \\
    \bottomrule
  \end{tabular}\qquad
  \begin{tabular}[t]{@{}lp{1.5em}l@{}}
    \toprule
    \verb|\textsection|               & \textsection               & 167 \\
    \verb|\textcopyright|             & \textcopyright             & 169 \\
    \verb|\textordfeminine|           & \textordfeminine           & 170 \\
    \verb|\textlnot|                  & \textlnot                  & 172 \\
    \verb|\textregistered|            & \textregistered            & 174 \\
    \verb|\textdegree|                & \textdegree                & 176 \\
    \verb|\textpm|                    & \textpm                    & 177 \\
    \verb|\textparagraph|             & \textparagraph             & 182 \\
    \verb|\textperiodcentered|        & \textperiodcentered        & 183 \\
    \verb|\textordmasculine|          & \textordmasculine          & 186 \\
    \verb|\textonequarter|            & \textonequarter            & 188 \\
    \verb|\textonehalf|               & \textonehalf               & 189 \\
    \verb|\textthreequarters|         & \textthreequarters         & 190 \\
    \verb|\texttimes|                 & \texttimes                 & 214 \\
    \verb|\textdiv|                   & \textdiv                   & 246 \\
    \null \\
    \bottomrule
  \end{tabular}
  \caption{Symbols available in all \texttt{TS1} sub-encodings}
  \label{tab:ts1-subset-always}
\end{table}



\begin{table}[!tbp]
  \centering
  \begin{tabular}[t]{@{}lll@{}}
    \toprule
    \verb*|\textcircled| & \textcircled{} & acc \\
    \bottomrule
  \end{tabular}
  \caption{Symbol unavailable in \texttt{TS1} sub-encoding \texttt{1}
    and higher}
  \label{tab:ts1-subset-1-disable}
\end{table}



\begin{table}[!tbp]
  \centering\footnotesize
  \begin{tabular}[t]{@{}lp{1.5em}l@{}}
    \toprule
    \verb|\capitalcedilla|       & \capitalcedilla{}       & acc \\
    \verb|\capitalogonek|        & \capitalogonek{}        & acc \\
    \verb|\capitalgrave|         & \capitalgrave{}         & 0   \\
    \verb|\capitalacute|         & \capitalacute{}         & 1   \\
    \verb|\capitalcircumflex|    & \capitalcircumflex{}    & 2   \\
    \verb|\capitaltilde|         & \capitaltilde{}         & 3   \\
    \verb|\capitaldieresis|      & \capitaldieresis{}      & 4   \\
    \verb|\capitalhungarumlaut|  & \capitalhungarumlaut{}  & 5   \\
    \verb|\capitalring|          & \capitalring{}          & 6   \\
    \verb|\capitalcaron|         & \capitalcaron{}         & 7   \\
    \verb|\capitalbreve|         & \capitalbreve{}         & 8   \\
    \verb|\capitalmacron|        & \capitalmacron{}        & 9   \\
    \verb|\capitaldotaccent|     & \capitaldotaccent{}     & 10  \\
    \verb|\capitaltie|           & \capitaltie{}           & 27  \\
    \verb|\newtie|               & \newtie{}               & 28  \\
    \verb|\capitalnewtie{}|      & \capitalnewtie{}        & 29  \\
    \verb|\textdblhyphen|        & \textdblhyphen          & 45  \\
    \verb|\textzerooldstyle|     & \textzerooldstyle       & 48  \\
    \verb|\textoneoldstyle|      & \textoneoldstyle        & 49  \\
    \verb|\texttwooldstyle|      & \texttwooldstyle        & 50  \\
    \verb|\textthreeoldstyle|    & \textthreeoldstyle      & 51  \\
    \verb|\textfouroldstyle|     & \textfouroldstyle       & 52  \\
    \verb|\textfiveoldstyle|     & \textfiveoldstyle       & 53  \\
    \verb|\textsixoldstyle|      & \textsixoldstyle        & 54  \\
    \verb|\textsevenoldstyle|    & \textsevenoldstyle      & 55  \\
    \verb|\texteightoldstyle|    & \texteightoldstyle      & 56  \\
    \verb|\textnineoldstyle|     & \textnineoldstyle       & 57  \\
    \verb|\textmho|              & \textmho                & 77  \\
    \verb|\textbigcircle|        & \textbigcircle          & 79  \\
    \verb|\textlbrackdbl|        & \textlbrackdbl          & 91  \\
    \verb|\textrbrackdbl|        & \textrbrackdbl          & 93  \\
    \verb|\textasciigrave|       & \textasciigrave         & 96  \\
    \verb|\textborn|             & \textborn               & 98  \\
    \bottomrule
  \end{tabular}\qquad
  \begin{tabular}[t]{@{}lp{1.5em}l@{}}
    \toprule
    \verb|\textdivorced|         & \textdivorced           & 99  \\
    \verb|\textdied|             & \textdied               & 100 \\
    \verb|\textleaf|             & \textleaf               & 108 \\
    \verb|\textmarried|          & \textmarried            & 109 \\
    \verb|\textmusicalnote|      & \textmusicalnote        & 110 \\
    \verb|\texttildelow|         & \texttildelow           & 126 \\
    \verb|\textdblhyphenchar|    & \textdblhyphenchar      & 127 \\
    \verb|\textasciibreve|       & \textasciibreve         & 128 \\
    \verb|\textasciicaron|       & \textasciicaron         & 129 \\
    \verb|\textacutedbl|         & \textacutedbl           & 130 \\
    \verb|\textgravedbl|         & \textgravedbl           & 131 \\
    \verb|\textdollaroldstyle|   & \textdollaroldstyle     & 138 \\
    \verb|\textcentoldstyle|     & \textcentoldstyle       & 139 \\
    \verb|\textnaira|            & \textnaira              & 143 \\
    \verb|\textguarani|          & \textguarani            & 144 \\
    \verb|\textpeso|             & \textpeso               & 145 \\
    \verb|\textrecipe|           & \textrecipe             & 147 \\
    \verb|\textpertenthousand|   & \textpertenthousand     & 152 \\
    \verb|\textpilcrow|          & \textpilcrow            & 153 \\
    \verb|\textbaht|             & \textbaht               & 154 \\
    \verb|\textdiscount|         & \textdiscount           & 156 \\
    \verb|\textopenbullet|       & \textopenbullet         & 158 \\
    \verb|\textservicemark|      & \textservicemark        & 159 \\
    \verb|\textlquill|           & \textlquill             & 160 \\
    \verb|\textrquill|           & \textrquill             & 161 \\
    \verb|\textasciidieresis|    & \textasciidieresis      & 168 \\
    \verb|\textcopyleft|         & \textcopyleft           & 171 \\
    \verb|\textcircledP|         & \textcircledP           & 173 \\
    \verb|\textasciimacron|      & \textasciimacron        & 175 \\
    \verb|\textasciiacute|       & \textasciiacute         & 180 \\
    \verb|\textreferencemark|    & \textreferencemark      & 184 \\
    \verb|\textsurd|             & \textsurd               & 187 \\
    \null \\
    \bottomrule
  \end{tabular}
  \caption{Symbols unavailable in \texttt{TS1} sub-encoding \texttt{2}
    and higher}
  \label{tab:ts1-subset-2-disable}
\end{table}


\begin{table}[!tbp]
  \centering\footnotesize
  \begin{tabular}[t]{@{}lp{1.5em}l@{}}
    \toprule
    \verb|textlangle|                & \textlangle               & 60 \\
    \bottomrule
  \end{tabular}\qquad
  \begin{tabular}[t]{@{}lp{1.5em}l@{}}
    \toprule
    \verb|\textrangle|               & \textrangle               & 62 \\
    \bottomrule
  \end{tabular}
  \caption{Symbols unavailable in \texttt{TS1} sub-encoding \texttt{3}
    and higher}
  \label{tab:ts1-subset-3-disable}
\end{table}


Thus, the symbols that are available in sub-encoding $x$ are the symbols
in table~\ref{tab:ts1-subset-always} (always available) and the symbols
that only become unavailable in sub-encodings $> x$.  The tables
\vrefrange{tab:ts1-subset-1-disable}{tab:ts1-subset-9-disable} show the symbols that
become unavailable in the different sub-encodings.  Again, the numbers
are the slots in the |TS1| encoding, |acc| indicates a `constructed'
accent.



\begin{table}[!tbp]
  \centering\footnotesize
  \begin{tabular}[t]{@{}lp{1.5em}l@{}}
    \toprule
    \verb|\textleftarrow|            & \textleftarrow            & 24  \\
    \verb|\textrightarrow|           & \textrightarrow           & 25  \\
    \verb|\textuparrow|              & \textuparrow              & 94  \\
    \verb|\textdownarrow|            & \textdownarrow            & 95 \\
    \bottomrule
  \end{tabular}\qquad
  \begin{tabular}[t]{@{}lp{1.5em}l@{}}
    \toprule
    \verb|\textcolonmonetary|        & \textcolonmonetary        & 141 \\
    \verb|\textwon|                  & \textwon                  & 142 \\
    \verb|\textlira|                 & \textlira                 & 146 \\
    \verb|\textdong|                 & \textdong                 & 150 \\
    \bottomrule
  \end{tabular}
  \caption{Symbols unavailable in \texttt{TS1} sub-encoding \texttt{4}
    and higher}
  \label{tab:ts1-subset-4-disable}
\end{table}


\begin{table}[!tbp]
  \centering\footnotesize
  \begin{tabular}[t]{@{}lp{1.5em}l@{}}
    \toprule
    \verb|\textnumero|               & \textnumero               & 155 \\
    \bottomrule
  \end{tabular}\qquad
  \begin{tabular}[t]{@{}lp{1.5em}l@{}}
    \toprule
    \verb|\textestimated|            & \textestimated            & 157 \\
    \bottomrule
  \end{tabular}
  \caption{Symbols unavailable in \texttt{TS1} sub-encoding \texttt{5}
    and higher}
  \label{tab:ts1-subset-5-disable}
\end{table}



\begin{table}[!tbp]
  \centering\footnotesize
  \begin{tabular}[t]{@{}lp{1.5em}l@{}}
    \toprule
    \verb|\textflorin|               & \textflorin               & 140 \\
    \bottomrule
  \end{tabular}\qquad
  \begin{tabular}[t]{@{}lp{1.5em}l@{}}
    \toprule
    \verb|\textcurrency|             & \textcurrency             & 164 \\
    \bottomrule
  \end{tabular}
  \caption{Symbols unavailable in \texttt{TS1} sub-encoding \texttt{6}
    and higher}
  \label{tab:ts1-subset-6-disable}
\end{table}



\begin{table}[!tbp]
  \centering\footnotesize
  \begin{tabular}[t]{@{}lp{1.5em}l@{}}
    \toprule
    \verb|\textfractionsolidus|      & \textfractionsolidus      & 47 \\
    \verb|\textminus|                & \textminus                & 61 \\
    \bottomrule
  \end{tabular}\qquad
  \begin{tabular}[t]{@{}lp{1.5em}l@{}}
    \toprule
    \verb|\textohm|                  & \textohm                  & 87 \\
    \verb|\textmu|                   & \textmu                   & 181 \\
    \bottomrule
  \end{tabular}
  \caption{Symbols unavailable in \texttt{TS1} sub-encoding \texttt{7}
    and higher}
  \label{tab:ts1-subset-7-disable}
\end{table}



\begin{table}[!tbp]
  \centering\footnotesize
  \begin{tabular}[t]{@{}lp{1.5em}l@{}}
    \toprule
    \verb|\textblank|                & \textblank                & 32 \\
    \verb|\textinterrobang|          & \textinterrobang          & 148 \\
    \bottomrule
  \end{tabular}\qquad
  \begin{tabular}[t]{@{}lp{1.5em}l@{}}
    \toprule
    \verb|\textinterrobangdown|      & \textinterrobangdown      & 149 \\
    \null \\
    \bottomrule
  \end{tabular}
  \caption{Symbols unavailable in \texttt{TS1} sub-encoding \texttt{8}
    and higher}
  \label{tab:ts1-subset-8-disable}
\end{table}


\begin{table}[!tbp]
  \centering\footnotesize
  \begin{tabular}[t]{@{}lp{1.5em}l@{}}
    \toprule
    \verb|\texttwelveudash|          & \texttwelveudash          & 21  \\
    \verb|\textthreequartersemdash|  & \textthreequartersemdash  & 22  \\
    \verb|\textbardbl|               & \textbardbl               & 134 \\
    \verb|\textcelsius|              & \textcelsius              & 137 \\
    \bottomrule
  \end{tabular}\qquad
  \begin{tabular}[t]{@{}lp{1.5em}l@{}}
    \toprule
    \verb|\texttwosuperior|          & \texttwosuperior          & 178 \\
    \verb|\textthreesuperior|        & \textthreesuperior        & 179 \\
    \verb|\textonesuperior|          & \textonesuperior          & 185 \\
    \null \\
    \bottomrule
  \end{tabular}
  \caption{Symbols unavailable in \texttt{TS1} sub-encoding \texttt{9}}
  \label{tab:ts1-subset-9-disable}
\end{table}



As an example,
%\begin{verbatim}
   \verb=\DeclareEncodingSubset{TS1}{foo}{5}=
%\end{verbatim}
indicates that the font family |foo| contains the always available
symbols (table~\vref{tab:ts1-subset-always}) and the ones disabled
in sub-encodings 6--9, i.e.,
tables~\vrefrange{tab:ts1-subset-6-disable}{tab:ts1-subset-9-disable}.

As these days many font families are set up to end in |-LF| (lining
figures), |-OsF| (oldstyle figures), etc. the declaration supports a
shortcut: if the \m{font family} name ends in |-*| then the star gets
replaced by these common ending, e.g.,
\begin{verbatim}
   \DeclareEncodingSubset{TS1}{Alegreya-*}{2}
\end{verbatim}
is the same as writing
\begin{verbatim}
   \DeclareEncodingSubset{TS1}{Alegreya-LF}  {2}
   \DeclareEncodingSubset{TS1}{Alegreya-OsF} {2}
   \DeclareEncodingSubset{TS1}{Alegreya-TLF} {2}
   \DeclareEncodingSubset{TS1}{Alegreya-TOsF}{2}
\end{verbatim}
If only some are needed then one can define them individually but in
many cases all four are wanted, hence the shortcut.

Maintainers of font bundles that include \texttt{TS1} encoded font files
should add an appropriate declaration into the corresponding
\texttt{ts1}\textit{family}\texttt{.fd} file, because otherwise the
default subencoding is assumed, which is probably disabling too many
glyphs that are actually available in the font.\footnote{The \LaTeX{}
  format contains declarations for many font families already, but this
  is really the wrong place for the declarations. Thus for new fonts
  they should be placed into the corresponding \texttt{.fd} file.}


\section{If you need to know more \ldots}

\NEWdescription{1996/06/01}
The |tracefnt| package provides for tracing the actions concerned with
loading, substituting and using fonts.  The package accepts the
following options:
\begin{description}
\item[errorshow] Write all information about font changes, etc.\ but
  only to the transcript file unless an error occurs. This means that
  information about font substitution will not be shown on the terminal.

\item[warningshow] Show all font warnings on the terminal. This setting
  corresponds to the default behavior when this \texttt{tracefnt}
  package is \emph{not} used!

\item[infoshow] Show all font warnings and all font info messages (that
  are normally only written to the transcript file) also on the
  terminal. This is the default when this \texttt{tracefnt} package is
  loaded.

\item[debugshow] In addition to what is shown by \texttt{infoshow}, show
  also changes of math fonts (as far as possible): beware, this option
  can produce a large amount of output.

\item[loading] Show the names of external font files when they are
  loaded.  This option shows only `newly loaded' fonts, not those
  already preloaded in the format or the class file before this
  \texttt{tracefnt} package becomes active.

\item[pausing] Turn all font warnings into errors so that \LaTeX{} will
  stop.
\end{description}

\emph{Warning}: The actions of this package can change the layout of a
document and even, in rare cases, produce clearly wrong output, so it
should not be used in the final formatting of `real documents'.

\begin{thebibliography}{1}

\bibitem{A-W:MG2004}
Frank Mittelbach and Michel Goossens.
\newblock {\em The {\LaTeX} Companion second edition}.
\newblock With Johannes Braams, David Carlisle, and Chris Rowley.
\newblock Addison-Wesley, Reading, Massachusetts, 2004.

\bibitem{tub:DKn89}
Donald~E. Knuth.
\newblock Typesetting concrete mathematics.
\newblock {\em {TUG}boat}, 10(1):31--36, April 1989.

\bibitem{A-W:LLa94}
Leslie Lamport.
\newblock {\em {\LaTeX:} A Document Preparation System}.
\newblock Addison-Wesley, Reading, Massachusetts, second edition, 1994.

\end{thebibliography}

\end{document}

%%% Local Variables:
%%% mode: latex
%%% TeX-master: t
%%% fill-column: 72
%%% End:
