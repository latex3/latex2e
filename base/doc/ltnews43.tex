% \iffalse meta-comment
%
% Copyright 2025
% The LaTeX Project and any individual authors listed elsewhere
% in this file.
%
% This file is part of the LaTeX base system.
% -——————————————
%
% It may be distributed and/or modified under the
% conditions of the LaTeX Project Public License, either version 1.3c
% of this license or (at your option) any later version.
% The latest version of this license is in
%    https://www.latex-project.org/lppl.txt
% and version 1.3c or later is part of all distributions of LaTeX
% version 2008 or later.
%
% This file has the LPPL maintenance status "maintained".
%
% The list of all files belonging to the LaTeX base distribution is
% given in the file `manifest.txt'. See also `legal.txt' for additional
% information.
%
% The list of derived (unpacked) files belonging to the distribution
% and covered by LPPL is defined by the unpacking scripts (with
% extension .ins) which are part of the distribution.
%
% \fi
% Filename: ltnews43.tex
%
% This is issue 43 of LaTeX News.

\NeedsTeXFormat{LaTeX2e}[2020-02-02]

\documentclass{ltnews}

%% Maybe needed only for Chris' inadequate system:
\providecommand\Dash {\unskip \textemdash}

%% NOTE: Chris' preferred hyphens!
%% \showhyphens{parameters}
%% \hyphenation{because}

\usepackage[T1]{fontenc}

\usepackage{lmodern,url,hologo}

\usepackage{csquotes}
\usepackage{multicol}
\usepackage{color}

\usepackage{amsmath}

\providecommand\hook[1]{\texttt{#1}}

\providecommand\meta[1]{$\langle$\textrm{\itshape#1}$\rangle$}
\providecommand\option[1]{\texttt{#1}}
\providecommand\env[1]{\texttt{#1}}
\providecommand\Arg[1]{\texttt\{\meta{#1}\texttt\}}


\providecommand\eTeX{\hologo{eTeX}}
\providecommand\XeTeX{\hologo{XeTeX}}
\providecommand\LuaTeX{\hologo{LuaTeX}}
\providecommand\pdfTeX{\hologo{pdfTeX}}
\providecommand\MiKTeX{\hologo{MiKTeX}}
\providecommand\CTAN{\textsc{ctan}}
\providecommand\TL{\TeX\,Live}

\providecommand\githubissue[2][]{\ifhmode\unskip\fi
     \quad\penalty500\strut\nobreak\hfill
     \mbox{\small\slshape(%
       \href{https://github.com/latex3/latex2e/issues/\getfirstgithubissue#2 \relax}%
          	    {github issue#1 #2}%
           )}%
     \par\smallskip}

% simple solution right now (just link to the first issue if there are more)
\def\getfirstgithubissue#1 #2\relax{#1}

% issues from the tagging-project:

\providecommand\taggingissue[2][]{\ifhmode\unskip\fi
     \quad\penalty500\strut\nobreak\hfill
     \mbox{\small\slshape(%
       \href{https://github.com/latex3/tagging-project/issues/\getfirstgithubissue#2 \relax}%
          	    {tagging-project issue#1 #2}%
           )}%
     \par\smallskip}

\providecommand\sxissue[1]{\ifhmode\unskip
     \else
       % githubissue preceding
       \vskip-\smallskipamount
       \vskip-\parskip
     \fi
     \quad\penalty500\strut\nobreak\hfill
     \mbox{\small\slshape(\url{https://tex.stackexchange.com/#1})}\par}

\providecommand\gnatsissue[2]{\ifhmode\unskip\fi
     \quad\penalty500\strut\nobreak\hfill
     \mbox{\small\slshape(%
       \href{https://www.latex-project.org/cgi-bin/ltxbugs2html?pr=#1\%2F\getfirstgithubissue#2 \relax}%
          	    {gnats issue #1/#2}%
           )}%
     \par}

\let\cls\pkg
\providecommand\env[1]{\texttt{#1}}
\providecommand\acro[1]{\textsc{#1}}

\vbadness=1400  % accept slightly empty columns


\let\finalpagebreak\pagebreak % for TUB (if they use it)
\let\finalvspace\vspace       % for document layout fixes

\makeatletter
% maybe not the greatest design but normally we wouldn't have subsubsections
\renewcommand{\subsubsection}{%
   \@startsection {subsubsection}{2}{0pt}{1.5ex \@plus 1ex \@minus .2ex}%
                  {-1em}{\@subheadingfont\colonize}%
}
\providecommand\colonize[1]{#1:}
\makeatother


% Undo ltnews's \verbatim@font with active < and >
\makeatletter
\def\verbatim@font{\normalsize\ttfamily}
\makeatother


%%%%%%%%%%%%%%%%%%%%%%%%%%%%%%%%%%%%%%%%%%%%%%%%%%%%%%%%%%%%%%%%%%%%%%%%%%%%%
\providecommand\tubcommand[1]{}
\tubcommand{\input{tubltmac}}

\publicationmonth{June}
\publicationyear{2026  --- DRAFT version for upcoming release}

\publicationissue{43}

\begin{document}

\maketitle
{\hyphenpenalty=10000 \exhyphenpenalty=10000 \spaceskip=3.33pt \hbadness=10000
\tableofcontents}



\setlength\rightskip{0pt plus 3em}

\section{Introduction}

% To write

\section{News from the Tagged PDF project}

\subsection{Improving tagging of floats}

The tagging support code for floats has been overhauled.  It now
allows to add tagging support for new float types like listings or
tcolorboxes.  By default float structures are deferred to the end of
the document but it is now possible to switch this on and off and to
output the floats in other places in the structure.  More details can
be found in \texttt{latex-lab-float.pdf}.

\subsection{Setting the language in \cs{DocumentMetadata}}

It is good practise to always set the main document language in
\cs{DocumentMetadata} explicitly using the \texttt{lang} key: It is
used to set the \texttt{/Lang} key in the PDF and its value can also
be used by packages and classes to adapt locale settings.
 
However, if there is no such setting then the code will use the main
language as set by \pkg{babel} or \pkg{polyglossia}. If they are not
used \texttt{lang=en} is applied as this has always been \LaTeX's
default.  When we introduced the \texttt{lang} key we added a warning
to the log in such cases but feedback we received indicated that it
caused concerns so now the fallback is applied silently.
%
\taggingissue{1115}




\section{New or improved commands}

\subsection{Recovering instance values}

In some cases, editing template instances requires knowing the
existing instance values. To support this, we have added the
expandable command
\cs{InstanceValue}\texttt{\meta{type}\meta{instance}\meta{key}}, which
returns the value if available or otherwise if empty (if the key or
instance does not exist).



\section{Code improvements}

\subsection{Revision of handling of \enquote{no value} concept}

The commands \cs{NewDocumentCommand}, etc., introduce the idea of
differentiating an absent optional argument from one which is simply
empty.  When an argument is entirely empty, it is given the special
\enquote{no value} marker. In previous releases, this was a
deliberately-awkward set of character tokens, which are therefore hard
to input accidentally.

Whilst this allowed us to easily detect \enquote{no value}, it turns
out there are places we want to be able to add such a value. This
comes up particularly in creating templates for some parts of the
document structure as part of the wider tagging project.

We have therefore changed the approach to use a marker token,
\cs{NoValue}, and updated the \cs{IfNoValue(TF)} test and
relatives. This means you now \emph{can} type in an optional argument
that is interpreted as \enquote{not present}, but this is not likely
to happen by accident.


\section{Bug fixes}

\subsection{Improve transparency of \cs{label}, \cs{index}, and friends}

Commands such as \cs{label} or \cs{index} are supposed to be
transparent with respect to surrounding spaces, i.e., spaces on both
sides should not lead to several spaces in typeset text. This always
worked reasonably well if there is only a single command. However, if
there are several of them in a row one could end up with a spurious
extra space. This has finally been corrected.
%
\githubissue{1910}




\section{Changes to packages in the \pkg{amsmath} category}

\subsection{Treat \cs{dots} before \cs{xrightarrow} correctly}

If one writes \verb=$ a \to \dots \to b $= the result is $ a \to \dots
\to b $, i.e., \cs{dots} are treated as binary dots. If you replace
\cs{to} with \cs{xrightarrow} then the dots suddenly become comma dots
and you have to use \cs{bdots} to get the binary dots back. This has
now been corrected for both \cs{xrightarrow} and \cs{xleftarrow}.
%
\githubissue{263}


\subsection{Don't lose a qed symbol with \texttt{fleqn}}

The \env{proof} environment of \pkg{amsthm} automatically puts a QED
symbol at the end of a proof.  Sometimes this is not the best place
and in that situation that author can direct \LaTeX{} to place the
symbol earlier by using \cs{qedhere} in an appropriate place. However,
when the \texttt{fleqn} option was in force this didn't always work
and the symbol got dropped in some cases. This has now been corrected.
%
\githubissue{783}



%\subsection{A fix}

%% Some text
%
%%\githubissue{XXXX}


%\section{Changes to packages in the \pkg{graphics} category}

%\section{Changes to packages in the \pkg{tools} category}

%\section{Changes to files in the \pkg{cyrillic} category}

\begin{thebibliography}{9}\frenchspacing

%\fontsize{9.3}{11.3}\selectfont

\bibitem{43:Lamport}
Leslie Lamport.
\newblock \emph{{\LaTeX}: {A} Document Preparation System: User's Guide and Reference
  Manual}.
\newblock \mbox{Addison}-Wesley, Reading, MA, USA, 2nd edition, 1994.
\newblock ISBN 0-201-52983-1.
\newblock Reprinted with corrections in 1996.

\bibitem{43:ltnews} \LaTeX{} Project Team.
  \emph{\LaTeXe{} news 1--42}. November, 2025.
  \url{https://latex-project.org/news/latex2e-news/ltnews.pdf}

\end{thebibliography}

\end{document}

