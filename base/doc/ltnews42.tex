% \iffalse meta-comment
%
% Copyright 2025
% The LaTeX Project and any individual authors listed elsewhere
% in this file.
%
% This file is part of the LaTeX base system.
% -——————————————
%
% It may be distributed and/or modified under the
% conditions of the LaTeX Project Public License, either version 1.3c
% of this license or (at your option) any later version.
% The latest version of this license is in
%    https://www.latex-project.org/lppl.txt
% and version 1.3c or later is part of all distributions of LaTeX
% version 2008 or later.
%
% This file has the LPPL maintenance status "maintained".
%
% The list of all files belonging to the LaTeX base distribution is
% given in the file `manifest.txt'. See also `legal.txt' for additional
% information.
%
% The list of derived (unpacked) files belonging to the distribution
% and covered by LPPL is defined by the unpacking scripts (with
% extension .ins) which are part of the distribution.
%
% \fi
% Filename: ltnews42.tex
%
% This is issue 42 of LaTeX News.

\NeedsTeXFormat{LaTeX2e}[2020-02-02]

\documentclass{ltnews}

%% Maybe needed only for Chris' inadequate system:
\providecommand\Dash {\unskip \textemdash}

%% NOTE: Chris' preferred hyphens!
%% \showhyphens{parameters}
%% \hyphenation{because}

\usepackage[T1]{fontenc}

\usepackage{lmodern,url,hologo}

\usepackage{csquotes}
\usepackage{multicol}
\usepackage{color}

\providecommand\hook[1]{\texttt{#1}}

\providecommand\meta[1]{$\langle$\textrm{\itshape#1}$\rangle$}
\providecommand\option[1]{\texttt{#1}}
\providecommand\env[1]{\texttt{#1}}
\providecommand\Arg[1]{\texttt\{\meta{#1}\texttt\}}


\providecommand\eTeX{\hologo{eTeX}}
\providecommand\XeTeX{\hologo{XeTeX}}
\providecommand\LuaTeX{\hologo{LuaTeX}}
\providecommand\pdfTeX{\hologo{pdfTeX}}
\providecommand\MiKTeX{\hologo{MiKTeX}}
\providecommand\CTAN{\textsc{ctan}}
\providecommand\TL{\TeX\,Live}

\providecommand\githubissue[2][]{\ifhmode\unskip\fi
     \quad\penalty500\strut\nobreak\hfill
     \mbox{\small\slshape(%
       \href{https://github.com/latex3/latex2e/issues/\getfirstgithubissue#2 \relax}%
          	    {github issue#1 #2}%
           )}%
     \par\smallskip}

% simple solution right now (just link to the first issue if there are more)
\def\getfirstgithubissue#1 #2\relax{#1}

% issues from the tagging-project:

\providecommand\taggingissue[2][]{\ifhmode\unskip\fi
     \quad\penalty500\strut\nobreak\hfill
     \mbox{\small\slshape(%
       \href{https://github.com/latex3/tagging-project/issues/\getfirstgithubissue#2 \relax}%
          	    {tagging-project issue#1 #2}%
           )}%
     \par\smallskip}

\providecommand\sxissue[1]{\ifhmode\unskip
     \else
       % githubissue preceding
       \vskip-\smallskipamount
       \vskip-\parskip
     \fi
     \quad\penalty500\strut\nobreak\hfill
     \mbox{\small\slshape(\url{https://tex.stackexchange.com/#1})}\par}

\providecommand\gnatsissue[2]{\ifhmode\unskip\fi
     \quad\penalty500\strut\nobreak\hfill
     \mbox{\small\slshape(%
       \href{https://www.latex-project.org/cgi-bin/ltxbugs2html?pr=#1\%2F\getfirstgithubissue#2 \relax}%
          	    {gnats issue #1/#2}%
           )}%
     \par}

\let\cls\pkg
\providecommand\env[1]{\texttt{#1}}
\providecommand\acro[1]{\textsc{#1}}

\vbadness=1400  % accept slightly empty columns


\let\finalpagebreak\pagebreak % for TUB (if they use it)
\let\finalvspace\vspace       % for document layout fixes

\makeatletter
% maybe not the greatest design but normally we wouldn't have subsubsections
\renewcommand{\subsubsection}{%
   \@startsection {subsubsection}{2}{0pt}{1.5ex \@plus 1ex \@minus .2ex}%
                  {-1em}{\@subheadingfont\colonize}%
}
\providecommand\colonize[1]{#1:}
\makeatother


% Undo ltnews's \verbatim@font with active < and >
\makeatletter
\def\verbatim@font{\normalsize\ttfamily}
\makeatother


%%%%%%%%%%%%%%%%%%%%%%%%%%%%%%%%%%%%%%%%%%%%%%%%%%%%%%%%%%%%%%%%%%%%%%%%%%%%%
\providecommand\tubcommand[1]{}
\tubcommand{\input{tubltmac}}

\publicationmonth{November}
\publicationyear{2025  --- DRAFT version for upcoming release}

\publicationissue{42}

\begin{document}

\maketitle
{\hyphenpenalty=10000 \exhyphenpenalty=10000 \spaceskip=3.33pt \hbadness=10000
\tableofcontents}

\setlength\rightskip{0pt plus 3em}

\medskip

\section{Introduction}

Fans of Douglas Adams will know that this must a special issue of the
\LaTeX{} news; unfortunately it doesn't give us the answer to
\enquote{Answer to the Ultimate Question of Life, the Universe, and
  Everything} yet \Dash we still expect a few more issues to each that
point.

However, with this release we have made further progress in the
generating tagged and accessible PDF documents from \LaTeX{} and to
indicate this we move away from calling it a prototype solution as by
now it is and can be used in production workflows (as long as one
restricts the documents to already supported packages). This does not
mean that \texttt{latex-lab} code (where the tagging support currently
resides) is no longer under development, but that the user-facing side
of the project is by now fairly stable and usable.

Outside of the tagging project we have added new functionality to the
font selection support as well as code improvements additional
functionality in a number of other places, for example, supporting
value expansion of key values at the time of declaration, which is
useful when specifying template instances and in similar places.

As usual, there have been a few bugs to take care of (odd ones for
sure this time) and we also decided to finally retire a few packages
from the tools collection. More exactly, we suggest that they should no
longer be used as there are better possibilities available by now.


\section{News from the Tagged PDF project}

\subsection{Expanding the \cs{DocumentMetadata} command}

In 2022 we introduced the \cs{DocumentMetadata} with a twofold
purpose: to provide a dedicated place for document wide settings and
metadata, and to act as a trigger command to identify documents that
want to load new code.  The latter allows the use of the new, extended
interfaces essential for the Tagging Project but also useful also
without tagging.

Initially, using \cs{DocumentMetadata} with an empty argument loaded
only the PDF management code and a new \pkg{hyperref} driver was used.
Since November 2024 \cs{DocumentMetadata} changes the default encoding
from OT1 to T1; and since June 2025 it also changes the default PDF
version from 1.7 to 2.0.

Additional code in latex-lab (needed, e.g., for the tagging project)
had to be loaded explicitly by using the \texttt{testphase} or the new
\texttt{tagging} key in the argument of \cs{DocumentMetadata}. Whilst
this allowed for the selective loading and testing of the new code, it
also produced problems for classes and packages adapting their code
for the tagging project since it was difficult to test which parts of
the latex-lab code were active.

In this release we therefore extend \cs{DocumentMetadata} even
further: it will now load directly all the code that one get also when
using the \texttt{tagging=off} or the \texttt{testphase=latest} key.

The values \texttt{phase-I}, \texttt{phase-II}, \texttt{phase-III} of
the \texttt{testphase} key will no longer load different code variants
but only activate tagging. Extra modules not yet incorporated in the
\texttt{latest} set of modules can still be loaded by using the
\texttt{testphase} key.

For documents that want to load the PDF management but do not want the
new tagging support code we provide a dedicated package. Such
documents should replace
\begin{verbatim}
\DocumentMetadata{pdfversion=1.7,
  pdfstandard=a-3b}
\end{verbatim}
by
\begin{verbatim}
\RequirePackage{pdfmanagement}
\SetKeys[document/metadata]{pdfversion=1.7,
  pdfstandard=a-3b}
\end{verbatim}

\subsection{Checking the compatibility with the Tagging support code}

We maintain a database showing the compatibility of classes and packages 
with the Tagging support code. This data can be viewed online~\cite{tagging-status}. 
We have now exported a part of the data into a small package \pkg{latex-tagging-status} 
and added a key \texttt{check-tagging-status} to \cs{DocumentMetadata}. 
When used the status of the packages and the class used by the document 
will be shown at the end of the log-file.

This is only a rough overview and a debugging aid, not a final report!
Using packages that are classified as incompatible or partially incompatible 
does not mean that the tagging is necessarily broken. For example, \pkg{hyperref}
is partially incompatible as the form fields are not properly tagged 
(this requires the use of the \pkg{l3pdffield} package), but in documents without 
form fields it is unproblematic. 
In case of partially-compatible or incompatible packages the full table should be checked 
as it often contains an explanation what is not yet working.
    
The package \pkg{latex-tagging-status} will be regularly updated to reflect changes
in packages and the status database. Erroneous messages should be reported at 
the github of the Tagging Project\cite{tagging-project}. It is also possible
to create a pull request to update or correct the data. 


\subsection{Requiring or testing for the Tagging support code}

Classes or packages that are written only for the new code loaded by
\cs{DocumentMetadata} can use the new command
\cs{NeedsDocumentMetadata} at the begin of the class or package
file. It will produce a suitable error message if the tagging support
code has not been loaded.

Classes and package that want to support both legacy documents and
newer documents using \cs{DocumentMetadata} can now use
\cs{IfDocumentMetadataTF} to test whether the new code has been loaded
-- eventually in combination with a test of the date of the format. To
test whether the PDF management has been loaded, the test
\cs{IfPDFManagementActiveTF} is provided.

 
\subsection{Moving paragraph tagging into sockets}

Paragraphs in \LaTeX{} can be nested, e.g., you can have a paragraph
containing a display quote, which in turn consists of more than one
(sub)paragraph, followed by some more text which all belongs to the
same outer paragraph.

To model such \enquote{semantic paragraphs} \LaTeX{} uses a structure
named \texttt{text-unit}\footnote{The name is under review and is
likely to change in future.}  and use \texttt{text} (rolemapped to
\texttt{P}) only for (portions of) the actual paragraph text\LaTeX{}.

This is semantically clear and allows processors who care to identify
the complete paragraphs by looking for \texttt{text-unit} tags.  But
we got also request for an option to disable the tagging of the
\enquote{semantic paragraphs}, so with this release we moved the
relevant tagging code into sockets. The \enquote{semantic paragraphs}
can now be disabled by assigning the \texttt{noop} plug to these
sockets:
\begin{verbatim}
\AssignTaggingSocketPlug{para/semantic/begin}
  {noop}
\AssignTaggingSocketPlug{para/semantic/end}
  {noop}
\end{verbatim}

\subsection{Hooks for \cs{includegraphics} keys}

The three keys definitions \texttt{alt}, \texttt{actualtext} and
\texttt{artifact} used by \cs{includegraphics} contain now hooks,
named \texttt{Gin/alt}, \texttt{Gin/actualtext}, and
\texttt{Gin/artifact}\footnote{\texttt{Gin} refers to the family name
used by keys in the \pkg{graphicx} package}.  The first two are hooks
with two arguments and get as first argument the (with
\verb+\text_purify:n+) purified value of the key which is also used in
the PDF and as second argument the raw value.  The hooks are processed
even if tagging is not activated.  With them it is, for example,
possible to store the alternative text:
\begin{verbatim}
  \AddToHookWithArguments{Gin/alt}
    {\gdef\myalttext{#2}}
  \includegraphics[alt=Hello World]
    {example-image}
  The alt text of the graphic was \myalttext.
\end{verbatim}


\subsection{Symbolic structure names}

 The names of structure elements tags may be taken from the standard
 PDF namespaces like \verb|Sect|, \verb|H1| or \verb|Figure| but they
 can also use alternative names provided the latter are role mapped to
 a standard name. The second approach is useful for three reasons:
  \begin{itemize}
  \item It looks nicer, if, e.g., a bible uses tag names such as
    \texttt{Testament}, \texttt{Chapter} or \texttt{Book} instead of
    \texttt{Sect}.
  \item It is possible to formulate additional constraints on such
    structures in a schema and thus ensure that there is no
    \texttt{Testament} inside a \texttt{Book}, something that cannot
    be done if \texttt{Sect} is used everywhere.
  \item We can provide a uniform LaTeX set of names for tags.
  \end{itemize}
  
  Currently it is difficult for document authors to change tag names
  as the tagging support code uses either a fixed name or some ad hoc
  internal variable. We therefore added three commands that offer an
  interface to declare, use and reassign \emph{symbolic structure
  names}. \verb+\NewStructureName+ takes one argument and declares a
  \emph{symbolic structure name}. The expandable command
  \verb+\UseStructureName+ takes one argument and allows to use name
  in a \cs{tagstructbegin} command. \verb+\AssignStructureRole+ allows
  to assign the symbolic structure name a role.

  In the coming months the various tag names in the tagging code will
  be replaced by such symbolic names. Once the process is finished,
  document and class authors will have a flexible tool to set up the
  tag names of their documents.


\subsection{Normalizing key names for block environments}

\emph{to write}


\subsection{Contexts in typesetting}

Sometimes document elements should change their layout depending on
where they are used, for example, lists might use less vertical space
when used inside a footnote or a float. To allow for such designs in a
consistent and easy way we introduce the concept of \enquote{named
  contexts}: while typesetting the document \LaTeX{} keeps track of a
current \enquote{primary context} and a current \enquote{secondary
  context}. They are changed automatically when certain commands, such
as \cs{footnote} or environments, such as floats, etc.\ are typeset.

For the primary context, \LaTeX{} distinguishes by default between,
typesetting material in the main galley (context name is
\meta{empty}), in a \texttt{footnote}, \texttt{marginal},
\texttt{float}, \texttt{caption}, \texttt{header}, or \texttt{footer}.

The \enquote{secondary context} is by default used to identify
typesetting in different font sizes and therefore knows knows about
\texttt{tiny}, \texttt{scriptsize}, \texttt{footnotesize},
\texttt{small}, \texttt{large}, \texttt{Large}, \texttt{LARGE},
\texttt{huge}, \texttt{Huge}, and \meta{empty} (denoting typesetting
in \cs{normalsize}).

In theory it would be possible for commands and environments to query
the current context and then alter their behavior, however that would
require comparably complex coding. Instead, the main usage for the
context is with template instances that are used to define layouts.
If a template instance is used via
\cs{UseInstance}\Arg{type}\Arg{inst-name} then this normally results
in calling up an instance of type \meta{type} with the name
\meta{inst-name}.\footnote{Such instances are defined with a
\cs{DeclareInstance} or \cs{DeclareInstanceCopy} declaration; see the
documentation in the file \texttt{lttemplates-doc.pdf}.}

However, when the \enquote{primary context} and/or the
\enquote{secondary context} is non-empty then \cs{UseInstance}
searches for an instance that is especially tailored to the current
context.  This works as follows:
\begin{itemize}
\item The string\texttt{:}\meta{primary
  context}\texttt{:}\meta{secondary context} is appended to
  \meta{inst-name} and if that instance exist it is
  used.\footnote{Note that this means that if the \meta{primary
    context} is empty we effectively append \texttt{::}\meta{secondary
    context}.}
\item
  If not then \meta{inst-name}\texttt{:}\meta{primary context} is
  tried next.
\item
  If that doesn't exist either then \meta{inst-name} is used as usual.
\end{itemize}
% 
This means it becomes trivial to alter the behavior of instances if
they appear in a special typesetting context. For example, if
\texttt{itemize-1} is the instance name for first-level itemize lists
then one can define another instance named \texttt{itemize-1:footnote}
to describe a special layout used in footnotes.  More details on this
can be found in \texttt{latex-lab-context.pdf} or by using
\verb*=texdoc latex-lab-context= on the command line.

At this point in time the mechanism is still rather experimental,
i.e., we provide and use it in \texttt{latex-lab} to gain experience
and we also encourage developers to experiment with it and provide
feedback. Details of the implementation are likely to change though.




\subsection{MathML intent attributes}

Two new commands, \cs{MathMLintent} and \cs{MathMLarg} are added. They
are defined in the format as no-op so they may be added to command
definitions in packages.  If \pkg{luamml} is enabled to generate
MathML, these commands allow \emph{intent} and \emph{arg} attributes
to be specified.

A  definition such as
\begin{verbatim}
\newcommand\abs[1]{%
\MathMLintent{absolute-value($x)}
             {{\lvert\MathMLarg{x}{#1}\rvert}}%
}
\end{verbatim}
would cause \verb|\abs{y}| to generate
\begin{verbatim}
 <mrow intent="absolute-value($x)">
   <mo>|</mo><mi arg="x">y</mi><mo>|</mo>
 </mrow>
\end{verbatim}
which will allow Assistive Technology (AT) to correctly read the
ambiguous notation $|y|$ as \enquote{the absolute value of y} or some
similar reading depending on the chosen language.


\subsection{Correctly handle tagging of math in tabular cells}

Mathematical content in tabular cells was not correctly tagged when a
MathML representation was automatically generated by \LuaTeX{}. Also
tabular preambles of the form \verb=>{$}c<{$}= or \verb=>{\(}c<{\)}=
failed. This has been corrected.
%
\taggingissue[s]{973 983}



\section{New or improved commands}

\subsection{Support separate font families for script fonts}

In \TeX's math processing separate fonts can be selected for text,
script and scriptscript sizes. \LaTeX's NFSS traditionally uses the
same font family at different sizes, handling adjustment needed for
making fonts appear better in a script location through the use of
optical sizes. This works great for traditional \TeX{} fonts, but for
OpenType fonts this leads to issues. OpenType MATH assumes the font in
a script location has separate features set and therefore received
specific adjustments.

To support this without relying on heuristics based on the font size,
a new command \verb+\DeclareMathScriptfontMapping+ has been added. It
takes 3 pairs of encoding/family arguments to indicate that for the
first pair when used as the math main font the second and the third
should be used as the script and scriptscript font, respectively.
%
\githubissue{1707}


\subsection[Programming support for font metafamilies]
           {Programming support for \LaTeX's font metafamilies}

\LaTeX{} knows three main document font families: \cs{rmfamily} for
the document's serifed font family, \cs{sffamily} for its sans serif
font family, and \cs{ttfamily} for its monospaced font family. In
addition, other font families can be used by the user or in a document
class or package by explicitly loading them through
\cs{fontfamily}\texttt\{\meta{name}\texttt\}\cs{selectfont}.

In some cases it is helpful to know which of the three metafamilies
(if any) is currently used for typesetting, and this information is
now made available for programmers in \cs{@currentmetafamily}. It
returns either \texttt{rm}, \texttt{sf}, \texttt{tt}, or \texttt{??}
(in case none of the metafamilies is currently used).

As a small application of this, the \LaTeX{} kernel now also contains
\cs{@restoremetafamily}. If the current metafamily is \meta{name} it
executes \cs{\meta{name}family}, e.g., \cs{sffamily}, and that then
executes the hook \texttt{\meta{name}family} besides other
re-initializations. This can be useful if that hook contains
conditional code and the condition has changed and therefore requires
re-initialization.

\subsection{Recovering the argument specifier for document commands}

In \LaTeX{} News 38~\cite{42:ltnews38} we explained that we had
\emph{removed} \cs{GetDocumentCommandArgSpec} since we felt that it
was only required for debugging. However, there are some specialist
use cases where access to the argument specification is useful: see,
for example, \url{https://github.com/latex3/latex3/pull/1799}. We have
therefore looked again at this area and added a \emph{code} interface
\cs{\detokenize{cmd_arg_spec:N}} for accessing the argument
specification. The use of a code-level rather than design-level name
here reflects the fact that this is an very specialized use case,
mainly of interest to package authors.

\section{Code improvements}

\subsection{Ensure that commands without arguments are not \cs{long}}

In its original implementation \cs{newcommand} or \cs{renewcommand}
always defined commands using \cs{long}\cs{def} even if the commands
had no arguments, i.e., in situations where the concept of \cs{long}
made no sense whatsoever.

The issue with that behavior is that commands differing only in their
\cs{long} status are nevertheless considered different when compared
with \cs{ifx} even if there are no arguments to which the \cs{long}
would apply. Thus, after \verb=\renewcommand\rmdefault{lmr}= and
\verb=\def\test{lmr}= the test \verb=\ifx\test\rmdefault= would be
\emph{false}, but it would be \emph{true} if \cs{rmdefault} had been
defined using \cs{def} (as many class files do).  This made comparing
commands without argument rather difficult.
%
We have therefore changed \cs{newcommand} and friends so that commands
without arguments are always defined without using the unnecessary
\cs{long} prefix.

Going forward, this will simplify package and kernel code as the code
can reliably assume that such macros are not \cs{long} regardless of
whether they are defined by \cs{renewcommand} or \cs{def}.

There is a small chance that this is a breaking change for some
package code (though we don't know of any case), i.e., if the code was
deliberately checking against \cs{long}\cs{def} only\Dash in that case
the test now needs to be made against the definition without \cs{long}
(or against both, which is what the NFSS implementation of the kernel
did in the past).
%
\githubissue{571}



\subsection{Avoid strange warnings about font substitutions}

A font series value such as \texttt{sbc} contains both the weight
(\texttt{sb}, i.e.\ ``semibold'') and the width (\texttt{c},
i.e.\ ``condensed'') of the font. If you want to reset only one of the
two to ``medium'' and keep the other, you can use
\verb+\fontseries{m?}+ or \verb+\fontseries{?m}+: The former switches
\texttt{sbc} to \texttt{c}, the latter switches \texttt{sbc} to
\texttt{sb}. However, if the resulting series does not exist, you got
strange warnings in the past, e.g.:
\begin{verbatim}
LaTeX Font Warning:
Font shape `OT1/cmss/c/n' undefined
using `OT1/cmss/m?/n' instead on input line 7.
LaTeX Font Warning:
Font shape `OT1/cmss/m?/n' undefined
using `OT1/cmss/m/n' instead on input line 7.
\end{verbatim}
This has now been corrected so that you get a single, more meaningful
warning:
\begin{verbatim}
LaTeX Font Warning:
Font shape `OT1/cmss/c/n' undefined
using `OT1/cmss/m/n' instead on input line 7.
\end{verbatim}
If the \texttt{m} series does not exist either, you will still get
strange warnings, but this should only affect very few fonts. The
source file was also tidied up a little on this occasion.
%
\githubissue{1727}


\subsection{Improved handling of infinite shrinkage errors}

In the June 2024 release~\cite{42:ltnews39} we described the improved
mark mechanism and the problems we had when working around \TeX's
\enquote{infinite shrinkage error}. By now the engines got a new
primitive \cs{ignoreprimitiveerror} which can be used to turn this
error into a warning, when, for example, you do only a trial splitting
of a box. This noticeably improves the output in the \texttt{.log}
file from
\begin{verbatim}
! Infinite glue shrinkage found in box being split.
<argument> Infinite shrink error above ignored !
l. ...   }
The box you are \vsplitting contains some
infinitely shrinkable glue, e.g., `\vss' or
`\vskip 0pt minus 1fil'. Such glue doesn't belong
there; but you can safely proceed, since the
offensive shrinkability has been made finite.
\end{verbatim}
to a simple
\begin{verbatim}
ignored error: Infinite glue shrinkage found in
box being split
\end{verbatim}
As an important side effect, the return code from the \TeX{} run stays
at 0 (unless there are real errors); so in workflows that want to test
whether a \TeX{} run ended without errors, you don't get a bogus
result because there is no longer an ignored error.
%
\githubissue{1750}


\subsection{Allow multiple family names in \cs{ProcessKeyOptions}}

The ability to process key--value options was introduced into the
kernel in the June 2022 release~\cite{42:ltnews35}, with the command
\cs{ProcessKeyOptions} carrying out the option assignment. In the
original version, this takes an optional argument which can select one
key family (namespace) for options. We have now extended this to take
a comma list of possible families.
%
\githubissue{1756}

\subsection{Control of value expansion in keys}

Normally, key--value input is treated ``as is'', with no expansion of
either key names or values. However, there are occasions when the
expansion of selected values is useful. We have now extended the key
handling for templates (\cs{DeclareInstance}, etc.) and for keys
created using the L3 programming layer to allow selective
expansion. In both cases, the syntax uses a trailing colon and a
single letter specifier: these letters are those used in
\cs{ExpandArgs} or the L3 programming layer. For example, to use the
values of the \LaTeXe{} variable \cs{@itemdepth}, one could have
settings
\begin{verbatim}
  key-a:c =  @itemdepth ,
  key-b:v =  @itemdepth
\end{verbatim}
This facility will \emph{automatically} be available in any package
setup macro using the L3 programming layer, for example \pkg{siunitx}.
%
\githubissue{1801}

\subsection{Support word exclusion in case changing}

Work on improving automatic case changing over previous releases has
continued.  We have now added the ability to `register' words for
exclusion from case changing, using \cs{DeclareLowercaseExclusions},
\cs{DeclareTitlecaseExclusions} and \cs{DeclareUppercaseExclusions}.


\subsection{Automatic insertion of \cs{par} tokens}

Since 2022 the major \TeX\ engines have a parameter,
\cs{partokencontext}, that controls whether a \cs{par} token is added
when \TeX\ is in horizontal mode at the end of \cs{vbox} and in
similar contexts. This gives more control than the classical behavior
where the internal \textit{end paragraph} routine is invoked with no
explicit token being added.

This allows the paragraph hooks to detect the end of paragraph even in
contexts such as at the end of a \cs{vbox}, where traditionally
package code has had to be modified to add an explicit \cs{par}. This
is expected to improve compatibility of existing packages with the
tagging code.

\LaTeX\ now sets this parameter to 2 by default to enable automatic
insertion of \cs{par}.
%
\githubissue{1864}


\section{Bug fixes}

\subsection{Support active characters correctly with \cs{DeclareRobustCommand}}

The mechanism used by \cs{DeclareRobustCommand} creates an internal
command which has a space added to the name of the document one: so
\verb*|\foo | for a command \verb|\foo|. That fails if applied to an
active character: unlike normal commands, these have to be exactly one
character long. Due to the way the implementation works, to date this
would result in redefining \verb*|\ | every time
\cs{DeclareRobustCommand} was used with an active character. This has
now been corrected: robust active characters are now created using the
engine \cs{protected} mechanism and do not use an internal
auxiliary. They still work in file names and labels to give the
character itself.
%
\githubissue{345}

\subsection{Avoid a \enquote{Corrupted NFSS tables} error}

When a character with an accent is typeset, say \enquote{ä} or
\enquote{é}, it might be the case that it doesn't exist in the font
but has to be constructed from the base character and a standalone
accent. If that accent is also not available in the font then \LaTeX{}
attempts to find it in a different font, typically one in a different
encoding, e.g., \texttt{OT1}. Unfortunately, when that involved font
substitutions it resulted in a loop generating the mentioned error.
This has now been corrected by adding necessary
\cs{DeclareFontSubstitution} statements.
%
\githubissue{1709}



%\subsection{A fix}

%% Some text
%
%%\githubissue{XXXX}

%\section{Changes to packages in the \pkg{amsmath} category}

%\section{Changes to packages in the \pkg{graphics} category}

\section{Changes to packages in the \pkg{tools} category}

\subsection{Updating the status of some components}

The \pkg{tools} bundle contains a range of packages with different
usage profiles. Some of these were necessary in the transition from
\LaTeX{}~2.09 to \LaTeXe{}, while others are very widely used in
current documents (for example \pkg{array}). We have therefore marked
a small number of packages in \pkg{tools} as \emph{retained only for
historical and stability reasons}, and where relevant pointed to more
up-to-date alternatives; the list is:
\begin{itemize}
  \item \pkg{enumerate}: use \pkg{enumitem} instead
  \item \pkg{rawfonts}: retained as part of \LaTeX{}~2.09 support
  \item \pkg{somedefs}: retained as part of \LaTeX{}~2.09 support
  \item \pkg{theorem}: use \pkg{amsthm} instead
  \item \pkg{verbatim}: use \pkg{fancyvrb} instead
\end{itemize}

%\section{Changes to files in the \pkg{cyrillic} category}

\begin{thebibliography}{9}\frenchspacing

%\fontsize{9.3}{11.3}\selectfont

\bibitem{42:Lamport}
Leslie Lamport.
\newblock \emph{{\LaTeX}: {A} Document Preparation System: User's Guide and Reference
  Manual}.
\newblock \mbox{Addison}-Wesley, Reading, MA, USA, 2nd edition, 1994.
\newblock ISBN 0-201-52983-1.
\newblock Reprinted with corrections in 1996.

\bibitem{42:ltnews} \LaTeX{} Project Team.
  \emph{\LaTeXe{} News 1--42}. November 2025.
  \url{https://latex-project.org/news/latex2e-news/ltnews.pdf}

\bibitem{42:ltnews35} \LaTeX{} Project Team.
  \emph{\LaTeXe{} News 35}. June 2022.
  \url{https://latex-project.org/news/latex2e-news/ltnews35.pdf}

\bibitem{42:ltnews38} \LaTeX{} Project Team.
  \emph{\LaTeXe{} News 38}. November 2023.
  \url{https://latex-project.org/news/latex2e-news/ltnews38.pdf}

\bibitem{42:ltnews39} \LaTeX{} Project Team.
  \emph{\LaTeXe{} News 39}. June 2024.
  \url{https://latex-project.org/news/latex2e-news/ltnews39.pdf}

\bibitem{42:ltnews41} \LaTeX{} Project Team.
  \emph{\LaTeXe{} News 41}. June 2025.
  \url{https://latex-project.org/news/latex2e-news/ltnews41.pdf}

\bibitem{tagging-status} \LaTeX{} Project Team.
  \emph{Tagging Status of \LaTeX{} Packages and Classes}. November 2025.
  \url{https://latex3.github.io/tagging-project/tagging-status}
  
\bibitem{tagging-project} \LaTeX{} Project Team.
  \emph{The \LaTeX{} Tagged PDF repository}. November 2025.
  \url{https://github.com/latex3/tagging-project/issues}

\end{thebibliography}

\end{document}
