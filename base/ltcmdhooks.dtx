% \iffalse meta-comment
%
%% From File: ltcmdhooks.dtx
%% Copyright (C) 2020-2025
%% Frank Mittelbach, Phelype Oleinik, The LaTeX Project
%
% It may be distributed and/or modified under the conditions of the
% LaTeX Project Public License (LPPL), either version 1.3c of this
% license or (at your option) any later version.  The latest version
% of this license is in the file
%
%    https://www.latex-project.org/lppl.txt
%
%
% \fi
%
% \iffalse
%%% From File: lthooks.dtx
%
%<*driver>
% \fi
\ProvidesFile{ltcmdhooks.dtx}
             [2025/06/27 v1.0l LaTeX Kernel (Command hooks)]
% \iffalse
%
\documentclass{l3doc}
\GetFileInfo{ltcmdhooks.dtx}
% \usepackage{ltcmdhooks}
\EnableCrossrefs
\CodelineIndex
\begin{document}
  \DocInput{ltcmdhooks.dtx}
\end{document}
%</driver>
%
% \fi
%
%
% \providecommand\hook[1]{\texttt{#1}}
% \providecommand\fmi[1]{\marginpar{\footnotesize FMi: #1}}
% \providecommand\pho[1]{\marginpar{\footnotesize PhO: #1}}
% \providecommand\phoinline[1]{\begin{quote}\itshape\footnotesize PhO: #1\end{quote}}
%
% \title{The \texttt{ltcmdhooks} module\thanks{This file has version
%    \fileversion\ dated \filedate, \copyright\ \LaTeX\
%    Project.}}
% \author{Frank Mittelbach \and Phelype Oleinik}
%
% \maketitle
%
%
% \tableofcontents
%
%
% \section{Introduction}
%
%    This file implements generic hooks for (arbitrary) commands.
%    In theory every command \tn[no-index]{\meta{name}} offers now two
%    associated hooks to which code can be added using
%    \tn{AddToHook},\!\footnote{In this documentation, when something is
%      being said about \tn{AddToHook}, the same will be valid for
%      \tn{AddToHookWithArguments}, unless that particular paragraph is
%      highlighting the differences between both.  The same is true for
%      the other hook-related functions and their
%      \texttt{\ldots WithArguments} counterparts.}
%    \tn{AddToHookNext}, \tn{AddToHookWithArguments}, and
%    \tn{AddToHookNextWithArguments}.\footnote{In practice this is not
%    supported
%    for all types of commands, see section~\ref{sec:look-ahead} for
%    the restrictions that apply and what happens if one tries to use
%    this with commands for which this is not supported.}
%
%    However, this is only true \enquote{in theory}. In practice there
%    are a number of restrictions that makes it impossible to use such
%    generic command hooks in a number of cases, so please read all
%    of section~\ref{sec:restrictions} to understand what may prevent
%    you from using them successfully.
%
%    The generic command hooks are:
%    \begin{description}
%    \item[\hook{cmd/\meta{name}/before}]
%
%       This hook is executed at the very start of the command, right
%       after its arguments (if any) are parsed.  The hook \meta{code}
%       runs in the command inside a call to \cs{UseHookWithArguments}.
%       Any code added to this hook using \tn{AddToHookWithArguments}
%       or \tn{AddToHookNextWithArguments} can access the command's
%       arguments using |#1|, |#2|, etc., up to the number of arguments
%       of the command.  If \tn{AddToHook} or \tn{AddToHookNext} are
%       used, the arguments cannot be accessed (see the \pkg{lthooks}
%       documentation\footnote{\texttt{texdoc lthooks-doc}} on hooks
%       with arguments).
%
%    \item[\hook{cmd/\meta{name}/after}]
%
%       This hook is similar to \hook{cmd/\meta{name}/before}, but it is
%       executed at the very end of the command body.  This hook is
%       implemented as a reversed hook.
%    \end{description}
%
%    The hooks are not physically present before
%    \verb=\begin{document}=\footnote{More specifically, they are
%    inserted in the commands after the \hook{begindocument} hook, so
%    they are also not present while \LaTeX{} is reading the
%    \texttt{.aux} file.} (i.e., using a command in the preamble
%    will never execute the hook) and if nobody has declared any code
%    for them, then they are not added to the command code ever.  For
%    example, if we have the following definition
%\begin{verbatim}
%    \newcommand\foo[2]{Code #1 for #2!}
%\end{verbatim}
%    then executing \verb=\foo{A}{B}= will simply run
%    \verb*=Code A for B!=
%    as it was always the case. However, if somebody, somewhere (e.g.,
%    in a package) adds
%\begin{verbatim}
%    \AddToHook{cmd/foo/before}{<before code>}
%\end{verbatim}
%    then, after |\begin{document}| the definition of \cs[no-index]{foo} will be:
%\begin{verbatim}
%    \renewcommand\foo[2]{%
%      \UseHookWithArguments{cmd/foo/before}{2}{#1}{#2}%
%      Code #1 for #2!}
%\end{verbatim}
%    and similarly \verb=\AddToHook{cmd/foo/after}{<after code>}=
%    alters the definition to
%\begin{verbatim}
%    \renewcommand\foo[2]{%
%      Code #1 for #2!%
%      \UseHookWithArguments{cmd/foo/after}{2}{#1}{#2}}
%\end{verbatim}
%
%    In other words, the mechanism is similar to what \pkg{etoolbox}
%    offers with \tn{pretocmd} and \tn{apptocmd} with the important
%    differences
%    \begin{itemize}
%    \item
%
%      that code can be prepended or appended (i.e., added to the
%      hooks) even if the command itself is not (yet) defined, because the
%      defining package has not been loaded at this point;
%
%    \item
%
%      and that by using the hook management interface it is now
%      possible to define how the code chunks added in these places
%      are ordered, if different packages want to add code at these
%      points.
%
%    \end{itemize}
%
%
%
%
% \section{Restrictions and Operational details}
% \label{sec:restrictions}
%
% Adding arbitrary material to commands is tricky because most of the
% time we do not know what the macro expects as arguments when expanding
% and \TeX{} doesn't have a reliable way to see that, so some guesswork
% has to be employed.
%
% We can do this in most cases when commands are defined using
% \cs{NewDocumentCommand} or \cs{newcommand} (with a few exceptions).
% For commands defined with \tn{def} the situation is less good.
% Common cases where the command hooks will not work are:
% \begin{itemize}
% \item
%
%   Commands that use special catcode settings within their
%   definition. In that case it is usually not possible to augment the
%   definition (see~\ref{sec:patching}).
%
% \item
%
%   If a command is defined while \cs{ExplSyntaxOn} is in force
%   \textbf{and} the command contains \verb=~= characters to represent
%   spaces, then it can't be patched to include the command hooks. In
%   fact in some very special circumstances you might even get a
%   low-level error rather than the information that the command can't
%   be patched (see,  for example,
%   \url{https://github.com/latex3/latex2e/issues/1430}.
%
% \item
%
%   Commands that have arguments as far as the user is concerned
%   (e.g., \cs{section} or \cs{caption}), but are defined in a way that these
%   arguments are not read by the user level command but only later
%   during the processing. In that case the \texttt{after} hook
%   doesn't work at all.  The \text{before} hook only works with
%   \cs{AddToHook} but not with \cs{AddToHookWithArguments} because the
%   arguments haven't been read at that point where the hook is
%   patched in. See
%   section~\ref{sec:look-ahead}.
%
%
% \item
%   Adding a specific  generic command hook is only attempted once per
%   command, thus after redefining a command such hooks will no longer
%   be there and will also not being re-added, see section~\ref{sec:timing}.
%
% \end{itemize}
% All this means that you have to have a good understanding of how
% commands are defined when you attempt to make use of such hooks and
% something goes wrong.
% What can help in that case is to turn on \cs{DebugHooksOn} in which
% case you get much more (low-level) details on why something fails and
% what was tried to enable the hooks.
%
%
% \subsection{Patching}\label{sec:patching}
%
% The code here tries to find out if a command was defined with
% \tn{newcommand} or \tn{DeclareRobustCommand} or
% \tn{NewDocumentCommand}, and if so it \emph{assumes} that the argument
% specification of the command is as expected (which is not fail-proof,
% if someone redefines the internals of these commands in devious ways,
% but is a reasonable assumption).
%
% If the command is one of the defined types, the code here does a
% sandboxed expansion of the command such that it can be redefined again
% exactly as before, but with the hook code added.
%
% If however the command is not a known type (it was defined with
% \tn{def}, for example), then the code uses an approach similar to
% \pkg{etoolbox}'s \tn{patchcmd} to retokenize the command with the hook
% code in place.  This procedure, however, is more likely to fail if the
% catcode settings are not the same as the ones at the time of command's
% definition, so not always adding a hook to a command will work.
%
% \subsubsection{Timing}\label{sec:timing}
%
% When \cs{AddToHook} (or its \pkg{expl3} equivalent) is called with
% a generic |cmd| hook, say, \hook{cmd/foo/before}, for the first time
% (that is, no code was added to that same hook before), in the preamble
% of a document, it will store a patch instruction for that command
% until |\begin{document}|, and only then all the commands which had
% hooks added will be patched in one go.  That means that no command in
% the preamble will have hooks patched into them.
%
% At |\begin{document}| all the delayed patches will be executed, and
% if the command doesn't exist the code is still added to the hook,
% but it will not be executed.  After |\begin{document}|, when
% \cs{AddToHook} is called with a generic |cmd| hook the first time,
% the command will be immediately patched to include the hook, and if
% it doesn't exist or if it can't be patched for any reason, an error
% is thrown; if \cs{AddToHook} was already used in the preamble no new
% patching is attempted.
%
% This has the consequence that a command defined or redefined after
% |\begin{document}| only uses generic |cmd| hook code if
% \cs{AddToHook} is called for the first time after the definition is
% made, or if the command explicitly uses the generic hook in its
% definition by declaring it with \cs{NewHookPair} adding \cs{UseHook} as
% part of the code.\footnote{We might change this behavior in the main
% document slightly after gaining some usage experience.}
%
%
% \subsection{Commands that look ahead}
% \label{sec:look-ahead}
%
% Some commands are defined in different ``steps'' and they look ahead
% in the input stream to find more arguments.  If you try to add some
% code to the \hook{cmd/\meta{name}/after} hook of such command, it will
% not work, and it is not possible to detect that programmatically, so
% the user has to know (or find out) which commands can or cannot have
% hooks attached to them.
%
% One good example is the \tn{section} command.  You can add something
% to the \hook{cmd/section/before} hook (but only with \cs{AddToHook}
% not \cs{AddToHookWithArguments}),
% but if you try to add anything to the \hook{cmd/section/after}
% hook, \tn{section} will no longer work at all.  That happens because the
% \tn{section} macro takes no argument, but instead calls a few
% internal \LaTeX{} macros to look for the optional and mandatory
% arguments.  By adding code to the \hook{cmd/section/after} hook, you
% get in the way of that scanning.
%
% In such a case, where it is known that a specific generic command
% hook does not work if code is added to it, the package author can
% add a \cs{DisableGenericHook}\footnote{Please use
% \cs{DisableGenericHook} if at all, only on hooks that you
% \enquote{own}, i.e., for commands your package or class defines and
% not second guess whether or not hooks of other packages should get
% disabled!} declaration to prevent this from happening in user
% documents and thereby avoiding obscure errors.
%
%
% \section{Package Author Interface}
% \label{sec:pkg-author}
%
% The \hook{cmd} hooks are, by default, available for all commands
% that can be patched to add the hooks.  For some commands, however,
% the very beginning or the very end of the code is not the best place
% to put the hooks, for example, if the command looks ahead for
% arguments (see section~\ref{sec:look-ahead}).
%
% If you are a package author and you want to add the hooks to your
% own commands in the proper position you can define the command and
% manually add the \cs{UseHookWithArguments} calls inside the command in
% the proper positions, and manually define the hooks with
% \cs{NewHookWithArguments} or \cs{NewReversedHookWithArguments}.
% When the hooks are explicitly defined,
% patching is not attempted so you can make sure your command works
% properly.  For example, an (admittedly not really useful) command
% that typesets its contents in a framed box with width optionally
% given in parentheses:
% \begin{verbatim}
%    \newcommand\fancybox{\@ifnextchar({\@fancybox}{\@fancybox(5cm)}}
%    \def\@fancybox(#1)#2{\fbox{\parbox{#1}{#2}}}
% \end{verbatim}
% If you try that definition, then add some code after it with
% \begin{verbatim}
%    \AddToHook{cmd/fancybox/after}{<code>}
% \end{verbatim}
% and then use the \cs[no-index]{fancybox} command you will see that it
% will be completely broken, because the hook will get executed in the
% middle of parsing for optional \texttt{(...)} argument.
%
% If, on the other hand, you want to add hooks to your command you can
% do something like:
% \begin{verbatim}
%   \newcommand\fancybox{\@ifnextchar({\@fancybox}{\@fancybox(5cm)}}
%   \def\@fancybox(#1)#2{\fbox{%
%                 \UseHookWithArguments{cmd/fancybox/before}{2}{#1}{#2}%
%                 \parbox{#1}{#2}%
%                 \UseHookWithArguments{cmd/fancybox/after}{2}{#1}{#2}}}
%   \NewHookWithArguments{cmd/fancybox/before}{2}
%   \NewReversedHookWithArguments{cmd/fancybox/after}{2}
% \end{verbatim}
% then the hooks will be executed where they should and no patching
% will be attempted.  It is important that the hooks are declared with
% \cs{NewHookWithArguments} or \cs{NewReversedHookWithArguments},
% otherwise the command hook
% code will try to patch the command.  Note also that the call to
% |\UseHookWithArguments{cmd/fancybox/before}| does not need to be in
% the definition of \cs[no-index]{fancybox}, but anywhere it makes sense
% to insert it (in this case in the internal
% \cs[no-index]{@fancybox}).
%
% Alternatively, if for whatever reason your command does not support
% the generic hooks provided here, you can disable a hook with
% \cs{DisableGenericHook}\footnote{Please use \cs{DisableGenericHook} if
% at all, only on hooks that you \enquote{own}, i.e., for commands
% your package or class defines and not second guess
% whether or not hooks of other packages should get disabled!}, so
% that when someone tries to add code to it they will get an error.
% Or if you don't want the error, you can simply declare the hook with
% \cs{NewHook} and never use it.
%
%
% The above approach is useful for really complex commands where for
% one or the other reason the hooks can't be placed at the very
% beginning and end of the command body and some hand-crafting is
% needed. However, in the example above the real (and in fact only)
% issue is the cascading argument parsing in the style developed long
% ago in \LaTeX~2.09. Thus, a much simpler solution for this case is
% to replace it with the modern \cs{NewDocumentCommand} syntax and
% define the command as follows:
% \begin{verbatim}
%  \DeclareDocumentCommand\fancybox{D(){5cm}m}{\fbox{\parbox{#1}{#2}}}
% \end{verbatim}
% If you do that then both hooks automatically work and are patched
% into the right places.
%
% \subsection{Arguments and redefining commands}
% \label{sec:redef-warn}
%
% The code in \pkg{ltcmdhooks} does its best to find out how many
% arguments a given command has, and to insert the appropriate call to
% \cs{UseHookWithArguments}, so that the arguments seen by the hook are
% exactly those grabbed by the command (the hook, after all, is a macro
% call, so the arguments have to be placed in the right order, or they
% won't match).
%
% When using the package writer interface, as discussed in
% section~\ref{sec:pkg-author}, to change the position of the hooks in
% your commands, you are also free to change how the hook code in your
% command sees its arguments.  When a \hook{cmd} hook is declared with
% \cs{NewHook} (or \cs{NewHookWithArguments} or other variations of
% that), it loses its \enquote{generic} nature and works as a regular
% hook.  This means that you may choose to declare it without
% arguments regardless if the command takes arguments or not, or
% declare it with arguments, even if the command takes none.
%
% However, this flexibility should not be abused.  When using a
% nonstandard configuration for the hook arguments, think reasonably:
% a user will expect that the argument \verb|#1| in the hook corresponds
% to the argument's first argument, and so on.  Any other configuration
% is likely to cause confusion and, if used, will have to be well
% documented.
%
% This flexibility, however, allows you to \enquote{correct} the
% arguments for the hooks.  For example, \LaTeX's \cs{refstepcounter}
% has a single argument, the name of the counter.  The \pkg{cleveref}
% package adds an optional argument to \cs{refstepcounter}, making the
% name of the counter argument \verb|#2|.  If the author of
% \pkg{cleveref} wanted, for whatever reason, to add hooks to
% \cs{refstepcounter}, to preserve compatibility he could write
% something along the lines of:
% \begin{verbatim}
%   \NewHookWithArguments{cmd/refstepcounter/before}{1}
%   \renewcommand\refstepcounter[2][<default>]{%
%     \UseHookWithArguments{cmd/refstepcounter/before}{1}{#2}%
%     <code for \refstepcounter>}
% \end{verbatim}
% so that the mandatory argument, which is arg \verb|#2| in the
% definition, would still be seen as \verb|#1| in the hook code.
%
% Another possibility would be to place the optional argument as the
% second argument for the hook, so that people looking for it would be
% able to use it.  In either case, it would have to be well documented
% to cause as little confusion as possible.
%
% \MaybeStop{\setlength\IndexMin{200pt}  \PrintIndex  }
%
%
%
% \section{The Implementation}
%
% \subsection{Execution plan}
%
% To add |before| and |after| hooks to a command we will need to peek
% into the definition of a command, which is always a tricky thing to
% do.  Some cases are easy because we know how the command was defined,
% so we can assume how its \meta{parameter text} looks like (for example
% a command defined with \tn{newcommand} may have an optional argument
% followed by a run of mandatory arguments), so we can just expand that
% command and make it grab |#1|, |#2|, etc.\@ as arguments and
% define it all back with the hooks added.
%
% Life's usually not that easy, so with some commands we can't do that
% (a |#1| might as well be |#|$_{12}$|1|$_{12}$ instead of the expected
% |#|$_{6}$|1|$_{12}$, for example) so we need to resort to ``patching''
% the command: read its \tn{meaning}, and tokenize it again with
% \tn{scantokens} and hope for the best.
%
% So the overall plan is:
% \begin{enumerate}
%   \item
%     Check if a command is of a known type (that is, defined with
%     \tn{newcommand}\footnote{It's not always possible to reliably
%     detect this case because a command defined with no optional
%     argument is indistinguishable from a \tn{def}ed command.},
%     \cs[no-index]{DeclareRobustCommand}, or
%     \cs[no-index]{New(Expandable)DocumentCommand}), and if is, take
%     appropriate action.
%   \item
%     If the command is not a known type, we'll check if the command can
%     be patched.  Two things will prevent a command from being
%     patched:  if it was defined in a nonstandard catcode setting, or
%     if it is an internal expl3 command with |__|\meta{module} in its
%     name, in which case we refuse to patch.
%   \item
%     If the command was defined in nonstandard catcode settings, we
%     will try a few standard ones to try our best to carry out the
%     pathing.  If this doesn't help either, the code will give up and
%     throw an error.
% \end{enumerate}
%
%
%    \begin{macrocode}
%<@@=hook>
%    \end{macrocode}
%
% \changes{v1.0b}{2021/05/24}{Use \cs{msg_...} instead of \cs{__kernel_msg...}}
% \changes{v1.0j}{2024/04/17}{Use \cs{__kernel_cs_parameter_spec:N} instead
%   of \cs{cs_argument_spec:N}/\cs{cs_parameter_spec:N}}
%
%    \begin{macrocode}
%<*2ekernel|latexrelease>
\ExplSyntaxOn
%<latexrelease>\NewModuleRelease{2021/06/01}{ltcmdhooks}
%<latexrelease>                 {The~hook~management~system~for~commands}
%    \end{macrocode}
%
% \subsection{Variables}
%
% \begin{macro}[int]{\g_hook_patch_action_list_tl}
%    Pairs of |\if<cmd>..\patch<cmd>| to be used with
%    \tn{robust@command@act} when looking for a known patching
%    rule. This token list is exposed because we see some future
%    applications (with very specialized packages, such as
%    \pkg{etoolbox} that may want to extend the pairs processed. It is
%    not meant for general use which is why it is not documented in
%    the interface documentation above.
%    \begin{macrocode}
\tl_new:N \g_hook_patch_action_list_tl
%    \end{macrocode}
% \end{macro}
%
% \begin{macro}{\l_@@_patch_num_args_int}
%   The number of arguments in a macro being patched.
%    \begin{macrocode}
\int_new:N \l_@@_patch_num_args_int
%    \end{macrocode}
% \end{macro}
%
% \begin{macro}{\l_@@_patch_prefixes_tl}
% \begin{macro}{\l_@@_param_text_tl}
% \begin{macro}{\l_@@_replace_text_tl}
%   The prefixes and parameters of the definition for the macro being
%   patched.
%    \begin{macrocode}
\tl_new:N \l_@@_patch_prefixes_tl
\tl_new:N \l_@@_param_text_tl
\tl_new:N \l_@@_replace_text_tl
%    \end{macrocode}
% \end{macro}
% \end{macro}
% \end{macro}
%
% \begin{macro}{\c_@@_hash_tl,\c_@@_hashes_tl}
%   Two constant token lists that contain one and two parameter tokens.
%   \changes{v1.0g}{2023/04/06}
%           {Rename to \cs{c__hook_hashes_tl} and add \cs{c__hook_hash_tl} (hook-args)}
%    \begin{macrocode}
\tl_const:Nn \c_@@_hash_tl { # }
\tl_const:Nn \c_@@_hashes_tl { # # }
%    \end{macrocode}
% \end{macro}
%
% \begin{macro}{\@@_exp_not:NN}
% \begin{macro}{\@@_def_cmd:w}
%   Two temporary macros that change depending on the macro being
%   patched.
%    \begin{macrocode}
\cs_new_eq:NN \@@_exp_not:NN ?
\cs_new_eq:NN \@@_def_cmd:w ?
%    \end{macrocode}
% \end{macro}
% \end{macro}
%
% \begin{macro}{\q_@@_recursion_tail,\q_@@_recursion_stop}
%   Internal quarks for recursion:  they can't appear in any macro being
%   patched.
%    \begin{macrocode}
\quark_new:N \q_@@_recursion_tail
\quark_new:N \q_@@_recursion_stop
%    \end{macrocode}
% \end{macro}
%
% \begin{macro}{\g_@@_delayed_patches_prop}
%   A list containing the patches delayed to |\begin{document}|, so that
%   patching is not attempted twice.
%    \begin{macrocode}
\prop_new:N \g_@@_delayed_patches_prop
%    \end{macrocode}
% \end{macro}
%
% \begin{macro}{\@@_patch_debug:x}
%   A helper for patching debug info.
%    \begin{macrocode}
\cs_new_protected:Npn \@@_patch_debug:x #1
  { \@@_debug:n { \iow_term:x { [lthooks]~#1 } } }
%    \end{macrocode}
% \end{macro}
%
% \subsection{Variants}
%
% \begin{macro}[int]{\tl_rescan:nV}
%   \pkg{expl3} function variants used throughout the code.
%    \begin{macrocode}
\cs_generate_variant:Nn \tl_rescan:nn { nV }
%    \end{macrocode}
% \end{macro}
%
% \subsection{Patching or delaying}
%
%   Before |\begin{document}| all patching is delayed.
%
% \begin{macro}{\@@_try_put_cmd_hook:n,\@@_try_put_cmd_hook:w}
%   This function is called from within \cs{AddToHook}, when code is
%   first added to a generic |cmd| hook.  
%   If it is called within in the preamble, it delays the action
%   until |\begin{document}|; 
%   otherwise it tries to update the hook.
% \changes{v1.0d}{2021/08/25}{Simplify generic hook detection}
%    \begin{macrocode}
%<latexrelease>\IncludeInRelease{2024/12/22}{\@@_try_put_cmd_hook:n}%
%<latexrelease>                 {Don't~define~command}
\cs_new_protected:Npn \@@_try_put_cmd_hook:n #1
  { \@@_try_put_cmd_hook:w #1 / / / \s_@@_mark {#1} }
\cs_new_protected:Npn \@@_try_put_cmd_hook:w
    #1 / #2 / #3 / #4 \s_@@_mark #5
  {
    \@@_debug:n { \iow_term:n { ->~Adding~cmd~hook~to~'#2'~(#3): } }
%    \end{macrocode}
%    \cs{@@_patch_cmd_or_delay:Nnn} expects the command to be patched
%    as its first argument so we need to construct it from its name
%    (\texttt{\#2}). However, at this moment it may not exist yet, so
%    using \cs{cs:w} would incorrectly turn it from
%    \enquote{undefined} into \cs{relax}. We  therefore use the
%    following curious construction: we start a group and expand out of
%    it to call \cs{cs:w}. If the command is now changed to \cs{relax}
%    the \cs{group_end:} will undo that change, but the token is
%    nevertheless there to be consumed by \cs{@@_patch_cmd_or_delay:Nnn}.
% \changes{v1.0k}{2024/12/22}{Avoid defining command while adding a cmd
%    hook (gh/1591)}
%    \begin{macrocode}
    \group_begin:
       \exp_after:wN
    \group_end:
       \exp_after:wN
    \@@_patch_cmd_or_delay:Nnn
       \cs:w #2\cs_end:
       {#2} {#3}
  }
%<latexrelease>\EndIncludeInRelease
%    \end{macrocode}
%
%    \begin{macrocode}
%<latexrelease>\IncludeInRelease{2021/11/15}{\@@_try_put_cmd_hook:n}%
%<latexrelease>                 {Standardise~generic~hook~names}
%<latexrelease>\cs_new_protected:Npn \@@_try_put_cmd_hook:n #1
%<latexrelease>  { \@@_try_put_cmd_hook:w #1 / / / \s_@@_mark {#1} }
%<latexrelease>\cs_new_protected:Npn \@@_try_put_cmd_hook:w
%<latexrelease>    #1 / #2 / #3 / #4 \s_@@_mark #5
%<latexrelease>  {
%<latexrelease>    \@@_debug:n { \iow_term:n { ->~Adding~cmd~hook~to~'#2'~(#3): } }
%<latexrelease>    \exp_args:Nc \__hook_patch_cmd_or_delay:Nnn {#2} {#2} {#3}
%<latexrelease>  }
%<latexrelease>\EndIncludeInRelease
%    \end{macrocode}
%
%    \begin{macrocode}
%<latexrelease>\IncludeInRelease{2021/06/01}{\@@_try_put_cmd_hook:n}%
%<latexrelease>                 {Standardise~generic~hook~names}
%<latexrelease>\cs_new_protected:Npn \@@_try_put_cmd_hook:n #1
%<latexrelease>  { \@@_try_put_cmd_hook:w #1 / / / \s_@@_mark {#1} }
%<latexrelease>\cs_new_protected:Npn \@@_try_put_cmd_hook:w
%<latexrelease>    #1 / #2 / #3 / #4 \s_@@_mark #5
%<latexrelease>  {
%<latexrelease>    \@@_debug:n { \iow_term:n { ->~Adding~cmd~hook~to~'#2'~(#3): } }
%<latexrelease>    \str_case:nnTF {#3}
%<latexrelease>        { { before } { } { after } { } }
%<latexrelease>      { \exp_args:Nc \@@_patch_cmd_or_delay:Nnn {#2} {#2} {#3} }
%<latexrelease>      { \msg_error:nnnn { hooks } { wrong-cmd-hook } {#2} {#3} }
%<latexrelease>  }
%<latexrelease>\EndIncludeInRelease
%    \end{macrocode}
% \end{macro}
%
% \begin{macro}{\@@_patch_cmd_or_delay:Nnn}
% \begin{macro}{\@@_cmd_begindocument_code:}
%   In the preamble, \cs{@@_patch_cmd_or_delay:Nnn} just adds the patch
%   instruction to a property list to be executed later.
%    \begin{macrocode}
\cs_new_protected:Npn \@@_patch_cmd_or_delay:Nnn #1 #2 #3
  {
    \@@_debug:n { \iow_term:n { ->~Add~generic~cmd~hook~for~#2~(#3). } }
    \@@_debug:n
      { \iow_term:n { !~In~the~preamble:~delaying. } }
    \prop_gput:Nnn \g_@@_delayed_patches_prop { #2 / #3 }
      { \@@_cmd_try_patch:nn {#2} {#3} }
  }
%    \end{macrocode}
%
%   The delayed patches are added to a property list to prevent
%   duplication, and the code stored in the property list for each
%   key is executed.  The function \cs{@@_patch_cmd_or_delay:Nnn} is
%   also redefined to be \cs{@@_patch_command:Nnn} so that no further
%   delaying is attempted.
%    \begin{macrocode}
\cs_new_protected:Npn \@@_cmd_begindocument_code:
  {
    \cs_gset_eq:NN \@@_patch_cmd_or_delay:Nnn \@@_patch_command:Nnn
    \prop_map_function:NN \g_@@_delayed_patches_prop { \use_ii:nn }
    \prop_gclear:N \g_@@_delayed_patches_prop
    \cs_undefine:N \@@_cmd_begindocument_code:
  }
\g@addto@macro \@kernel@after@begindocument
  { \@@_cmd_begindocument_code: }
%    \end{macrocode}
% \end{macro}
% \end{macro}
%
% \begin{macro}{\@@_cmd_try_patch:nn}
%   At |\begin{document}| tries patching the command if the hook
%   was not manually created in the meantime.  If the document does not
%   exist, no error is raised here as it may hook into a package that
%   wasn't loaded.  Hooks added to commands in the document body still
%   raise an error if the command is not defined.
%    \begin{macrocode}
\cs_new_protected:Npn \@@_cmd_try_patch:nn #1 #2
  {
    \@@_debug:n
      { \iow_term:x { ->~\string\begin{document}~try~cmd / #1 / #2. } }
    \@@_if_declared:nTF { cmd / #1 / #2 }
      {
        \@@_debug:n
          { \iow_term:n { .->~Giving~up:~hook~already~created. } }
      }
      {
        \cs_if_exist:cT {#1}
          { \exp_args:Nc \@@_patch_command:Nnn {#1} {#1} {#2} }
      }
  }
%    \end{macrocode}
% \end{macro}
%
%
%
%
%
% \subsection{Patching commands}
%
% \begin{macro}{\@@_patch_command:Nnn}
% \begin{macro}{\@@_patch_check:NNnn}
% \begin{macro}[TF]{\@@_if_public_command:N}
% \begin{macro}{\@@_if_public_command:w}
%   \cs{@@_patch_command:Nnn} will do some sanity checks on the
%   argument to detect if it is possible to add hooks to the command,
%   and raises an error otherwise.  If the command can contain hooks,
%   then it uses \tn{robust@command@act} to find out what type is the
%   command, and patch it accordingly.
%    \begin{macrocode}
\cs_new_protected:Npn \@@_patch_command:Nnn #1 #2 #3
  {
    \@@_patch_debug:x { analyzing~'\token_to_str:N #1' }
    \@@_patch_debug:x { \token_to_str:N #1 = \token_to_meaning:N #1 }
    \@@_patch_check:NNnn \cs_if_exist:NTF #1 { undef }
      {
        \@@_patch_debug:x { ++~control~sequence~is~defined }
        \@@_patch_check:NNnn \token_if_macro:NTF #1 { macro }
          {
            \@@_patch_debug:x { ++~control~sequence~is~a~macro }
            \@@_patch_check:NNnn \@@_if_public_command:NTF #1 { expl3 }
              {
                \@@_patch_debug:x { ++~macro~is~not~private }
                \robust@command@act
                  \g_hook_patch_action_list_tl #1
                  \@@_retokenize_patch:Nnn { #1 {#2} {#3} }
              }
          }
      }
  }
%    \end{macrocode}
%
%   And here's the auxiliary used above:
%    \begin{macrocode}
\cs_new_protected:Npn \@@_patch_check:NNnn #1 #2 #3 #4
  {
    #1 #2 {#4}
      {
        \msg_error:nnxx { hooks } { cant-patch }
          { \token_to_str:N #2 } {#3}
      }
  }
%    \end{macrocode}
%   and a conditional \cs{@@_if_public_command:NTF} to check if a command
%   has |__| in its name (no other checking is performed).  Primitives
%   with |:D| in their name could be included here, but they are already
%   discarded in the \cs{token_if_macro:NTF} test above.
%    \begin{macrocode}
\use:x
  {
    \prg_new_protected_conditional:Npnn
        \exp_not:N \@@_if_public_command:N ##1 { TF }
      {
        \exp_not:N \exp_last_unbraced:Nf
          \exp_not:N \@@_if_public_command:w
            { \exp_not:N \cs_to_str:N ##1 }
          \tl_to_str:n { _ _ } \s_@@_mark
      }
  }
\exp_last_unbraced:NNNNo
\cs_new_protected:Npn \@@_if_public_command:w
    #1 \tl_to_str:n { _ _ } #2 \s_@@_mark
  {
    \tl_if_empty:nTF {#2}
      { \prg_return_true: }
      { \prg_return_false: }
  }
%    \end{macrocode}
% \end{macro}
% \end{macro}
% \end{macro}
% \end{macro}
%
%
%
%
%
%
%
% \subsubsection{Patching by expansion and redefinition}
%
% \begin{macro}[int]{\g_hook_patch_action_list_tl}
%    This is the list of known command types and the function that
%    patches the command hooks into them.  The conditionals are taken
%    from \tn{ShowCommand}, \tn{NewCommandCopy} and
%    \cs{__kernel_cmd_if_xparse:NTF} defined in \texttt{ltcmd}.
%    \begin{macrocode}
\tl_gset:Nn \g_hook_patch_action_list_tl
  {
    { \@if@DeclareRobustCommand \@@_patch_DeclareRobustCommand:Nnn }
    { \@if@newcommand \@@_patch_newcommand:Nnn }
    { \__kernel_cmd_if_xparse:NTF \@@_cmd_patch_xparse:Nnn }
  }
%    \end{macrocode}
% \end{macro}
%
%
%
%
% \begin{macro}{\@@_patch_DeclareRobustCommand:Nnn}
%   At this point we know that the commands can be patched by expanding
%   then redefining.  These are the cases of commands defined with
%   \tn{newcommand} with an optional argument or with
%   \tn{DeclareRobustCommand}.
%
%   With \cs{@@_patch_DeclareRobustCommand:Nnn} we check if the command
%   has an optional argument (with a test counter-intuitively called
%   \tn{@if@newcommand}; also make sure the command doesn't take args by
%   calling \cs{robust@command@chk@safe}).  If so, we pass the patching action
%   to \cs{@@_patch_newcommand:Nnn}, otherwise we call the patching engine
%   \cs{@@_patch_expand_redefine:NNnn} with a \cs{c_false_bool} to
%   indicate that there is no optional argument.
%
%   \changes{v1.0c}{2021/07/20}
%           {Use \cs{robust@command@chk@safe} before \cs{@if@newcommand}.}
%    \begin{macrocode}
\cs_new_protected:Npn \@@_patch_DeclareRobustCommand:Nnn #1
  {
    \exp_args:Nc \@@_patch_DeclareRobustCommand_aux:Nnn
      { \cs_to_str:N #1 ~ }
  }
\cs_new_protected:Npn \@@_patch_DeclareRobustCommand_aux:Nnn #1
  {
    \robust@command@chk@safe #1
      { \@if@newcommand #1 }
      { \use_ii:nn }
        { \@@_patch_newcommand:Nnn }
        { \@@_patch_expand_redefine:NNnn \c_false_bool }
          #1
  }
%    \end{macrocode}
% \end{macro}
%
%
%
% \begin{macro}{\@@_patch_newcommand:Nnn}
%   If the command was defined with \tn{newcommand} and an optional
%   argument, call the patching engine with a \cs{c_true_bool} to flag
%   the presence of an optional argument, and with
%   \cs[no-index]{\string\command} to patch the actual code for
%   \cs[no-index]{command}.
%    \begin{macrocode}
\cs_new_protected:Npn \@@_patch_newcommand:Nnn #1
  {
    \exp_args:NNc \@@_patch_expand_redefine:NNnn \c_true_bool
      { \c_backslash_str \cs_to_str:N #1 }
  }
%    \end{macrocode}
% \end{macro}
%
%  \begin{macro}{\@@_cmd_patch_xparse:Nnn}
%    And for commands defined by the \pkg{xparse} commands use this
%    for patching:
%    \begin{macrocode}
\cs_new_protected:Npn \@@_cmd_patch_xparse:Nnn #1
  {
    \exp_args:NNc \@@_patch_expand_redefine:NNnn \c_false_bool
      { \cs_to_str:N #1 ~ code }
  }
%    \end{macrocode}
%  \end{macro}
%
%
%
%
%
% \begin{macro}{\@@_patch_expand_redefine:NNnn}
% \begin{macro}{\@@_redefine_with_hooks:Nnnn}
% \begin{macro}[EXP]{\@@_make_prefixes:w}
%   Now the real action begins.  Here we have in |#1| a boolean
%   indicating if the command has a leading |[|\ldots|]|-delimited
%   argument, in |#2| the command control sequence, in |#3| the name of
%   the command (note that |#1|${}\ne{}$|\csname#2\endcsname| at this
%   point!), and in |#4| the hook position, either |before| or |after|.
%
% \changes{v1.0f}{2021/10/20}
%         {Correct patching by expansion+redefinition when the macro
%          contains a parameter token (gh/697)}
%   Patching with expansion+redefinition is trickier than it looks like
%   at first glance.  Suppose the simple definition:
% \begin{verbatim}
%   \def\foo#1{#1##2}
% \end{verbatim}
%   When defined, its \meta{replacement text} will be a token list
%   containing:
%   \begin{quote}
%     \itshape
%     out\_param |1|, mac\_param |#|, character |2|
%   \end{quote}
%
%   Then, after expanding \cs{foo}|{##1}| (here |##| denotes a single
%   |#|$_6$) we end up with a token list with \textit{out\_param}~|1|
%   replaced:
%   \begin{quote}
%     \itshape
%     mac\_param |#|, character |1|, mac\_param |#|, character |2|
%   \end{quote}
%   that is, the definition would be:
% \begin{verbatim}
%   \def\foo#1{#1#2}
% \end{verbatim}
%   which obviously fails, because the original input in the definition
%   was |##| but \TeX{} reduced that to a single parameter token |#|$_6$
%   when carrying out the definition.  That leaves no room for a clever
%   solution with (say) \cs{unexpanded}, because anything that would
%   double the second |#|$_6$, would also (incorrectly) double the
%   first, so there's not much to do other than a manual solution.
%
%   There are three cases we can distinguish to make things hopefully
%   faster on simpler cases:
%   \begin{enumerate}
%     \item a macro with no parameters;
%     \item a macro with no parameter tokens in its definition;
%     \item a macro with parameters \emph{and} parameter tokens.
%   \end{enumerate}
%
%   The first case is trivial:  if the macro has no parameters, we can
%   just use \cs{unexpanded} around it, and if there is a parameter
%   token in it, it is handled correctly (the macro can be treated as a
%   |tl| variable).
%
%   The second case requires looking at the \meta{replacement text} of
%   the macro to see if it has a parameter token in there.  If it does
%   not, then there is no worry, and the macro can be redefined normally
%   (without \cs{unexpanded}).
%
%   The third case, as usual, is the devious one.  Here we'll have to
%   loop through the definition token by token, and double every
%   parameter token, so that this case can be handled like the previous
%   one.
%    \begin{macrocode}
%<latexrelease>\IncludeInRelease{2023/06/01}{\@@_patch_expand_redefine:NNnn}
%<latexrelease>                 {cmd~hooks~with~args}
\cs_new_protected:Npn \@@_patch_expand_redefine:NNnn #1 #2 #3 #4
  {
    \@@_patch_debug:x { ++~command~can~be~patched~without~rescanning }
%    \end{macrocode}
%   We'll start by counting the number of arguments in the command by
%   counting the number of characters in the \cs{cs_parameter_spec:N} of
%   the macro, divided by two, and subtracting one if the command has an
%   optional argument (that is, an extra |[]| in its
%   \meta{parameter text}).
%    \begin{macrocode}
    \int_set:Nn \l_@@_patch_num_args_int
      {
        \exp_args:Nf \str_count:n { \__kernel_cs_parameter_spec:N #2 } / 2
        \bool_if:NT #1 { -1 }
      }
%    \end{macrocode}
%   Now build two token lists:
%   \begin{description}
%     \item[\cs{l_@@_param_text_tl}] will contain the
%       \meta{parameter text} to be used when redefining the macro.  It
%       should be identical to the \meta{parameter text} used when
%       originally defining that macro.
%     \item[\cs{l_@@_replace_text_tl}] will contain braced pairs of
%       \cs{c_@@_hashes_tl}\meta{num} to feed to the macro when expanded.
%       This token list as well as the previous will have the first item
%       surrounded by |[|\ldots|]| in the case of an optional argument.
%   \end{description}
%
%   The use of \cs{c_@@_hashes_tl} here is to differentiate actual
%   parameters in the macro from parameter tokens in the original
%   definition of the macro.  Later on, \cs{c_@@_hashes_tl} is either
%   replaced by actual parameter tokens, or expanded into them.
%    \begin{macrocode}
    \int_compare:nNnTF { \l_@@_patch_num_args_int } > { \c_zero_int }
      {
%    \end{macrocode}
%   We'll first check if the command has any parameter token in its
%   definition (feeding it empty arguments), and set \cs{@@_exp_not:n}
%   accordingly.  \cs{@@_exp_not:n} will be used later to either leave
%   \cs{c_@@_hashes_tl} or expand it, and also to remember the result of
%   \cs{@@_if_has_hash:nTF} to avoid testing twice (the test can be
%   rather slow).
%    \begin{macrocode}
        \tl_set:Nx \l_@@_tmpa_tl { \bool_if:NTF #1 { [ ] } { { } } }
        \int_step_inline:nnn { 2 } { \l_@@_patch_num_args_int }
          { \tl_put_right:Nn \l_@@_tmpa_tl { { } } }
        \exp_args:NNo \exp_args:No \@@_if_has_hash:nTF
            { \exp_after:wN #2 \l_@@_tmpa_tl }
          { \cs_set_eq:NN \@@_exp_not:n \exp_not:n }
          { \cs_set_eq:NN \@@_exp_not:n \use:n }
        \cs_set_protected:Npn \@@_tmp:w ##1 ##2
          {
            ##1 \l_@@_param_text_tl   { \use:n ##2 }
            ##1 \l_@@_replace_text_tl { \@@_exp_not:n {##2} }
          }
%    \end{macrocode}
%   Here we'll conditionally add |[|\ldots|]| around the first
%   parameter:
%   \changes{v1.0g}{2023/04/06}
%           {Rename to \cs{c__hook_hashes_tl} (hook-args)}
%    \begin{macrocode}
        \bool_if:NTF #1
          { \@@_tmp:w \tl_set:Nx { [ \c_@@_hashes_tl 1 ] } }
          { \@@_tmp:w \tl_set:Nx { { \c_@@_hashes_tl 1 } } }
%    \end{macrocode}
%   Then, for every parameter from the second, just add it normally:
%    \begin{macrocode}
        \int_step_inline:nnn { 2 } { \l_@@_patch_num_args_int }
          { \@@_tmp:w \tl_put_right:Nx { { \c_@@_hashes_tl ##1 } } }
%    \end{macrocode}
%   Now, if the command has any parameter token in its definition
%   (then \cs{@@_exp_not:n} is \cs{exp_not:n}), call
%   \cs{@@_double_hashes:n} to double them, and replace every
%   \cs{c_@@_hashes_tl} by |#|:
%    \begin{macrocode}
        \tl_set:Nx \l_@@_replace_text_tl
          { \exp_not:N #2 \exp_not:V \l_@@_replace_text_tl }
        \tl_set:Nx \l_@@_replace_text_tl
          {
            \token_if_eq_meaning:NNTF \@@_exp_not:n \exp_not:n
              { \exp_args:NNV \exp_args:No \@@_double_hashes:n }
              { \exp_args:NV \exp_not:o }
                  \l_@@_replace_text_tl
          }
%    \end{macrocode}
%   And now, set a few auxiliaries for the case that the macro has
%   parameters, so it won't be passed through \cs{unexpanded} (twice):
%    \begin{macrocode}
        \cs_set_eq:NN \@@_def_cmd:w \tex_gdef:D
        \cs_set_eq:NN \@@_exp_not:NN \prg_do_nothing:
      }
      {
%    \end{macrocode}
%   In the case the macro has no parameters, we'll treat it as a token
%   list and things are much simpler (expansion control looks a bit
%   complicated, but it's just a pair of \cs{exp_not:N} preventing
%   another \cs{exp_not:n} from expanding):
%    \begin{macrocode}
        \tl_clear:N \l_@@_param_text_tl
        \tl_set_eq:NN \l_@@_replace_text_tl #2
        \cs_set_eq:NN \@@_def_cmd:w \tex_xdef:D
        \cs_set:Npn \@@_exp_not:NN ##1 { \exp_not:N ##1 \exp_not:N }
      }
%    \end{macrocode}
%   Before redefining, we need to also get the prefixes used when
%   defining the command.  Here we ensure that the \tn{escapechar} is
%   printable, otherwise a macro defined with prefixes
%   |\protected \long| will have it \tn{meaning} printed as
%   |protectedlong|, making life unnecessarily complicated.  Here the
%   \tn{escapechar} is changed to |/|, then we loop between pairs of
%   |/|\ldots|/| extracting the prefixes.
%    \begin{macrocode}
    \group_begin:
      \int_set:Nn \tex_escapechar:D { `\/ }
      \use:x
        {
    \group_end:
    \tl_set:Nx \exp_not:N \l_@@_patch_prefixes_tl
      { \exp_not:N \@@_make_prefixes:w \cs_prefix_spec:N #2 / / }
        }
%    \end{macrocode}
%   Here we redefine the hook to have the right number of arguments.
%   Disabling the hook, undefining the \verb|parameter| token list then
%   calling \cs{@@_make_usable:nn} are enough to redefine the hook to
%   the extent we want.  Code stored in the hook and other metadata
%   about it are not lost in the process.
% \changes{v1.0h}{2023/05/21}
%         {Changes to allow support arguments in cmd hooks (cmd-args)}
%    \begin{macrocode}
    \@@_disable:n { cmd / #3 / #4 }
    \cs_undefine:c { c_@@_cmd / #3 / #4_parameter_tl }
    \@@_make_usable:nn { cmd / #3 / #4 } { \l_@@_patch_num_args_int }
%    \end{macrocode}
%   Now call \cs{@@_redefine_with_hooks:Nnnn} with the macro being
%   redefined in |#1|, then \cs{UseHook}|{cmd/<name>/before}| in |#2| or
%   \cs{UseHook}|{cmd/<name>/after}| in |#3| (one is always empty), and
%   in |#4| the \meta{replacement text} of the macro.
%    \begin{macrocode}
    \use:e
      {
        \@@_redefine_with_hooks:Nnnn \exp_not:N #2
        \str_if_eq:nnTF {#4} { after }
          { \use_ii_i:nn }
          { \use:nn }
            { {
                \@@_exp_not:NN \exp_not:N \UseHookWithArguments
                  { cmd / #3 / #4 } { \int_use:N \l_@@_patch_num_args_int }
                  \@@_braced_parameter:n { cmd / #3 / #4 }
            } }
            { { } }
            { \@@_exp_not:NN \exp_not:V \l_@@_replace_text_tl }
      }
%    \end{macrocode}
%   Finally, update the hook code.
%    \begin{macrocode}
    \@@_update_hook_code:n { cmd / #3 / #4 }
  }
%<latexrelease>\EndIncludeInRelease
%<latexrelease>\IncludeInRelease{2021/06/01}{\@@_patch_expand_redefine:NNnn}
%<latexrelease>                 {cmd~hooks~with~args}
%<latexrelease>\cs_gset_protected:Npn \@@_patch_expand_redefine:NNnn #1 #2 #3 #4
%<latexrelease>  {
%<latexrelease>    \@@_patch_debug:x { ++~command~can~be~patched~without~rescanning }
%<latexrelease>    \int_set:Nn \l_@@_patch_num_args_int
%<latexrelease>      {
%<latexrelease>        \exp_args:Nf \str_count:n { \__kernel_cs_parameter_spec:N #2 } / 2
%<latexrelease>        \bool_if:NT #1 { -1 }
%<latexrelease>      }
%<latexrelease>    \int_compare:nNnTF { \l_@@_patch_num_args_int } > { \c_zero_int }
%<latexrelease>      {
%<latexrelease>        \tl_set:Nx \l_@@_tmpa_tl { \bool_if:NTF #1 { [ ] } { { } } }
%<latexrelease>        \int_step_inline:nnn { 2 } { \l_@@_patch_num_args_int }
%<latexrelease>          { \tl_put_right:Nn \l_@@_tmpa_tl { { } } }
%<latexrelease>        \exp_args:NNo \exp_args:No \@@_if_has_hash:nTF
%<latexrelease>            { \exp_after:wN #2 \l_@@_tmpa_tl }
%<latexrelease>          { \cs_set_eq:NN \@@_exp_not:n \exp_not:n }
%<latexrelease>          { \cs_set_eq:NN \@@_exp_not:n \use:n }
%<latexrelease>        \cs_set_protected:Npn \@@_tmp:w ##1 ##2
%<latexrelease>          {
%<latexrelease>            ##1 \l_@@_param_text_tl   { \use:n ##2 }
%<latexrelease>            ##1 \l_@@_replace_text_tl { \@@_exp_not:n {##2} }
%<latexrelease>          }
%<latexrelease>        \bool_if:NTF #1
%<latexrelease>          { \@@_tmp:w \tl_set:Nx { [ \c_@@_hash_tl 1 ] } }
%<latexrelease>          { \@@_tmp:w \tl_set:Nx { { \c_@@_hash_tl 1 } } }
%<latexrelease>        \int_step_inline:nnn { 2 } { \l_@@_patch_num_args_int }
%<latexrelease>          { \@@_tmp:w \tl_put_right:Nx { { \c_@@_hash_tl ##1 } } }
%<latexrelease>        \tl_set:Nx \l_@@_replace_text_tl
%<latexrelease>          { \exp_not:N #2 \exp_not:V \l_@@_replace_text_tl }
%<latexrelease>        \tl_set:Nx \l_@@_replace_text_tl
%<latexrelease>          {
%<latexrelease>            \token_if_eq_meaning:NNTF \@@_exp_not:n \exp_not:n
%<latexrelease>              { \exp_args:NNV \exp_args:No \@@_double_hashes:n }
%<latexrelease>              { \exp_args:NV \exp_not:o }
%<latexrelease>                  \l_@@_replace_text_tl
%<latexrelease>          }
%<latexrelease>        \cs_set_eq:NN \@@_def_cmd:w \tex_gdef:D
%<latexrelease>        \cs_set_eq:NN \@@_exp_not:NN \prg_do_nothing:
%<latexrelease>      }
%<latexrelease>      {
%<latexrelease>        \tl_clear:N \l_@@_param_text_tl
%<latexrelease>        \tl_set_eq:NN \l_@@_replace_text_tl #2
%<latexrelease>        \cs_set_eq:NN \@@_def_cmd:w \tex_xdef:D
%<latexrelease>        \cs_set:Npn \@@_exp_not:NN ##1 { \exp_not:N ##1 \exp_not:N }
%<latexrelease>      }
%<latexrelease>    \group_begin:
%<latexrelease>      \int_set:Nn \tex_escapechar:D { `\/ }
%<latexrelease>      \use:x
%<latexrelease>        {
%<latexrelease>    \group_end:
%<latexrelease>    \tl_set:Nx \exp_not:N \l_@@_patch_prefixes_tl
%<latexrelease>      { \exp_not:N \@@_make_prefixes:w \cs_prefix_spec:N #2 / / }
%<latexrelease>        }
%<latexrelease>    \use:x
%<latexrelease>      {
%<latexrelease>        \@@_redefine_with_hooks:Nnnn \exp_not:N #2
%<latexrelease>        \str_if_eq:nnTF {#4} { after }
%<latexrelease>          { \use_ii_i:nn }
%<latexrelease>          { \use:nn }
%<latexrelease>            { { \@@_exp_not:NN \exp_not:N \UseHook { cmd / #3 / #4 } } }
%<latexrelease>            { { } }
%<latexrelease>            { \@@_exp_not:NN \exp_not:V \l_@@_replace_text_tl }
%<latexrelease>      }
%<latexrelease>  }
%<latexrelease>\EndIncludeInRelease
%    \end{macrocode}
%
%   Now that all the needed tools are ready, without further ado we'll
%   redefine the command.  The definition uses the prefixes gathered in
%   \cs{l_@@_patch_prefixes_tl}, a primitive \cs{@@_def_cmd:w} (which is
%   \cs{tex_gdef:D} or \cs{tex_xdef:D}) to avoid adding extra prefixes,
%   and the \meta{parameter text} from \cs{l_@@_param_text_tl}.
%
%   Then finally, in the body of the definition, we insert |#2|, which
%   is \hook{cmd/\#1/before} or empty, |#4| which is the
%   \meta{replacement text}, and |#3| which is \hook{cmd/\#1/after} or
%   empty.
%
% \changes{v1.0e}{2021/09/28}
%                {Make patching of commands a global operation (gh/674)}
%    \begin{macrocode}
\cs_new_protected:Npn \@@_redefine_with_hooks:Nnnn #1 #2 #3 #4
  {
    \l_@@_patch_prefixes_tl
      \exp_after:wN \@@_def_cmd:w
        \exp_after:wN #1 \l_@@_param_text_tl
      { #2  #4  #3 }
  }
%    \end{macrocode}
%
%   Here's the auxiliary that makes the prefix control sequences for the
%   redefinition.  Each item has to be \cs{tl_trim_spaces:n}'d because
%   the last item (and not any other) has a trailing space.
%    \begin{macrocode}
\cs_new:Npn \@@_make_prefixes:w / #1 /
  {
    \tl_if_empty:nF {#1}
      {
        \exp_not:c { tex_ \tl_trim_spaces:n {#1} :D }
        \@@_make_prefixes:w /
      }
  }
%    \end{macrocode}
% \end{macro}
% \end{macro}
% \end{macro}
%
%
% Here are some auxiliaries for the contraption above.
%
% \begin{macro}[pTF]{\@@_if_has_hash:n}
% \begin{macro}{\@@_if_has_hash:w,\@@_if_has_hash_check:w}
%   \cs{@@_if_has_hash:nTF} searches the token list |#1| for a catcode~6
%   token, and if any is found, it returns |true|, and |false|
%   otherwise.  The searching doesn't care about preserving groups or
%   spaces:  we can ignore those safely (braces are removed) so that
%   searching is as fast as possible.
%    \begin{macrocode}
\prg_new_conditional:Npnn \@@_if_has_hash:n #1 { TF }
  { \@@_if_has_hash:w #1 ## \s_@@_mark }
\cs_new:Npn \@@_if_has_hash:w #1
  {
    \tl_if_single_token:nTF {#1}
      {
        \token_if_eq_catcode:NNTF ## #1
          { \@@_if_has_hash_check:w }
          { \@@_if_has_hash:w }
      }
      { \@@_if_has_hash:w #1 }
  }
\cs_new:Npn \@@_if_has_hash_check:w #1 \s_@@_mark
  { \tl_if_empty:nTF {#1} { \prg_return_false: } { \prg_return_true: } }
%    \end{macrocode}
% \end{macro}
% \end{macro}
%
%
% \begin{macro}[rEXP]{\@@_double_hashes:n}
% \begin{macro}[rEXP]{
%     \@@_double_hashes:w,
%     \@@_double_hashes_output:N,
%     \@@_double_hashes_output_aux:N,
%     \@@_double_hashes_stop:w,
%     \@@_double_hashes_group:n,
%     \@@_double_hashes_space:w,
%   }
%   \cs{@@_double_hashes:n} loops through the token list |#1| and
%   duplicates any catcode~6 token, and expands tokens \cs{ifx}-equal to
%   \cs{c_@@_hashes_tl}, and leaves all other tokens \cs{notexpanded} with
%   \cs{exp_not:N}.  Unfortunately pairs of explicit catcode~1 and
%   catcode~2 character tokens are normalised to |{|$_1$ and |}|$_1$
%   because it's not feasible to expandably detect the character code
%   (\emph{maybe} it could be done using something along the lines of
%   \url{https://tex.stackexchange.com/a/527538}, but it's far too much
%   work for close to zero benefit).
%
%   \cs{@@_double_hashes:w} is the tail-recursive loop macro, that tests
%   which of the three types of item is in the head of the token list.
%    \begin{macrocode}
\cs_new:Npn \@@_double_hashes:n #1
  { \@@_double_hashes:w #1 \q_@@_recursion_tail \q_@@_recursion_stop }
\cs_new:Npn \@@_double_hashes:w #1 \q_@@_recursion_stop
  {
    \tl_if_head_is_N_type:nTF {#1}
      { \@@_double_hashes_output:N }
      {
        \tl_if_head_is_group:nTF {#1}
          { \@@_double_hashes_group:n }
          { \@@_double_hashes_space:w }
      }
    #1 \q_@@_recursion_stop
  }
%    \end{macrocode}
%
%   \cs{@@_double_hashes_output:N} checks for the end of the token list,
%   then checks if the token is \cs{c_@@_hashes_tl}, and if so just leaves
%   it.
%   \changes{v1.0g}{2023/04/06}
%           {Add case for \cs{c__hook_hashes_tl} (hook-args)}
%    \begin{macrocode}
\cs_new:Npn \@@_double_hashes_output:N #1
  {
    \if_meaning:w \q_@@_recursion_tail #1
      \@@_double_hashes_stop:w
    \fi:
    \if:w ?
      \if_meaning:w \c_@@_hash_tl #1 ! \fi:
      \if_meaning:w \c_@@_hashes_tl #1 ! \fi:
          ?
    \else:
%    \end{macrocode}
%   (this \cs{use_i:nnnn} uses \cs{fi:} and consumes \cs{use:n}, the
%   whole \cs{if_catcode:w} block, and the \cs{exp_not:N}, leaving just
%   |#1| which is \cs{c_@@_hashes_tl}.)
%    \begin{macrocode}
      \use_i:nnnn
    \fi:
    \use:n
      {
%    \end{macrocode}
%   If |#1| is not \cs{c_@@_hashes_tl}, then check if its catcode is~6,
%   and if so, leave it doubled in \cs{exp_not:n} and consume the
%   following |\exp_not:N #1|.
%    \begin{macrocode}
        \@@_double_hashes_output_aux:N #1
        \use_none:n
          { \exp_not:n { #1 #1 } }
      }
%    \end{macrocode}
%   If both previous tests returned |false|, then leave the token
%   unexpanded and resume the loop.
%    \begin{macrocode}
    \exp_not:N #1
    \@@_double_hashes:w
  }
%    \end{macrocode}
% \changes{v1.0l}{2025/06/27}{Support \cs{alignmark}}
%    \begin{macrocode}
\cs_new:Npn \@@_double_hashes_output_aux:N #1
  {
    \if_catcode:w ## \exp_not:N #1
      \exp_after:wN \use_ii:nnnn
    \fi:
  }
\sys_if_engine_luatex:T
  {
    \cs_gset:Npn \@@_double_hashes_output_aux:N #1
      {
        \if_catcode:w ## \exp_not:N #1
          \exp_after:wN \use_ii:nnnn
        \else:
          \if_meaning:w \tex_alignmark:D \tex_alignmark:D #1
            \exp_after:wN \exp_after:wN \exp_after:wN \use_ii:nnnn
          \fi:
        \fi:
      }    
  }
\cs_new:Npn \@@_double_hashes_stop:w #1 \q_@@_recursion_stop { \fi: }
%    \end{macrocode}
%
%   Dealing with spaces and grouped tokens is trivial:
%    \begin{macrocode}
\cs_new:Npn \@@_double_hashes_group:n #1
  { { \@@_double_hashes:n {#1} } \@@_double_hashes:w }
\exp_last_unbraced:NNo
\cs_new:Npn \@@_double_hashes_space:w \c_space_tl
  { ~ \@@_double_hashes:w }
%    \end{macrocode}
% \end{macro}
% \end{macro}
%
%
% \subsubsection{Patching by retokenization}
%
% At this point we've drained the possibilities of patching a command by
% expansion-and-redefinition, so we have to resort to patching by
% retokenizing the command.  Patching by retokenization is done by
% getting the \tn{meaning} of the command, doing the necessary
% manipulations on the generated string, and the retokenizing that again
% by using \tn{scantokens}.
%
% Patching by retokenization is definitely a riskier business, because
% it relies that the tokens printed by \tn{meaning} produce the exact
% same tokens as the ones in the original definition.  That is, the
% catcode régime must be exactly(ish) the same, and there is no way of
% telling except by trial and error.
%
% \begin{macro}{\@@_retokenize_patch:Nnn}
%   This is the macro that will control the whole process.  First we'll
%   try out one final, rather trivial case, of a command with no
%   arguments;  that is, a token list.  This case can be patched with
%   the expand-and-redefine routine but it has to be the very last case
%   tested for, because most (all?) robust commands start with a
%   top-level macro with no arguments, so testing this first would
%   short-circuit \tn{robust@command@act} and the top-level macros would
%   be incorrectly patched.  In that case, we just check if the
%   \cs{cs_parameter_spec:N} is empty, and call
%   \cs{@@_patch_expand_redefine:NNnn}.
%    \begin{macrocode}
\cs_new_protected:Npn \@@_retokenize_patch:Nnn #1 #2 #3
  {
    \str_if_eq:eeTF { \__kernel_cs_parameter_spec:N #1 } { }
      { \@@_patch_expand_redefine:NNnn \c_false_bool #1 {#2} {#3} }
      {
        \@@_patch_debug:x { ..~command~can~only~be~patched~by~rescanning }
%    \end{macrocode}
%
%   Otherwise, we start the actual patching by retokenization job.  The
%   code calls \cs{@@_try_patch_with_catcodes:Nnnnw} with a different
%   catcode setting:
%   \begin{itemize}
%     \item The current catcode setting;
%     \item Switching the catcode of |@|;
%     \item Switching the \pkg{expl3} syntax on or off;
%     \item Both of the above.
%   \end{itemize}
%
%   If patching succeeds, \cs{@@_try_patch_with_catcodes:Nnnnw} has the
%   side-effect of patching the macro |#1| (which may be an internal
%   from the command whose name is~|#2|).
%    \begin{macrocode}
        \tl_set:Nx \l_@@_tmpa_tl
          {
            \int_compare:nNnTF { \char_value_catcode:n {`\@ } } = { 12 }
              { \exp_not:N \makeatletter } { \exp_not:N \makeatother }
          }
        \tl_set:Nx \l_@@_tmpb_tl
          {
            \bool_if:NTF \l__kernel_expl_bool
              { \ExplSyntaxOff } { \ExplSyntaxOn }
          }
        \use:x
          {
            \exp_not:N \@@_try_patch_with_catcodes:Nnnnw
                \exp_not:n { #1 {#2} {#3} }
              { \prg_do_nothing: }
              { \exp_not:V \l_@@_tmpa_tl } % @
              { \exp_not:V \l_@@_tmpb_tl } % _:
              {
                \exp_not:V \l_@@_tmpa_tl   % @
                \exp_not:V \l_@@_tmpb_tl   % _:
              }
          }
              \q_recursion_tail \q_recursion_stop
%    \end{macrocode}
%
%   If no catcode setting succeeds, give up and raise an error.  The
%   command isn't changed in any way in that case.
%    \begin{macrocode}
          {
            \msg_error:nnxx { hooks } { cant-patch }
              { \c_backslash_str #2 } { retok }
          }
      }
  }
%    \end{macrocode}
% \end{macro}
%
%
%
% \begin{macro}{\@@_try_patch_with_catcodes:Nnnnw}
%   This function is a simple wrapper around
%   \cs{@@_cmd_if_scanable:NnTF} and \cs{@@_patch_retokenize:Nnnn} if
%   the former returns \meta{true}, plus some debug messages.
%    \begin{macrocode}
%<latexrelease>\IncludeInRelease{2023/06/01}{\@@_try_patch_with_catcodes:Nnnnw}
%<latexrelease>                 {cmd~hooks~with~args}
\cs_new_protected:Npn \@@_try_patch_with_catcodes:Nnnnw #1 #2 #3 #4
  {
    \quark_if_recursion_tail_stop_do:nn {#4} { \use:n }
    \@@_patch_debug:x { ++~trying~to~patch~by~retokenization }
    \@@_cmd_if_scanable:NnTF {#1} {#4}
      {
        \@@_patch_debug:x { ++~macro~can~be~retokenized~cleanly }
        \@@_patch_debug:x { ==~retokenizing~macro~now }
        \@@_patch_retokenize:Nnnn #1 { cmd / #2 / #3 } {#3} {#4}
        \use_i_delimit_by_q_recursion_stop:nw \use_none:n
      }
      {
        \@@_patch_debug:x { --~macro~cannot~be~retokenized~cleanly }
        \@@_try_patch_with_catcodes:Nnnnw #1 {#2} {#3}
      }
  }
%<latexrelease>\EndIncludeInRelease
%<latexrelease>\IncludeInRelease{2021/06/01}{\@@_try_patch_with_catcodes:Nnnnw}
%<latexrelease>                 {cmd~hooks~with~args}
%<latexrelease>\cs_gset_protected:Npn \@@_try_patch_with_catcodes:Nnnnw #1 #2 #3 #4
%<latexrelease>  {
%<latexrelease>    \quark_if_recursion_tail_stop_do:nn {#4} { \use:n }
%<latexrelease>    \@@_patch_debug:x { ++~trying~to~patch~by~retokenization }
%<latexrelease>    \@@_cmd_if_scanable:NnTF {#1} {#4}
%<latexrelease>      {
%<latexrelease>        \@@_patch_debug:x { ++~macro~can~be~retokenized~cleanly }
%<latexrelease>        \@@_patch_debug:x { ==~retokenizing~macro~now }
%<latexrelease>        \@@_patch_retokenize:Nnnn #1 {#2} {#3} {#4}
%<latexrelease>        \use_i_delimit_by_q_recursion_stop:nw \use_none:n
%<latexrelease>      }
%<latexrelease>      {
%<latexrelease>        \@@_patch_debug:x { --~macro~cannot~be~retokenized~cleanly }
%<latexrelease>        \@@_try_patch_with_catcodes:Nnnnw #1 {#2} {#3}
%<latexrelease>      }
%<latexrelease>  }
%<latexrelease>\EndIncludeInRelease
%    \end{macrocode}
% \end{macro}
%
%
%
%
% \begin{macro}[int]{\kerneltmpDoNotUse}
%   This is an oddity required to be safe (as safe as reasonably
%   possible) when patching the command.  The entirety of
%   \begin{quote}
%     \meta{prefixes} \tn{def} \meta{cs} \meta{parameter text}
%       |{|\meta{replacement text}|}|
%   \end{quote}
%   will go through \tn{scantokens}.  The \meta{parameter text} and
%   \meta{replacement text} are what we are trying to retokenize, so not
%   much worry there.  The other items, however, should ``just work'',
%   so some care is needed to not use too fancy catcode settings.
%   Therefore we can't use an \pkg{expl3}-named macro for \meta{cs}, nor
%   the \pkg{expl3} versions of \tn{def} or the \meta{prefixes}.
%   That is why the definitions that will eventually go into
%   \tn{scantokens} will use the oddly (but hopefully clearly)-named
%   \cs{kerneltmpDoNotUse}:
%    \begin{macrocode}
\cs_new_eq:NN \kerneltmpDoNotUse !
%    \end{macrocode}
%   \phoinline{Maybe this can be avoided by running the \meta{parameter text}
%     and the \meta{replacement text} separately through \tn{scantokens}
%     and then putting everything together at the end.}
% \end{macro}
%
%
%
% \begin{macro}{\@@_patch_required_catcodes:}
%   Here are the catcode settings that are \emph{mandatory} when
%   retokenizing commands.  These are the minimum necessary settings to
%   perform the definitions:  they identify control sequences, which
%   must be escaped with |\|$_0$, delimit the definition with |{|$_1$
%   and |}|$_2$, and mark parameters with |#|$_6$.  Everything else may
%   be changed, but not these.
%    \begin{macrocode}
\cs_new_protected:Npn \@@_patch_required_catcodes:
  {
    \char_set_catcode_escape:N \\
    \char_set_catcode_group_begin:N \{
    \char_set_catcode_group_end:N \}
    \char_set_catcode_parameter:N \#
    % \int_set:Nn \tex_endlinechar:D { -1 }
    % \int_set:Nn \tex_newlinechar:D { -1 }
  }
%    \end{macrocode}
%   \phoinline{\pkg{etoolbox} sets the \tn{endlinechar} and \tn{newlinechar}
%     when patching, but as far as I tested these didn't make much of
%     a difference, so I left them out for now.  Maybe
%     \tn{newlinechar}|=-1| avoids a space token being added after the
%     definition.}
%   \phoinline{If the patching is split by \meta{parameter text} and
%     \meta{replacement text}, then only \# will have to stay in that
%     list.}
%   \phoinline{Actually now that we patch
%     \texttt{\cs{UseHook}\{cmd/foo/before\}}, all the tokens there need
%     to have the right catcodes, so this list now includes all
%     lowercase letters, U and H, the slash, and whatever characters in
%     the command name\ldots sigh\ldots}
% \end{macro}
%
%
%
%
% \begin{macro}[TF]{\@@_cmd_if_scanable:Nn}
%   Here we'll do a quick test if the command being patched can in fact
%   be retokenized with the specific catcode setting without changing
%   in meaning.  The test is straightforward:
%   \begin{enumerate}
%     \item apply \tn{meaning} to the command;
%     \item split the \meta{prefixes}, \meta{parameter text} and
%       \meta{replacement text} and arrange them as
%       \begin{quote}
%         \meta{prefixes}\tn{def}\cs{kerneltmpDoNotUse}%^^A
%           \meta{parameter text}|{|\meta{replacement text}|}|
%       \end{quote}
%     \item rescan that with the given catcode settings, and do
%       the definition; then finally
%     \item compare \cs{kerneltmpDoNotUse} with the original command.
%   \end{enumerate}
%   If both are \tn{ifx}-equal, the command can be safely patched.
%    \begin{macrocode}
\prg_new_protected_conditional:Npnn \@@_cmd_if_scanable:Nn #1 #2 { TF }
  {
    \cs_set_eq:NN \kerneltmpDoNotUse \scan_stop:
    \cs_set_eq:NN \@@_tmp:w \scan_stop:
    \use:x
      {
        \cs_set:Npn \@@_tmp:w
            ####1 \tl_to_str:n { macro: } ####2 -> ####3 \s_@@_mark
          { ####1 \def \kerneltmpDoNotUse ####2   {####3} }
        \tl_set:Nx \exp_not:N \l_@@_tmpa_tl
          { \exp_not:N \@@_tmp:w \token_to_meaning:N #1 \s_@@_mark }
      }
    \tl_rescan:nV { #2 \@@_patch_required_catcodes: } \l_@@_tmpa_tl
    \token_if_eq_meaning:NNTF #1 \kerneltmpDoNotUse
      { \prg_return_true: }
      { \prg_return_false: }
  }
%    \end{macrocode}
% \end{macro}
%
%
% \begin{macro}{\@@_guess_arg_count:NN}
% \begin{macro}{\@@_guess_arg_count:wN}
% \begin{macro}{\@@_guess_arg_count:nw}
%   Looks at the parameter text of a macro, and counts the parameters
%   by looking at the number after a \verb|#|, and checking if they are
%   sequential.  This macro assumes that all parameters are marked with
%   hashes, and not other characters, and that there is no
%   \enquote{trick parameter}.
% \changes{v1.0h}{2023/05/21}
%         {Macro added (cmd-args)}
%    \begin{macrocode}
%<latexrelease>\IncludeInRelease{2023/06/01}{\@@_guess_arg_count:NN}
%<latexrelease>                 {cmd~hooks~with~args}
\cs_new_protected:Npn \@@_guess_arg_count:NN #1
  {
    \exp_after:wN \@@_guess_arg_count:wN
      \token_to_meaning:N #1 \s_@@_mark
  }
\exp_last_unbraced:NNNNo
\cs_new_protected:Npx \@@_guess_arg_count:wN
    #1 { \tl_to_str:n { macro: } } #2 \s_@@_mark #3
  {
    \int_set:Nn #3
      {
        \exp_not:N \@@_guess_arg_count:nw { 0 } #2
          \c_hash_str 0 \s_@@_mark
      }
  }
\use:e
  { \cs_new:Npn \exp_not:N \@@_guess_arg_count:nw #1 #2 \c_hash_str #3 }
  {
    \int_compare:nNnTF { #1 + 1 } = {#3}
      { \@@_guess_arg_count:nw {#3} }
      { #1 \@@_use_none_delimit_by_s_mark:w }
  }
%<latexrelease>\EndIncludeInRelease
%<latexrelease>\IncludeInRelease{2021/06/01}{\@@_guess_arg_count:NN}
%<latexrelease>                 {cmd~hooks~with~args}
%<latexrelease>\cs_undefine:N \@@_guess_arg_count:NN
%<latexrelease>\EndIncludeInRelease
%    \end{macrocode}
% \end{macro}
% \end{macro}
% \end{macro}
%
%
% \begin{macro}{\@@_patch_retokenize:Nnnn}
%   Then, if \cs{@@_cmd_if_scanable:NnTF} returned true, we can go on
%   and patch the command.
%    \begin{macrocode}
%<latexrelease>\IncludeInRelease{2023/06/01}{\@@_patch_retokenize:Nnnn}
%<latexrelease>                 {cmd~hooks~with~args}
\cs_new_protected:Npn \@@_patch_retokenize:Nnnn #1 #2 #3 #4
  {
%    \end{macrocode}
%   Here, when patching by retokenization, we can only guess the number
%   of arguments of the macro.
% \changes{v1.0h}{2023/05/21}
%         {Changes to allow support arguments in cmd hooks (cmd-args)}
%    \begin{macrocode}
    \@@_guess_arg_count:NN #1 \l_@@_patch_num_args_int
%    \end{macrocode}
%
%   Then we redefine the hook to have the right number of arguments.
%   Disabling the hook, undefining the \verb|parameter| token list then
%   calling \cs{@@_make_usable:nn} are enough to redefine the hook to
%   the extent we want.  Code stored in the hook and other metadata
%   about it are not lost in the process.
%    \begin{macrocode}
    \@@_disable:n {#2}
    \cs_undefine:c { c_@@_#2_parameter_tl }
    \@@_make_usable:nn {#2} { \l_@@_patch_num_args_int }
    \tl_set:Ne \l_@@_tmpa_tl
      { \exp_args:Ne \tl_to_str:n { \@@_braced_parameter:n {#2} } }
    \use:x
      {
        \str_replace_all:Nnn \exp_not:N \l_@@_tmpa_tl
          { #### } { \c_hash_str }
      }
%    \end{macrocode}
%   Then, make some things \tn{relax} to avoid lots of
%   \tn{noexpand} below.
%    \begin{macrocode}
    \cs_set_eq:NN \kerneltmpDoNotUse \scan_stop:
    \cs_set_eq:NN \@@_tmp:w \scan_stop:
    \use:x
      {
%    \end{macrocode}
%   Now we'll define \cs{@@_tmp:w} such that it splits the \tn{meaning}
%   of the macro (|#1|) into its three parts:
%   \begin{enumerate}
%     \def\makelabel#1{\texttt{\#\#\#\##1}}
%     \item \meta{prefixes}
%     \item \meta{parameter text}
%     \item \meta{replacement text}
%   \end{enumerate}
%   and arrange that a complete definition, then place the |before|
%   or |after| hooks around the \meta{replacement text}:
%   accordingly.
%    \begin{macrocode}
        \cs_set:Npn \@@_tmp:w
            ####1 \tl_to_str:n { macro: } ####2 -> ####3 \s_@@_mark
          {
            ####1 \def \kerneltmpDoNotUse ####2
              {
                \str_if_eq:nnT {#3} { before }
                  {
                    \token_to_str:N \UseHookWithArguments {#2}
                      { \int_use:N \l_@@_patch_num_args_int }
                      \l_@@_tmpa_tl
                  }
                ####3
                \str_if_eq:nnT {#3} { after }
                  {
                    \token_to_str:N \UseHookWithArguments {#2}
                      { \int_use:N \l_@@_patch_num_args_int }
                      \l_@@_tmpa_tl
                  }
              }
          }
%    \end{macrocode}
%   Now we just have to get the \tn{meaning} of the command being
%   patched and pass it through the meat grinder above.
%    \begin{macrocode}
        \tl_set:Nx \exp_not:N \l_@@_tmpa_tl
          { \exp_not:N \@@_tmp:w \token_to_meaning:N #1 \s_@@_mark }
      }
%    \end{macrocode}
%   Now rescan with the given catcode settings (overridden by the
%   \cs{@@_patch_required_catcodes:}), and implicitly (by using the
%   rescanned token list) carry out the definition from above.
%    \begin{macrocode}
    \tl_rescan:nV { #4 \@@_patch_required_catcodes: } \l_@@_tmpa_tl
%    \end{macrocode}
%   And to close, copy the newly-defined command into the old name and
%   the patching is finally completed:
%
% \changes{v1.0e}{2021/09/28}
%                {Make patching of commands a global operation (gh/674)}
%    \begin{macrocode}
    \cs_gset_eq:NN #1 \kerneltmpDoNotUse
%    \end{macrocode}
%   Finally, update the hook code.
%    \begin{macrocode}
    \@@_update_hook_code:n {#2}
  }
%<latexrelease>\EndIncludeInRelease
%<latexrelease>\IncludeInRelease{2021/06/01}{\@@_patch_retokenize:Nnnn}
%<latexrelease>                 {cmd~hooks~with~args}
%<latexrelease>\cs_gset_protected:Npn \@@_patch_retokenize:Nnnn #1 #2 #3 #4
%<latexrelease>  {
%<latexrelease>    \cs_set_eq:NN \kerneltmpDoNotUse \scan_stop:
%<latexrelease>    \cs_set_eq:NN \@@_tmp:w \scan_stop:
%<latexrelease>    \use:x
%<latexrelease>      {
%<latexrelease>        \cs_set:Npn \@@_tmp:w
%<latexrelease>            ####1 \tl_to_str:n { macro: } ####2 -> ####3 \s_@@_mark
%<latexrelease>          {
%<latexrelease>            ####1 \def \kerneltmpDoNotUse ####2
%<latexrelease>              {
%<latexrelease>                \str_if_eq:nnT {#3} { before }
%<latexrelease>                  { \token_to_str:N \UseHook { cmd / #2 / #3 } }
%<latexrelease>                ####3
%<latexrelease>                \str_if_eq:nnT {#3} { after }
%<latexrelease>                  { \token_to_str:N \UseHook { cmd / #2 / #3 } }
%<latexrelease>              }
%<latexrelease>          }
%<latexrelease>        \tl_set:Nx \exp_not:N \l_@@_tmpa_tl
%<latexrelease>          { \exp_not:N \@@_tmp:w \token_to_meaning:N #1 \s_@@_mark }
%<latexrelease>      }
%<latexrelease>    \tl_rescan:nV { #4 \@@_patch_required_catcodes: } \l_@@_tmpa_tl
%<latexrelease>    \cs_gset_eq:NN #1 \kerneltmpDoNotUse
%<latexrelease>  }
%<latexrelease>\EndIncludeInRelease
%    \end{macrocode}
% \end{macro}
%
% \subsection{Messages}
%
%    \begin{macrocode}
%<latexrelease>\IncludeInRelease{2023/06/01}{wrong-cmd-hook}%
%<latexrelease>                 {Standardise~generic~hook~names}
%<latexrelease>\EndIncludeInRelease
%<latexrelease>\IncludeInRelease{2021/06/01}{wrong-cmd-hook}%
%<latexrelease>                 {Standardise~generic~hook~names}
%<latexrelease>\msg_new:nnnn { hooks } { wrong-cmd-hook }
%<latexrelease>  {
%<latexrelease>    Generic~hook~`cmd/#1/#2'~is~invalid.
%<latexrelease>%    The~hook~should~be~`cmd/#1/before'~or~`cmd/#1/after'.
%<latexrelease>  }
%<latexrelease>  {
%<latexrelease>    You~tried~to~add~a~generic~hook~to~command~\iow_char:N \\#1,~but~`#2'~
%<latexrelease>    is~an~invalid~component.~Only~`before'~or~`after'~are~allowed.
%<latexrelease>  }
%<latexrelease>\EndIncludeInRelease
\msg_new:nnnn { hooks } { cant-patch }
  {
    Generic~hooks~cannot~be~added~to~'#1'.
  }
  {
    You~tried~to~add~a~hook~to~'#1',~but~LaTeX~was~unable~to~
    patch~the~command~because~it~\@@_unpatchable_cases:n {#2}.
  }
\cs_new:Npn \@@_unpatchable_cases:n #1
  {
    \str_case:nn {#1}
      {
        { undef } { doesn't~exist }
        { macro } { is~not~a~macro }
        { expl3 } { is~a~private~expl3~macro }
        { retok } { can't~be~retokenized~cleanly }
      }
  }
%    \end{macrocode}
%
%
%    \begin{macrocode}
%<latexrelease>\IncludeInRelease{0000/00/00}{ltcmdhooks}%
%<latexrelease>                 {The~hook~management~system~for~commands}
%<latexrelease>
%    \end{macrocode}
%    The command \cs{@@_cmd_begindocument_code:} is used in an
%    internal hook, so we need to make sure it has a harmless
%    definition after rollback as that will not remove it from the
%    kernel hook.
%    \begin{macrocode}
%<latexrelease>\cs_set_eq:NN \@@_cmd_begindocument_code: \prg_do_nothing:
%<latexrelease>
%<latexrelease>\EndModuleRelease
\ExplSyntaxOff
%</2ekernel|latexrelease>
%    \end{macrocode}
%
%    \begin{macrocode}
%<@@=>
%    \end{macrocode}
%
% \Finale
%
%%%%%%%%%%%%%%%%%%%%%%%%%%%%%%%%%%%%%%%%%%%%%
\endinput

^^A  Needed for emacs
^^A
^^A  Local Variables: 
^^A  mode: latex
^^A  coding: utf-8-unix
^^A  End: 
