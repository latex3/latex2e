% \iffalse meta-comment
%
% File: ltproperties.dtx
%
% Copyright (C) 2023-2024 The LaTeX Project
%
% It may be distributed and/or modified under the conditions of the
% LaTeX Project Public License (LPPL), either version 1.3c of this
% license or (at your option) any later version.  The latest version
% of this license is in the file
%
%    http://www.latex-project.org/lppl.txt
%
% This file is part of the LaTeX base system. (The Work in LPPL)
% and all files in that bundle must be distributed together.
%
% -----------------------------------------------------------------------
%
% The development version of the bundle can be found at
%
%    https://github.com/latex3/latex2e
%
% for those people who are interested.
%
% \fi
%
% \iffalse
%%% From File: ltproperties.dtx
%
%<*driver> 
% \fi
\ProvidesFile{ltproperties.dtx}
             [2024/10/21 v1.0h LaTeX Kernel (Properties)]
% \iffalse
%
\documentclass[full]{l3doc}
\GetFileInfo{ltproperties.dtx}

\EnableCrossrefs
\CodelineIndex
\begin{document}
  \DocInput{ltproperties.dtx}
\end{document}
%</driver>
% \fi
%
% \title{^^A
%   Recording and cross-referencing document properties^^A
%  \thanks{This module has version
%    \fileversion\ dated \filedate, \copyright\ The \LaTeX\
%    Project.}
% }
%
% \author{^^A
%  The \LaTeX\ Project\thanks
%    {^^A
%      E-mail:
%        \href{mailto:latex-team@latex-project.org}
%          {latex-team@latex-project.org}^^A
%    }^^A
% }
%
% \maketitle
%
% \begin{documentation}
% \begin{abstract}
% This code implements command to record and (expandably) reference 
% document properties. It extends the standard \cs{label}/\cs{ref}/\cs{pageref}
% commands.
% \end{abstract}
%  
% \tableofcontents
%  
% \section{Introduction}
% 
% The module allows to record the \enquote{current state} of various 
% document properties (typically the content of macros and values of counters)
% and to access them in other places through a label.
% The list of properties that can be recorded and retrieved
% are not fix and can be extended by the user. The values of the properties 
% are recorded in the \texttt{.aux} file and can be retrieved at the second compilation.
% 
%
% The module uses the ideas of properties and labels. A label is
% a document reference point: a name for the user. An property is something
% that \LaTeX{} can track, such as a page number, section number or name.
% The names of labels and properties may be arbitrary. Note that there is
% a single namespace for each.
%
% \section{Design discussion}
%
% The design here largely follows ideas from \pkg{zref}. In particular, there
% are two independent concepts: properties that can be recorded between runs,
% and labels which consist of lists of these properties. The reason for the
% split is that individual labels will want to record some but not all
% properties. For examples, a label concerned with position would track
% the $x$ and $y$ coordinates of the current point, but not for example
% the page number.
%
% In the current implementation, properties share a single namespace. This
% allows multiple lists to re-use the same properties, for example page number,
% absolute page number, etc. This does mean that \emph{changing} a standard
% property is an issue. However, some properties have complex definitions
% (again, see \pkg{zref} at present): having them in a single shared space
% avoids the need to copy code.
%
% Labels could be implemented as |prop| data. That is not done at present as
% there is no obvious need to map to or copy the data. As such, faster
% performance is available using a hash table approach as in a \enquote{classical}
% set up. Data written to the |.aux| file uses simple paired \emph{balanced
% text} not keyvals: this avoids any restrictions on names and again offers
% increased performance.
%
% The \pkg{expl3} versions of the label command do not
% use \cs{@bsphack}/\cs{@esphack} to avoid double spaces, 
% but the \LaTeXe{} command does as it lives at the document command level.  
% 
% The reference commands are expandable.
% 
% Currently the code has nearly no impact on the main \cs{label} and \cs{ref} commands as
% too many external packages rely on the concrete implementation.
% There is one exception:
% the label names share the same namespace. That means that if both |\label{ABC}| and
% |\RecordProperties{ABC}{page}| are used there is a warning 
% \texttt{Label `ABC' multiply defined}. 
%
% \section{Handling unknown labels and properties}
% With the standard \cs{label}/\cs{ref} commands the requested label is 
% either in the |.aux|-file (and so known) or not. 
% In the first case the stored value can be used,
% in the second case the reference commands print two question marks.
% 
% With flexible property lists a reference commands asks for the 
% value of a specific property stored under a label name
% and we have to consider more variants: 
% \begin{itemize}
% \item If the requested property is unknown (not declared) the system 
% is not correctly set up and an error is issued. 
% \item If the label is unknown, the default of the property is used. 
% \item If the label is known, but doesn't provide a value for the 
% property then again the default of the property is used.
% \item The command |\property_ref:nnn| allows to give a local default
% which is used instead of the property default in the two cases before.  
% \end{itemize}
% 
% \section{Rerun messages}
% 
% As the reference commands are expandable they can neither issue a message that
% the label or the label-property combination is unknown, nor can they trigger the
% rerun message at the end of the \LaTeX{} run.
% 
% Where needed such messages must therefore be triggered manually. For this two commands
% are provided: \cs{property_ref_undefined_warn:} and  \cs{property_ref_undefined_warn:nn}.
% See below for a description.
% 
% \section{Open points}
% 
% \begin{itemize}
%  \item The \texttt{xpos} and \texttt{ypos} properties require that the position is 
% stored first but there is no (public) engine independent interface yet. Code must 
% use \cs{tex_savepos:D}. 
% \end{itemize}
% 
% \section{Code interfaces}
%
% \begin{function}{\property_new:nnnn,\property_gset:nnnn}
%   \begin{syntax}
%     \cs{property_new:nnnn}  \Arg{property} \Arg{setpoint} \Arg{default} \Arg{code}
%     \cs{property_gset:nnnn} \Arg{property} \Arg{setpoint} \Arg{default} \Arg{code}
%   \end{syntax}
%  \LaTeXe-interface: see \cs{NewProperty}, \cs{SetProperty}.\\ 
%   Sets the \meta{property} to have the \meta{default} specified, and at the
%   \meta{setpoint} (either |now| or |shipout|) to write the result of the
%   \meta{code} as part of a label. The \meta{code} should be expandable. The expansion
%   of \meta{code} (the value of the property) is written to the |.aux| file and read
%   back from there at the next compilation. Values should assume that the 
%   standard \LaTeX{} catcode régime with |@| a letter is active then. 
%   
%   If the property is declared within a package it is suggested
%   that its name is build from letters, hyphens and slashes,
%   and is always structured as follows:\\ 
%   \meta{package-name}\texttt{/}\meta{property-name}. 
% \end{function}
%
% \begin{function}
%   {
%     \property_record:nN,
%     \property_record:nn, \property_record:nV, \property_record:ee
%   }
%   \begin{syntax}
%      \cs{property_record:nN} \Arg{label} \meta{clist var}
%      \cs{property_record:nn} \Arg{label} \Arg{clist}
%   \end{syntax}
%   \LaTeXe{}-interface: see \cs{RecordProperties}.\\  
%   Writes the list of properties given by the \meta{clist} to the |.aux|
%   file with the \meta{label} specified.
% \end{function}
%
% \begin{function}[EXP]{\property_ref:nn,\property_ref:ee}
%   \begin{syntax}
%      \cs{property_ref:nn} \Arg{label} \Arg{property}
%   \end{syntax}
%   \LaTeXe{}-interface: see \cs{RefProperty}.\\  
%   Expands to the value of the \meta{property} for the \meta{label}, if
%   available, and the default value of the property otherwise.
%   If \meta{property} has not been declared with |\property_new:nnnn| 
%   an error is issued. The command raises an internal, expandable, local flag 
%   if the reference can not be resolved.
% \end{function}
% 
% \begin{function}[EXP]{\property_ref:nnn,\property_ref:een}
%   \begin{syntax}
%     \cs{property_ref:nnn} \Arg{label} \Arg{property} \Arg{local default}
%   \end{syntax}
%  \LaTeXe{}-interface: see \cs{RefProperty}.\\
%  Expands to the value of the \meta{property} for the \meta{label}, if
%  available, and to \meta{local default} otherwise.
%  If \meta{property} has not been declared with |\property_new:nnnn| 
%  an error is issued. The command raises an internal, expandable local flag 
%  if the reference can not be resolved.
% \end{function}
% 
% \begin{function}{\property_ref_undefined_warn:}
%   \begin{syntax}
%     \cs{property_ref_undefined_warn:}
%   \end{syntax}
%  \LaTeXe{}-interface: not provided.\\ 
%  The commands triggers the standard warning
%  \\
%  \hspace*{1em}\texttt{LaTeX Warning: There were undefined references.}
%  \\ 
%  at the end of the document if there was a recent
%  \cs{property_ref:nn} or \cs{property_ref:nnn} which couldn't be resolved
%  and so raised the flag. \enquote{Recent} means in the same group 
%  or in some outer group!
% \end{function}
%
% \begin{function}{\property_ref_undefined_warn:n,\property_ref_undefined_warn:e}
%   \begin{syntax}
%     \cs{property_ref_undefined_warn:n} \Arg{label}
%   \end{syntax}
% \LaTeXe{}-interface: not provided.\\
%  The commands triggers the standard warning\\
%  \hspace*{1em}\texttt{LaTeX Warning: There were undefined references.}
%  \\ 
%  at the end of the document if \meta{label} is not known.
%  At the point where it is called it also issues the warning\\
%  \hspace*{1em}%
%  \texttt{Reference~`\meta{label}'~on~page~\meta{page}\space undefined}.
% \end{function}
% 
% \begin{function}{\property_ref_undefined_warn:nn,\property_ref_undefined_warn:ee}
%   \begin{syntax}
%     \cs{property_ref_undefined_warn:nn} \Arg{label} \Arg{property}
%   \end{syntax}
%  \LaTeXe{}-interface: see \cs{RefUndefinedWarn}.\\
%  The commands triggers the standard warning\\
%  \hspace*{1em}%  
%  \texttt{LaTeX Warning: There were undefined references.}\\ 
%  at the end of the document if the reference can not be resolved.
%  At the point where it is called it also issues the warning\\
%  \hspace*{1em}%
%  \texttt{Reference~`\meta{label}'~on~page~\meta{page}\space undefined}\\
%   if the label
%  is unknown, or the more specific\\
%  \hspace*{1em}%
%  \texttt{Property `\meta{property}' undefined for reference
%    `\meta{label}' on page \meta{page}}\\
%  if the label is known but doesn't provide a value for the requested property.   
% \end{function}
% 
% \begin{function}[pTF]{\property_if_exist:n,\property_if_exist:e}
%   \begin{syntax}
%     \cs{property_if_exist_p:n} \Arg{property}
%     \cs{property_if_exist:nTF} \Arg{property} \Arg{true code} \Arg{false code}
%   \end{syntax}
%  \LaTeXe{}-interface: \cs{IfPropertyExistsTF}.\\   
%   Tests if the \meta{property} has been declared.
% \end{function}
%
% \begin{function}[pTF]{\property_if_recorded:n,\property_if_recorded:e}
%   \begin{syntax}
%     \cs{property_if_recorded_p:n} \Arg{label} 
%     \cs{property_if_recorded:nTF} \Arg{label} \Arg{true code} \Arg{false code}
%   \end{syntax}
%  \LaTeXe{}-interface: \cs{IfLabelExistsTF}\\
%   Tests if the \meta{label} is known. This is also true if the label has been
%   set with the standard \cs{label} command.
% \end{function}
%
% \begin{function}[pTF]{\property_if_recorded:nn,\property_if_recorded:ee}
%   \begin{syntax}
%     \cs{property_if_recorded_p:nn} \Arg{label} \Arg{property}
%     \cs{property_if_recorded:nnTF} \Arg{label} \Arg{property} \Arg{true code} \Arg{false code}
%   \end{syntax}
%   \LaTeXe{}-interface: \cs{IfPropertyRecordedTF}.\\  
%   Tests if the label \meta{label} is known and if it provides a value of the \meta{property}.
% \end{function}
%
% \section{Auxiliary file interfaces}
%
% \begin{function}{\new@label@record}
%   \begin{syntax}
%      \cs{new@label@record} \Arg{label} \Arg{data}
%   \end{syntax}
%   This is a command only for use in the |.aux| file. It loads the key--value
%   list of \meta{data} to be available for the \meta{label}.
% \end{function}
%
% \section{\LaTeXe{} interface}
% 
% The LaTeXe{} interfaces always expand label and property arguments. 
% This means that one must be careful when using active chars or commands in the
% names. UTF8-chars are protected and should be safe, similar most babel shorthands.
% 
% \begin{function}{\NewProperty,\SetProperty}
%   \begin{syntax}
%     \cs{NewProperty} \Arg{property} \Arg{setpoint} \Arg{default} \Arg{code}
%     \cs{SetProperty} \Arg{property} \Arg{setpoint} \Arg{default} \Arg{code}
%   \end{syntax}
%   Sets the \meta{property} to have the \meta{default} specified, and at the
%   \meta{setpoint} (either |now| or |shipout|) to write the result of the
%   \meta{code} as part of a label. The \meta{code} should be expandable. The expansion
%   of \meta{code} (the value of the property) is written to the |.aux| file and read
%   back from there at the next compilation (at which point normally
%   the standard \LaTeX{} catcode régime with |@| a letter is active).
%   
% \end{function}
% 
% \begin{function}{\RecordProperties}
%   \begin{syntax}
%      \cs{RecordProperties} \Arg{label} \Arg{clist}
%   \end{syntax}
%   Writes the list of properties given by the \meta{clist} to the |.aux|
%   file with the \meta{label} specified. Similar to the standard \cs{label} command
%   the arguments are expanded. So \meta{clist} can be a macro containing a list
%   of properties. Also similar to the standard \cs{label} command, the command is surrounded
%   by an \cs{@bsphack}/\cs{@esphack} pair to preserve spacing.
% \end{function}
%
% \begin{function}[EXP]{\RefProperty}
%   \begin{syntax}
%      \cs{RefProperty} \oarg{local default} \Arg{label} \Arg{property}
%   \end{syntax}
%   Expands to the value of the \meta{property} for the \meta{label}, if
%   available, and the default value of the property or -- if given -- 
%   to \meta{local default} otherwise.
%   If \Arg{property} has not been declared an error is issued.
% \end{function}
% 
% \begin{function}{\IfPropertyExistsTF,\IfPropertyExistsT,\IfPropertyExistsF}
%   \begin{syntax}
%     \cs{IfPropertyExistsTF} \Arg{property} \Arg{true code} \Arg{false code}
%   \end{syntax}
%   Tests if the \meta{property} has been declared.
% \end{function}
%
% \begin{function}{\IfLabelExistsTF,\IfLabelExistsT,\IfLabelExistsF}
%   \begin{syntax}
%     \cs{IfLabelExistsTF} \Arg{label} \Arg{true code} \Arg{false code}
%   \end{syntax}
%   Tests if the \meta{label} has been recorded. This is also true if a label
%   has been set with the standard \cs{label} command.
% \end{function}
% 
% \begin{function}{\IfPropertyRecordedTF,\IfPropertyRecordedT,\IfPropertyRecordedF}
%   \begin{syntax}
%     \cs{IfPropertyRecordedTF} \Arg{label} \Arg{property} \Arg{true code} \Arg{false code}
%   \end{syntax}
%   Tests if the label and a value of the \meta{property} for the \meta{label} are both known.
% \end{function}
%
% \begin{function}{\RefUndefinedWarn}
%   \begin{syntax}
%      \cs{RefUndefinedWarn} \Arg{label} \Arg{property}
%   \end{syntax}
%  This commands triggers the standard warning\\
%  \hspace*{1em}%
%  \texttt{LaTeX Warning: There were undefined references.}\\ 
%  at the end of the document if the reference for \meta{label} and \meta{property}
%  can not be resolved.
%  At the point where it is called it also issues the warning\\
%  \hspace*{1em}%
%  \texttt{Reference `\meta{label}' on page \meta{page} undefined}\\
%  if the label  is unknown, or the more specific\\
%  \hspace*{1em}%
%  \texttt{Property `\meta{property}' undefined for reference 
%  `\meta{label}' on page \meta{page}}
%  if the label is known but doesn't provide a value for the requested property.   
% \end{function}
%
% \section{Pre-declared properties}
%
% \begin{variable}{abspage}
%   (shipout) The absolute value of the current page: starts at $1$ and increases
%   monotonically at each shipout.
% \end{variable}
%
% \begin{variable}{page}
%   (shipout) The current page as given by \cs{thepage}: this may or may not
%   be a numerical value, depending on the current style. Contrast with
%   \cs{abspage}. You get this value also  with the standard \cs{label}/\cs{pageref}.
% \end{variable}
%
% \begin{variable}{pagenum}
%   (shipout) The current page as arabic number. This is suitable for integer operations and
%   comparisons.
% \end{variable}
% 
% \begin{variable}{label}
%   (now) The content of \cs{@currentlabel}. This is the value that
%   you get also with the standard \cs{label}/\cs{ref}. 
% \end{variable}
% 
% \begin{variable}{title}
%    (now) The content of \cs{@currentlabelname}. 
%   This command is filled beside others by the \pkg{nameref} package and some
%   classes (e.g.~\pkg{memoir}).
% \end{variable}
% 
% \begin{variable}{target}
%   (now)  The content of \cs{@currentHref}. 
%   This command is normally filled by for example 
%   \pkg{hyperref} and gives the name of the last destination it created. 
% \end{variable}
% 
% \begin{variable}{pagetarget}
%   (shipout)  The content of \cs{@currentHpage}. 
%   This command is filled for example by a recent version of 
%   \pkg{hyperref} and then gives the name of the last page 
%   destination it created. 
% \end{variable}
% 
% \begin{variable}{counter}
%   (now) The content of \cs{@currentcounter}. 
%   This command contains after a \cs{refstepcounter} the name of the counter.
% \end{variable}
% 
% \begin{variable}{xpos,ypos}
%   (shipout)  This stores the $x$~and $y$ coordinates of a point previously 
%   stored with \cs{pdfsavepos}/\cs{savepos}. 
%   E.g.~(if bidi is used it can be necessary to save the position
%   before and after the label):
%   \begin{verbatim}
%     \tex_savepos:D 
%     \property_record:nn{myposition}{xpos,ypos}
%     \tex_savepos:D 
%   \end{verbatim}  
% \end{variable}
% \end{documentation}
%
% \begin{implementation}
%
% \section{The Implementation}
%
%    \begin{macrocode}
%<*2ekernel|latexrelease>
%    \end{macrocode}
%    \begin{macrocode}
\ExplSyntaxOn
%    \end{macrocode}
%
%    \begin{macrocode}
%<@@=property>
%    \end{macrocode}
%
%    \begin{macrocode}
%<latexrelease>\NewModuleRelease{2023/11/01}{ltproperties}
%<latexrelease>                 {Cross-referencing~properties}
%    \end{macrocode}
%
% The approach here is based closely on that from \pkg{zref}; separate out
% lists of properties and the properties themselves, so the latter can be
% used multiple times and in varying combinations. 
% However, not everything is a straight copy. Firstly,
% we treat lists of properties as simple comma lists: that allows us to have
% either saved or dynamic lists and to avoid another data structure. The cost
% is that errors are detected at point-of-use, but in any real case that should
% be true anyway (and is true for \tn{zref@labelbyprop} already). Secondly,
% we allow properties to have
% arbitrary names, as the code does not require them to tokenize as control
% sequences.
%
% \begin{macro}{\property_new:nnnn, \property_gset:nnnn, \@@_gset:nnnn}
%   As properties can be reset, they are not constants. But they also have
%   various pieces of required data. So we use the same approach as color and
%   make them declarations. Data-wise, we need the detail of the implementation,
%   the default and a flag to show if the code works now or at shipout. This
%   last entry is done using text so needs a check. We could use a set of
%   |prop| here, but as we never need to map or copy the lists, we can gain
%   performance using the hash table approach.
%    \begin{macrocode}
\cs_new_protected:Npn \property_new:nnnn #1#2#3#4
  {
    \cs_if_free:cTF { @@_code_ #1 : }
      {
        \exp_args:Nx \@@_gset:nnnn { \tl_to_str:n {#1} }
         {#2} {#3} {#4}
      }
      {
        \msg_error:nn { property }{ exists }{#1}
      }   
  }
\cs_new_protected:Npn \property_gset:nnnn #1#2#3#4
  {
    \@@_gset:ennn { \tl_to_str:n {#1} }
      {#2} {#3} {#4}
  }
\cs_new_protected:Npn \@@_gset:nnnn #1#2#3#4
  {
    \cs_gset:cpn { @@_code_ #1 : } {#4}
    \tl_gclear_new:c { g_@@_default_ #1 _tl }
    \tl_gset:cn { g_@@_default_ #1 _tl } {#3}
    \bool_if_exist:cF { g_@@_shipout_ #1 _tl }
      { \bool_new:c { g_@@_shipout_ #1 _tl } }
    \str_case:nnF {#2}
      {
        { now } { { \bool_gset_false:c { g_@@_shipout_ #1 _tl } } }
        { shipout }
          { \bool_gset_true:c { g_@@_shipout_ #1 _tl } }
      }
      { \msg_error:nnnn { property } { unknown-setpoint } {#1} {#2} }
  }
\cs_generate_variant:Nn \@@_gset:nnnn {ennn}
%    \end{macrocode}
% \end{macro}
% 
% \begin{macro}{\NewProperty,\SetProperty}
% For consistency we expand the property name, but this doesn't warant a 
% variant of the L3-commands.
% \changes{v1.0c}{2023-09-20}{use \cs{protected@edef}} 
%    \begin{macrocode}
\cs_new_protected:Npn \NewProperty #1#2#3#4
  {
    \protected@edef\reserved@a{#1}
    \exp_args:No \property_new:nnnn {\reserved@a} {#2}{#3}{#4}
  }  
\cs_new_protected:Npn \SetProperty #1#2#3#4
  {
    \protected@edef\reserved@a{#1}
    \exp_args:No \property_gset:nnnn {\reserved@a} {#2}{#3}{#4}
  }    
%    \end{macrocode}
% \end{macro}
% 
% \begin{macro}{\property_record:nN}
% \begin{macro}
%   {
%     \property_record:nn,
%     \property_record:nV,
%     \property_record:ee,
%     \property_record:oo,
%     \@@_record:nn,
%     \@@_record:en
%   }
% \begin{macro}[EXP]
%   {\@@_record_value:n, \@@_record_value_aux:n, \@@_record_value_aux:e}
%   Writing data when it is labelled means expanding at this stage and possibly
%   later too. That is all pretty easy using \pkg{expl3}: we accept a stray
%   comma at the end of the list as that is easier to deal with than trying
%   to tidy up, and there is no real downside.
% \changes{v1.0d}{2024-01-17}{Use \cs{protected@write}}%   
% \changes{v1.0f}{2024-09-05}{Remove \cs{if@filesw} test to be in line with \cs{label}, 
%   tagging-project issue 696}%
%    \begin{macrocode}
\cs_new_protected:Npn \property_record:nN #1#2
  { \property_record:nV {#1} #2 }
\cs_new_protected:Npn \property_record:nn #1#2
  { \@@_record:en { \tl_to_str:n {#1} } {#2} }
\cs_generate_variant:Nn \property_record:nn { nV , ee, oo }
\cs_new_protected:Npn \@@_record:nn #1#2
  {
    \protected@write \@auxout {}
      {
        \token_to_str:N \new@label@record 
          {#1}
          { \clist_map_function:nN {#2} \@@_record_value:n }
      }
  }
\cs_generate_variant:Nn \@@_record:nn { e }
\cs_new:Npn \@@_record_value:n #1
  { \@@_record_value_aux:e { \tl_to_str:n {#1} } }
\cs_new:Npn \@@_record_value_aux:n #1
  {
    \cs_if_exist:cTF { @@_code_ #1 : }
      {
        {#1}
        {
          \bool_if:cTF { g_@@_shipout_ #1 _tl }
            { \exp_not:c }
            { \use:c }
              { @@_code_ #1 : }
        }
      }
      { \msg_expandable_error:nnn { property } { not-declared } {#1} }
  }
\cs_generate_variant:Nn \@@_record_value_aux:n { e }
%    \end{macrocode}
% \end{macro}
% \end{macro}
% \end{macro}
%
% \begin{macro}{\RecordProperties}
% \changes{v1.0c}{2023-09-20}{use \cs{protected@edef} for safer handling of active chars.} % 
%    \begin{macrocode}
\NewDocumentCommand\RecordProperties { m m }
  {
    \@bsphack
    \protected@edef\reserved@a{#1}
    \protected@edef\reserved@b{#2}
    \property_record:oo {\reserved@a}{\reserved@b}
    \@esphack
  }  
%    \end{macrocode}
% \end{macro}
%
% \subsection{Reference commands}
%
% \begin{variable}{ l_@@_ref_flag }
%   A flag that is set if a reference couldn't be resolved.
%    \begin{macrocode}
\flag_new:n { l_@@_ref_flag }
%    \end{macrocode}
% \end{variable}
%
% \begin{macro}[EXP]{\property_ref:nn,\property_ref:ee}
%   Search for the label/property combination, and if not found fall back
%   to the default of the property.
%    \begin{macrocode}
\cs_new:Npn \property_ref:nn #1#2
  { 
    \@@_ref:een 
      { \tl_to_str:n {#1} } 
      { \tl_to_str:n {#2} }
      { \tl_use:c { g_@@_default_ #2 _tl } } 
  }
\cs_generate_variant:Nn \property_ref:nn {ee}  
%    \end{macrocode}
% \end{macro}
% 
% \begin{macro}[EXP]{\property_ref:nnn,\property_ref:een, \@@_ref:nnn, \@@_ref:een}
%   This allows to set a local default value which overrides the default value
%   of the property.
%    \begin{macrocode}
\cs_new:Npn \property_ref:nnn #1#2#3
  {
    \@@_ref:een
      { \tl_to_str:n {#1} }
      { \tl_to_str:n {#2} }
      {#3}
  }
\cs_new:Npn \@@_ref:nnn #1#2#3
  {
    \tl_if_exist:cTF { g_@@_label_ #1 _ #2 _tl }
      { \tl_use:c { g_@@_label_ #1 _ #2 _tl } }
      {
        \flag_if_raised:nF 
          { l_@@_ref_flag } { \flag_raise:n { l_@@_ref_flag } }
%    \end{macrocode}
%   We test for the default of the property only to check if the property has
%   been declared.
%    \begin{macrocode}
        \tl_if_exist:cTF { g_@@_default_ #2 _tl }
          { #3 }
          { \msg_expandable_error:nnn { property } { not-declared } {#2} }
      }
  }
\cs_generate_variant:Nn \@@_ref:nnn { ee }
\cs_generate_variant:Nn \property_ref:nnn {een} 
%    \end{macrocode}
% \end{macro}
% 
% \begin{macro}[EXP]{\RefProperty}
%   Search for the label/property combination, and if not found fall back
%   to the default of the property or the given default.
%    \begin{macrocode}
\NewExpandableDocumentCommand \RefProperty { o m m }
  {
    \IfNoValueTF {#1}
      {
        \property_ref:ee {#2}{#3}
      }
      {
        \property_ref:een {#2}{#3}{#1}
      }
  }
%    \end{macrocode}
% \end{macro}
% 
% \begin{macro}{\new@label@record}
% \begin{macro}{\@@_data:nnn}
%   A standard recursion loop.
%    \begin{macrocode}
\cs_new_protected:Npn \new@label@record #1#2
  {
    \tl_if_exist:cTF { r@#1 }
      {
        \gdef \@multiplelabels
          { \@latex@warning@no@line { There~were~multiply-defined~labels } }
        \@latex@warning@no@line { Label~`#1'~multiply~defined }
      }
      {
        \tl_new:c { r@#1 }
        \tl_gset:cn { r@#1 }{#2}
      }
    \@@_data:nnn {#1} #2 { \q_recursion_tail } { ? } \q_recursion_stop
  }
\cs_new_protected:Npn \@@_data:nnn #1#2#3
  {
    \quark_if_recursion_tail_stop:n {#2}
    \tl_gclear_new:c { g_@@_label_ \tl_to_str:n {#1} _ \tl_to_str:n {#2} _tl }
    \tl_gset:cn { g_@@_label_ \tl_to_str:n {#1} _ \tl_to_str:n {#2} _tl } {#3}
    \@@_data:nnn {#1}
  }  
%    \end{macrocode}
%
% This command is used in \cs{enddocument} to test if some label values have changed.
%    \begin{macrocode}
\cs_new_protected:Npn \@kernel@new@label@record@testdef #1 #2
  {
    \tl_if_eq:cnF { r@#1 } {#2} 
      { \@tempswatrue }
  }
%    \end{macrocode}
% \end{macro}
% \end{macro}
%
% \subsection{Tests and warnings}
%
% \begin{macro}[pTF]{\property_if_exist:n}
%   Tests if property has been declared.
%    \begin{macrocode}
\prg_new_conditional:Npnn \property_if_exist:n #1 { p , T , F,  TF }  
  % #1 property
  {
    \cs_if_exist:cTF { @@_code_ #1 : }
      {
        \prg_return_true:
      }
      {
        \prg_return_false:
      }
  }
\prg_generate_conditional_variant:Nnn \property_if_exist:n {e} {TF}
%    \end{macrocode}
% \end{macro}
%
% \begin{macro}{\IfPropertyExistsTF,\IfPropertyExistsT,\IfPropertyExistsF}
% \changes{v1.0e}{2024-04-17}{Renamed \cs{IfPropertyExistTF} to
%                             \cs{IfPropertyExistsTF} (gh/1262)} 
%    \begin{macrocode}
\cs_new_eq:NN \IfPropertyExistsTF \property_if_exist:eTF  
\cs_new:Npn   \IfPropertyExistsT #1#2 {\property_if_exist:eTF {#1}{#2}{} }  
\cs_new:Npn   \IfPropertyExistsF #1   {\property_if_exist:eTF {#1}{} }  
%    \end{macrocode}
% \end{macro}
% 
% \begin{macro}[pTF]{\property_if_recorded:n}
%   Tests if the label has been set.
%   This can then be used to setup e.g.~rerun messages.
%    \begin{macrocode}
\prg_new_conditional:Npnn \property_if_recorded:n #1 { p , T , F,  TF }  
  % #1 label 
  {
    \tl_if_exist:cTF { r@#1 }
      {
        \prg_return_true:
      }
      {
        \prg_return_false:
      }
  }
\prg_generate_conditional_variant:Nnn \property_if_recorded:n {e} {TF}  
%    \end{macrocode}
% \end{macro}
%
% \begin{macro}{\IfLabelExistsTF,\IfLabelExistsT,\IfLabelExistsF}
% \changes{v1.0e}{2024-04-17}{Renamed \cs{IfLabelExistTF} to
%                             \cs{IfLabelExistsTF} (gh/1262)}
% \changes{v1.0g}{2024-09-25}{Fixed definitions of \cs{IfLabelExistsT}
%                                              and \cs{IfLabelExistsF}}
%    \begin{macrocode}
\cs_new_eq:NN \IfLabelExistsTF \property_if_recorded:eTF  
\cs_new:Npn   \IfLabelExistsT #1#2 {\property_if_recorded:eTF {#1}{#2}{} }  
\cs_new:Npn   \IfLabelExistsF #1   {\property_if_recorded:eTF {#1}{} }  
%    \end{macrocode}
% \end{macro} 
%
% \begin{macro}[pTF]{\property_if_recorded:nn}
% tests if the label/property combination has been set
% This can then be used to setup e.g.~rerun messages.
%    \begin{macrocode}
\prg_new_conditional:Npnn \property_if_recorded:nn #1#2 { p , T , F,  TF }  
  % #1 label #2 property
  {
    \tl_if_exist:cTF { g_@@_label_ \tl_to_str:n {#1} _ \tl_to_str:n {#2} _tl }
      {
        \prg_return_true:
      }
      {
        \prg_return_false:
      }
  }
\prg_generate_conditional_variant:Nnn \property_if_recorded:nn {ee} {TF}     
%    \end{macrocode}
% \end{macro}
%
% \begin{macro}{\IfPropertyRecordedTF,\IfPropertyRecordedT,\IfPropertyRecordedF}
% \changes{v1.0h}{2024-10-21}{Define \cs{IfPropertyRecordedT}, \cs{IfPropertyRecordedF}}
%    \begin{macrocode}
\cs_new_eq:NN \IfPropertyRecordedTF \property_if_recorded:eeTF
\cs_new:Npn \IfPropertyRecordedT #1#2#3 { \property_if_recorded:eeTF {#1}{#2}{#3}{} }
\cs_new:Npn \IfPropertyRecordedF #1#2#3 { \property_if_recorded:eeTF {#1}{#2}{}{#3} }
%    \end{macrocode}
% \end{macro}
% 
% \begin{macro}{\property_ref_undefined_warn:}
% \cs{G@refundefinedtrue} is defined in \texttt{ltxref} and redefines a warning message.
%    \begin{macrocode}
\cs_new_protected:Npn \property_ref_undefined_warn:
 {
   \flag_if_raised:nT { l_@@_ref_flag }
    {
      \G@refundefinedtrue
    }
 }
%    \end{macrocode}
% \end{macro}
%
% \begin{macro}{\property_ref_undefined_warn:n}
%    \begin{macrocode}
\cs_new_protected:Npn \property_ref_undefined_warn:n #1 %#1 label
  {
   \property_if_recorded:nF {#1}
    {
       \G@refundefinedtrue
       \@latex@warning{Reference~`#1'~on~page~\thepage\space undefined}%
    }         
  }
%    \end{macrocode}
% \end{macro}
%
% \begin{macro}{\property_ref_undefined_warn:nn,
%               \property_ref_undefined_warn:ee,
%               \RefUndefinedWarn}
%    \begin{macrocode}
\cs_new_protected:Npn \property_ref_undefined_warn:nn #1#2 %#1 label, #2 property
 {
   \property_if_recorded:nTF {#1}
    {
      \property_if_recorded:nnF {#1}{#2}
        {
          \G@refundefinedtrue
          \@latex@warning
           { Property~`#2'~undefined~for~reference~`#1'~on~page~\thepage }
        }
    }
    {
       \G@refundefinedtrue
       \@latex@warning { Reference~`#1'~on~page~\thepage\space undefined }%
    }         
 }
\cs_generate_variant:Nn \property_ref_undefined_warn:nn {ee}
\cs_set_eq:NN \RefUndefinedWarn \property_ref_undefined_warn:ee
%    \end{macrocode}
% \end{macro}
%
% \subsection{Predeclared properties}
%
% \begin{variable}{abspage}
%    \begin{macrocode}
\property_new:nnnn { abspage } { shipout }
  { 0 } { \int_use:N \g_shipout_readonly_int }
%    \end{macrocode}
% \end{variable}
%
% \begin{variable}{page}
%    \begin{macrocode}
\property_new:nnnn  { page } { shipout } { 0 } { \thepage }
%    \end{macrocode}
% \end{variable}
% 
% \begin{variable}{pagenum}
%    \begin{macrocode}
\property_new:nnnn  { pagenum } { shipout } { 0 } { \the \value { page } }
%    \end{macrocode}
% \end{variable}
% 
% \begin{variable}{label}
%    \begin{macrocode}
\property_new:nnnn  { label } { now } { ?? } { \@currentlabel }
%    \end{macrocode}
% \end{variable}
% 
% \begin{variable}{title}
%    \begin{macrocode}
\property_new:nnnn  { title } { now } 
  { \exp_not:n { \textbf { ?? } } } { \@currentlabelname }
%    \end{macrocode}
% \end{variable}
% 
% \begin{variable}{target}
%    \begin{macrocode}
\property_new:nnnn  { target } { now } { } { \@currentHref }
%    \end{macrocode}
% \end{variable}
%
% \begin{variable}{target}
%    \begin{macrocode}
\newcommand\@currentHpage{}
\property_new:nnnn  { pagetarget } { shipout } { } { \@currentHpage }
%    \end{macrocode}
% \end{variable}

% \begin{variable}{counter}
%    \begin{macrocode}
\property_new:nnnn  { counter } { now }  { } { \@currentcounter }
%    \end{macrocode}
% \end{variable}
% 
% \begin{variable}{xpos,ypos}
%    \begin{macrocode}
\property_new:nnnn  { xpos } { shipout } { 0} { \int_use:N \tex_lastxpos:D }
\property_new:nnnn  { ypos } { shipout } { 0} { \int_use:N \tex_lastypos:D } 
%    \end{macrocode}
% \end{variable}
%  
%  \subsection{Messages}
%    \begin{macrocode}
\msg_new:nnnn { property } { exists }
    { Property~'#1'~ has~ already~ been~ declared. }
    { There~ already~ exists~ a~ property~ declaration~ with~ this~
      name.\\
      Please~ use~ a~ different~ name~ for~ your~ property.}
      
\msg_new:nnnn { property } { not-declared }
  { Property~'#1'~not~declared. }
  {
    LaTeX~has~been~asked~to~use~property~'#1',~but~this~
    name~has~not~been~declared.
  }
\msg_new:nnnn { property } { unknown-setpoint }
  { Unknown~keyword~'#2'~for~setting~property~'#1'. }
  {
    LaTeX~has~been~asked~to~set~the~property~'#1',~but~the~keyword~
    '#2'~is~not~one~of~the~two~known~values:~'now'~or~'shipout'.
  }
%    \end{macrocode}
%
%    \begin{macrocode}
%
%<latexrelease>\IncludeInRelease{0000/00/00}{ltproperties}
%<latexrelease>                 {cross-referencing~properties~(undo)}%
%<latexrelease>
%<latexrelease>\let \NewProperty  \@undefined
%<latexrelease>\let \SetProperty \@undefined
%<latexrelease>
%<latexrelease>\let \RecordProperties  \@undefined
%<latexrelease>\let \RefProperty \@undefined
%<latexrelease>\let \RefUndefinedWarn  \@undefined
%<latexrelease>
%<latexrelease>\let \IfPropertyExistsTF \@undefined
%<latexrelease>\let \IfLabelExistsTF \@undefined
%<latexrelease>\let \IfPropertyRecordedTF  \@undefined
%<latexrelease>
%<latexrelease>\let\new@label@record \@undefined
%<latexrelease>\let\@kernel@new@label@record@testdef\@undefined
%<latexrelease>\EndModuleRelease
%    \end{macrocode}
%
%    \begin{macrocode}
\ExplSyntaxOff
%    \end{macrocode}
%
%    \begin{macrocode}
%</2ekernel|latexrelease>
%    \end{macrocode}
%
%    Reset module prefix:
%    \begin{macrocode}
%<@@=>
%    \end{macrocode}
%
%
% \end{implementation}
%
% \Finale
