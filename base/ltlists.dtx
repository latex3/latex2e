% \iffalse meta-comment
%
% Copyright (C) 1993-2025
% The LaTeX Project and any individual authors listed elsewhere
% in this file.
%
% This file is part of the LaTeX base system.
% -------------------------------------------
%
% It may be distributed and/or modified under the
% conditions of the LaTeX Project Public License, either version 1.3c
% of this license or (at your option) any later version.
% The latest version of this license is in
%    https://www.latex-project.org/lppl.txt
% and version 1.3c or later is part of all distributions of LaTeX
% version 2008 or later.
%
% This file has the LPPL maintenance status "maintained".
%
% The list of all files belonging to the LaTeX base distribution is
% given in the file `manifest.txt'. See also `legal.txt' for additional
% information.
%
% The list of derived (unpacked) files belonging to the distribution
% and covered by LPPL is defined by the unpacking scripts (with
% extension .ins) which are part of the distribution.
%
% \fi
% \iffalse
%%% From File: ltlists.dtx
%<*driver>
% \fi
\ProvidesFile{ltlists.dtx}
             [2025/01/26 v1.0v LaTeX Kernel (List Environments)]
% \iffalse
\documentclass{ltxdoc}
\GetFileInfo{ltlists.dtx}
\title{\filename}
\date{\filedate}
 \author{%
  Johannes Braams\and
  David Carlisle\and
  Alan Jeffrey\and
  Leslie Lamport\and
  Frank Mittelbach\and
  Tobias Oetiker\thanks{Tobi has converted the documentation to
    doc.sty standard}\and
  Chris Rowley\and
  Rainer Sch\"opf}

\begin{document}
 \MaintainedByLaTeXTeam{latex}
 \maketitle
 \DocInput{\filename}
\end{document}
%</driver>
% \fi
%
%
% \changes{v1.0b}{1994/03/28}{Improve documentation}
% \changes{v1.0f}{1994/05/21}{Use new error commands}
% \changes{v1.0f}{1995/05/21}{Moved to doc.sty standard}
%
% \section{List, and related environments}
%
% The generic commands for creating an indented environment --
% |enumerate|, |itemize|, |quote|, etc -- are:
% \begin{quote}
%        |\list|\marg{LABEL}\marg{COMMANDS} ... |\endlist|
% \end{quote}
% which can be invoked by the user as the list environment.  The LABEL
% argument specifies item labeling.  COMMANDS contains commands for
% changing the horizontal and vertical spacing parameters.
%
% Each item of the environment is begun by the command
% |\item[|ITEMLABEL|]|
% which produces an item labeled by ITEMLABEL.  If the argument is
% missing, then the LABEL argument of the |\list| command is used as the
% item label.
%
% The label is formed by putting |\makelabel|\marg{ITEMLABEL} in an hbox
% whose width is either its natural width or else |\labelwidth|,
% whichever is larger.  The |\list| command defines |\makelabel| to have
% the default  definition:
% \begin{quote}
%     |\makelabel|\marg{ARG} == BEGIN |\hfil| ARG END
% \end{quote}
% which, for a label of width less than |\labelwidth|, puts the label
% flushright, |\labelsep| to the left of the item's text.  However,
% |\makelabel| can be |\let| to another command by the |\list|'s
% COMMANDS argument.
%
% A |\usecounter|\marg{foo} command in the second argument causes the
% counter \emph{foo} to be initialized to zero, and stepped by every
% |\item| command without an argument.  (|\label| commands within the
% list refer to this counter.)
%
% When you leave a list environment, returning either to an enclosing
% list or normal text mode, LaTeX begins a new paragraph if and only if
% you leave a blank line after the |\end| command.  This is accomplished
% by the |\@endparenv| command.
%
% Blank lines are ignored every other reasonable place--i.e.:
% \begin{itemize}
%  \item  Between the |\begin{list}| and the first |\item|,
%  \item  Between the |\item| and the text of that item.
%  \item Between the end of the last item and the |\end{list}|.
% \end{itemize}
%
% For an environment like quotation, in which items are not labeled,
% the entire environment is a single item.  It is defined by
% letting |\quotation| == |\list{}{...}\item\relax|.  (Note the
% |\relax|, there in case the first character in the environment is a
% '['.)  The spacing parameters provide a great deal of flexibility in
% designing the format, including the ability to let the indentation of
% the first paragraph be different from that of the subsequent ones.
%
% The trivlist environment is equivalent to a list environment
% whose second argument sets the following parameter values:
% \begin{description}
% \item[\cs{leftmargin} = 0:] causes no indentation of left margin
% \item[\cs{labelwidth} = 0:] see below for precise effect this has.
% \item[\cs{itemindent} = 0:] with a null label, makes first paragraph
%        have no indentation.  Succeeding paragraphs have |\parindent|
%        indentation.  To give first paragraph same indentation, set
%        |\itemindent| = |\parindent| before the |\item[]|.
% \end{description}
%
% Every |\item| in a trivlist environment must have an argument---in
% many cases, this will be the null argument (|\item[]|).  The trivlist
% environment is mainly used for paragraphing environments, like
% verbatim, in which there is no margin change.  It provides the same
% vertical spacing as the list environment, and works reasonably well
% when it occurs immediately after an |\item| command in an enclosing
% list.
%
% \MaybeStop{}
%
%
% \changes{v1.0a}{1994/03/07}{Initial version, split from latex.dtx}
% \changes{v1.0a}{1994/03/07}{Long lines wrapped to 72 columns}
%
%
% \subsection{List and Trivlist}
%
%
% The following variables are used inside a list environment:
% \begin{description}
% \item[\cs{@totalleftmargin}] The distance that the prevailing left
%     margin is indented from the outermost left margin,
% \item[\cs{linewidth}] The width of the current line.  Must be
%     initialized to |\hsize|.
% \item[\cs{@listdepth}] A count for holding current list nesting depth.
% \item[\cs{makelabel}] A macro with a single argument, used to
%   generate the label from the argument (given or implied)
%   of the |\item| command. Initialized to |\@mklab| by the |\list|
%   command.  This command must produce  some stretch---i.e., an
%   |\hfil|.
% \item[\cs{@inlabel}] A switch that is false except between the time
%   an |\item| is encountered and the time that \TeX{}
%   actually enters horizontal mode.  Should be tested by commands
%   that can be messed up by the list environment's use of |\everypar|.
% \item[\cs{box}\cs{@labels}] When |@inlabel = true|, it holds the labels
%   to be put out by |\everypar|.
% \item[\texttt{@noparitem}] A switch set by |\list| when
%   |@inlabel = true|.
%   Handles the case of a |\list| being the first thing in an item.
% \item[\texttt{@noparlist}] A switch set true for a list that begins an
%   item.  No |\topsep| space is added before or after |\item|'s such a
%   list.
% \item[\texttt{@newlist}] Set true by |\list|, set false by the first
%   text (by |\everypar|).
% \item[\texttt{@noitemarg}]  Set true when executing an |\item| with no
%   explicit argument.  Used to save space. To save time, make two
%   separate  |\@item| commands.
% \item[\texttt{@nmbrlist}] Set true by |\usecounter| command, causes
%   list to be numbered.
% \item[\cs{@listctr}] |\def|'ed by |\usecounter| to name of counter.
% \item[\cs{@noskipsec}] A switch set true by a sectioning command when
%    it is creating an in-text heading with |\everypar|.
% \end{description}
%
% Throughout a list environment, |\hsize| is the width of the current
% line, measured from the outermost left margin to the outermost right
% margin.  Environments like tabbing should use |\linewidth| instead of
% |\hsize|.
%
% Here are the parameters of a list that can be set by commands in
% the |\list|'s COMMANDS argument.  These parameters are all \TeX{}
% skips or dimensions (defined by |\newskip| or |\newdimen|), so the
% usual \TeX\ or \LaTeX\ commands can be used to set them.  The
% commands will be executed in vmode if and only if the |\list| was
% preceded by a |\par| (or something like an |\end{list}|), so the
% spacing parameters can be set according to whether the list is
% inside a paragraph or is its own paragraph.
%
%
% \subsection{Vertical Spacing (skips)}
% \begin{description}
% \item[\cs{topsep}:]  Space between first item and preceding paragraph.
% \item[\cs{partopsep}:] Extra space added to \cs{topsep} when
%        environment starts a new paragraph (is called in vmode).
% \item[\cs{itemsep}:] Space between successive items.
% \item[\cs{parsep}:] Space between paragraphs within an item -- the
%                 \cs{parskip} for this environment.
% \end{description}
%
% \subsection{Penalties}
% \begin{description}
%
% \item[\cs{@beginparpenalty}:] put at the beginning of a list
% \item[\cs{@endparpenalty}:] put at end of list
%  \item[\cs{@itempenalty}:] put between items.
%  \end{description}
%
% \subsection{Horizontal Spacing (dimens)}
% \begin{description}
% \item[\cs{leftmargin}:] space between left margin of enclosing
%   environment (or of page if top level list) and left margin of
%                     this list.  Must be nonnegative.
%  \item[\cs{rightmargin}:] analogous.
%  \item[\cs{listparindent}:] extra indentation at beginning of every
%     paragraph of a list except the one started by the \cs{item}
%                      command.  May be negative!  Usually, labeled
%                       lists have \cs{listparindent} equal to zero.
%   \item[\cs{itemindent}:] extra indentation added right BEFORE an item
%                      label.
%  \item[\cs{labelwidth}:] nominal width of box that contains the label.
%                      If the natural width of the
%                         label $\leq$ \cs{labelwidth},
%                      then the label is flushed right inside a box
%                      of width \cs{labelwidth} (with an \cs{hfil}).
%                      Otherwise,
%                      a box of the natural width is employed, which
%                       causes an indentation of the text on that line.
%     \item[\cs{labelsep}:] space between end of label box and text of
%                      first item.
%  \end{description}
% \subsection{Default Values}
%      Defaults for the list environment are set as follows.
%      First, \cs{rightmargin}, \cs{listparindent} and \cs{itemindent}
%      are set
%      to 0pt.  Then, one of the commands
%      \cs{@listi}, \cs{@listii}, ... , \cs{@listvi}
%      is called, depending upon the current level of the list.
%      The \cs{@list} \ldots commands should be defined by the document
%      style.  A convention that the document style should follow is
%      to set \cs{leftmargin} to
%      \cs{leftmargini},\ldots, \cs{leftmarginvi} for
%      the appropriate level.  Items that aren't changed may be left
%      alone, but everything that could possibly be changed must be
%      reset.
%  \begin{oldcomments}
%  \list{LABEL}{COMMANDS} ==
%   BEGIN
%     if \@listdepth > 5
%       then LaTeX error: 'Too deeply nested'
%       else \@listdepth :=G \@listdepth + 1
%     fi
%     \rightmargin     := 0pt
%     \listparindent   := 0pt
%     \itemindent      := 0pt
%     \eval(@list \romannumeral\the\@listdepth)  %% Set default values:
%     \@itemlabel      :=L LABEL
%     \makelabel       == \@mklab
%     @nmbrlist        :=L false
%     COMMANDS
%
%     \@trivlist                % commands common to \list and \trivlist
%
%     \parskip          :=L \parsep
%     \parindent        :=L \listparindent
%     \linewidth        :=L \linewidth - \rightmargin -\leftmargin
%     \@totalleftmargin :=L \@totalleftmargin + \leftmargin
%     \parshape 1 \@totalleftmargin \linewidth
%     \ignorespaces                    % gobble space up to \item
%    END
%
% \endlist == BEGIN \@listdepth :=G \@listdepth -1
%                   \endtrivlist
%             END
%
% \@trivlist ==
%  BEGIN
%     if @newlist = T then \@noitemerr fi
%                     %% This command removed for some forgotten reason.
%     \@topsepadd :=L \topsep
%     if @noskipsec then leave vertical mode fi  %% Added 11 Jun 85
%     if vertical mode
%       then \@topsepadd :=L \@topsepadd + \partopsep
%       else \unskip \par            % remove glue from end of last line
%     fi
%     if @inlabel = true
%        then @noparitem :=L true
%             @noparlist :=L true
%        else @noparlist :=L false
%             \@topsep   :=L \@topsepadd
%     fi
%     \@topsep         :=L \@topsep + \parskip  %% Change 4 Sep 85
%     \leftskip        :=L 0pt         % Restore paragraphing parameters
%     \rightskip       :=L \@rightskip
%     \parfillskip     :=L 0pt + 1fil
%
%   NOTE: \@setpar called on every \list in case \par has been
%   temporarily  munged before the \list command.
%     \@setpar{if @newlist = false then {\@@par} fi}
%     \@newlist         :=G T
%     \@outerparskip    :=L \parskip
% END
%
% \trivlist  ==
% BEGIN
%  \parsep     := \parskip
%  @nmbrlist := F
%  \@trivlist
%  \labelwidth := 0
%  \leftmargin := 0
%  \itemindent := \parindent
%  \@itemlabel :=L "empty"          %% added 93/12/13
%  \makelabel{LABEL} == LABEL
% END
%
% \endtrivlist ==
%   BEGIN
%     if @inlabel = T then \indent fi
%     if horizontal mode then \unskip \par fi
%     if @noparlist = true
%       else if \lastskip > 0
%               then \@tempskipa := \lastskip
%                    \vskip - \lastskip
%                    \vskip \@tempskipa -\@outerparskip + \parskip
%            fi
%            \@endparenv
%     fi
%   END
%
% \@endparenv ==
%   BEGIN
%    \addpenalty{@endparpenalty}
%    \addvspace{\@topsepadd}
%    \endgroup    %% ends the \begin command's \begingroup
%    \par  ==  BEGIN
%                \@restorepar
%                \everypar{}
%                \par
%              END
%    \everypar == BEGIN remove \lastbox \everypar{} END
%    \begingroup  %% to match the \end commands \endgroup
%   END
%
% \item == BEGIN if math mode then WARNING fi
%                  if  next char = [
%                  then  \@item
%                  else  @noitemarg := true
%                        \@item[@itemlabel]
%          END
%
% \@item[LAB] ==
%    BEGIN
%     if @noparitem = true
%       then @noparitem := false
%                % NOTE: then clause  hardly every taken,
%                %  so made a macro \@donoparitem
%            \box\@labels :=G \hbox{\hskip -\leftmargin
%                                   \box\@labels
%                                   \hskip \leftmargin }
%            if @minipage = false then
%               \@tempskipa := \lastskip
%               \vskip -\lastskip
%               \vskip \@tempskipa + \@outerparskip - \parskip
%            fi
%       else if @inlabel = true
%              then \indent \par   % previous item empty.
%            fi
%            if hmode then 2 \unskip's
%                           % To remove any space at end of prev.
%                           % paragraph that could cause a blank line.
%                     \par
%            fi
%            if @newlist = T
%               then if @nobreak = T   % Kludge if list follows \section
%                      then \addvspace{\@outerparskip - \parskip}
%                      else \addpenalty{\@beginparpenalty}
%                           \addvspace{\@topsep}
%                           \addvspace{-\parskip}   %% added 4 Sep 85
%                    fi
%               else \addpenalty{\@itempenalty}
%                    \addvspace{\itemsep}
%            fi
%            @inlabel :=G true
%     fi
%
%     \everypar{ @minipage :=G F
%                @newlist :=G F
%                if  @inlabel = true
%                  then @inlabel :=G false
%                       \hskip -\parindent
%                       \box\@labels
%                       \penalty 0
%                             %% 3 Oct 85  -- allow line break here
%                       \box\@labels :=G null
%                fi
%                \everypar{} }
%     @nobreak :=G false
%     if  @noitemarg = true
%       then @noitemarg := false
%            if @nmbrlist
%              then \refstepcounter{\@listctr}
%     fi     fi
%     \@tempboxa   :=L \hbox{\makelabel{LAB}}
%     \box\@labels :=G \@labels \hskip \itemindent
%                       \hskip - (\labelwidth + \labelsep)
%                       if \wd \@tempboxa > \labelwidth
%                          then \box\@tempboxa
%                          else \hbox to \labelwidth {\unhbox\@tempboxa}
%                       fi
%                       \hskip\labelsep
%     \ignorespaces                        %gobble space up to text
%   END
%
%   \makelabel{LABEL} == ERROR   %% default to catch lonely \item
%
%
%   \usecounter{CTR} == BEGIN  @nmbrlist :=L true
%                              \@listctr == CTR
%                              \setcounter{CTR}{0}
%                       END
%
% DEFINE \dimen's and \count
% \end{oldcomments}
% \begin{macro}{\topsep}
% \begin{macro}{\partopsep}
% \begin{macro}{\itemsep}
% \begin{macro}{\parsep}
% \begin{macro}{\@topsep}
% \begin{macro}{\@topsepadd}
% \begin{macro}{\@outerparskip}
%    \begin{macrocode}
%<*2ekernel>
\newskip\topsep
\newskip\partopsep
\newskip\itemsep
\newskip\parsep
\newskip\@topsep
\newskip\@topsepadd
\newskip\@outerparskip
%    \end{macrocode}
% \end{macro}\end{macro}\end{macro}\end{macro}\end{macro}\end{macro}
% \end{macro}
% \begin{macro}{\leftmargin}\begin{macro}{\rightmargin}
% \begin{macro}{\listparindent}\begin{macro}{\itemindent}
% \begin{macro}{\labelwidth}\begin{macro}{\labelsep}
% \begin{macro}{\linewidth}
% \begin{macro}{\@totalleftmargin}
%    \begin{macrocode}
\newdimen\leftmargin
\newdimen\rightmargin
\newdimen\listparindent
\newdimen\itemindent
\newdimen\labelwidth
\newdimen\labelsep
\newdimen\linewidth
\newdimen\@totalleftmargin \@totalleftmargin=\z@
%    \end{macrocode}
% \end{macro}\end{macro}\end{macro}\end{macro}\end{macro}
% \end{macro}\end{macro}\end{macro}
%
% \begin{macro}{\leftmargini}
% \begin{macro}{\leftmarginii}
% \begin{macro}{\leftmarginiii}
% \begin{macro}{\leftmarginiv}
% \begin{macro}{\leftmarginv}
% \begin{macro}{\leftmarginvi}
%    \begin{macrocode}
\newdimen\leftmargini
\newdimen\leftmarginii
\newdimen\leftmarginiii
\newdimen\leftmarginiv
\newdimen\leftmarginv
\newdimen\leftmarginvi
%    \end{macrocode}
% \end{macro}\end{macro}\end{macro}
% \end{macro}\end{macro}\end{macro}
%
% \begin{macro}{\@listdepth}\begin{macro}{\@itempenalty}
% \begin{macro}{\@beginparpenalty}\begin{macro}{\@endparpenalty}
%    \begin{macrocode}
\newcount\@listdepth \@listdepth=0
\newcount\@itempenalty
\newcount\@beginparpenalty
\newcount\@endparpenalty
%    \end{macrocode}
% \end{macro}\end{macro}\end{macro}\end{macro}
%
% \begin{macro}{\@labels}
%    \begin{macrocode}
\newbox\@labels
%    \end{macrocode}
% \end{macro}
%
% \begin{macro}{\if@inlabel}
% \begin{macro}{\@inlabelfalse}
% \begin{macro}{\@inlabeltrue}
%    \begin{macrocode}
\newif\if@inlabel \@inlabelfalse
%    \end{macrocode}
% \end{macro}\end{macro}\end{macro}
%
% \begin{macro}{\if@newlist}
% \begin{macro}{\@newlistfalse}
% \begin{macro}{\@newlisttrue}
%    \begin{macrocode}
\newif\if@newlist   \@newlistfalse
%    \end{macrocode}
% \end{macro}\end{macro}\end{macro}
%
% \begin{macro}{\if@noparitem}
% \begin{macro}{\@noparitemfalse}
% \begin{macro}{\@noparitemtrue}
%    \begin{macrocode}
\newif\if@noparitem \@noparitemfalse
%    \end{macrocode}
% \end{macro}\end{macro}\end{macro}
%
% \begin{macro}{\if@noparlist}
% \begin{macro}{\@noparlistfalse}
% \begin{macro}{\@noparlisttrue}
%    \begin{macrocode}
\newif\if@noparlist \@noparlistfalse
%    \end{macrocode}
% \end{macro}\end{macro}\end{macro}
%
% \begin{macro}{\if@noitemarg}
% \begin{macro}{\@noitemargfalse}
% \begin{macro}{\@noitemargtrue}
%    \begin{macrocode}
\newif\if@noitemarg \@noitemargfalse
%    \end{macrocode}
% \end{macro}\end{macro}\end{macro}
%
% \begin{macro}{\if@newlist}
% \begin{macro}{\@newlistfalse}
% \begin{macro}{\@newlisttrue}
%    \begin{macrocode}
\newif\if@nmbrlist  \@nmbrlistfalse
%    \end{macrocode}
% \end{macro}\end{macro}\end{macro}
%
% \begin{environment}{list}
% \begin{macro}{\list}
%    \begin{macrocode}
\def\list#1#2{%
  \ifnum \@listdepth >5\relax
    \@toodeep
  \else
    \global\advance\@listdepth\@ne
  \fi
  \rightmargin\z@
  \listparindent\z@
  \itemindent\z@
  \csname @list\romannumeral\the\@listdepth\endcsname
  \def\@itemlabel{#1}%
  \let\makelabel\@mklab
  \@nmbrlistfalse
  #2\relax
  \@trivlist
  \parskip\parsep
  \parindent\listparindent
  \advance\linewidth -\rightmargin
  \advance\linewidth -\leftmargin
  \advance\@totalleftmargin \leftmargin
  \parshape \@ne \@totalleftmargin \linewidth
  \ignorespaces}
%    \end{macrocode}
% \end{macro}
% \end{environment}
%
% \begin{macro}{\par@deathcycles}
%    \begin{macrocode}
\newcount\par@deathcycles
%    \end{macrocode}
% \end{macro}
%
% \begin{macro}{\@trivlist}
% \changes{v1.0e}{1994/12/02}{RmS: Added check for looping}
% \changes{v1.0p}{1996/10/31}{Added check for missing item in outer
%                             list}
% \changes{v1.0q}{1996/11/04}{Moved check for missing item: only checked
%                 when not inlabel flag is false}
% Because |\par| is sometimes made a no-op it is possible for a missing
% |\item| to produce a loop that does not fill memory and so never gets
% trapped by \TeX.  We thus need to trap this here by setting |\par| to
% count the number of times a paragraph ii is called with no progress
% being made started.
%    \begin{macrocode}
\def\@trivlist{%
  \if@noskipsec \leavevmode \fi
  \@topsepadd \topsep
  \ifvmode
    \advance\@topsepadd \partopsep
  \else
    \unskip \par
  \fi
  \if@inlabel
    \@noparitemtrue
    \@noparlisttrue
  \else
    \if@newlist \@noitemerr \fi
    \@noparlistfalse
    \@topsep \@topsepadd
  \fi
  \advance\@topsep \parskip
  \leftskip \z@skip
  \rightskip \@rightskip
  \parfillskip \@flushglue
  \par@deathcycles \z@
  \@setpar{\if@newlist
             \advance\par@deathcycles \@ne
             \ifnum \par@deathcycles >\@m
               \@noitemerr
               {\@@par}%
             \fi
           \else
             {\@@par}%
           \fi}%
  \global \@newlisttrue
  \@outerparskip \parskip}
%    \end{macrocode}
% \end{macro}
%
% \begin{environment}{trivlist}
% \begin{macro}{\trivlist}
% \changes{0.0}{1992/03/18}{RmS: added \cs{@nmbrlistfalse}}
%    \begin{macrocode}
\def\trivlist{%
  \parsep\parskip
  \@nmbrlistfalse
  \@trivlist
  \labelwidth\z@
  \leftmargin\z@
  \itemindent\z@
%    \end{macrocode}
%
%    We initialise |\@itemlabel| so that a \texttt{trivlist} with
%    an |\item| not having an optional argument doesn't produce an
%    error message.
% \changes{latex2e}{1993/12/13}{Initialised \cs{@itemlabel}}
%    \begin{macrocode}
  \let\@itemlabel\@empty
  \def\makelabel##1{##1}}
%    \end{macrocode}
% \end{macro}
% \end{environment}
%
% \begin{macro}{\endlist}
%    \begin{macrocode}
\def\endlist{%
  \global\advance\@listdepth\m@ne
  \endtrivlist}
%    \end{macrocode}
% \end{macro}
%
%    The definition of \cs{trivlist} used to be in ltspace.dtx
%    so that other commands could be `let to it'.
%    They now use \cs{def}.
% \begin{macro}{\endtrivlist}
% \changes{v1.2b ltspace}{1994/11/12}{Changed order of tests to make
% \cs{@noitemerror} correct: end of an era.}
% \changes{v1.0i}{1995/05/25}{Macros moved from ltspace.dtx}
% \changes{v1.0n}{1996/10/25}{Change \cs{indent} to \cs{leavevmode}}
% \changes{v1.0n}{1996/10/25}{Reset flags explicitly}
% \changes{v1.0o}{1996/10/26}{Correct typo}
%    \begin{macrocode}
\def\endtrivlist{%
  \if@inlabel
    \leavevmode
    \global \@inlabelfalse
  \fi
  \if@newlist
    \@noitemerr
    \global \@newlistfalse
  \fi
  \ifhmode\unskip \par
%    \end{macrocode}
%    We also check if we are in math mode and issue an error message
%    if so (hoping that |\@currenvir| resolves suitably). Otherwise
%    the usual ``perhaps a missing item'' error will get triggered
%    later which is confusing.
% \changes{v1.0s}{2002/10/28}{Check for math mode (pr/3437)}
%    \begin{macrocode}
  \else
    \@inmatherr{\end{\@currenvir}}%
  \fi
  \if@noparlist \else
    \ifdim\lastskip >\z@
      \@tempskipa\lastskip \vskip -\lastskip
      \advance\@tempskipa\parskip \advance\@tempskipa -\@outerparskip
      \vskip\@tempskipa
    \fi
    \@endparenv
  \fi
}
%    \end{macrocode}
% \end{macro}
%
%
%
% \begin{macro}{\@endparenv}
% \begin{macro}{\@doendpe}
% To suppress the paragraph indentation in text immediately following
% a paragraph-making environment, \cs{everypar} is changed to remove the
% space, and \cs{par} is redefined to restore \cs{everypar}.  Instead of
% redefining \cs{par} and \cs{everypar}, \cs{@endparenv} was changed to
% set the @endpe switch, letting \cs{end} redefine \cs{par} and
% \cs{everypar}.
%
% This allows paragraph-making environments to work right when called
% by other environments. (Changed 27 Oct 86)
%
% In 2024 this logic was partially replaced with a new algorithm:
% \begin{itemize}
% \item
%   \cs{if@endpe} is now set globally to \texttt{true} or \texttt{false}.
% \item
%    In addition \cs{@endpetrue} initiates an \cs{aftergroup} call
%    to \cs{propagate@doendpe} if it is used inside a group.
% \item
%    \cs{propagate@doendpe} in turn checks the status of \cs{if@endpe}
%    and if that is \texttt{true} it calls \cs{@doendpe} otherwise it
%    does nothing.
% \item
%    \cs{@doendpe} in turn calls \cs{@endpetrue} and also makes the
%    necessary changes to
%    \cs{par} and \cs{everypar} so that they handle as before any
%    empty line that follows the environment.
% \item
%    Because of the \cs{@endpetrue} we get another \cs{aftergroup}, so the
%    mechanisnm slowly migrates out of several groups if those follow
%    immediately after the end of the environment. If, however, there is a new
%    paragraph started or an explicit \cs{par} before the next group ends then this will
%    result in a call to \cs{@endpefalse} and the migration stops
%    (note that \cs{propagate@doendpe} is still called once after the group but does nothing).
% \item
%    Using this approach something like
%\begin{verbatim}
% {%  some customization here
%  \begin{equation}
%    x=y
%  \end{equation}}
% some text
%\end{verbatim}
%    is still correctly identified as a paragraph continuation so that
%    there is no indentation before \texttt{some text}.
% \item
%    We can get away with using global settings of \cs{if@endpe} even
%    in nested situations (without keeping track of the status in a
%    stack), because the switch change is made only at
%    the very end of such environments, basically directly before the
%    \cs{endgroup} in \cs{end}, and it is later set back to \texttt{false} by
%    the next \cs{everypar} or the next \cs{par}. Even if the
%    environment is called without using \cs{begin} \cs{end}, the
%    situation doesn't change (or rather cause a problem).
% \item
%    However, there is one scenario where the new approach would
%    change the behavior. If a box is being built, e.g., with
%    \cs{setbox}, we have now the case that a \cs{@endpetrue}
%    inside would migrate out into a context in which it should not be
%    true (because the box might get used elsewhere, e.g., a float).
%    In the past, due to local switch changes, that didn't happen, i.e.,
%    \cs{if@endpe} would revert to \texttt{false} at the end of the box
%    definition.
% \item
%    Thus, to avoid that one has to explicitly set it back to false at
%    the end of such constructions, just as  we also need to prevent
%    colors from
%    migrating out. Thus the correct  place to do this is
%    in \cs{color@endgroup} because that is always called at that point.
% \end{itemize}
%    \begin{macrocode}
\def\@endparenv{%
  \addpenalty\@endparpenalty\addvspace\@topsepadd\@endpetrue}
%    \end{macrocode}
%
%    \begin{macrocode}
%<latexrelease>\IncludeInRelease{2015/01/01}{\@doendpe}{clubpenalty fix}%
\def\@doendpe{\@endpetrue
     \def\par{\@restorepar
%    \end{macrocode}
%    If a section heading changes |\clubpenalty| to keep lines
%    after it together then this modification is restored via the
%   |\everypar| mechanism at the start of the next paragraph. As we
%   destroy the contents of this token here we explicitly set
%   |\clubpenalty| back to its default.
% \changes{v1.0t}{2015/05/10}{Explicitly reset \cs{clubpenalty} before
%  clearing \cs{everypar}; see also pr/0462 and pr/4065}
% \changes{v1.0u}{2024/06/23}{Set \cs{if@endpe} to \texttt{false}
%    before calling \cs{par} (needed for tagging)}
%    \begin{macrocode}
              \clubpenalty\@clubpenalty
              \@endpefalse
              \everypar{}\par}%
%    \end{macrocode}
%
%    Use |\setbox0=\lastbox| instead of   |\hskip -\parindent|
%    so that a \cs{noindent} becomes a no-op when used before
%    a line immediately following a list environment(23 Oct 86).
% \changes{v1.0k}{1995/11/07}{Enclosed \cs{setbox0} assignment by a
% group so that it leaves the contents of box $0$ intact.}
%    \begin{macrocode}
     \everypar
         {{\setbox\z@\lastbox}%
          \everypar{}\@endpefalse}}
%<latexrelease>\EndIncludeInRelease
%    \end{macrocode}
%
%    \begin{macrocode}
%<latexrelease>\IncludeInRelease{0000/00/00}{\@doendpe}{clubpenalty fix}%
%<latexrelease>\def\@doendpe{\@endpetrue
%<latexrelease>    \def\par{\@restorepar\everypar{}\par\@endpefalse}\everypar
%<latexrelease>               {{\setbox\z@\lastbox}\everypar{}\@endpefalse}}
%<latexrelease>\EndIncludeInRelease
%    \end{macrocode}
% \end{macro}
% \end{macro}
%
%
% \begin{macro}{\if@endpe}
% \begin{macro}{\@endpefalse}
% \begin{macro}{\@endpetrue}
% \begin{macro}{\propagate@doendpe}
%    As outlined above these are no longer simple switches, but we
%    keep the name because it is used all over the place.
%    \begin{macrocode}
\newif\if@endpe
%    \end{macrocode}
%    
% \changes{v1.0u}{2024/06/23}{Set \cs{if@endpe} globally and also set
%    up migration to the outside}
%    \begin{macrocode}
%</2ekernel>
%<*2ekernel|latexrelease>
%<latexrelease>\IncludeInRelease{2024/11/01}%
%<latexrelease>                 {\@endpetrue}{New @endpe handling}%
%    \end{macrocode}
%
%    \begin{macrocode}
\def\@endpefalse{\global\let\if@endpe\iffalse}
%    \end{macrocode}
%
%    \begin{macrocode}
\def\@endpetrue {%
  \global\let\if@endpe\iftrue
%    \end{macrocode}
%    If we are inside a simple or a semi-simple group then propagate
%    to the outside, for all other group types do nothing. Normally, we
%    would start out in the group opened by \cs{begin} (type 14). When we
%    migrate out of that we are either on top-level (type 0) or in
%    another semi-simple group (type 14) or
%    in some other group. Thus, the best order of tests is to first
%    test for 14, then for 0 and finally for 1 (simple group).
% \changes{v1.0v}{2025/01/26}{Only migrate \cs{@doendpe} out of simple
%    and semi-simple groups (gh1641)}
%    \begin{macrocode}
  \ifnum\currentgrouptype =14        % semi-simple group
    \aftergroup\propagate@doendpe
  \else
    \ifnum\currentgrouptype =\z@     % no group: top-level
    \else
      \ifnum\currentgrouptype =\@ne  % simple group
        \aftergroup\propagate@doendpe
     \fi
    \fi
  \fi
}
%    \end{macrocode}
%    If \cs{if@endpe} is still true after the group ends, we run \cs{@doendpe}
%    that in turn runs another \cs{@endpetrue} (besides other things),
%    thus propagating further if necessary.
%    However, if the endpe situation got resolved and \cs{if@endpe} is \texttt{false}
%    then nothing further happens.
%    \begin{macrocode}
\def\propagate@doendpe{\if@endpe \@doendpe \fi}
%    \end{macrocode}
%    \begin{macrocode}
%</2ekernel|latexrelease>
%<latexrelease>\EndIncludeInRelease
%<latexrelease>\IncludeInRelease{0000/00/00}%
%<latexrelease>                 {\@endpetrue}{New @endpe handling}%
%<latexrelease>
%<latexrelease>
%<latexrelease>\def\@endpefalse{\let\if@endpe\iffalse}
%<latexrelease>\def\@endpetrue{\let\if@endpe\iftrue}
%<latexrelease>
%<latexrelease>\EndIncludeInRelease
%<*2ekernel>
%    \end{macrocode}
% \end{macro}\end{macro}\end{macro}\end{macro}
%
%
% \begin{macro}{\@mklab}
%    \begin{macrocode}
\def\@mklab#1{\hfil #1}
%    \end{macrocode}
% \end{macro}
%
% \begin{macro}{\item}
% \changes{LaTeX2.09}{1992/09/18}
%     {(RmS) Added warning if \cs{item} is used in math mode}
% \changes{v1.0c}{1994/04/28}
%     {Replaced \cs{@ltxnomath} by \cs{@inmatherr}}
% \changes{v1.0d}{1994/05/03}
%     {Removed superfluous braces}
%    \begin{macrocode}
\def\item{%
  \@inmatherr\item
  \@ifnextchar [\@item{\@noitemargtrue \@item[\@itemlabel]}}
%    \end{macrocode}
% \end{macro}
% \begin{macro}{\@donoparitem}
%    \begin{macrocode}
\def\@donoparitem{%
  \@noparitemfalse
  \global\setbox\@labels\hbox{\hskip -\leftmargin
                               \unhbox\@labels
                                \hskip \leftmargin}%
  \if@minipage\else
    \@tempskipa\lastskip
    \vskip -\lastskip
    \advance\@tempskipa\@outerparskip
    \advance\@tempskipa -\parskip
    \vskip\@tempskipa
  \fi}
%    \end{macrocode}
% \end{macro}
%
% \begin{macro}{\@item}
% \changes{v1.0l}{1996/07/26}{Remove unnecessary \cs{global} before
%                 \cs{@minipage...}}
%    \begin{macrocode}
\def\@item[#1]{%
  \if@noparitem
    \@donoparitem
  \else
    \if@inlabel
      \indent \par
    \fi
    \ifhmode
      \unskip\unskip \par
    \fi
    \if@newlist
      \if@nobreak
        \@nbitem
      \else
        \addpenalty\@beginparpenalty
        \addvspace\@topsep
        \addvspace{-\parskip}%
      \fi
    \else
      \addpenalty\@itempenalty
      \addvspace\itemsep
    \fi
    \global\@inlabeltrue
  \fi
  \everypar{%
    \@minipagefalse
    \global\@newlistfalse
%    \end{macrocode}
%    This |\if@inlabel| check is needed in case an item starts of
%    inside a group so that |\everypar| does not become empty
%    outside that group.
%    \begin{macrocode}
    \if@inlabel
      \global\@inlabelfalse
%    \end{macrocode}
%    The paragraph indent is now removed by using |\setbox...| since
%    this makes |\noindent| a no-op here, as it should be. Thus the
%    following comment is redundant but is left here for the sake of
%    future historians:
%    this next command was changed from an hskip to a kern to avoid
%    a break point after the parindent box: the skip could cause a
%    line-break if a very long label occurs in raggedright setting.
% \changes{v1.0d}{1994/05/03}{\cs{hskip} changed to \cs{kern}}
% \changes{v1.0m}{1996/10/23}{\cs{kern...} changed to \cs{setbox...}}
% \changes{v1.0r}{1997/02/21}
%    {\cs{ifvoid} check added for \cs{noindent}. latex/2414}
% If |\noindent| was used after |\item| want to cancel the |\itemindent|
% skip. This case can be detected as the indentation box will be void.
%    \begin{macrocode}
      {\setbox\z@\lastbox
       \ifvoid\z@
         \kern-\itemindent
       \fi}%
%    \end{macrocode}
%
%    \begin{macrocode}
      \box\@labels
      \penalty\z@
    \fi
%    \end{macrocode}
%    This code is intended to prevent a page break after the first
%    line of an item that comes immediately after a section title. It
%    may be sensible to always forbid a page break after one line of
%    an item?  As with all such settings of |\clubpenalty| it is local
%    so will have no effect if the item starts in a group.
%
%    Only resetting |\@nobreak| when it is true is now
%    essential since now it is sometimes set locally.
% \changes{v1.0m}{1996/10/23}{Added setting of \cs{clubpenalty} and
%    set \cs{@nobreakfalse} only when necessary}
%    \begin{macrocode}
    \if@nobreak
      \@nobreakfalse
      \clubpenalty \@M
    \else
      \clubpenalty \@clubpenalty
      \everypar{}%
    \fi}%
%    \end{macrocode}
% \changes{v1.0l}{1996/07/26}{Remove unnecessary \cs{global} before
%                 \cs{@nobreak...}}
% \changes{v1.0m}{1996/10/23}{\cs{@nobreak...} moved into the
%          \cs{everypar} and not executed unconditionally, see above}
%    \begin{macrocode}
  \if@noitemarg
    \@noitemargfalse
    \if@nmbrlist
%    \end{macrocode}
% \changes{v1.0g}{1995/05/17}{Removed surplus braces}
%    \begin{macrocode}
      \refstepcounter\@listctr
    \fi
  \fi
%    \end{macrocode}
%    We use |\sbox| to support colour commands.
% \changes{LaTeX2e}{1993/12/08}{use \cs{sbox} to support colour}
%    \begin{macrocode}
  \sbox\@tempboxa{\makelabel{#1}}%
  \global\setbox\@labels\hbox{%
    \unhbox\@labels
    \hskip \itemindent
    \hskip -\labelwidth
    \hskip -\labelsep
    \ifdim \wd\@tempboxa >\labelwidth
      \box\@tempboxa
%    \end{macrocode}
% \changes{LaTeX2.09}{1991/11/22}
%         {(RmS) Changed second call to \cs{makelabel} to
%           \cs{unhbox}\cs{@tempboxa}.
%          Avoids problems with side effects in \cs{makelabel} and is
%               more efficient.}
%    \begin{macrocode}
    \else
      \hbox to\labelwidth {\unhbox\@tempboxa}%
    \fi
    \hskip \labelsep}%
  \ignorespaces}
%    \end{macrocode}
% \end{macro}
%
% \begin{macro}{\makelabel}
% \changes{LaTeX2.09}{1991/11/04}
%         {(RmS) added default definition for \cs{makelabel},
%               to produce an error message.}
%    \begin{macrocode}
\def\makelabel#1{%
  \@latex@error{Lonely \string\item--perhaps a missing
        list environment}\@ehc}
%    \end{macrocode}
% \end{macro}
%
% \begin{macro}{\@nbitem}
% \changes{v1.0g}{1995/05/17}{Removed surplus braces}
%    \begin{macrocode}
\def\@nbitem{%
  \@tempskipa\@outerparskip
  \advance\@tempskipa -\parskip
  \addvspace\@tempskipa}
%    \end{macrocode}
% \end{macro}
%
% \begin{macro}{\usecounter}
%    \begin{macrocode}
\def\usecounter#1{\@nmbrlisttrue\def\@listctr{#1}\setcounter{#1}\z@}
%    \end{macrocode}
% \end{macro}
%
%
% \subsection{Itemize and Enumerate}
%
%  Enumeration is done with four counters: |enumi|, |enumii|, |enumiii|
%  and |enumiv|, where |enum|N controls the numbering of the Nth level
%  enumeration.  The label is generated by the commands
%  \cs{labelenumi} \ldots{} \cs{labelenumiv}, which should be defined
%  by the document style.
%  Note that \cs{p@enum}N\cs{theenum}N defines the output
%  of a \cs{ref} command.  A typical definition might be:
% \begin{verbatim}
%     \def\theenumii{\alph{enumii}}
%     \def\p@enumii{\theenumi}
%     \def\labelenumii{(\theenumii)}
% \end{verbatim}
% which will print the labels as `(a)', `(b)', \ldots
% and print a \cs{ref} as `3a'.
%
% The item numbers are moved to the right of the label box, so they are
% always a distance of \cs{labelsep} from the item.
%
% \cs{@enumdepth} holds the current enumeration nesting depth.
%
% Itemization is controlled by four commands: \cs{labelitemi},
% \cs{labelitemii},
% \cs{labelitemiii}, and \cs{labelitemiv}.
% To cause the second-level list to be
% bulleted, you just define \cs{labelitemii}
% to be $\bullet$.  \cs{@itemspacing}
% and \cs{@itemdepth} are the analogs of \cs{@enumspacing} and
% \cs{@enumdepth}.
%
% \begin{oldcomments}
% \enumerate ==
%   BEGIN
%     if \@enumdepth > 3
%       then errormessage: ``Too deeply nested''.
%       else \@enumdepth :=L \@enumdepth + 1
%            \@enumctr :=L eval(enum@\romannumeral\the\@enumdepth)
%            \list{\label(\@enumctr)}
%                 {\usecounter{\@enumctr}
%                  \makelabel{LABEL} ==  \hss \llap{LABEL}}
%     fi
%   END
%
% \endenumerate == \endlist
% \end{oldcomments}
%
% \begin{macro}{\@enumdepth}
%    \begin{macrocode}
\newcount\@enumdepth \@enumdepth = 0
%    \end{macrocode}
% \end{macro}
%
% \begin{macro}{\c@enumi}
% \begin{macro}{\c@enumii}
% \begin{macro}{\c@enumii}
% \begin{macro}{\c@enumiv}
%    \begin{macrocode}
\@definecounter{enumi}
\@definecounter{enumii}
\@definecounter{enumiii}
\@definecounter{enumiv}
%    \end{macrocode}
% \end{macro}
% \end{macro}
% \end{macro}
% \end{macro}
%
% \begin{environment}{enumerate}
% \changes{v1.0g}{1995/05/17}{Use \cs{thr@@} and remove surplus braces}
%    \begin{macrocode}
\def\enumerate{%
  \ifnum \@enumdepth >\thr@@\@toodeep\else
    \advance\@enumdepth\@ne
    \edef\@enumctr{enum\romannumeral\the\@enumdepth}%
%    \end{macrocode}
%
% \changes{v1.0j}{1995/07/09}{Use \cs{expandafter}}
%    \begin{macrocode}
      \expandafter
      \list
        \csname label\@enumctr\endcsname
        {\usecounter\@enumctr\def\makelabel##1{\hss\llap{##1}}}%
  \fi}
%    \end{macrocode}
%
%    \begin{macrocode}
\let\endenumerate =\endlist
%    \end{macrocode}
% \end{environment}
%
%
% \begin{oldcomments}
%  \itemize ==
%    BEGIN
%      if \@itemdepth > 3
%        then  errormessage: 'Too deeply nested'.
%        else \@itemdepth :=L \@itemdepth + 1
%             \@itemitem  == eval(labelitem\romannumeral\the\@itemdepth)
%             \list{\@nameuse{\@itemitem}}
%                   {\makelabel{LABEL} ==  \hss \llap{LABEL}}
%      fi
%    END
%
%  \enditemize ==  \endlist
%
% \end{oldcomments}
%
% \begin{macro}{\@itemdepth}
%    \begin{macrocode}
\newcount\@itemdepth \@itemdepth = 0
%    \end{macrocode}
% \end{macro}
%
% \begin{environment}{itemize}
% \changes{v1.0g}{1995/05/17}{Use \cs{thr@@}}
%    \begin{macrocode}
\def\itemize{%
  \ifnum \@itemdepth >\thr@@\@toodeep\else
    \advance\@itemdepth\@ne
    \edef\@itemitem{labelitem\romannumeral\the\@itemdepth}%
%    \end{macrocode}
%
% \changes{v1.0j}{1995/07/09}{Use \cs{expandafter}}
%    \begin{macrocode}
    \expandafter
    \list
      \csname\@itemitem\endcsname
      {\def\makelabel##1{\hss\llap{##1}}}%
  \fi}
%    \end{macrocode}
%
%    \begin{macrocode}
\let\enditemize =\endlist
%</2ekernel>
%    \end{macrocode}
% \end{environment}
%
% \Finale
%
