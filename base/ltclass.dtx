% \iffalse meta-comment
%
% Copyright (C) 1993-2025
% The LaTeX Project and any individual authors listed elsewhere
% in this file.
%
% This file is part of the LaTeX base system.
% -------------------------------------------
%
% It may be distributed and/or modified under the
% conditions of the LaTeX Project Public License, either version 1.3c
% of this license or (at your option) any later version.
% The latest version of this license is in
%    https://www.latex-project.org/lppl.txt
% and version 1.3c or later is part of all distributions of LaTeX
% version 2008 or later.
%
% This file has the LPPL maintenance status "maintained".
%
% The list of all files belonging to the LaTeX base distribution is
% given in the file `manifest.txt'. See also `legal.txt' for additional
% information.
%
% The list of derived (unpacked) files belonging to the distribution
% and covered by LPPL is defined by the unpacking scripts (with
% extension .ins) which are part of the distribution.
%
% \fi
%
% \iffalse
%%% From File: ltclass.dtx
%
%<*driver>
% \fi
\ProvidesFile{ltclass.dtx}
             [2025/05/10 v1.5m LaTeX Kernel (Class & Package Interface)]
% \iffalse
\documentclass{ltxdoc}
\GetFileInfo{ltclass.dtx}
\begin{document}
\title{The main structure of documents}
\author{Frank Mittelbach\and Chris Rowley\and Alan Jeffrey\and
        David Carlisle}
\date{\filedate}
 \MaintainedByLaTeXTeam{latex}
 \maketitle

 \providecommand\pkg[1]{\texttt{#1}}

 \DocInput{\filename}
\end{document}
%</driver>
% \fi
%
% \iffalse
% (C) Copyright Frank Mittelbach, Chris Rowley,
%               Alan Jeffrey and David Carlisle 1993-1998.
% All rights reserved.
% \fi
%
%
% \changes{v0.2f}{1993/11/22}
%         {\cs{@unknownversion} removed}
% \changes{v1.0j}{1994/10/18}
%         {Move \cs{listfiles} to ltfiles.dtx}
% \changes{v1.0f}{1994/05/22}{Use new warning and error commands}
% \changes{v1.0l}{1994/11/17}{\cs{@tempa} to \cs{reserved@a}}
% \changes{v1.0z}{1998/03/21}{Added to documentation of filecontents}
% \changes{v1.1c}{1998/08/17}{(RmS) Minor documentation fixes.}
% \changes{v1.3o}{2020/08/21}{Integration of new hook management interface}
% \changes{v1.4f}{2021/08/25}{Standardise generic hook names (gh/648)}
%
%
% \section{Introduction}
%
% This file implements the following declarations, which replace
% |\documentstyle| in  \LaTeXe\ documents.
%
% Note that old documents containing |\documentstyle| will be run using
% a compatibility option---thus keeping everyone happy, we hope!
%
% The overall idea is that there are two types of `style files':
% `class files' which define elements and provide a default formatting
% for them;  and `packages' which provide extra functionality.  One
% difference between \LaTeXe\ and \LaTeX2.09 is that \LaTeXe\ packages
% may have options. Note that options to classes packages may be
% implemented such that they input files, but these file names are not
% necessarily directly related to the option name.
%
% \section{User interface}
%
% |\documentclass[|\meta{main-option-list}|]{|^^A
%   \meta{class}|}[|\meta{version}|]|
%
% There must be exactly one such declaration, and it must come first.
% The \meta{main-option-list} is a list of options which can modify the
% formatting of elements which are defined in the \meta{class} file
% as well as in all following |\usepackage| declarations (see below).
% The \meta{version} is a version number, beginning with a date in the
% format |YYYY/MM/DD|.  If an older version of the class is found, a
% warning is issued.
%
% \bigskip
%
% |\documentstyle[|\meta{main-option-list}|]{|^^A
%   \meta{class}|}[|\meta{version}|]|
%
% The |\documentstyle| declaration is kept in order to maintain upward
% compatibility with \LaTeX2.09 documents.  It is similar to
% |\documentclass|, but it causes all options in
% \meta{main-option-list} that the \meta{class} does not use to be
% passed to |\RequirePackage| after the options have been processed.
% This maintains compatibility with the 2.09 behaviour. Also a flag is
% set to indicate that the document is to be processed in \LaTeX2.09
% compatibility mode.  As far as most packages are concerned, this
% only affects the warnings and errors \LaTeX\ generates. This flag
% does affect the definition of font commands, and |\sloppy|.
%
% \bigskip
%
% |\usepackage[|\meta{package-option-list}|]{|^^A
%    \meta{package-list}|}[|\meta{version}|]|
%
% There can be any number of these declarations. All packages in
% \meta{package-list} are called with the same options.
%
% Each \meta{package} file defines new elements (or modifies those
% defined in the \meta{class}), and thus extends the range of documents
% which can be processed.
% The \meta{package-option-list} is a list of options which can modify
% the formatting of elements defined in the \meta{package} file.
% The \meta{version} is a version number, beginning with a date in the
% format |YYYY/MM/DD|.  If an older version of the package is found, a
% warning is issued.
%
% Each package is loaded only once.  If the same package is requested
% more than once, nothing happens, unless the package has been requested
% with options that were not given the first time it was loaded, in
% which case an error is produced.
%
% As well as processing the options given in the
% \meta{package-option-list}, each package processes the
% \meta{main-option-list}.  This means that options that affect all
% of the packages can be given globally, rather than repeated for every
% package.
%
% Note that class files have the extension |.cls|, packages have the
% extension |.sty|.
%
% \DescribeEnv{filecontents}
% The environment |filecontents| is intended for passing the contents
% of packages, options, or other files along with a document in a
% single file.
% It has one argument, which is the name of the file to create. If that
% file already exists (maybe only in the current directory if the OS
% supports a notion of a `current directory' or `default directory')
% then nothing happens
% (except for an information message) and the body of the environment
% is bypassed. Otherwise, the body of the environment is written
% verbatim to the file name given as the first argument, together with
% some comments about how it was produced.
%
% The environment can also be called with an optional argument which is
% used to alter some of its behavior: option \texttt{force} or
% \texttt{overwrite} will allow for overwriting existing files,
% option \texttt{nosearch} will only check the current directory
% when looking if the file exists. This can be useful if you want to
% generate a local (modified) copy of some file that is already in the
% search tree of \TeX{}. Finally, you can use \texttt{noheader} to
% prevent it from writing the standard blurb at the top of the file
% (this is actually the same as using the star form of the environment).
%
% The environment is now allowed anywhere in the document, but to ensure
% that all packages or options necessary are available when the
% document is run, it is normally best to place it at the top of your
% file (before \cs{documentclass}).
% A possible use case for using it inside the document body is if you
% want to reuse some text several times in the document you could then
% write it and later use \cs{input} to retrieve it where needed.
%
%
% The begin and end tags should each be on a
% line by itself.
%
% \subsection{Option processing}
%
% When the options are processed, they are divided into two types: {\em
% local\/} and {\em global}:
% \begin{itemize}
%
% \item For a class, the options in the |\documentclass| command are
%    local.
%
% \item For a package, the options in the |\usepackage| command are
%    local, and the options in the |\documentclass| command are global.
%
% \end{itemize}
% The options for |\documentclass| and |\usepackage|
% are processed in the following way:
% \begin{enumerate}
%
% \item The local and global options that have been declared
%   (using |\DeclareOption| as  described below) are processed
%   first.
%
%  In the case of |\ProcessOptions|, they are processed in the order
%  that they were declared in the class or package.
%
%  In the case of |\ProcessOptions*|, they are processed in the order
%  that they appear in the option-lists. First the global options, and
%  then the local ones.
%
% \item Any remaining local options are dealt with using the default
%   option (declared using the |\DeclareOption*| declaration described
%   below).  For document classes, this usually does nothing, but
%   records the option on a list of unused options.
%   For packages, this usually produces an error.
%
% \end{enumerate}
% Finally, when |\begin{document}| is reached, if there are any global
% options which have not been used by either the class or any package,
% the system will produce a warning.
%
%
% \section{Class and Package interface}
%
% \subsection{Class name and version}
%
% \DescribeMacro\ProvidesClass
% A class can identify itself with the
% |\ProvidesClass{|\meta{name}|}[|\meta{version}|]| command.  The
% \meta{version} should begin with a date in the format |YYYY/MM/DD|.
%
% \subsection{Package name and version}
%
% \DescribeMacro\ProvidesPackage
% A package can identify itself with the
% |\ProvidesPackage|\marg{name}\oarg{version} command.  The
% \meta{version} should begin with a date in the format |YYYY/MM/DD|.
%
% \subsection{Requiring other packages}
%
% \DescribeMacro\RequirePackage
% Packages or classes can load other packages using\\
% |\RequirePackage|\oarg{options}\marg{name}\oarg{version}.\\
% If the package has already been loaded, then nothing happens unless
% the requested options are not a subset of the options with which it
% was loaded, in which case an error is called.
%
% \DescribeMacro\LoadClass
%  Similar to |\RequirePackage|, but for classes, may not be used in
%  package files.
%
% \DescribeMacro\PassOptionsToPackage
% Packages can pass options to other packages using:\\
% |\PassOptionsToPackage{|\meta{options}|}{|\meta{package}|}|.\\
% \DescribeMacro\PassOptionsToClass
% This adds the \meta{options} to the options list of any future
% |\RequirePackage| or |\usepackage| command.  For example:
% \begin{verbatim}
%    \PassOptionsToPackage{foo,bar}{fred}
%    \RequirePackage[baz]{fred}\end{verbatim}
% is the same as:
%\begin{verbatim}
%    \RequirePackage[foo,bar,baz]{fred}
%\end{verbatim}
%
% \DescribeMacro\LoadClassWithOptions
% |\LoadClassWithOptions|\marg{name}\oarg{version}:\\
% This is similar to
% |\LoadClass|, but it always calls class \meta{name} with
% exactly the same option list that is being used by the current class,
% rather than an option explicitly  supplied or passed on by
% |\PassOptionsToClass|.
% \DescribeMacro\RequirePackageWithOptions
% |\RequirePackageWithOptions| is the analogous command for packages.
%
% This is mainly intended to allow one class to simply build on another,
% for example:
%\begin{verbatim}
%   \LoadClassWithOptions{article}
%\end{verbatim}
%
% This should be contrasted with the slightly different construction
%\begin{verbatim}
%   \DeclareOption*{\PassOptionsToClass{\CurrentOption}{article}}
%   \ProcessOptions
%   \LoadClass{article}
%\end{verbatim}
%
% As used here, the effects are more or less the same, but the
% version using |\LoadClassWithOptions| is slightly quicker
% (and less to type).
% If, however, the class declares options of its own then
% the two constructions are different; compare, for example:
%\begin{verbatim}
%   \DeclareOption{landscape}{...}
%   \ProcessOptions
%   \LoadClassWithOptions{article}
%\end{verbatim}
% with:
%\begin{verbatim}
%   \DeclareOption{landscape}{...}
%   \DeclareOption*{\PassOptionsToClass{\CurrentOption}{article}}
%   \ProcessOptions
%   \LoadClass{article}
%\end{verbatim}
% In the first case, the \textsf{article} class will be called with
% option |landscape| precisely when the current class is called with
% this option; but in the second example it will
% not as in that case \textsf{article} is only passed options by the
% default option handler, which is not used for |landscape| as that
% option is explicitly declared.
%
% \DescribeMacro\IfPackageLoadedTF
% \DescribeMacro\IfClassLoadedTF
% \DescribeMacro\@ifpackageloaded
% \DescribeMacro\@ifclassloaded
% To find out if a package has already been loaded, use
% \begin{quote}
% |\IfPackageLoadedTF{|\meta{package}|}{|\meta{true}|}{|\meta{false}|}|\\
% \end{quote}
% or the old name \cs{@ifpackageloaded}.
%
% \DescribeMacro\IfPackageAtLeastTF
% \DescribeMacro\IfClassAtLeastTF
% \DescribeMacro\IfFileAtLeastTF
% \DescribeMacro\@ifpackagelater
% \DescribeMacro\@ifclasslater
% \changes{v1.1i}{2013/07/07}{Correctly describe how the date in
%       \cs{@ifpackagelater} is used}
% To find out if a package has already been loaded with a version
% equal to or more
% recent than \meta{date}, use
% \begin{quote}
% |\IfPackageAtLeastTF{|\meta{package}|}{|\meta{date}|}{|^^A
% \meta{true}|}{|\meta{false}|}|
% \end{quote}
% or the old name \cs{@ifpackagelater}.

% \DescribeMacro\IfFormatAtLeastTF
% To test the format date use
% \begin{quote}
% |\IfFormatAtLeastTF{|\meta{date}|}{|^^A
% \meta{true}|}{|\meta{false}|}|
% \end{quote}
%
% \DescribeMacro\IfPackageLoadedWithOptionsTF
% \DescribeMacro\IfClassLoadedWithOptionsTF
% \DescribeMacro\@ifpackagewith
% \DescribeMacro\@ifclasswith
% To find out if a package has already been loaded with at least the
% options \meta{options}, use
% \begin{quote}
% |\IfPackageLoadedWithOptionsTF{|\meta{package}|}{|\meta{options}|}{|^^A
% \meta{true}|}{|\meta{false}|}|
% \end{quote}
% or the old name \cs{@ifpackagewith}.
%
% There exists one package that can't be tested with the above
% commands: the \texttt{fontenc} package pretends that it was never
% loaded to allow for repeated reloading with different options (see
% \texttt{ltoutenc.dtx} for details).
%
%
% \subsection{Declaring new options}
%
% Options for classes and packages are built using the same macros.
%
% \DescribeMacro\DeclareOption To define a builtin option, use
% |\DeclareOption{|\meta{name}|}{|\meta{code}|}|.
%
% \DescribeMacro{\DeclareOption*} To define the default action to
% perform for local options which have not been declared, use
% |\DeclareOption*{|\meta{code}|}|.
%
% {\em Note\/}: there should be no use of\\
%  |\RequirePackage|, |\DeclareOption|, |\DeclareOption*| or
%   |\ProcessOptions|\\
% inside |\DeclareOption| or |\DeclareOption*|.
%
% Possible uses for |\DeclareOption*| include:
%
% |\DeclareOption*{}|\\
%    Do nothing. Silently accept unknown options. (This suppresses the
%    usual warnings.)
%
% |\DeclareOption*{\@unkownoptionerror}|\\
%     Complain about unknown local options. (The initial setting for
%       package files.)
%
% |\DeclareOption*{\PassOptionsToPackage{\CurrentOption}|^^A
%                                     |{|\meta{pkg-name}|}}|\\
% Handle the current option by passing it on to the package
% \meta{pkg-name}, which will presumably be loaded via
% |\RequirePackage| later in the file. This is useful for building
% `extension' packages, that perhaps handle a couple of new options,
% but then pass everything else on to an existing package.
%
% |\DeclareOption*{\InputIfFileExists{xx-\CurrentOption.yyy}%|\\
% |               {}%|\\
% |               {\OptionNotUsed}}|\\
%  Handle the option foo by loading the file |xx-foo.yyy| if it
%  exists, otherwise do nothing, but declare that the option was not
%  used.
%  Actually the |\OptionNotUsed| declaration is only needed if this is
%  being used in class files, but does no harm in package files.
%
%
% \subsection{Safe Input Macros}
% \DescribeMacro{\InputIfFileExists}
%  |\InputIfFileExists{|\meta{file}|}{|\meta{then}|}{|\meta{else}|}|\\
% Inputs \meta{file} if it exists. Immediately before the input,
% \meta{then} is executed. Otherwise \meta{else} is executed.
%
% \DescribeMacro{\IfFileExists}
% As above, but does not input the file.
%
% One thing you might like to put in the \meta{else} clause is
%
% \DescribeMacro{\@missingfileerror}
% This starts an interactive request for a filename, supplying default
% extensions. Just hitting return causes the whole input to be skipped
% and entering |x| quits the current run,
%
% \DescribeMacro{\input}
% This has been redefined from the \LaTeX2.09 definition, in terms of
% the new commands |\InputIfFileExists| and |\@missingfileerror|.
%
%
% \DescribeMacro{\listfiles} Giving this declaration in the preamble
% causes a list of all files input via the `safe input' commands to be
% listed at the end. Any strings specified in the optional argument to
% |\ProvidesPackage| are listed alongside the file name. So files in
% standard (and other non-standard) distributions can put informative
% strings in this argument.
%
% \MaybeStop{}
%
% \section{Implementation}
%
%    \begin{macrocode}
%<*2ekernel>
%    \end{macrocode}
%
%
% \changes{v0.2g}{1993/11/23}
%         {Various macros now moved to latex.tex.}
% \changes{v0.2g}{1993/11/23}
%         {Warnings and errors now directly coded.}
% \changes{v0.2h}{1993/11/28}
%         {Primitive filenames now terminated by space not \cs{relax}.}
% \changes{v0.2h}{1993/11/28}
%         {Directory syntax checking moved to dircheck.dtx}
% \changes{v0.2h}{1993/11/28}
%         {Assorted commands now in the kernel removed.}
% \changes{v0.2i}{1993/12/03}
%         {\cs{@onlypreamble}: Many commands declared.}
% \changes{v0.2i}{1993/12/03}
%         {Removed obsolete \cs{@documentclass}}
% \changes{v0.2o}{1993/12/13}
%         {Removed setting \cs{errorcontextlines}\ (now in latex.tex)}
% \changes{v0.2p}{1993/12/15}
%         {Removed extra `.'s from \cs{@@warning}s}
% \changes{v0.2s}{1994/01/17}
%         {Added many more \cs{@onlypreamble} commands}
% \changes{v0.2s}{1994/01/17}
%         {Wrapped long lines to column 72}
% \changes{v0.3a}{1994/03/02}
%         {Remove need for driver file}
% \changes{v0.3b}{1994/03/08}
%         {Modify driver code into `new style'}
% \changes{v0.3c}{1994/03/12}
%         {Change name from docclass to ltclass}
% \changes{v0.3h}{1994/04/25}
%         {Removed spurious extra `.'s at the end of error messages}
% \changes{v1.0a}{1994/04/29}
%         {Change version number to 1 (no other change)}
% \changes{v1.0k}{1994/11/03}
%         {Move \cs{@missingfileerror} to ltfiles}
%
% \begin{macro}{\if@compatibility}
%    The flag for compatibility mode.
%    \begin{macrocode}
\newif\if@compatibility
%    \end{macrocode}
% \end{macro}
%
% \begin{macro}{\@documentclasshook}
%    This legacy hook is called after the first |\documentclass| command.
%    It is \emph{not} integrated with the new 2020 hook management system!
%    By
%    default this checks to see if |\@normalsize| is undefined, and if
%    so, sets it to |\normalsize|.
% \changes{v0.2q}{1993/12/17}
%         {Macro added}
% \changes{v0.2z}{1994/02/10}
%         {Changed the name from \cs{@compatibility} to
%          \cs{@documentclasshook}, and added the check for whether
%          \cs{@normalsize} has been defined.  ASAJ.}
%    \begin{macrocode}
\def\@documentclasshook{%
   \ifx\@normalsize\@undefined
      \let\@normalsize\normalsize
   \fi
}
%    \end{macrocode}
% \end{macro}
%
%  \begin{macro}{\@declaredoptions}
%    This list is automatically built by |\DeclareOption|.
%    It is the list of options (separated by commas) declared in
%    the class or package file and it defines the order in which
%    the corresponding |\ds@|\meta{option} commands are executed.
%    All local \meta{option}s which are not declared will be processed
%    in the order defined by the optional argument of |\documentclass|
%    or |\usepackage|.
%    \begin{macrocode}
\let\@declaredoptions\@empty
%    \end{macrocode}
%  \end{macro}
%
%  \begin{macro}{\@classoptionslist}
%    List of options of the main class.
% \changes{v1.0u}{1996/07/26}{made only preamble}
% \changes{v1.4e}{2021/07/19}{Drop \cs{@onlypreamble}}
%    \begin{macrocode}
\let\@classoptionslist\relax
%\@onlypreamble\@classoptionslist
%    \end{macrocode}
%  \end{macro}
%
%  \begin{macro}{\@raw@classoptionslist}
%    List of options of the main class (unprocessed).
% \changes{v1.4b}{2021/05/18}{Initialise to \cs{relax} to match \cs{@classoptionslist}}
%    \begin{macrocode}
\let\@raw@classoptionslist\relax
%    \end{macrocode}
%  \end{macro}
%
%  \begin{macro}{\@unusedoptionlist}
% \changes{v1.0u}{1996/07/26}{made only preamble}
%    List of options of the main class that haven't been declared or
%    loaded as class option files.
% \changes{v1.4e}{2021/07/19}{Drop \cs{@onlypreamble}}
%    \begin{macrocode}
\let\@unusedoptionlist\@empty
%\@onlypreamble\@unusedoptionlist
%    \end{macrocode}
%  \end{macro}
%
%  \begin{macro}{\CurrentOption}
%    Name of current package or option.
% \changes{v0.2c}{1993/11/17}
%         {Name changed from \cs{@curroption}}
%    \begin{macrocode}
\let\CurrentOption\@empty
%    \end{macrocode}
%  \end{macro}
%
% \begin{macro}{\@currpath}
%   Path to the current file if explicitly given.
%   \changes{v1.3u}{2020/11/20}{Macro added}
%    \begin{macrocode}
%</2ekernel>
%<*2ekernel|latexrelease>
%<latexrelease>
%<latexrelease>\IncludeInRelease{2020/10/01}{\@currpath}%
%<latexrelease>  {Add \@currpath}%
\let\@currpath\@empty
%<latexrelease>\EndIncludeInRelease
%
%<latexrelease>\IncludeInRelease{0000/00/00}{\@currpath}%
%<latexrelease>  {Add \@currpath}%
%<latexrelease>\let\@currpath\@undefined
%<latexrelease>\EndIncludeInRelease
%</2ekernel|latexrelease>
%<*2ekernel>
%    \end{macrocode}
% \end{macro}
%
%  \begin{macro}{\@currname}
%    Name of current package or option.
%    \begin{macrocode}
\let\@currname\@empty
%    \end{macrocode}
%  \end{macro}
%
% \begin{macro}{\@currext}
%    The current file extension.
% \changes{v0.2a}{1993/11/14}{Name changed from \cs{@currextension}}
%    \begin{macrocode}
\global\let\@currext=\@empty
%    \end{macrocode}
% \end{macro}
%
% \begin{macro}{\@clsextension}
% \begin{macro}{\@pkgextension}
%    The two possible values of |\@currext|.
% \changes{v1.4e}{2021/07/19}{Drop \cs{@onlypreamble}}
%    \begin{macrocode}
\def\@clsextension{cls}
\def\@pkgextension{sty}
%    \end{macrocode}
% \end{macro}
% \end{macro}
%
% \begin{macro}{\@pushfilename}
% \begin{macro}{\@popfilename}
% \begin{macro}{\@currnamestack}
% Commands to push and pop the file name and extension. \\
% |#1| current name.      \\
% |#2| current extension. \\
% |#3| current catcode of |@|. \\
% |#4| Rest of the stack.
% \changes{v1.3l}{2020/06/05}{Added \cs{@expl@push@filename@@}
%          and \cs{@expl@push@filename@aux@@}}
% \changes{v1.3s}{2020/10/08}{Added missing 2020/02/02 \cs{IncludeInRelease}}
% \changes{v1.3v}{2020/12/14}{Removed \cs{@expl@@@hook@curr@name@push@@n}}
%    \begin{macrocode}
%</2ekernel>
%<*2ekernel|latexrelease>
%<latexrelease>
%<latexrelease>\IncludeInRelease{2020/10/01}{\@pushfilename}%
%<latexrelease>  {Add \@expl@push@filename@@ and \@expl@push@filename@aux@@}%
\def\@pushfilename{%
%    \end{macrocode}
%   The push and pop macros are injected in \cs{@pushfilename} and
%   \cs{@popfilename} so that they correctly keep track of the hook
%   labels.
%
%   This needs cleanup with the \pkg{expl3} interfaces also playing
%   here, e.g., \cs{@expl@push@filename@@} needs cleanup and (and
%   should probably not have this name either).
%    \begin{macrocode}
  \@expl@push@filename@@
  \xdef\@currnamestack{%
    {\@currname}%
    {\@currext}%
    {\the\catcode`\@}%
    \@currnamestack}%
%    \end{macrocode}
%   Temporarily add a stack for \cs{@currpath} here.  This should be
%   integrated in the main file stack eventually, but other packages
%   rely on \cs{@currnamestack} having three elements per file, so that
%   isn't a trivial change.  The prefix \cs{@kernel@...} hopefully
%   discourages people from using it.
%    \begin{macrocode}
  \xdef\@kernel@currpathstack{%
    {\@currpath}%
    \@kernel@currpathstack}%
  \@expl@push@filename@aux@@}
%<latexrelease>\EndIncludeInRelease
%    \end{macrocode}
%
%   The following version of \cs{@pushfilename} didn't formally exist in
%   this file, but in the 2020/02/02 release, \pkg{expl3} was preloaded
%   and it patched \cs{@pushfilename} (and \cs{@popfilename}) by adding
%   some hooks in there.  But rolling back to 2020/02/02, \pkg{expl3}
%   doesn't patch these macros again, so rolling back has to take those
%   hooks into account.  Same goes for \cs{@popfilename}.
%    \begin{macrocode}
%<latexrelease>
%<latexrelease>\IncludeInRelease{2020/02/02}{\@pushfilename}%
%<latexrelease>  {Add \@expl@push@filename@@}%
%<latexrelease>\def\@pushfilename{%
%<latexrelease>  \@expl@push@filename@@
%<latexrelease>  \xdef\@currnamestack{%
%<latexrelease>    {\@currname}%
%<latexrelease>    {\@currext}%
%<latexrelease>    {\the\catcode`\@}%
%<latexrelease>    \@currnamestack}%
%<latexrelease>    \@expl@push@filename@aux@@}
%<latexrelease>\EndIncludeInRelease
%<latexrelease>
%    \end{macrocode}
%
% When we roll back from a release that has \pkg{expl3} preloaded, the
% definitions of \cs{@pushfilename} and \cs{@popfilename} can't be
% completely rolled back otherwise \pkg{expl3}-based packages won't
% have the automatic \cs{ExplSyntaxOff} at the end.  Here and below for
% \cs{@popfilename}, we don't roll back all the way through if coming
% from \LaTeX${}>2020-02-02$.
% \changes{v1.4a}{2021/03/27}
%         {Do not completely roll back if \pkg{expl3} is loaded.}
%    \begin{macrocode}
%<latexrelease>\IncludeInRelease{0000/00/00}{\@pushfilename}%
%<latexrelease>  {Add \@expl@push@filename@@ and \@expl@push@filename@aux@@}%
%<latexrelease>\ifnum\sourceLaTeXdate<20200202\relax
%<latexrelease>  \GenericInfo{}{Defining 00-00-00\string\@pushfilename.}
%<latexrelease>\def\@pushfilename{%
%<latexrelease>  \xdef\@currnamestack{%
%<latexrelease>    {\@currname}%
%<latexrelease>    {\@currext}%
%<latexrelease>    {\the\catcode`\@}%
%<latexrelease>    \@currnamestack}}
%<latexrelease>\else
%<latexrelease>  \GenericInfo{}{Defining 2020-02-02\string\@pushfilename.}
%<latexrelease>\def\@pushfilename{%
%<latexrelease>  \@expl@push@filename@@
%<latexrelease>  \xdef\@currnamestack{%
%<latexrelease>    {\@currname}%
%<latexrelease>    {\@currext}%
%<latexrelease>    {\the\catcode`\@}%
%<latexrelease>    \@currnamestack}%
%<latexrelease>    \@expl@push@filename@aux@@}
%<latexrelease>\fi
%<latexrelease>\EndIncludeInRelease
\@onlypreamble\@pushfilename
%    \end{macrocode}
%
%
%
%
%
% \changes{v1.3l}{2020/06/05}{Added \cs{@expl@pop@filename@@}}
%    \begin{macrocode}
%<latexrelease>
%<latexrelease>\IncludeInRelease{2020/10/01}{\@popfilename}%
%<latexrelease>  {Add \@expl@pop@filename@@}%
\def\@popfilename{\@expl@@@hook@curr@name@pop@@
  \expandafter\@p@pfilename\@currnamestack\@nil
%    \end{macrocode}
%   Same for popping:
%    \begin{macrocode}
  \expandafter\@p@pfilepath\@kernel@currpathstack\@nil
  \@expl@pop@filename@@}
%<latexrelease>\EndIncludeInRelease
%<latexrelease>
%<latexrelease>\IncludeInRelease{2020/02/02}{\@popfilename}%
%<latexrelease>  {Add \@expl@push@filename@@}%
%<latexrelease>\def\@popfilename{\expandafter\@p@pfilename\@currnamestack\@nil
%<latexrelease>  \@expl@pop@filename@@}
%<latexrelease>\EndIncludeInRelease
%<latexrelease>
%    \end{macrocode}
%
% \changes{v1.4a}{2021/03/27}
%         {Do not completely roll back if \pkg{expl3} is loaded.}
%    \begin{macrocode}
%<latexrelease>\IncludeInRelease{0000/00/00}{\@popfilename}%
%<latexrelease>  {Add \@expl@push@filename@@ and \@expl@push@filename@aux@@}%
%<latexrelease>\ifnum\sourceLaTeXdate<20200202\relax
%<latexrelease>  \GenericInfo{}{Defining 00-00-00\string\@popfilename.}
%<latexrelease>\def\@popfilename{\expandafter\@p@pfilename\@currnamestack\@nil}
%<latexrelease>\else
%<latexrelease>  \GenericInfo{}{Defining 2020-02-02\string\@popfilename.}
%<latexrelease>\def\@popfilename{\expandafter\@p@pfilename\@currnamestack\@nil
%<latexrelease>  \@expl@pop@filename@@}
%<latexrelease>\fi
%<latexrelease>\EndIncludeInRelease
\@onlypreamble\@popfilename
%    \end{macrocode}
%
%    \begin{macrocode}
%</2ekernel|latexrelease>
%<*2ekernel>
%    \end{macrocode}
%
%
%
%    \begin{macrocode}
\def\@p@pfilename#1#2#3#4\@nil{%
  \gdef\@currname{#1}%
  \gdef\@currext{#2}%
  \catcode`\@#3\relax
  \gdef\@currnamestack{#4}}
\@onlypreamble\@p@pfilename
%    \end{macrocode}
%
%    \begin{macrocode}
\gdef\@currnamestack{}
\@onlypreamble\@currnamestack
%    \end{macrocode}
% \end{macro}
% \end{macro}
% \end{macro}
%
%
% \begin{macro}{\@kernel@currpathstack}
%   Path to the current file if explicitly given.  The auxiliary is
%   needed here to insert a \cs{@empty} to prevent the loss of braces.
%   \changes{v1.3u}{2020/11/20}{Macro added}
%   \changes{v1.3w}{2021/01/21}{Add empty entry for latexrelease}
%   \changes{v1.4c}{2021/06/03}%
%           {Take care of \cs{@kernel@currpathstack} when rolling
%            back/forward.}
%    \begin{macrocode}
%</2ekernel>
%<*2ekernel|latexrelease>
%<latexrelease>
%<latexrelease>\IncludeInRelease{2020/10/01}{\@kernel@currpathstack}%
%<latexrelease>  {Add \@kernel@currpathstack}%
%    \end{macrocode}
%   If rolling backwards to this release, \cs{@kernel@currpathstack}
%   will be defined, so the \cs{gdef} line should not be executed, thus
%   the \cs{@gobblethree} will take it out, so the satck isn't touched.
%    \begin{macrocode}
%<latexrelease>\@ifundefined{@kernel@currpathstack}{}{\@gobblethree}
\gdef\@kernel@currpathstack{}%
%    \end{macrocode}
%   If rolling forward to this release, then the \cs{gdef} line above
%   will define the path stack to be empty (which it can't be, inside a
%   file), so the code below will traverse the \cs{@currnamestack}, and
%   add as many empty items to \cs{@kernel@currpathstack} as there are
%   items in \cs{@currnamestack}, so both are back in sync.  Most of the
%   time \pkg{latexrelease} is loaded on top-level, so only one item is
%   needed, but \pkg{platexrelease} loads it internally, so the more
%   complicated loop is needed.
%    \begin{macrocode}
%<latexrelease>\ifx\@kernel@currpathstack\@empty
%<latexrelease>  \def\reserved@a#1#2#3{%
%<latexrelease>    \ifx\relax#3\else
%<latexrelease>      \g@addto@macro\@kernel@currpathstack{{}}%
%<latexrelease>      \expandafter\reserved@a
%<latexrelease>    \fi}%
%<latexrelease>  \expandafter\reserved@a\@currnamestack{}{}{\relax}%
%<latexrelease>\fi
\def\@p@pfilepath#1{%
  \gdef\@currpath{#1}\@p@pfilepath@aux\@empty}
\def\@p@pfilepath@aux#1\@nil{%
  \xdef\@kernel@currpathstack{#1}}
%<latexrelease>\EndIncludeInRelease
%
%<latexrelease>\IncludeInRelease{0000/00/00}{\@kernel@currpathstack}%
%<latexrelease>  {Add \@kernel@currpathstack}%
%<latexrelease>\let\@kernel@currpathstack\@undefined
%<latexrelease>\let\@p@pfilepath\@undefined
%<latexrelease>\let\@p@pfilepath@aux\@undefined
%<latexrelease>\EndIncludeInRelease
%</2ekernel|latexrelease>
%<*2ekernel>
%    \end{macrocode}
% \end{macro}
%
%
% \begin{macro}{\@ptionlist}
%    Returns the option list of the file.
% \changes{v1.4e}{2021/07/19}{Drop \cs{@onlypreamble}}
%    \begin{macrocode}
\def\@ptionlist#1{%
  \@ifundefined{opt@#1}\@empty{\csname opt@#1\endcsname}}
%\@onlypreamble\@ptionlist
%    \end{macrocode}
% \end{macro}
%
% \begin{macro}{\@ifpackageloaded}
% \begin{macro}{\@ifclassloaded}
%   |\@ifpackageloaded{|\meta{name}|}|
%  Checks to see whether a file has been loaded.
% \changes{v0.2t}{1994/01/18}
%         {Fix typo \cs{@pkgetension}}
% \changes{v1.4e}{2021/07/19}{Drop \cs{@onlypreamble}}
%    \begin{macrocode}
\def\@ifpackageloaded{\@ifl@aded\@pkgextension}
\def\@ifclassloaded{\@ifl@aded\@clsextension}
%    \end{macrocode}
%
%    \begin{macrocode}
\def\@ifl@aded#1#2{%
  \expandafter\ifx\csname ver@#2.#1\endcsname\relax
    \expandafter\@secondoftwo
  \else
    \expandafter\@firstoftwo
  \fi}
%    \end{macrocode}
% \end{macro}
% \end{macro}
%
%
%
% \begin{macro}{\@ifpackagelater}
% \begin{macro}{\@ifclasslater}
% |\@ifpackagelater{|\meta{name}|}{YYYY/MM/DD}{|\meta{true
%    code}|}{|\meta{false code}|}|
%     Checks that the package loaded is more recent or equal to the
%    given date.
%    A better name for it  would therefore been
%    |\@ifpackagelaterorequal| but it is in use for more than 30
%    years, so \ldots
% \changes{v1.4e}{2021/07/19}{Drop \cs{@onlypreamble}}
%    \begin{macrocode}
\def\@ifpackagelater{\@ifl@ter\@pkgextension}
\def\@ifclasslater{\@ifl@ter\@clsextension}
%    \end{macrocode}
% \end{macro}
% \end{macro}
%
%
%
%
%  \begin{macro}{\IfPackageAtLeastTF}
%  \begin{macro}{\IfClassAtLeastTF}
%  \begin{macro}{\IfFileAtLeastTF}
%  \begin{macro}{\IfFormatAtLeastTF}
% |\IfFormatAtLeastTF{YYYY/MM/DD}{|\meta{true
%    code}|}{|\meta{false code}|}|
%    Test if the format is later or equal to the given date.
% \changes{v1.3k}{2020/04/07}{Macro added; also in rollback (gh/168)}
% \changes{v1.4e}{2021/07/19}{Drop \cs{@onlypreamble}}
% \changes{v1.5g}{2023/03/28}{Added \cs{IfFileAtLeastTF} (gh/1015)}
%    \begin{macrocode}
%</2ekernel>
%<*2ekernel|latexrelease>
%<latexrelease>\IncludeInRelease{2020/10/01}%
%<latexrelease>                 {\IfFormatAtLeastTF}{Test format date}%
\def\IfFormatAtLeastTF{\@ifl@t@r\fmtversion}
\let\IfPackageAtLeastTF\@ifpackagelater
\let\IfClassAtLeastTF\@ifclasslater
\def\IfFileAtLeastTF#1{\expandafter\@ifl@t@r\csname ver@#1\endcsname}
%    \end{macrocode}
%    For rollback pretend it was available since the beginning of dawn.
%    \begin{macrocode}
%</2ekernel|latexrelease>
%<latexrelease>\EndIncludeInRelease
%<latexrelease>\IncludeInRelease{0000/00/00}%
%<latexrelease>                 {\IfFormatAtLeastTF}{Test format date}%
%<latexrelease>\def\IfFormatAtLeastTF{\@ifl@t@r\fmtversion}
%<latexrelease>\let\IfPackageAtLeastTF\@ifpackagelater
%<latexrelease>\let\IfClassAtLeastTF\@ifclasslater
%<latexrelease>\def\IfFileAtLeastTF#1{\expandafter\@ifl@t@r\csname ver@#1\endcsname}
%<latexrelease>\EndIncludeInRelease
%<*2ekernel>
%    \end{macrocode}
%  \end{macro}
%  \end{macro}
%  \end{macro}
%  \end{macro}
%
% \begin{macro}{\@ifl@ter}
% \changes{v1.4e}{2021/07/19}{Drop \cs{@onlypreamble}}
%    \begin{macrocode}
\def\@ifl@ter#1#2{%
  \expandafter\@ifl@t@r
    \csname ver@#2.#1\endcsname}
%</2ekernel>
%    \end{macrocode}
%
% This internal macro is also used in |\NeedsTeXFormat|.
% \changes{v0.2f}{1993/11/22}
%         {Added //00 so parsing never produces a runaway argument.}
% \changes{v1.2d}{2018/02/18}
%         {Added 0 up front to make bad data come out as 0.}
% \changes{v1.2g}{2018/04/08}
%         {Strip leading spaces from dates.}
%    \begin{macrocode}
%<latexrelease>\IncludeInRelease{2018/04/01}%
%<latexrelease>                 {\@ifl@t@r}{Guard against bad input}%
%<*2ekernel|latexrelease>
\def\@ifl@t@r#1#2{%
  \ifnum\expandafter\@parse@version@#1//00\@nil<%
        \expandafter\@parse@version@#2//00\@nil
    \expandafter\@secondoftwo
  \else
    \expandafter\@firstoftwo
  \fi}
\def\@parse@version@#1{\@parse@version0#1}
%</2ekernel|latexrelease>
%<latexrelease>\EndIncludeInRelease
%<latexrelease>\IncludeInRelease{0000/00/00}%
%<latexrelease>                 {\@ifl@t@r}{Guard against bad input}%
%<latexrelease>\def\@ifl@t@r#1#2{%
%<latexrelease>  \ifnum\expandafter\@parse@version#1//00\@nil<%
%<latexrelease>        \expandafter\@parse@version#2//00\@nil
%<latexrelease>    \expandafter\@secondoftwo
%<latexrelease>  \else
%<latexrelease>    \expandafter\@firstoftwo
%<latexrelease>  \fi}
%<latexrelease>\let\@parse@version@\@undefined
%<latexrelease>\EndIncludeInRelease
%<*2ekernel>
%    \end{macrocode}
%
% \end{macro}
%
% \changes{v1.1j}{2016/06/20}
%         {don't declare as \cs{@onlypreamble}}
% \changes{v1.2c}{2017/03/08}
%         {add \cs{@parse@version@dash} to support yyyy-mm-dd as well as yyyy/mm/dd }
%    \begin{macrocode}
%</2ekernel>
%<*2ekernel|latexreleasefirst>
\def\@parse@version#1/#2/#3#4#5\@nil{%
\@parse@version@dash#1-#2-#3#4\@nil
}
%    \end{macrocode}
%
% The |\if| test here ensures that an argument with no |/|  or |-| produces 0 (actually 00).
%    \begin{macrocode}
\def\@parse@version@dash#1-#2-#3#4#5\@nil{%
  \if\relax#2\relax\else#1\fi#2#3#4 }
%</2ekernel|latexreleasefirst>
%<*2ekernel>
%    \end{macrocode}
%
%
%
% \begin{macro}{\@ifpackagewith}
% \begin{macro}{\@ifclasswith}
% |\@ifpackagewith{|\meta{name}|}{|\meta{option-list}|}|
% Checks that \meta{option-list} is a subset of the options
% \textbf{with} which \meta{name} was loaded.
% \changes{v1.4e}{2021/07/19}{Drop \cs{@onlypreamble}}
%    \begin{macrocode}
\def\@ifpackagewith{\@if@ptions\@pkgextension}
\def\@ifclasswith{\@if@ptions\@clsextension}
%    \end{macrocode}
%
%    \begin{macrocode}
\def\@if@ptions#1#2{%
  \@expandtwoargs\@if@pti@ns{\@ptionlist{#2.#1}}}
%    \end{macrocode}
%
% Probably shouldn't use |\CurrentOption| here\ldots (changed to
% |\reserved@b|.)
% \changes{v0.2y}{1994/02/07}
%         {Add extra ,s so `two' is not matched with `twocolumn'}
% \changes{v1.1i}{2011/08/19}
%         {Re-jig definition after more stringent \cs{in@} test.}
% \changes{v1.4e}{2021/07/19}{Drop \cs{@onlypreamble}}
%    \begin{macrocode}
%</2ekernel>
%<latexrelease>\IncludeInRelease{2017/01/01}%
%<latexrelease>                 {\@if@pti@ns}{Spaces in option clash check}%
%<*2ekernel|latexrelease>
\def\@if@pti@ns#1#2{%
 \let\reserved@a\@firstoftwo
%    \end{macrocode}
% \changes{v1.2a}{2016/10/02}
%         {Ignore spaces while checking for option clash}
%    \begin{macrocode}
 \edef\reserved@b{\zap@space#2 \@empty}%
 \@for\reserved@b:=\reserved@b\do{%
   \ifx\reserved@b\@empty
   \else
     \expandafter\in@\expandafter{\expandafter,\reserved@b,}{,#1,}%
     \ifin@
     \else
       \let\reserved@a\@secondoftwo
     \fi
   \fi
 }%
 \reserved@a}
%</2ekernel|latexrelease>
%<latexrelease>\EndIncludeInRelease
%<latexrelease>\IncludeInRelease{0000/00/00}%
%<latexrelease>                 {\@if@pti@ns}{Spaces in option clash check}%
%<latexrelease>\def\@if@pti@ns#1#2{%
%<latexrelease> \let\reserved@a\@firstoftwo
%<latexrelease> \@for\reserved@b:=#2\do{%
%<latexrelease>  \ifx\reserved@b\@empty
%<latexrelease>   \else
%<latexrelease>   \expandafter\in@\expandafter
%<latexrelease>                   {\expandafter,\reserved@b,}{,#1,}%
%<latexrelease>    \ifin@
%<latexrelease>    \else
%<latexrelease>     \let\reserved@a\@secondoftwo
%<latexrelease>    \fi
%<latexrelease>  \fi
%<latexrelease> }%
%<latexrelease> \reserved@a}
%<latexrelease>\EndIncludeInRelease
%<*2ekernel>
%    \end{macrocode}
%
% \end{macro}
% \end{macro}
%
%
%
%
%  \begin{macro}{\IfPackageLoadedTF,\IfPackageLoadedWithOptionsTF,
%                \IfClassLoadedTF,\IfClassLoadedWithOptionsTF}
%    More public names for the commands already available for a long time.
%    \begin{macrocode}
%</2ekernel>
%<*2ekernel|latexrelease>
%<latexrelease>\IncludeInRelease{2024/06/01}%
%<latexrelease>                 {\IfPackageLoadedTF}{Test package loading}%
\let \IfPackageLoadedTF            \@ifpackageloaded
\let \IfClassLoadedTF              \@ifclassloaded
\let \IfPackageLoadedWithOptionsTF \@ifpackagewith
\let \IfClassLoadedWithOptionsTF   \@ifclasswith
%    \end{macrocode}
%    For rollback/rollforward pretend everything was available since
%    the beginning of dawn.
%    \begin{macrocode}
%</2ekernel|latexrelease>
%<latexrelease>\EndIncludeInRelease
%<latexrelease>\IncludeInRelease{0000/00/00}%
%<latexrelease>                 {\IfPackageLoadedTF}{Test package loading}%
%<latexrelease>
%<latexrelease>\let \IfPackageLoadedTF            \@ifpackageloaded
%<latexrelease>\let \IfClassLoadedTF              \@ifclassloaded
%<latexrelease>\let \IfPackageLoadedWithOptionsTF \@ifpackagewith
%<latexrelease>\let \IfClassLoadedWithOptionsTF   \@ifclasswith
%<latexrelease>
%<latexrelease>\EndIncludeInRelease
%<*2ekernel>
%    \end{macrocode}
%  \end{macro}
%
%
%
%
%  \begin{macro}{
%    \IfPackageLoadedT,\IfPackageLoadedF,
%    \IfPackageAtLeastT,\IfPackageAtLeastF,
%    \IfClassAtLeastT,\IfClassAtLeastF,
%    \IfFileAtLeastT,\IfFileAtLeastF,
%    \IfFormatAtLeastT,\IfFormatAtLeastF,
%    \IfPackageLoadedWithOptionsT,\IfPackageLoadedWithOptionsF,
%    \IfClassLoadedT,\IfClassLoadedF,
%    \IfClassLoadedWithOptionsF,\IfClassLoadedWithOptionsTF
%   }
%    A few more conditionals for convenience
% \changes{v1.5k}{2024/04/10}{Provide T and F conditionals not just TF
%    (gh/1262)}
% \changes{v1.5m}{2025/05/10}{avoid errors for par in code argument of T versions (gh/1733)}
%    \begin{macrocode}
%</2ekernel>
%<*2ekernel|latexrelease>
%<latexrelease>\IncludeInRelease{2024/06/01}%
%<latexrelease>                 {\IfPackageLoadedT}{More conditionals}%
\def\IfPackageLoadedT   #1{\IfPackageLoadedTF{#1}\@firstofone\@gobble}
\def\IfPackageLoadedF   #1{\IfPackageLoadedTF{#1}{}}
\def\IfClassLoadedT     #1{\IfClassLoadedTF{#1}\@firstofone\@gobble}
\def\IfClassLoadedF     #1{\IfClassLoadedTF{#1}{}}
\def\IfPackageAtLeastT#1#2{\IfPackageAtLeastTF{#1}{#2}\@firstofone\@gobble}
\def\IfPackageAtLeastF#1#2{\IfPackageAtLeastTF{#1}{#2}{}}
\def\IfClassAtLeastT    #1#2{\IfClassAtLeastTF{#1}{#2}\@firstofone\@gobble}
\def\IfClassAtLeastF    #1#2{\IfClassAtLeastTF{#1}{#2}{}}
\def\IfFileAtLeastT   #1#2{\IfFileAtLeastTF{#1}{#2}\@firstofone\@gobble}
\def\IfFileAtLeastF   #1#2{\IfFileAtLeastTF{#1}{#2}{}}
\def\IfFormatAtLeastT   #1{\IfFormatAtLeastTF{#1}\@firstofone\@gobble}
\def\IfFormatAtLeastF   #1{\IfFormatAtLeastTF{#1}{}}
\def\IfPackageLoadedWithOptionsT #1#2{\IfPackageLoadedWithOptionsTF{#1}{#2}\@firstofone\@gobble}
\def\IfPackageLoadedWithOptionsF #1#2{\IfPackageLoadedWithOptionsTF{#1}{#2}{}}
\def\IfClassLoadedWithOptionsT #1#2{\IfClassLoadedWithOptionsTF{#1}{#2}\@firstofone\@gobble}
\def\IfClassLoadedWithOptionsF #1#2{\IfClassLoadedWithOptionsTF{#1}{#2}{}}
%    \end{macrocode}
%    
%  \begin{macro}{\IfFileLoadedTF,\IfFileLoadedT,\IfFileLoadedF}
%    These three commands haven't been there at all in the past.
% \changes{v1.5k}{2024/04/10}{Provide \cs{IfFileLoadedTF} and variants
%    (gh/1222)}
%    \begin{macrocode}
\def\IfFileLoadedTF#1{%
  \expandafter\ifx\csname ver@#1\endcsname\relax
    \expandafter\@secondoftwo
  \else
    \expandafter\@firstoftwo
  \fi}
\def\IfFileLoadedT  #1{\IfFileLoadedTF{#1}\@firstofone\@gobble}
\def\IfFileLoadedF  #1{\IfFileLoadedTF{#1}{}}
%    \end{macrocode}
%    For rollback/rollforward pretend everything was available since
%    the beginning of dawn.
%    \begin{macrocode}
%</2ekernel|latexrelease>
%<latexrelease>\EndIncludeInRelease
%<latexrelease>\IncludeInRelease{0000/00/00}%
%<latexrelease>                 {\IfPackageLoadedT}{More conditionals}%
%<latexrelease>
%<latexrelease>\def\IfPackageLoadedT #1#2{\IfPackageLoadedTF{#1}{#2}{}}
%<latexrelease>\def\IfPackageLoadedF   #1{\IfPackageLoadedTF{#1}{}}
%<latexrelease>\def\IfClassLoadedT   #1#2{\IfClassLoadedTF{#1}{#2}{}}
%<latexrelease>\def\IfClassLoadedF     #1{\IfClassLoadedTF{#1}{}}
%<latexrelease>\def\IfPackageAtLeastT#1#2#3{\IfPackageAtLeastTF{#1}{#2}{#3}{}}
%<latexrelease>\def\IfPackageAtLeastF  #1#2{\IfPackageAtLeastTF{#1}{#2}{}}
%<latexrelease>\def\IfClassAtLeastT  #1#2#3{\IfClassAtLeastTF{#1}{#2}{#3}{}}
%<latexrelease>\def\IfClassAtLeastF    #1#2{\IfClassAtLeastTF{#1}{#2}{}}
%<latexrelease>\def\IfFileAtLeastT   #1#2#3{\IfFileAtLeastTF{#1}{#2}{#3}{}}
%<latexrelease>\def\IfFileAtLeastF     #1#2{\IfFileAtLeastTF{#1}{#2}{}}
%<latexrelease>\def\IfFormatAtLeastT   #1#2{\IfFormatAtLeastTF{#1}{#2}{}}
%<latexrelease>\def\IfFormatAtLeastF     #1{\IfFormatAtLeastTF{#1}{}}
%<latexrelease>\def\IfPackageLoadedWithOptionsT #1#2#3{\IfPackageLoadedWithOptionsTF{#1}{#2}{#3}{}}
%<latexrelease>\def\IfPackageLoadedWithOptionsF   #1#2{\IfPackageLoadedWithOptionsTF{#1}{#2}{}}
%<latexrelease>\def\IfClassLoadedWithOptionsT #1#2#3{\IfClassLoadedWithOptionsTF{#1}{#2}{#3}{}}
%<latexrelease>\def\IfClassLoadedWithOptionsF   #1#2{\IfClassLoadedWithOptionsTF{#1}{#2}{}}
%<latexrelease>
%<latexrelease>\def\IfFileLoadedTF#1{%
%<latexrelease>  \expandafter\ifx\csname ver@#1\endcsname\relax
%<latexrelease>    \expandafter\@secondoftwo
%<latexrelease>  \else
%<latexrelease>    \expandafter\@firstoftwo
%<latexrelease>  \fi}
%<latexrelease>\def\IfFileLoadedT#1#2{\IfFileLoadedTF{#1}{#2}{}}
%<latexrelease>\def\IfFileLoadedF  #1{\IfFileLoadedTF{#1}{}}
%<latexrelease>
%<latexrelease>\EndIncludeInRelease
%<*2ekernel>
%    \end{macrocode}
%  \end{macro}
%  \end{macro}
%
%
%
%
% \begin{macro}{\ProvidesPackage}
%    Checks that the current filename is correct, and defines
%    |\ver@filename|.
% \changes{v0.3c}{1994/03/12}
%         {Add \cs{wlog}}
% \changes{v0.3c}{1994/03/12}
%         {use \cs{@gtempa}}
%    \begin{macrocode}
%</2ekernel>
%<latexrelease>\IncludeInRelease{2020/10/01}%
%<latexrelease>  {\ProvidesPackage}{Check name with \strcmp}%
%<*2ekernel|latexrelease>
\def\ProvidesPackage#1{%
  \xdef\@gtempa{#1}%
%    \end{macrocode}
% \changes{v1.3u}{2020/11/20}
%         {Use string comparison instead of \cs{ifx}}
%   Here \cs{@currpath} is explicitly added to the file name to report
%   when a package or class is loaded using an explicit path.  Loading
%   using a path in the argument is supported but not encouraged.
%    \begin{macrocode}
  \@expandtwoargs\@expl@str@if@eq@@nnTF
      {\@gtempa}{\@currpath\@currname}{}{%
    \@latex@warning@no@line{You have requested
      \@cls@pkg\space`\@currpath\@currname',\MessageBreak
       but the \@cls@pkg\space provides `#1'}%
    }%
  \@ifnextchar[\@pr@videpackage{\@pr@videpackage[]}}%]
\@onlypreamble\ProvidesPackage
%</2ekernel|latexrelease>
%<latexrelease>\EndIncludeInRelease
%
%<latexrelease>\IncludeInRelease{0000/00/00}%
%<latexrelease>  {\ProvidesPackage}{Undo: check name with \strcmp}%
%<latexrelease>\def\ProvidesPackage#1{%
%<latexrelease>  \xdef\@gtempa{#1}%
%<latexrelease>  \ifx\@gtempa\@currname\else
%<latexrelease>    \@latex@warning@no@line{You have requested
%<latexrelease>      \@cls@pkg\space`\@currname',\MessageBreak
%<latexrelease>       but the \@cls@pkg\space provides `#1'}%
%<latexrelease>  \fi
%<latexrelease>  \@ifnextchar[\@pr@videpackage{\@pr@videpackage[]}}%]
%<latexrelease>\EndIncludeInRelease
%<*2ekernel>
%    \end{macrocode}
%  \end{macro}
%
%
%
%  \begin{macro}{\@pr@videpackage}
%    This is the helper command for \cs{ProvidesPackage}. It tries to
%    be cautious when handling the identification string in case it
%    contains UTF-8 characters.
% \changes{v1.3e}{2019/11/29}{Protect package info text (gh/52)}
% \changes{v1.3r}{2020/10/01}{Allow for package substitution}
%    \begin{macrocode}
%</2ekernel>
%<*2ekernel|latexrelease>
%<latexrelease>\IncludeInRelease{2020/10/01}%
%<latexrelease>                 {\@pr@videpackage}{Allow for package substitution}%
\def\@pr@videpackage[#1]{%
  \expandafter\protected@xdef                %     <-- protected...
     \csname ver@\@currname.\@currext\endcsname{#1}% Loaded package
  \expandafter\let
    \csname ver@\@currpkg@reqd\expandafter\endcsname % Requested package
    \csname ver@\@currname.\@currext\endcsname
  \ifx\@currext\@clsextension
    \typeout{Document Class: \@gtempa\space#1}%
  \else
    \protected@wlog{Package: \@gtempa\space#1}%   <--- protected
  \fi}
%    \end{macrocode}
% \end{macro}
%
%
%
%  \begin{macro}{\protected@wlog}
%    This is like plain \TeX's \cs{wlog} but gracefully handles
%    protected commands.
%    \begin{macrocode}
\long\def\protected@wlog#1{\begingroup
  \set@display@protect
  \immediate \write \m@ne {#1}\endgroup }
%    \end{macrocode}
%  \end{macro}
%
%
%    \begin{macrocode}
%</2ekernel|latexrelease>
%<latexrelease>\EndIncludeInRelease
%<latexrelease>\IncludeInRelease{2020/02/02}%
%<latexrelease>                 {\@pr@videpackage}{Protection for package info}%
%<latexrelease>
%<latexrelease>\def\@pr@videpackage[#1]{%
%<latexrelease>  \expandafter\protected@xdef                %     <-- protected...
%<latexrelease>     \csname ver@\@currname.\@currext\endcsname{#1}%
%<latexrelease>\ifx\@currext\@clsextension
%<latexrelease>    \typeout{Document Class: \@gtempa\space#1}%
%<latexrelease>  \else
%<latexrelease>    \protected@wlog{Package: \@gtempa\space#1}%   <--- protected
%<latexrelease>  \fi}
%<latexrelease>
%<latexrelease>\EndIncludeInRelease
%<latexrelease>\IncludeInRelease{0000/00/00}%
%<latexrelease>                 {\@pr@videpackage}{Protection for package info}%
%<latexrelease>
%<latexrelease>\def\@pr@videpackage[#1]{%
%<latexrelease>  \expandafter\xdef\csname ver@\@currname.\@currext\endcsname{#1}%
%<latexrelease>  \ifx\@currext\@clsextension
%<latexrelease>    \typeout{Document Class: \@gtempa\space#1}%
%<latexrelease>  \else
%<latexrelease>    \wlog{Package: \@gtempa\space#1}%
%<latexrelease>  \fi}
%<latexrelease>\let\protected@wlog\@undefined
%<latexrelease>
%<latexrelease>\EndIncludeInRelease
%<*2ekernel>
%    \end{macrocode}
%
%    \begin{macrocode}
\@onlypreamble\@pr@videpackage
%    \end{macrocode}
%
%
%
% \begin{macro}{\ProvidesClass}
%    Like |\ProvidesPackage|, but for classes.
%    This needs a dummy \pkg{latexrelease} block to copy the definition
%    of \cs{ProvidesPackage} as it changes across releases.
%    \begin{macrocode}
%</2ekernel>
%<latexrelease>\IncludeInRelease{0000/00/00}%
%<latexrelease>  {\ProvidesClass}{Track \ProvidesPackage}%
%<*2ekernel|latexrelease>
\let\ProvidesClass\ProvidesPackage
\@onlypreamble\ProvidesClass
%</2ekernel|latexrelease>
%<latexrelease>\EndIncludeInRelease
%<*2ekernel>
%    \end{macrocode}
% \end{macro}
%
% \begin{macro}{\ProvidesFile}
%    Like |\ProvidesPackage|, but for arbitrary files. Do not apply
%    |\@onlypreamble| to these, as we may want to label files input
%    during the document.
% \changes{v0.2l}{1993/12/07}
%         {Macro added}
% \changes{v0.3c}{1994/03/12}
%         {Add \cs{wlog}}
% \changes{v0.3g}{1994/04/11}
%         {Protect against weird catcodes.}
% \begin{macro}{\@providesfile}
% \changes{v1.0r}{1995/10/17}
%         {Delay definition of \cs{ProvidesFile} till ltfinal}
% \changes{v1.1a}{1998/03/21}
%         {Allow \&. Internal/2702}
% \changes{v1.1d}{2001/05/25}{Explicitly set catcode of
%                              \cs{endlinechar} to 10 (pr/3334)}
% \changes{v1.1e}{2001/06/04}{But only if it is a char (pr/3334)}
% \changes{v1.1f}{2001/08/26}{Readded setting of space char (pr/3353)}
%    \begin{macrocode}
\def\ProvidesFile#1{%
  \begingroup
    \catcode`\ 10 %
    \ifnum \endlinechar<256 %
      \ifnum \endlinechar>\m@ne
        \catcode\endlinechar 10 %
      \fi
    \fi
    \@makeother\/%
    \@makeother\&%
%    \end{macrocode}
% \changes{v1.1g}{2004/01/28}{Use kernel version of
%                             \cs{@ifnextchar} (pr/3501)}
%    \begin{macrocode}
    \kernel@ifnextchar[{\@providesfile{#1}}{\@providesfile{#1}[]}}
%    \end{macrocode}
%
% During initex a special version of |\@providesfile| is used.
% The real definition is installed right at the end, in |ltfinal.dtx|.
%\begin{verbatim}
%\def\@providesfile#1[#2]{%
%    \wlog{File: #1 #2}%
%    \expandafter\xdef\csname ver@#1\endcsname{#2}%
%  \endgroup}
%\end{verbatim}
% \end{macro}
% \end{macro}
%
% \begin{macro}{\PassOptionsToPackage}
% \begin{macro}{\PassOptionsToClass}
%   If the package has been loaded, we check that it was first loaded with
%   the options.  Otherwise we add the option list to that of the package.
% \changes{v1.3t}{2020/10/18}{Drop path from \cs{input@path} (gh/414).}
% \changes{v1.3x}{2021/02/18}{save raw option lists (gh/85)}
%    \begin{macrocode}
%</2ekernel>
%<latexrelease>\IncludeInRelease{2021/06/01}%
%<latexrelease>                 {\@pass@ptions}{Raw option lists}%
%<*2ekernel|latexrelease>
\def\@pass@ptions#1#2#3{%
  \@expl@@@filehook@set@curr@file@@nNN
    {\@expl@@@filehook@resolve@file@subst@@w #3.#1\@nil}%
      \reserved@a\reserved@b
  \@expl@@@filehook@clear@replacement@flag@@
%    \end{macrocode}
% \changes{v1.5d}{2022/10/10}{Use \cs{protected@xdef}.}
%    \begin{macrocode}
  \expandafter\protected@xdef\csname opt@\reserved@a\endcsname{%
    \@ifundefined{opt@\reserved@a}\@empty
      {\csname opt@\reserved@a\endcsname,}%
    \zap@space#2 \@empty}%
%    \end{macrocode}
% \changes{v1.3u}{2020/11/20}
%         {Copy option list to the requested package.}
%    \begin{macrocode}
  \expandafter\let
    \csname opt@#3.#1\expandafter\endcsname
    \csname opt@\reserved@a\endcsname
%    \end{macrocode}
% Extend raw option list
% \changes{v1.4c}{2021/06/06}
%         {apply \cs{expandafter} to raw options for gh/580}
%    \begin{macrocode}
  \@ifundefined{@raw@opt@#3.#1}%
    {\expandafter\gdef\csname @raw@opt@#3.#1\expandafter\endcsname
              \expandafter{#2}}%
    {\expandafter\g@addto@macro\csname @raw@opt@#3.#1\expandafter\endcsname
              \expandafter{\expandafter,#2}}%
}
%</2ekernel|latexrelease>
%<latexrelease>\EndIncludeInRelease
%    \end{macrocode}
%
%    \begin{macrocode}
%<latexrelease>\IncludeInRelease{2020/10/01}{\@pass@ptions}
%<latexrelease>  {Add file replacement in \@pass@ptions}%
%<latexrelease>
%<latexrelease>\def\@pass@ptions#1#2#3{%
%<latexrelease>  \@expl@@@filehook@set@curr@file@@nNN
%<latexrelease>    {\@expl@@@filehook@resolve@file@subst@@w #3.#1\@nil}%
%<latexrelease>      \reserved@a\reserved@b
%<latexrelease>  \@expl@@@filehook@clear@replacement@flag@@
%<latexrelease>  \expandafter\xdef\csname opt@\reserved@a\endcsname{%
%<latexrelease>    \@ifundefined{opt@\reserved@a}\@empty
%<latexrelease>      {\csname opt@\reserved@a\endcsname,}%
%<latexrelease>    \zap@space#2 \@empty}%
%<latexrelease>  \expandafter\let
%<latexrelease>    \csname opt@#3.#1\expandafter\endcsname
%<latexrelease>    \csname opt@\reserved@a\endcsname}
%<latexrelease>\EndIncludeInRelease
%    \end{macrocode}
%
%    \begin{macrocode}
%<latexrelease>\IncludeInRelease{0000/00/00}{\@pass@ptions}
%<latexrelease>  {\@pass@ptions}%
%<latexrelease>
%<latexrelease>\def\@pass@ptions#1#2#3{%
%<latexrelease>  \expandafter\xdef\csname opt@#3.#1\endcsname{%
%<latexrelease>    \@ifundefined{opt@#3.#1}\@empty
%<latexrelease>      {\csname opt@#3.#1\endcsname,}%
%<latexrelease>    \zap@space#2 \@empty}}
%<latexrelease>\EndIncludeInRelease
%<*2ekernel>
%    \end{macrocode}
%
%    \begin{macrocode}
\@onlypreamble\@pass@ptions
%    \end{macrocode}
%
%    \begin{macrocode}
\def\PassOptionsToPackage{\@pass@ptions\@pkgextension}
\def\PassOptionsToClass{\@pass@ptions\@clsextension}
\@onlypreamble\PassOptionsToPackage
\@onlypreamble\PassOptionsToClass
%    \end{macrocode}
% \end{macro}
% \end{macro}
%
% \begin{macro}{\DeclareOption}
% \begin{macro}{\DeclareOption*}
%    Adds an option as a |\ds@| command, or the default |\default@ds|
%    command.
% \changes{v0.2c}{1993/11/17}
%         {Error checking added}
% \changes{v1.0m}{1995/04/21}
%         {Made long /1498}
% \changes{v1.0n}{1995/05/12}
%         {Use \cs{toks@} to remove need to double hash /1557}
%    \begin{macrocode}
\def\DeclareOption{%
  \let\@fileswith@pti@ns\@badrequireerror
  \@ifstar\@defdefault@ds\@declareoption}
\long\def\@declareoption#1#2{%
   \xdef\@declaredoptions{\@declaredoptions,#1}%
   \toks@{#2}%
   \expandafter\edef\csname ds@#1\endcsname{\the\toks@}}
\long\def\@defdefault@ds#1{%
  \toks@{#1}%
  \edef\default@ds{\the\toks@}}
\@onlypreamble\DeclareOption
\@onlypreamble\@declareoption
\@onlypreamble\@defdefault@ds
%    \end{macrocode}
% \end{macro}
% \end{macro}
%
% \begin{macro}{\OptionNotUsed}
% \changes{v1.3x}{2021/02/18}{filter out =value from unused option list (gh/85)}
% \begin{macro}{\@remove@eq@value}
% \changes{v1.3x}{2021/02/18}{macro added (gh/85)}
% If we are in a class file, add |\CurrentOption| to the list of
% unused options. Otherwise, in a package file do nothing.
%    \begin{macrocode}
%</2ekernel>
%<latexrelease>\IncludeInRelease{2021/06/01}%
%<latexrelease>                 {\OptionNotUsed}{filter unused option list}%
%<*2ekernel|latexrelease>
\ExplSyntaxOn
\def\@remove@eq@value#1=#2\@nil{\tl_trim_spaces:n{#1}}
\ExplSyntaxOff
%    \end{macrocode}
%
%    \begin{macrocode}
\def\OptionNotUsed{%
  \ifx\@currext\@clsextension
    \xdef\@unusedoptionlist{%
      \ifx\@unusedoptionlist\@empty\else\@unusedoptionlist,\fi
      \expandafter\@remove@eq@value\CurrentOption=\@nil}%
  \fi}
%</2ekernel|latexrelease>
%<latexrelease>\EndIncludeInRelease
%<latexrelease>\IncludeInRelease{0000/00/00}%
%<latexrelease>                 {\OptionNotUsed}{filter unused option list}%
%<latexrelease>\let\@remove@eq@value\@undefined
%    \end{macrocode}
%
%    \begin{macrocode}
%<latexrelease>\def\OptionNotUsed{%
%<latexrelease>  \ifx\@currext\@clsextension
%<latexrelease>    \xdef\@unusedoptionlist{%
%<latexrelease>      \ifx\@unusedoptionlist\@empty\else\@unusedoptionlist,\fi
%<latexrelease>      \CurrentOption}%
%<latexrelease>  \fi}
%<latexrelease>\EndIncludeInRelease
%<*2ekernel>
%    \end{macrocode}
%
%    \begin{macrocode}
\@onlypreamble\OptionNotUsed
%    \end{macrocode}
% \end{macro}
% \end{macro}
%
% \begin{macro}{\default@ds}
% The default option code.
% Set by |\@onefilewithoptions| to either |\OptionNotUsed| for
% classes, or |\@unknownoptionerror| for packages. This may be reset
% in either case with |\DeclareOption*|.
%    \begin{macrocode}
% \let\default@ds\OptionNotUsed
%    \end{macrocode}
% \end{macro}
%
% \begin{macro}{\ProcessOptions}
% \begin{macro}{\ProcessOptions*}
% |\ProcessOptions| calls |\ds@option| for each known package option,
% then calls |\default@ds| for each option on the local options list.
% Finally resets all the declared options to |\relax|. The empty option
% does nothing, this has to be reset on the off chance it's set to
% |\relax| if an empty element gets into the |\@declaredoptions| list.
%
% The star form is similar but executes options given in the order
% specified in the document, not the order they are declared in the
% file. In the case of packages, global options are executed before
% local ones.
% \changes{v0.2a}{1993/11/14}
%         {Stop adding the global option list inside class files.}
% \changes{v0.2a}{1993/11/14}
%         {Optimize `empty option' code.}
% \changes{v0.2b}{1993/11/15}
%         {Star form added.}
% \changes{v0.2c}{1993/11/17}
%         {restoring \cs{@fileswith@pti@ns} added.}
% \changes{v1.5d}{2022/10/10}
%         {Use \cs{protected@edef}.}
%    \begin{macrocode}
\def\ProcessOptions{%
  \let\ds@\@empty
  \protected@edef\@curroptions{\@ptionlist{\@currname.\@currext}}%
  \@ifstar\@xprocess@ptions\@process@ptions}
\@onlypreamble\ProcessOptions
%    \end{macrocode}
%
% \changes{v0.2y}{1994/02/07}
%         {Add extra ,s so `two' is not matched with `twocolumn'}
%    \begin{macrocode}
\def\@process@ptions{%
  \@for\CurrentOption:=\@declaredoptions\do{%
    \ifx\CurrentOption\@empty\else
      \@expandtwoargs\in@{,\CurrentOption,}{%
         ,\ifx\@currext\@clsextension\else\@classoptionslist,\fi
         \@curroptions,}%
      \ifin@
        \@use@ption
        \expandafter\let\csname ds@\CurrentOption\endcsname\@empty
      \fi
    \fi}%
  \@process@pti@ns}
\@onlypreamble\@process@ptions
%    \end{macrocode}
%
% \changes{v0.2y}{1994/02/07}
%         {Add extra ,s so `two' is not matched with `twocolumn'}
% \changes{v1.3z}{2021/03/05}{modify so braces to not give errors (gh/513)}
% \changes{v1.5e}{2022/10/22}{Use \cs{detokenize}}
%    \begin{macrocode}
%</2ekernel>
%<latexrelease>\IncludeInRelease{2021/06/01}%
%<latexrelease>                 {\@xprocess@ptions}{safer @xprocess@ptions}%
%<*2ekernel|latexrelease>
\def\@xprocess@ptions{%
  \ifx\@currext\@clsextension\else
   \ifx\@classoptionslist\relax\else
    \@for\CurrentOption:=\@classoptionslist\do{%
      \ifx\CurrentOption\@empty\else
        \@ifundefined{ds@\detokenize\expandafter{\CurrentOption}}{}{%
          \@use@ption
          \expandafter\let\csname ds@\CurrentOption\endcsname\@empty
        }%
      \fi}%
    \fi
  \fi
  \@process@pti@ns}
%</2ekernel|latexrelease>
%<latexrelease>\EndIncludeInRelease
%<latexrelease>\IncludeInRelease{0000/00/00}%
%<latexrelease>                 {\@xprocess@ptions}{safer @xprocess@ptions}%
%<latexrelease>\let\@remove@eq@value\@undefined
%    \end{macrocode}
%
%    \begin{macrocode}
%<latexrelease>\def\@xprocess@ptions{%
%<latexrelease>  \ifx\@currext\@clsextension\else
%<latexrelease>    \@for\CurrentOption:=\@classoptionslist\do{%
%<latexrelease>      \ifx\CurrentOption\@empty\else
%<latexrelease>        \@expandtwoargs\in@{,\CurrentOption,}{,\@declaredoptions,}%
%<latexrelease>        \ifin@
%<latexrelease>          \@use@ption
%<latexrelease>          \expandafter\let\csname ds@\CurrentOption\endcsname\@empty
%<latexrelease>        \fi
%<latexrelease>      \fi}%
%<latexrelease>  \fi
%<latexrelease>  \@process@pti@ns}
%<latexrelease>\EndIncludeInRelease
%<*2ekernel>
%    \end{macrocode}
%
%    \begin{macrocode}
\@onlypreamble\@xprocess@ptions
%    \end{macrocode}
%
% The common part of |\ProcessOptions| and |\ProcessOptions*|.
% \changes{v1.5e}{2022/10/22}{Use \cs{detokenize}}
%    \begin{macrocode}
%</2ekernel>
%<*2ekernel|latexrelease>
%<latexrelease>\IncludeInRelease{2020/10/01}%
%<latexrelease>                 {\@process@pti@ns}{Unused options issue}%
\def\@process@pti@ns{%
  \@for\CurrentOption:=\@curroptions\do{%
    \@ifundefined{ds@\detokenize\expandafter{\CurrentOption}}%
      {\@use@ption
       \default@ds}%
%    \end{macrocode}
% There should not be any non-empty definition of |\CurrentOption| at
% this point, as all the declared options were executed earlier. This is
% for compatibility with 2.09 styles which use |\def\ds@|\ldots\
% directly, and so have options which do not appear in
% |\@declaredoptions|.
%    \begin{macrocode}
      \@use@ption}%
%    \end{macrocode}
% Clear all the definitions for option code. First set all the declared
% options to |\relax|, then reset the `default' and `empty' options. and
% the lst of declared options.
%    \begin{macrocode}
  \@for\CurrentOption:=\@declaredoptions\do{%
    \expandafter\let\csname ds@\CurrentOption\endcsname\relax}%
%    \end{macrocode}
% \changes{v1.0r}{1995/10/17}
%         {Reset \cs{CurrentOption} for graphics/1873}
% \changes{v1.3k}{2020/04/07}{Use different method to ignore
%    unprocessed options (gh/22)}
%    \begin{macrocode}
  \let\CurrentOption\@empty
  \let\@fileswith@pti@ns\@@fileswith@pti@ns
  \AtEndOfPackage{\expandafter\let
                     \csname unprocessedoptions-\@currname.\@currext\endcsname
                     \relax}}
\@onlypreamble\@process@pti@ns
%</2ekernel|latexrelease>
%<latexrelease>\EndIncludeInRelease
%<latexrelease>\IncludeInRelease{0000/00/00}%
%<latexrelease>                 {\@process@pti@ns}{Unused options issue}%
%<latexrelease>
%<latexrelease>\def\@process@pti@ns{%
%<latexrelease>  \@for\CurrentOption:=\@curroptions\do{%
%<latexrelease>    \@ifundefined{ds@\CurrentOption}%
%<latexrelease>      {\@use@ption
%<latexrelease>       \default@ds}%
%<latexrelease>      \@use@ption}%
%<latexrelease>  \@for\CurrentOption:=\@declaredoptions\do{%
%<latexrelease>    \expandafter\let\csname ds@\CurrentOption\endcsname\relax}%
%<latexrelease>  \let\CurrentOption\@empty
%<latexrelease>  \let\@fileswith@pti@ns\@@fileswith@pti@ns
%<latexrelease>  \AtEndOfPackage{\let\@unprocessedoptions\relax}}
%<latexrelease>\EndIncludeInRelease
%<*2ekernel>
%    \end{macrocode}
% \end{macro}
% \end{macro}
%
% \begin{macro}{\@options}
% |\@options| is a synonym for |\ProcessOptions*| for upward
% compatibility with \LaTeX2.09 style files.
%    \begin{macrocode}
\def\@options{\ProcessOptions*}
\@onlypreamble\@options
%    \end{macrocode}
% \end{macro}
%
% \begin{macro}{\@use@ption}
% Execute the code for the current option.
% \changes{v0.2g}{1993/11/23}
%         {Name changed from \cs{@executeoption}}
% \changes{v1.0e}{1994/05/17}
%         {Execute option after removing from list, not before}
% \changes{v1.3x}{2021/02/18}{filter out =value from unused option list (gh/85)}
% \changes{v1.5e}{2022/10/22}{Use \cs{detokenize}}
%    \begin{macrocode}
%</2ekernel>
%<latexrelease>\IncludeInRelease{2021/06/01}%
%<latexrelease>                 {\@use@ption}{filter unused option list}%
%<*2ekernel|latexrelease>
\def\@use@ption{%
  \@expandtwoargs\@removeelement
     {\expandafter\@remove@eq@value\CurrentOption=\@nil}%
  \@unusedoptionlist\@unusedoptionlist
  \csname ds@\detokenize\expandafter{\CurrentOption}\endcsname}
%</2ekernel|latexrelease>
%<latexrelease>\EndIncludeInRelease
%<latexrelease>\IncludeInRelease{0000/00/00}%
%<latexrelease>                 {\@use@ption}{filter unused option list}%
%<latexrelease>\def\@use@ption{%
%<latexrelease>  \@expandtwoargs\@removeelement\CurrentOption
%<latexrelease>  \@unusedoptionlist\@unusedoptionlist
%<latexrelease>  \csname ds@\CurrentOption\endcsname}
%<latexrelease>\EndIncludeInRelease
%<*2ekernel>
%    \end{macrocode}
%
%    \begin{macrocode}
\@onlypreamble\@use@ption
%    \end{macrocode}
% \end{macro}
%
% \begin{macro}{\ExecuteOptions}
% |\ExecuteOptions{|\meta{option-list}|}| executes the code declared
% for each option.
% \changes{v0.2d}{1993/11/18}
%         {Use \cs{CurrentOption} not \cs{reserved@a}}
% \changes{v0.2k}{1993/12/06}
%         {Preserve \cs{CurrentOption}.}
%    \begin{macrocode}
%</2ekernel>
%<latexrelease>\IncludeInRelease{2017/01/01}%
%<latexrelease>                 {\ExecuteOptions}{Spaces in \ExecuteOptions}%
%<*2ekernel|latexrelease>
\def\ExecuteOptions#1{%
%    \end{macrocode}
% \changes{v1.2a}{2016/10/02}
%         {Ignore spaces in argument}
% Use |\@fortmp| here as it is anyway cleared during |\@for| loop
% so does not change any existing names.
%    \begin{macrocode}
  \edef\@fortmp{\zap@space#1 \@empty}%
  \def\reserved@a##1\@nil{%
    \@for\CurrentOption:=\@fortmp\do
             {\csname ds@\CurrentOption\endcsname}%
    \edef\CurrentOption{##1}}%
  \expandafter\reserved@a\CurrentOption\@nil}
%</2ekernel|latexrelease>
%<latexrelease>\EndIncludeInRelease
%<latexrelease>\IncludeInRelease{0000/00/00}%
%<latexrelease>                 {\ExecuteOptions}{Spaces in \ExecuteOptions}%
%<latexrelease>\def\ExecuteOptions#1{%
%<latexrelease>  \def\reserved@a##1\@nil{%
%<latexrelease>    \@for\CurrentOption:=#1\do
%<latexrelease>             {\csname ds@\CurrentOption\endcsname}%
%<latexrelease>    \edef\CurrentOption{##1}}%
%<latexrelease>  \expandafter\reserved@a\CurrentOption\@nil}
%<latexrelease>\EndIncludeInRelease
%<*2ekernel>
%    \end{macrocode}
%
%    \begin{macrocode}
\@onlypreamble\ExecuteOptions
%    \end{macrocode}
% \end{macro}
%
% The top-level commands, which just set some parameters then call
% the internal command, |\@fileswithoptions|.
% \begin{macro}{\documentclass}
% \changes{v1.0q}{1995/06/19}
%         {Don't redefine \cs{usepackage} in compat mode for /1634}
% The main new-style class declaration.
%    \begin{macrocode}
\def\documentclass{%
  \let\documentclass\@twoclasseserror
  \if@compatibility\else\let\usepackage\RequirePackage\fi
  \@fileswithoptions\@clsextension}
\@onlypreamble\documentclass
%    \end{macrocode}
% \end{macro}
%
% \begin{macro}{\documentstyle}
% 2.09 style class `style' declaration.
% \changes{v0.2a}{1993/11/14}
%         {Added \cs{RequirePackage} \cs{@unusedoptionlist} stuff.}
% \changes{v0.2b}{1993/11/15}
%         {Modified to match \cs{ProcessOption*}}
% \changes{v0.2d}{1993/11/18}
%         {Modified \cs{RequirePackage} stuff.}
% \changes{v0.2n}{1993/12/09}
%         {input 209 compatibility file.}
% \changes{v0.2o}{1993/12/13}
%         {compatibility file now latex209.sty.}
% \changes{v0.2q}{1993/12/17}
%         {Match Alan's new code.}
% \changes{v0.2u}{1994/01/21}
%         {compatibility file now latex209.def.}
%    \begin{macrocode}
\def\documentstyle{%
  \makeatletter%%
%% This file will generate fast loadable files and documentation
%% driver files from the doc files in this package when run through
%% LaTeX or TeX.
%%
%% Copyright (C) 1993-2024
%% The LaTeX Project and any individual authors listed elsewhere
%% in this file.
%%
%% This file is part of the LaTeX base system.
%% -------------------------------------------
%%
%% It may be distributed and/or modified under the
%% conditions of the LaTeX Project Public License, either version 1.3c
%% of this license or (at your option) any later version.
%% The latest version of this license is in
%%    https://www.latex-project.org/lppl.txt
%% and version 1.3c or later is part of all distributions of LaTeX
%% version 2008 or later.
%%
%% This file has the LPPL maintenance status "maintained".
%%
%% As this file contains legal notices, it is NOT PERMITTED to modify
%% this file in any way that the legal information placed into
%% generated files is changed (i.e., the files generated when the
%% original file is executed). This restriction does not apply if
%% (parts of) the content is reused in a different WORK producing its
%% own generated files.
%%
%% The list of all files belonging to the LaTeX base distribution is
%% given in the file `manifest.txt'. See also `legal.txt' for additional
%% information.
%%
%%
%%
%%
%% --------------- start of docstrip commands ------------------
%%

\input docstrip

\preamble

This is a generated file.

The source is maintained by the LaTeX Project team and bug
reports for it can be opened at https://latex-project.org/bugs.html
(but please observe conditions on bug reports sent to that address!)


Copyright (C) 1993-2024
The LaTeX Project and any individual authors listed elsewhere
in this file.

This file was generated from file(s) of the LaTeX base system.
--------------------------------------------------------------

It may be distributed and/or modified under the
conditions of the LaTeX Project Public License, either version 1.3c
of this license or (at your option) any later version.
The latest version of this license is in
   https://www.latex-project.org/lppl.txt
and version 1.3c or later is part of all distributions of LaTeX
version 2008 or later.

This file has the LPPL maintenance status "maintained".

This file may only be distributed together with a copy of the LaTeX
base system. You may however distribute the LaTeX base system without
such generated files.

The list of all files belonging to the LaTeX base distribution is
given in the file `manifest.txt'. See also `legal.txt' for additional
information.

The list of derived (unpacked) files belonging to the distribution
and covered by LPPL is defined by the unpacking scripts (with
extension .ins) which are part of the distribution.
\endpreamble

\keepsilent
\usedir{tex/latex/base}

\Msg{*** Generating compatibility mode files ***}

\generate{
        \usedir{tex/latex/base}
        \file{latex209.def}{
             \from{latex209.dtx}{head}
             \from{oldlfont.dtx}{latex209}
             \from{latex209.dtx}{tail}}
        \file{article.sty}{
             \from{latex209.dtx}{article}}
        \file{book.sty}{
             \from{latex209.dtx}{book}}
        \file{report.sty}{
             \from{latex209.dtx}{report}}
        \file{letter.sty}{
             \from{latex209.dtx}{letter}}
        \file{slides.sty}{
             \from{latex209.dtx}{slides}}
        \file{fleqn.sty}{
             \from{latex209.dtx}{fleqn}}
        \file{leqno.sty}{
             \from{latex209.dtx}{leqno}}
        \file{openbib.sty}{
             \from{latex209.dtx}{openbib}}
        \file{bezier.sty}{
             \from{latex209.dtx}{bezier}}
        \file{t1enc.sty}{
             \from{latex209.dtx}{t1enc}}
        }

\ifToplevel{
\Msg{***********************************************************}
\Msg{*}
\Msg{* To finish the installation you have to move the following}
\Msg{* files into a directory searched by TeX:}
\Msg{*}
\Msg{* \space\space latex209.def *.sty }
\Msg{*}
\Msg{* To produce the documentation run the files ending with}
\Msg{* `.drv' through LaTeX.}
\Msg{*}
\Msg{* Happy TeXing}
\Msg{***********************************************************}
}

\endbatchfile
\makeatother
  \documentclass}
\@onlypreamble\documentstyle
%    \end{macrocode}
% \end{macro}
%
% \begin{macro}{\RequirePackage}
% Load package if not already loaded.
%    \begin{macrocode}
\def\RequirePackage{%
  \@fileswithoptions\@pkgextension}
\@onlypreamble\RequirePackage
%    \end{macrocode}
% \end{macro}
%
% \begin{macro}{\LoadClass}
% Load class.
%    \begin{macrocode}
\def\LoadClass{%
  \ifx\@currext\@pkgextension
     \@latex@error
      {\noexpand\LoadClass in package file}%
      {You may only use \noexpand\LoadClass in a class file.}%
  \fi
  \@fileswithoptions\@clsextension}
\@onlypreamble\LoadClass
%    \end{macrocode}
% \end{macro}
%
% \begin{macro}{\@loadwithoptions}
% \changes{v1.0t}{1995/11/14}{macro added}
% \changes{v1.4c}{2021/06/06}
%         {handle raw options for gh/580}
% Pass the current option list on to a class or package.
% |#1| is |\@|\emph{cls-or-pkg}|extension|,
% |#2| is |\RequirePackage| or |\LoadClass|,
% |#3| is the class or package to be loaded.
%    \begin{macrocode}
%</2ekernel>
%<latexrelease>\IncludeInRelease{2021/06/01}%
%<latexrelease>                 {\@loadwithoptions}{Raw option lists load with options}%
%<*2ekernel|latexrelease>
\def\@loadwithoptions#1#2#3{%
  \expandafter\let\csname opt@#3.#1\expandafter\endcsname
       \csname opt@\@currname.\@currext\endcsname
  \expandafter\let\csname @raw@opt@#3.#1\expandafter\endcsname
       \csname @raw@opt@\@currname.\@currext\endcsname
   #2{#3}}
%</2ekernel|latexrelease>
%<latexrelease>\EndIncludeInRelease
%    \end{macrocode}
%
%    \begin{macrocode}
%<latexrelease>\IncludeInRelease{0000/00/00}
%<latexrelease>                 {\@loadwithoptions}{Raw option lists load with options}%
%<latexrelease>\def\@loadwithoptions#1#2#3{%
%<latexrelease>  \expandafter\let\csname opt@#3.#1\expandafter\endcsname
%<latexrelease>       \csname opt@\@currname.\@currext\endcsname
%<latexrelease>   #2{#3}}
%<latexrelease>\EndIncludeInRelease
%<*2ekernel>
%    \end{macrocode}
%
%    \begin{macrocode}
\@onlypreamble\@loadwithoptions
%    \end{macrocode}
% \end{macro}
%
%
% \begin{macro}{\LoadClassWithOptions}
% \changes{v1.0t}{1995/11/14}{macro added}
% Load class `|#1|' with the current option list.
%    \begin{macrocode}
\def\LoadClassWithOptions{%
  \@loadwithoptions\@clsextension\LoadClass}
\@onlypreamble\LoadClassWithOptions
%    \end{macrocode}
% \end{macro}
%
% \begin{macro}{\RequirePackageWithOptions}
% \changes{v1.0t}{1995/11/14}{macro added}
% \changes{v1.0v}{1996/10/04}{Reset \cs{@unprocessedoptions} for /2269}
% Load package `|#1|' with the current option list.
%    \begin{macrocode}
%</2ekernel>
%<*2ekernel|latexrelease>
%<latexrelease>\IncludeInRelease{2020/10/01}%
%<latexrelease>                 {\RequirePackageWithOptions}{Unused options issue}%
\def\RequirePackageWithOptions{%
%    \end{macrocode}
%    The resetting of the unprocessed options is now done on a par package basis.
% \changes{v1.3k}{2020/04/07}{Use different method to ignore
%    unprocessed options (gh/22)}
%    \begin{macrocode}
  \AtEndOfPackage{\expandafter\let
                    \csname unprocessedoptions-\@currname.\@currext\endcsname
                    \relax}%
  \@loadwithoptions\@pkgextension\RequirePackage}
\@onlypreamble\RequirePackageWithOptions
%</2ekernel|latexrelease>
%<latexrelease>\EndIncludeInRelease
%    \end{macrocode}
%
%    \begin{macrocode}
%<latexrelease>\IncludeInRelease{0000/00/00}%
%<latexrelease>                 {\RequirePackageWithOptions}{Unused options issue}%
%<latexrelease>
%<latexrelease>\def\RequirePackageWithOptions{%
%<latexrelease>  \AtEndOfPackage{\let\@unprocessedoptions\relax}%
%<latexrelease>  \@loadwithoptions\@pkgextension\RequirePackage}
%<latexrelease>\EndIncludeInRelease
%<*2ekernel>
%    \end{macrocode}
% \end{macro}
%
%






%
% \begin{macro}{\usepackage}
%    To begin with, |\usepackage| produces an error.  This is reset by
%    |\documentclass|.
% \changes{v0.2o}{1993/12/13}
%         {Fixed error handling}
% \changes{v1.0h}{1994/05/23}{Remove argument if possible}
%    \begin{macrocode}
\def\usepackage#1#{%
  \@latex@error
    {\noexpand \usepackage before \string\documentclass}%
    {\noexpand \usepackage may only appear in the document
      preamble, i.e.,\MessageBreak
      between \noexpand\documentclass and
      \string\begin{document}.}%
  \@gobble}
\@onlypreamble\usepackage
%    \end{macrocode}
% \end{macro}
%
% \begin{macro}{\NeedsTeXFormat}
% Check that the document is running on the correct system.
% \changes{v0.2a}{1993/11/14}
%         {made more robust for alternative syntax for other formats.}
% \changes{v0.2c}{1993/11/17}
%         {Name changed from \cs{NeedsFormat}}
% \changes{v0.2d}{1993/11/18}
%         {\cs{fmtname} \cs{fmtversion} not \cs{@}\ldots}
%    \begin{macrocode}
\def\NeedsTeXFormat#1{%
  \def\reserved@a{#1}%
  \ifx\reserved@a\fmtname
    \expandafter\@needsformat
  \else
     \@latex@error{This file needs format `\reserved@a'%
       \MessageBreak but this is `\fmtname'}{%
       The current input file will not be processed
       further,\MessageBreak
       because it was written for some other flavor of
       TeX.\MessageBreak\@ehd}%
%    \end{macrocode}
%    If the file is not meant to be processed by \LaTeXe{} we stop
%    inputting it, but we do not end the run. We just end inputting
%    the current file.
% \changes{v1.0h}{1994/05/23}
%     {Don't stop completely when format is wrong}
%    \begin{macrocode}
     \endinput \fi}
\@onlypreamble\NeedsTeXFormat
%    \end{macrocode}
%
%    \begin{macrocode}
\def\@needsformat{%
  \@ifnextchar[%]
    \@needsf@rmat
    {}}
\@onlypreamble\@needsformat
%    \end{macrocode}
%
% \changes{v1.0b}{1994/05/04}
%         {Changed wording of the warning}
%    \begin{macrocode}
\def\@needsf@rmat[#1]{%
    \@ifl@t@r\fmtversion{#1}{}%
    {\@latex@warning@no@line
        {You have requested release `#1' of LaTeX,\MessageBreak
         but only release `\fmtversion' is available}}}
\@onlypreamble\@needsf@rmat
%    \end{macrocode}
% \end{macro}
%
% \begin{macro}{\zap@space}
% |\zap@space foo|\meta{space}|\@empty| removes all spaces from |foo|
% that are not protected by |{ }| groups.
%    \begin{macrocode}
\def\zap@space#1 #2{%
  #1%
  \ifx#2\@empty\else\expandafter\zap@space\fi
  #2}
%    \end{macrocode}
% \end{macro}
%
% \begin{macro}{\@fileswithoptions}
% \changes{v1.5i}{2024/01/30}{Test group level}
% The common part of |\documentclass| and |\usepackage|.
%    \begin{macrocode}
%</2ekernel>
%<latexrelease>\IncludeInRelease{2024/06/01}%
%<latexrelease>                 {\@fileswithoptions}{Check Group}%
%<*2ekernel|latexrelease>
\def\@fileswithoptions#1{%
    \ifnum\currentgrouplevel>\z@
     \@latex@error
      {Loading a class or package in a group}%
      {Classes and packages should only be loaded at the top level}%
  \fi
  \@ifnextchar[%]
    {\@fileswith@ptions#1}%
    {\@fileswith@ptions#1[]}}
%</2ekernel|latexrelease>
%<latexrelease>\EndIncludeInRelease
%<latexrelease>\IncludeInRelease{0000/00/00}%
%<latexrelease>                 {\@fileswithoptions}{Check Group}%
%<latexrelease>\def\@fileswithoptions#1{%
%<latexrelease>  \@ifnextchar[%]
%<latexrelease>    {\@fileswith@ptions#1}%
%<latexrelease>    {\@fileswith@ptions#1[]}}
%<latexrelease>\EndIncludeInRelease
%<*2ekernel>
%    \end{macrocode}
%    \begin{macrocode}
\@onlypreamble\@fileswithoptions
%    \end{macrocode}
%
% \changes{v0.2f}{1993/11/22}
%         {Made the default [] not [\cs{@unknownversion}]}
% \changes{v1.1h}{2007/08/05}
%         {Prevent loss of brackets PR/3965}
%    \begin{macrocode}
\def\@fileswith@ptions#1[#2]#3{%
  \@ifnextchar[%]
  {\@fileswith@pti@ns#1[{#2}]#3}%
  {\@fileswith@pti@ns#1[{#2}]#3[]}}
\@onlypreamble\@fileswith@ptions
%    \end{macrocode}
% Then we do some work.
%
% First of all, we define the global variables.
% Then we look to see if the file has already been loaded.
% If it has, we check that it was first loaded with at least the current
% options.
% If it has not, we add the current options to the package options,
% set the default version to be |0000/00/00|, and load the file if we
% can find it.
% Then we check the version number.
%
% Finally, we restore the old file name, reset the default option,
% and we set the catcode of |@|.
%
% For classes, we can immediately process the file. For other types,
% |#2| could be a comma separated list, so loop through, processing
% each one separately.
% \changes{v0.2q}{1993/12/17}
%         {Add \cs{@compatibility} hook}
% \changes{v0.2s}{1994/01/17}
%         {Modify to reduce parameter stack usage}
% \changes{v0.2y}{1994/02/07}
%         {Run \cs{@compatibility} on the first class to start
%          (not the first to finish) }
% \changes{v0.2z}{1994/02/10}
%         {Renamed \cs{@compatibility} to \cs{@documentclasshook}.
%          ASAJ.}
% \changes{v1.1h}{2007/08/05}
%         {Prevent loss of brackets PR/3965}
% \changes{v2.1b}{2016/11/09}
%         {Improve \cs{ifx} tests PR/4497}
% \changes{v1.3x}{2021/02/18}{save raw class option list (gh/85)}
% \changes{v1.5e}{2022/10/22}{Use \cs{protected@xdef}.}
%    \begin{macrocode}
%</2ekernel>
%<latexrelease>\IncludeInRelease{2020/10/01}%
%<latexrelease>        {\@fileswith@pti@ns}{ifx tests in \@fileswith@pti@ns}%
%<*2ekernel|latexrelease>
\def\@fileswith@pti@ns#1[#2]#3[#4]{%
  \ifx#1\@clsextension
    \ifx\@classoptionslist\relax
      \protected@xdef\@classoptionslist{\zap@space#2 \@empty}%
%    \end{macrocode}
% Save raw class list.
%    \begin{macrocode}
      \gdef\@raw@classoptionslist{#2}%
%    \end{macrocode}
%
%    \begin{macrocode}
      \def\reserved@a{%
        \@onefilewithoptions#3[{#2}][{#4}]#1%
        \@documentclasshook}%
    \else
      \def\reserved@a{%
        \@onefilewithoptions#3[{#2}][{#4}]#1}%
    \fi
  \else
%    \end{macrocode}
% build up a list of calls to |\@onefilewithoptions|
% (one for each package) without thrashing the parameter stack.
%    \begin{macrocode}
    \def\reserved@b##1,{%
%    \end{macrocode}
% If |#1| is |\@nnil| we have reached the end of the list
% (older version used |\@nil| here but |\@nil| is undefined so |\ifx|
% equal to all undefined commands)
%    \begin{macrocode}
      \ifx\@nnil##1\relax\else
%    \end{macrocode}
%  If |\ifx\@nnil##1\@nnil| is true then |#1| is (presumably) empty
% (Older code used |\relax| which is slightly easier to get into |#1|
% by mistake, which would spoil this test.)
%    \begin{macrocode}
        \ifx\@nnil##1\@nnil\else
%    \end{macrocode}
%
% \changes{v1.4d}{2021/07/12}{add \cs{unexpanded}}
%    \begin{macrocode}
         \noexpand\@onefilewithoptions##1[{\unexpanded{#2}}][{#4}]%
         \noexpand\@pkgextension
        \fi
        \expandafter\reserved@b
      \fi}%
      \edef\reserved@a{\zap@space#3 \@empty}%
      \edef\reserved@a{\expandafter\reserved@b\reserved@a,\@nnil,}%
  \fi
  \reserved@a}
%</2ekernel|latexrelease>
%<latexrelease>\EndIncludeInRelease
%<latexrelease>\IncludeInRelease{2017/01/01}%
%<latexrelease>        {\@fileswith@pti@ns}{ifx tests in \@fileswith@pti@ns}%
%<latexrelease>\def\@fileswith@pti@ns#1[#2]#3[#4]{%
%<latexrelease>  \ifx#1\@clsextension
%<latexrelease>    \ifx\@classoptionslist\relax
%<latexrelease>      \xdef\@classoptionslist{\zap@space#2 \@empty}%
%<latexrelease>      \def\reserved@a{%
%<latexrelease>        \@onefilewithoptions#3[{#2}][{#4}]#1%
%<latexrelease>        \@documentclasshook}%
%<latexrelease>    \else
%<latexrelease>      \def\reserved@a{%
%<latexrelease>        \@onefilewithoptions#3[{#2}][{#4}]#1}%
%<latexrelease>    \fi
%<latexrelease>  \else
%<latexrelease>    \def\reserved@b##1,{%
%<latexrelease>      \ifx\@nnil##1\relax\else
%<latexrelease>        \ifx\@nnil##1\@nnil\else
%<latexrelease>         \noexpand\@onefilewithoptions##1[{#2}][{#4}]%
%<latexrelease>         \noexpand\@pkgextension
%<latexrelease>        \fi
%<latexrelease>        \expandafter\reserved@b
%<latexrelease>      \fi}%
%<latexrelease>      \edef\reserved@a{\zap@space#3 \@empty}%
%<latexrelease>      \edef\reserved@a{\expandafter\reserved@b\reserved@a,\@nnil,}%
%<latexrelease>  \fi
%<latexrelease>  \reserved@a}
%    \end{macrocode}
%
%    \begin{macrocode}
%<latexrelease>\EndIncludeInRelease
%<latexrelease>\IncludeInRelease{0000/00/00}%
%<latexrelease>        {\@fileswith@pti@ns}{ifx tests in \@fileswith@pti@ns}%
%<latexrelease>\def\@fileswith@pti@ns#1[#2]#3[#4]{%
%<latexrelease>  \ifx#1\@clsextension
%<latexrelease>    \ifx\@classoptionslist\relax
%<latexrelease>      \xdef\@classoptionslist{\zap@space#2 \@empty}%
%<latexrelease>      \def\reserved@a{%
%<latexrelease>        \@onefilewithoptions#3[{#2}][{#4}]#1%
%<latexrelease>        \@documentclasshook}%
%<latexrelease>    \else
%<latexrelease>      \def\reserved@a{%
%<latexrelease>        \@onefilewithoptions#3[{#2}][{#4}]#1}%
%<latexrelease>    \fi
%<latexrelease>  \else
%<latexrelease>    \def\reserved@b##1,{%
%<latexrelease>      \ifx\@nil##1\relax\else
%<latexrelease>        \ifx\relax##1\relax\else
%<latexrelease>         \noexpand\@onefilewithoptions##1[{#2}][{#4}]%
%<latexrelease>         \noexpand\@pkgextension
%<latexrelease>        \fi
%<latexrelease>        \expandafter\reserved@b
%<latexrelease>      \fi}%
%<latexrelease>      \edef\reserved@a{\zap@space#3 \@empty}%
%<latexrelease>      \edef\reserved@a{%
%<latexrelease>        \expandafter\reserved@b\reserved@a,\@nil,}%
%<latexrelease>  \fi
%<latexrelease>  \reserved@a}
%<latexrelease>\EndIncludeInRelease
%<*2ekernel>
%    \end{macrocode}
%
%    \begin{macrocode}
\@onlypreamble\@fileswith@pti@ns
%    \end{macrocode}
%
% \begin{macro}{\load@onefilewithoptions}
%   This macro is used when loading packages or classes.
%
% Have the main argument as |#1|, so we only need one |\expandafter|
% above.
% \changes{v0.2a}{1993/11/14}
%         {Moved resetting of \cs{default@ds}, \cs{ds@} and
%         \cs{@declaredoptions} here, from the end of
%         \cs{ProcessOptions}.}
% \changes{v0.2f}{1993/11/22}
%         {Made the initial version [] not [\cs{@unknownversion}]}
% \changes{v0.2m}{1993/12/07}
%         {Reset \cs{CurrentOption}}
% \changes{v1.3d}{2019/10/18}{Initialize \cs{...-h@@k} only when loading
%                             the package or class (gh/198)}
% \changes{v1.5h}{2023/04/14}{Define \cs{load@onefilewithoptions} when
%                             in \pkg{latexrelease} (gh/992)}
%    \begin{macrocode}
%</2ekernel>
%<*2ekernel|latexrelease>
%<latexrelease>\IncludeInRelease{2020/10/01}%
%<latexrelease>      {\@onefilewithoptions}{Hooks and unused options issue}%
%    \end{macrocode}
%
%   Here this macro is called \cs{@onefilewithoptions}, but further
%   ahead in this file it is renamed to \cs{load@onefilewithoptions},
%   and \cs{@onefilewithoptions} becomes a wrapper around this, used for
%   bookkeeping when rolling back.  Therefore, when in
%   \pkg{latexrelease}, we need to define \cs{load@onefilewithoptions}
%   instead, thus the extra guarded \cs{def} line below:
%    \begin{macrocode}
%<*2ekernel>
\def\@onefilewithoptions#1[#2][#3]#4{%
%</2ekernel>
%<latexrelease>\def\load@onefilewithoptions#1[#2][#3]#4{%
%    \end{macrocode}
%
%    We have to sanitise file names, so that something like
% \begin{verbatim}
%   \usepackage{some/local/path/array}
%   \usepackage{array}
% \end{verbatim}
%    won't load |array.sty| twice.  It is remotely possible that
%    those are two different files, but as a matter of principles, we
%    will consider that the base file name uniquely identifies a
%    package, regardless of where it lives.  This assumption already
%    holds for file hooks, for example, which address the hook to a file
%    by its base name only.
%
%    We'll use \cs{@expl@@@filehook@set@curr@file@@nNN} to parse the
%    file name and return the \meta{path} and \meta{base+ext} in
%    separate token lists.  Further ahead, most operations use
%    \cs{@currname} which doesn't have a path attached to it;  only few
%    actions prepend \cs{@currpath} to \cs{@currname} (namely loading,
%    as we have to respect the given path).
%
%    A file substitution isn't followed just yet because at this point
%    we are parsing user input, so the file is still what the user
%    asked for, and not the file actually loaded.
%    \begin{macrocode}
  \@expl@@@filehook@set@curr@file@@nNN{#1.#4}\reserved@a\reserved@b
  \edef\reserved@c{\def\noexpand\reserved@c####1%
    \detokenize\expandafter{\expanded{.#4}}%
    \noexpand\@nil{\def\noexpand\reserved@a{####1}}}\reserved@c
  \expandafter\reserved@c\reserved@a\@nil
  \@pushfilename
  \xdef\@currname{\string@makeletter\reserved@a}%
  \xdef\@currpath{\ifx\reserved@b\@empty\else\reserved@b/\fi}%
  \global\let\@currext#4%
%    \end{macrocode}
%    The command \cs{ver@\meta{file}.\meta{ext}} is used to signal that
%    a package is already loaded, either because it is in fact loaded, or
%    because it's loading was suppressed.  In minimal installations, said
%    package may not exist but still have its loading suppressed with
%    \cs{ver@\meta{file}.\meta{ext}}, so before checking if the file
%    exists we have to check that we do need to load it with
%    \cs{@ifl@aded}.  If we don't, then there's no point in checking for
%    a typo or load-disabling.
%    \begin{macrocode}
  \@ifl@aded\@currext\@currname
%    \end{macrocode}
% \changes{v1.5b}{2022/03/18}{Switch to \cs{ProcessKeyOptions}}
% \changes{v1.5c}{2022/06/20}{Pass raw options to \cs{ProcessKeyOptions}}
% \changes{v1.5e}{2022/10/20}
%         {Define key option handler in \pkg{ltkeys}}
%    In the current preferred approach, a key family name will exist for
%    processing using \pkg{ltkeys}. In that case, we replace the previous
%    package options with the new ones, then call the key handler.
%    Otherwise, we use the more classical clash handler.
%    \begin{macrocode}
    {%
      \@ifundefined{opt@handler@\@currname.\@currext}
        {\@onefilewithoptions@clashchk{#2}}
        {%
%    \end{macrocode}
% \changes{v1.5d}{2022/10/10}{Use \cs{protected@edef}.}
%    \begin{macrocode}
          \expandafter\protected@edef
            \csname opt@\@currname.\@currext\endcsname
            {\zap@space#2 \@empty}%
%    \end{macrocode}
% \changes{v1.5j}{2024/03/22}
%         {Apply one-step expansion to raw option list,
%          to be consistent with change for gh/580 (gh/1298).}
%    \begin{macrocode}
          \expandafter\def
            \csname @raw@opt@\@currname.\@currext\expandafter\endcsname
            \expandafter{#2}%
          \@nameuse{opt@handler@\@currname.\@currext}%
        }%
    }%
    {\makeatletter
%    \end{macrocode}
%    The next line seems to be necessary for 2.09 compatibility (the
%    way the code is written there) This seems questionable and should be
%    look at as in 2e it is definitely unnecessary at this point!
%    \begin{macrocode}
     \@reset@ptions
%    \end{macrocode}
%    First we take the \meta{name} and \meta{ext} given in the argument
%    and check if the file exists, and issue an error otherwise asking
%    for a correction with \cs{@missingfileerror}.  For checking if the
%    file exists we use \cs{@currpath} (usually empty) before
%    \cs{@currname}.
%    \begin{macrocode}
     \IfFileExists{\@currpath\@currname.\@currext}{}%
       {\@missing@onefilewithoptions{#2}}%
%    \end{macrocode}
%    If \cs{@currname} is empty (the user replied to the ``Enter file
%    name'' prompt with \meta{RETURN}), so stop here
%    (do \cs{@popfilename} to pop the item just added above).
%
%    This \cs{@gobble} omits the date check at the end.
%    \begin{macrocode}
     \ifx\@currname\@empty
       \expandafter\@gobble
     \else
%    \end{macrocode}
%    If the file exists, check if it was load-prevented, and otherwise
%    do the bookkeeping with \cs{@filehook@file@push}
%    then call \cs{set@curr@file} to set \cs{@curr@file} (and do any
%    required substitution), then actually load the class/package with
%    \cs{load@onefile@withoptions}.  \cs{set@curr@file} also needs the
%    file path.
%    \begin{macrocode}
       \@disable@packageload@do{\@currname.\@currext}%
         {\@expl@@@filehook@file@push@@
          \set@curr@file{\@currpath\@currname.\@currext}%
          \@filehook@set@CurrentFile
%    \end{macrocode}
%    \changes{v1.3q}{2020/09/06}
%         {Save \cs{@currpkg@reqd} so that we don't lose track of
%          package substitutions.}
%    The \cs{set@curr@file} line above might have replaced the file, so
%    \cs{@currname} and \cs{@currext} may no longer hold the actual
%    package being loaded, so in that case we need to update these two
%    token lists (\cs{@curr@file} holds the file name after replacement,
%    so we parse that).
%
%    The requested file is saved in \cs{@currpkg@reqd} to be used in
%    \cs{InputIfFileExists} later:  if the updated \cs{@currname} and
%    \cs{@currext} are used we lose track of the substitution, so
%    \cs{CurrentFile} and \cs{CurrentFileUsed} will be (incorrectly)
%    the same.
%
%    \changes{v1.3t}{2020-10-11}{Restore \cs{@currpkg@reqd} after
%      finished loading a package file (gh/408).}
%    \begin{macrocode}
          \expandafter\@swaptwoargs\expandafter
            {\expandafter{\@currpkg@reqd}}%
            {% <
%    \end{macrocode}
%    \cs{@currpkg@reqd} doesn't take a path because it is used later to
%    assign \cs[no-index]{opt@...} and \cs[no-index]{ver@...}.
%    \begin{macrocode}
          \edef\@currpkg@reqd{\@currname.\@currext}%
          \ifx\CurrentFile\CurrentFileUsed
          \else
            \filename@parse\@curr@file
            \edef\@currpath{\string@makeletter\filename@area}%
            \edef\@currname{\string@makeletter\filename@base}%
            \edef\@currext{\string@makeletter\filename@ext}%
          \fi
          \load@onefile@withoptions{#2}%
          \def\@currpkg@reqd%{\@currpkg@reqd}
            }% >
%    \end{macrocode}
%    Now just clean up and exit.
%    \begin{macrocode}
          \@expl@@@filehook@file@pop@@}%
       \expandafter\@firstofone
     \fi}%
%    \end{macrocode}
%   Except in the case where \cs{@currname} is empty, the date is
%   checked against the date marked in the package file:
%    \begin{macrocode}
    {\@ifl@ter\@currext{\@currname}{#3}{}%
      {\@latex@warning@no@line
        {You have requested,\on@line,
         version\MessageBreak
           `#3' of \@cls@pkg\space \@currname,\MessageBreak
         but only version\MessageBreak
          `\csname ver@\@currname.\@currext\endcsname'\MessageBreak
         is available}}%
%    \end{macrocode}
% \changes{v0.2c}{1993/11/17}
%         {Added trap for two \cs{LoadClass} commands.}
%    \begin{macrocode}
     \ifx\@currext\@clsextension\let\LoadClass\@twoloadclasserror\fi}%
    \@popfilename
    \@reset@ptions}
%    \end{macrocode}
%
% \begin{macro}{\@onefilewithoptions@clashchk}
%    If the package is already loaded, check that there were no option
%    clashes.
% \changes{v1.1b}{1998/05/07}
%         {Modify help message for latex/2805}
% \changes{v1.5a}{2021/11/30}
%         {Separated out from \cs{@onefilewithoptions}}
%    \begin{macrocode}
\def\@onefilewithoptions@clashchk#1{%
  \@if@ptions\@currext{\@currname}{#1}{}%
      {\@latex@error
        {Option clash for \@cls@pkg\space \@currname}%
        {The package \@currname\space has already been loaded
         with options:\MessageBreak
         \space\space[\@ptionlist{\@currname.\@currext}]\MessageBreak
         There has now been an attempt to load it
          with options\MessageBreak
         \space\space[#1]\MessageBreak
         Adding the global options:\MessageBreak
         \space\space
              \@ptionlist{\@currname.\@currext},#1\MessageBreak
         to your \noexpand\documentclass declaration may fix this.%
         \MessageBreak
         Try typing \space <return> \space to proceed.}}%
     \@firstofone}
%    \end{macrocode}
% \end{macro}
%
%    \begin{macrocode}
\let\@currpkg@reqd\@empty
%    \end{macrocode}
%
%    \begin{macrocode}
\@onlypreamble\@onefilewithoptions
%    \end{macrocode}
%
%    The kernel no longer uses \cs{@unprocessedoptions}
%    \begin{macrocode}
\let\@unprocessedoptions\@undefined
%    \end{macrocode}
% \end{macro}
%
% \begin{macro}{\@missing@onefilewithoptions}
%    Now the action taken when a file is not found.  Path must be
%    included here as it eventually leads to a file lookup.
%    \begin{macrocode}
\def\@missing@onefilewithoptions#1{%
  \@missingfileerror{\@currpath\@currname}\@currext
  \global\let\@currpath\@missingfile@area
  \global\let\@currname\@missingfile@base
  \global\let\@currext\@missingfile@ext}
%    \end{macrocode}
% \end{macro}
%
% \begin{macro}{\load@onefile@withoptions}
%    Now the code that actually does the file loading:
%    \begin{macrocode}
\def\load@onefile@withoptions#1{%
  \let\CurrentOption\@empty
  \@reset@ptions
%    \end{macrocode}
% Grab everything in a macro, so the parameter stack is popped before
% any processing begins.
% \changes{v0.2s}{1994/01/17}
%         {Modify to reduce parameter stack usage}
%    \begin{macrocode}
  \def\reserved@a{%
    \@pass@ptions\@currext{#1}{\@currname}%
%    \end{macrocode}
% \changes{v1.3u}{2020/11/20}
%         {Copy option list to the requested package.}
% \changes{v1.4c}{2021/06/06}
%         {Copy raw options for gh/580}
%    \begin{macrocode}
    \expandafter\let
      \csname opt@\@currpkg@reqd\expandafter\endcsname
      \csname opt@\@currname.\@currext\endcsname
    \expandafter\let
      \csname @raw@opt@\@currpkg@reqd\expandafter\endcsname
      \csname @raw@opt@\@currname.\@currext\endcsname
    \global\expandafter
    \let\csname ver@\@currname.\@currext\endcsname\@empty
%    \end{macrocode}
%    We initialize \cs{...-h@@k} here and only if we load the file so that it
%    remains undefined otherwise.
%    \begin{macrocode}
    \expandafter\let\csname\@currname.\@currext-h@@k\endcsname\@empty
%    \end{macrocode}
%    When the current extension is \cs{@pkgextension} we are loading a
%    package otherwise, if it is \cs{@clsextension}, a class, so
%    depending on that we execute different hooks.  If the extension is
%    neither, then it is another type of file without special hooks.
% \changes{v1.4e}{2021/07/23}{Make class/name/before a one-time hook}
% \changes{v1.4e}{2021/07/23}{Make package/name/before a one-time hook}
%    \begin{macrocode}
%-----------------------------------------
    \ifx\@currext\@pkgextension
      \UseHook{package/before}%
      \UseOneTimeHook{package/\@currname/before}%
    \else
      \ifx\@currext\@clsextension
        \UseHook{class/before}%
        \UseOneTimeHook{class/\@currname/before}%
      \fi
    \fi
%    \end{macrocode}
%    Now actually load the file (at this point we are certain it exists,
%    but use \cs{InputIfFileExists} so that file hooks are executed).
%    \cs{@currpath} is needed here too.
%    \begin{macrocode}
    \InputIfFileExists{\@currpath\@currpkg@reqd}{}%
      {\@latex@error
        {The \@cls@pkg\space\@currpkg@reqd\space failed to load}\@ehd}%
%-----------------------------------------
%    \end{macrocode}
%    In older versions of the code |\@unprocessedoptions| would
%    generate an error for each specified
%    option in a package unless a |\ProcessOptions| has appeared in the
%    package file.
% \changes{v0.2v}{1994/01/29}
%         {All options raise error if no \cs{ProcessOptions} appears}
% \changes{v0.2x}{1994/02/02}
%         {Only run the hook and options check if the file was
%    loaded.}
%
%    This has changed in 2020. We now use a separate macro per package
%    to avoid interference in case of nested packages.  The whole
%    code for handling this issue (GitHub 22) was provided by Hironobu
%    Yamashita, thanks for that.
% \changes{v1.3k}{2020/04/07}{Use different method to ignore
%    unprocessed options (gh/22)}
%    \begin{macrocode}
    \expandafter\let\csname unprocessedoptions-\@currname.\@currext\endcsname
                    \@@unprocessedoptions
    \csname\@currname.\@currext-h@@k\endcsname
    \expandafter\let\csname\@currname.\@currext-h@@k\endcsname
              \@undefined
%    \end{macrocode}
%    Catch the case where the packages has handled the options and
%    redefined \cs{@unprocessedoptions} to \cs{relax} (old interface).
%    In that case no error should be produced.
% \changes{v1.3k}{2020/04/07}{Use different method to ignore
%    unprocessed options (gh/22)}
%    \begin{macrocode}
    \ifx\@unprocessedoptions\relax
      \let\@unprocessedoptions\@undefined
%    \end{macrocode}
%    Otherwise run the per package set of unused options.
%    \begin{macrocode}
    \else
      \csname unprocessedoptions-\@currname.\@currext\endcsname
    \fi
%    \end{macrocode}
%    In either case we drop the macro afterwards as it is no longer needed.
%    \begin{macrocode}
    \expandafter\let
        \csname unprocessedoptions-\@currname.\@currext\endcsname
       \@undefined
%    \end{macrocode}
%    And same procedure, James, when we are finished loading, except
%    that the hook order is now reversed.
% \changes{v1.4e}{2021/07/23}{Make class/name/after a one-time hook}
% \changes{v1.4e}{2021/07/23}{Make package/name/after a one-time hook}
%    \begin{macrocode}
%-----------------------------------------
    \ifx\@currext\@pkgextension
      \UseOneTimeHook{package/\@currname/after}%
      \UseHook{package/after}%
    \else
      \ifx\@currext\@clsextension
        \UseOneTimeHook{class/\@currname/after}%
        \UseHook{class/after}%
      \fi
    \fi}%
%-----------------------------------------
  \@ifl@aded\@currext\@currname{}{\reserved@a}}
%    \end{macrocode}
%
% \changes{v1.4f}{2021/08/25}{Declare non-generic package and class hooks}
%   Now declare the non-generic package and class hooks used above:
%    \begin{macrocode}
\NewHook{package/before}
\NewHook{class/before}
\NewReversedHook{package/after}
\NewReversedHook{class/after}
%    \end{macrocode}
%  \end{macro}
%
%    \begin{macrocode}
%</2ekernel|latexrelease>
%<latexrelease>\EndIncludeInRelease
%<latexrelease>\IncludeInRelease{0000/00/00}%
%<latexrelease>      {\@onefilewithoptions}{Hooks and unused options issue}%
%<latexrelease>
%    \end{macrocode}
%    Because of the way \cs{@onfilewithoptions} is changed for
%    rollback handling below we have to define
%    \cs{load@onefilewithoptions} when rolling back!
%    \begin{macrocode}
%<latexrelease>\def\load@onefilewithoptions#1[#2][#3]#4{%
%<latexrelease>  \@pushfilename
%<latexrelease>  \xdef\@currname{#1}%
%<latexrelease>  \global\let\@currext#4%
%<latexrelease>  \let\CurrentOption\@empty
%<latexrelease>  \@reset@ptions
%<latexrelease>  \makeatletter
%<latexrelease>  \def\reserved@a{%
%<latexrelease>    \@ifl@aded\@currext{#1}%
%<latexrelease>      {\@if@ptions\@currext{#1}{#2}{}%
%<latexrelease>        {\@latex@error
%<latexrelease>            {Option clash for \@cls@pkg\space #1}%
%<latexrelease>            {The package #1 has already been loaded
%<latexrelease>             with options:\MessageBreak
%<latexrelease>             \space\space[\@ptionlist{#1.\@currext}]\MessageBreak
%<latexrelease>             There has now been an attempt to load it
%<latexrelease>              with options\MessageBreak
%<latexrelease>             \space\space[#2]\MessageBreak
%<latexrelease>             Adding the global options:\MessageBreak
%<latexrelease>             \space\space
%<latexrelease>                  \@ptionlist{#1.\@currext},#2\MessageBreak
%<latexrelease>             to your \noexpand\documentclass declaration may fix this.%
%<latexrelease>             \MessageBreak
%<latexrelease>             Try typing \space <return> \space to proceed.}}}%
%<latexrelease>      {\@pass@ptions\@currext{#2}{#1}%
%<latexrelease>       \global\expandafter
%<latexrelease>       \let\csname ver@\@currname.\@currext\endcsname\@empty
%<latexrelease>       \expandafter\let\csname\@currname.\@currext-h@@k\endcsname\@empty
%<latexrelease>       \InputIfFileExists
%<latexrelease>         {\@currname.\@currext}%
%<latexrelease>         {}%
%<latexrelease>         {\@missingfileerror\@currname\@currext}%
%<latexrelease>    \let\@unprocessedoptions\@@unprocessedoptions
%<latexrelease>    \csname\@currname.\@currext-h@@k\endcsname
%<latexrelease>    \expandafter\let\csname\@currname.\@currext-h@@k\endcsname
%<latexrelease>              \@undefined
%<latexrelease>    \@unprocessedoptions}%
%<latexrelease>    \@ifl@ter\@currext{#1}{#3}{}%
%<latexrelease>      {\@latex@warning@no@line
%<latexrelease>         {You have requested,\on@line,
%<latexrelease>          version\MessageBreak
%<latexrelease>            `#3' of \@cls@pkg\space #1,\MessageBreak
%<latexrelease>          but only version\MessageBreak
%<latexrelease>           `\csname ver@#1.\@currext\endcsname'\MessageBreak
%<latexrelease>          is available}}%
%<latexrelease>    \ifx\@currext\@clsextension\let\LoadClass\@twoloadclasserror\fi
%<latexrelease>    \@popfilename
%<latexrelease>    \@reset@ptions}%
%<latexrelease>  \reserved@a}
%<latexrelease>
%<latexrelease>\let \load@onefile@withoptions    \@undefined
%<latexrelease>\let \@missing@onefilewithoptions \@undefined
%<latexrelease>
%<latexrelease>\EndIncludeInRelease
%<*2ekernel>
%    \end{macrocode}
% \end{macro}
%
%
%
%
%
%
% \begin{macro}{\@@fileswith@pti@ns}
% Save the definition (for error checking).
% \changes{v0.2c}{1993/11/17}
%         {Macro added}
%    \begin{macrocode}
\let\@@fileswith@pti@ns\@fileswith@pti@ns
\@onlypreamble\@@fileswith@pti@ns
%    \end{macrocode}
% \end{macro}
%
% \begin{macro}{\@reset@ptions}
% Reset the default option, and clear lists of declared options.
% \changes{v0.2a}{1993/11/14}{macro added}
%    \begin{macrocode}
\def\@reset@ptions{%
  \global\ifx\@currext\@clsextension
    \let\default@ds\OptionNotUsed
   \else
    \let\default@ds\@unknownoptionerror
  \fi
  \global\let\ds@\@empty
  \global\let\@declaredoptions\@empty}
\@onlypreamble\@reset@ptions
%    \end{macrocode}
% \end{macro}
%
%
%
%
% \subsection{Hooks}
%
% Allow code to be saved to be executed at specific later times.
%
% Here we save things in macros. I considered using toks registers (and
% |\addto@hook| from the NFSS code), but that would require stacking the
% contents in the case of required packages, so just generate a new
% macro for each package.
% \begin{macro}{\@begindocumenthook}
% \changes{v1.0s}{1995/10/20}
%         {Make setting conditional, for autoload version}
% \begin{macro}{\@enddocumenthook}
% Stuff to appear at the beginning or end of the document.
%    \begin{macrocode}
\ifx\@begindocumenthook\@undefined
  \let\@begindocumenthook\@empty
\fi
\let\@enddocumenthook\@empty
%    \end{macrocode}
% \end{macro}
% \end{macro}
%
%
% \begin{macro}{\AtEndOfPackage}
% \begin{macro}{\AtEndOfClass}
% \begin{macro}{\AtBeginDocument}
% \begin{macro}{\AtEndDocument}
% The access functions.
% \changes{v0.2a}{1993/11/14}
%         {Included extension in the generated macro name for package
%         and class hooks.}
%    \begin{macrocode}
\def\AtEndOfPackage{%
  \expandafter\g@addto@macro\csname\@currname.\@currext-h@@k\endcsname}
\let\AtEndOfClass\AtEndOfPackage
\@onlypreamble\AtEndOfPackage
\@onlypreamble\AtEndOfClass
%    \end{macrocode}
%
%    \begin{macrocode}
%</2ekernel>
%<*2ekernel|latexrelease>
%<latexrelease>\IncludeInRelease{2020/10/01}%
%<latexrelease>                 {\AtBeginDocument}{Use hook system}%
\DeclareRobustCommand\AtBeginDocument{\AddToHook{begindocument}}
\DeclareRobustCommand\AtEndDocument  {\AddToHook{enddocument}}
%\DeclareRobustCommand\AtEndDocument {\AddToHook{env/document/end}} % alternative impl
%    \end{macrocode}
%
%    \begin{macrocode}
%</2ekernel|latexrelease>
%<latexrelease>\EndIncludeInRelease
%<latexrelease>\IncludeInRelease{0000/00/00}%
%<latexrelease>                 {\AtBeginDocument}{Use hook system}%
%<latexrelease>
%<latexrelease>\DeclareRobustCommand\AtBeginDocument{\g@addto@macro\@begindocumenthook}
%<latexrelease>\DeclareRobustCommand\AtEndDocument{\g@addto@macro\@enddocumenthook}
%<latexrelease>
%<latexrelease>\EndIncludeInRelease
%<*2ekernel>
%    \end{macrocode}
%
%    In its initial implementation (not using the hook system)
%    \cs{AtBeginDocument} was made \cs{@onlypreamble} because using it
%    later had no effect whatsoever, thus was most certainly an
%    unintended programming error. With the reimplementation, using
%    the \texttt{begindocument} hook internally, this has changed
%    because adding to a onetime hook after it has already been used
%    simply executes the additional code immediately. We therefore no
%    longer generate an error if it is used inside the document so
%    that \verb=\AddToHook{begindocument}= and \cs{AtBeginDocument}
%    are truly equivalent (as claimed in the hook documentation).
% \changes{v1.5n}{2025/01/02}{Do not make \cs{AtBeginDocument}
%    preamble only (gh/1604)}
%    \begin{macrocode}
%\@onlypreamble\AtBeginDocument
%    \end{macrocode}
% \end{macro}
% \end{macro}
% \end{macro}
% \end{macro}
%
%
% \begin{macro}{\@cls@pkg}
%    The current file type.
% \changes{v0.2i}{1993/12/03}
%         {Name changed to avoid clash with output routine.}
% \changes{v1.5l}{2024/06/04}
%         {Allow for extensions other than "cls" and "pkg"}
%    \begin{macrocode}
%</2ekernel>
%<*2ekernel|latexrelease>
%<latexrelease>\IncludeInRelease{2024/11/01}%
%<latexrelease>                 {\@cls@pkg}{Allow for more extensions}%
\def\@cls@pkg{%
  \ifx\@currext\@clsextension
    document class%
  \else
    \ifx\@currext\@pkgextension
      package%
    \else
      file%
    \fi
  \fi}
%</2ekernel|latexrelease>
%<latexrelease>\EndIncludeInRelease
%    \end{macrocode}
%
%    \begin{macrocode}
%<latexrelease>\IncludeInRelease{0000/00/00}%
%<latexrelease>                 {\@cls@pkg}{Allow for more extensions}%
%<latexrelease>
%<latexrelease>\def\@cls@pkg{%
%<latexrelease>  \ifx\@currext\@clsextension
%<latexrelease>    document class%
%<latexrelease>  \else
%<latexrelease>    package%
%<latexrelease>  \fi}
%<latexrelease>\EndIncludeInRelease
%<*2ekernel>
%    \end{macrocode}
%
%    \begin{macrocode}
\@onlypreamble\@cls@pkg
%    \end{macrocode}
% \end{macro}
%
% \begin{macro}{\@unknownoptionerror}
% Bad option.
%    \begin{macrocode}
\def\@unknownoptionerror{%
  \@latex@error
    {Unknown option `\CurrentOption' for \@cls@pkg\space`\@currname'}%
    {The option `\CurrentOption' was not declared in
     \@cls@pkg\space`\@currname', perhaps you\MessageBreak
      misspelled its name.
     Try typing \space <return>
     \space to proceed.}}
\@onlypreamble\@unknownoptionerror
%    \end{macrocode}
% \end{macro}
%
% \begin{macro}{\@@unprocessedoptions}
% Declare an error for each option, unless a |\ProcessOptions| occurred.
% \changes{v0.2v}{1994/01/29}
%         {Macro added.}
% \changes{v1.0t}{1995/11/14}{Allow empty option}
% \changes{v1.5d}{2022/10/10}{Use \cs{protected@edef}.}
%    \begin{macrocode}
\def\@@unprocessedoptions{%
  \ifx\@currext\@pkgextension
    \protected@edef\@curroptions{\@ptionlist{\@currname.\@currext}}%
    \@for\CurrentOption:=\@curroptions\do{%
        \ifx\CurrentOption\@empty\else\@unknownoptionerror\fi}%
  \fi}
\@onlypreamble\@unprocessedoptions
\@onlypreamble\@@unprocessedoptions
%    \end{macrocode}
% \end{macro}
%
% \begin{macro}{\@badrequireerror}
% |\RequirePackage| or |\LoadClass| occurs in the options section.
% \changes{v0.2c}{1993/11/17}
%         {Macro added}
%    \begin{macrocode}
\def\@badrequireerror#1[#2]#3[#4]{%
  \@latex@error
    {\noexpand\RequirePackage or \noexpand\LoadClass
         in Options Section}%
    {The \@cls@pkg\space `\@currname' is defective.\MessageBreak
     It attempts to load `#3' in the options section, i.e.,\MessageBreak
     between \noexpand\DeclareOption and \string\ProcessOptions.}}
\@onlypreamble\@badrequireerror
%    \end{macrocode}
% \end{macro}
%
% \begin{macro}{\@twoloadclasserror}
% Two |\LoadClass| in a class.
% \changes{v0.2c}{1993/11/17}
%         {Macro added}
%    \begin{macrocode}
\def\@twoloadclasserror{%
  \@latex@error
    {Two \noexpand\LoadClass commands}%
    {You may only use one \noexpand\LoadClass in a class file}}
\@onlypreamble\@twoloadclasserror
%    \end{macrocode}
% \end{macro}
%
% \begin{macro}{\@twoclasseserror}
% Two |\documentclass| or |\documentstyle|.
% \changes{v0.2h}{1993/11/28}
%         {Macro added}
%    \begin{macrocode}
\def\@twoclasseserror#1#{%
  \@latex@error
    {Two \noexpand\documentclass or \noexpand\documentstyle commands}%
    {The document may only declare one class.}\@gobble}
\@onlypreamble\@twoclasseserror
%    \end{macrocode}
% \end{macro}
%
% \subsection{Providing shipment}
%
% \begin{macro}{\two@digits}
% Prefix a number less than 10 with `0'.
%    \begin{macrocode}
\def\two@digits#1{\ifnum#1<10 0\fi\number#1}
%    \end{macrocode}
% \end{macro}
%
%  \begin{environment}{filecontents}
%  \begin{macro}{\filecontents}
%  \begin{macro}{\endfilecontents}
%    This environment implements inline files.
%    The star-form does not write extra comments into the file.
%
% \changes{v0.2h}{1993/11/28}
%         {Don't globally allocate a write stream (always use 15)}
% \changes{v0.2r}{1993/12/19}{Different message when ignoring a file}
% \changes{v0.3g}{1994/04/11}
%         {Add star form,
%          don't write \cs{endinput} at the end of the file.}
% \changes{v1.0c}{1994/05/11}
%         {Add checks for form feed and tab}
% \changes{v1.0m}{1995/04/21}
%         {Close input check stream: latex/1487}
% \changes{v1.0p}{1995/05/25}{Delete \cs{filec@ntents} after preamble}
% \changes{v1.3a}{2019/07/01}{Support UTF8 and spaces in
%                             filecontents environment file name}
% \changes{v1.3b}{2019/08/27}{Make various commands robust}
% \changes{v1.3c}{2019/09/11}{Support optional argument for filecontents}
% \changes{v1.3f}{2020/01/05}{Support more write streams in LuaTeX gh/238}
%    \begin{macrocode}
%</2ekernel>
%<*2ekernel|latexrelease>
%<latexrelease>\IncludeInRelease{2020/10/01}%
%<latexrelease>                 {\filec@ntents}{Define \q@curr@file directly (gh/220)}%
%
%    \end{macrocode}
%    We use |@tempswa| to mean no preamble writing and reuse |@filesw|
%    to indicate no overwriting:
%    \begin{macrocode}
\def\filecontents{\@tempswatrue\@fileswtrue
  \@ifnextchar[\filec@ntents@opt\filec@ntents
}
\@namedef{filecontents*}{\@tempswafalse\@fileswtrue
  \@ifnextchar[\filec@ntents@opt\filec@ntents
}
%    \end{macrocode}
%    To handle the optional argument we execute for each option the
%    command \cs{filec@ntents@OPTION} if it exist or complain about
%    unknown option.
% \changes{v1.3h}{2020/01/28}{Allow spaces in option string and display
%     only unknown options not the whole option list (gh/256)}
% \changes{v1.5f}{2022/11/16}{Introduce key 'nowarn' on filecontents (gh/958)}
%    \begin{macrocode}
\def\filec@ntents@opt[#1]{%
  \edef\@fortmp{\zap@space#1 \@empty}%
  \@for\reserved@a:=\@fortmp\do{%
    \ifcsname filec@ntents@\reserved@a\endcsname
      \csname filec@ntents@\reserved@a\endcsname
    \else
    \@latex@error{Unknown filecontents option \reserved@a}%
       {Valid options are force (or overwrite), nosearch, noheader, nowarn}%
    \fi}%
  \filec@ntents
}
%    \end{macrocode}
%    Option \texttt{force} (or \texttt{overwrite}) changes the
%    overwriting switch
%    \begin{macrocode}
\let\filec@ntents@force\@fileswfalse
\let\filec@ntents@overwrite\@fileswfalse  % alternative name
%    \end{macrocode}
%    and option \texttt{noheader} the preamble switch (which is
%    equivalent to using the star form of the environment).
%    \begin{macrocode}
\let\filec@ntents@noheader\@tempswafalse
%    \end{macrocode}
%    Option \texttt{nosearch} only checks the current directory not
%    the whole \TeX{} tree for the existence of the file to write.
%    \begin{macrocode}
\def\filec@ntents@nosearch{%
  \let\filec@ntents@checkdir\@currdir
  \def\filec@ntents@where{in current directory}}
%    \end{macrocode}
%    By default we search the whole tree:
%    \begin{macrocode}
\let\filec@ntents@checkdir\@empty
\def\filec@ntents@where{exists on the system}
%    \end{macrocode}
%    Option \texttt{nowarn} does not show any warning on the terminal
%    but still writes it to the \texttt{.log}.
% \changes{v1.5f}{2022/11/16}{Introduce key 'nowarn' on filecontents (gh/958)}
%    \begin{macrocode}
\def\filec@ntents@nowarn{%
  \let\filec@ntents@warning\@latex@note@no@line
}
%    \end{macrocode}
%    By default we show terminal warnings.
%    \begin{macrocode}
\let\filec@ntents@warning\@latex@warning@no@line
%    \end{macrocode}
%
%    \begin{macrocode}
\begingroup%
\@tempcnta=1
\loop
  \catcode\@tempcnta=12  %
  \advance\@tempcnta\@ne %
\ifnum\@tempcnta<32      %
\repeat                  %
\catcode`\*=11 %
\catcode`\^^M\active%
\catcode`\^^L\active\let^^L\relax%
\catcode`\^^I\active%
%    \end{macrocode}
%
% \changes{v1.3m}{2020-08-08}{define \cs{q@curr@file} directly as the
%    quotes have already been removed (gh/220)}
%    \begin{macrocode}
\gdef\filec@ntents#1{%
  \set@curr@file{\filec@ntents@checkdir#1}%
  \edef\q@curr@file{"\@curr@file"}%
%    \end{macrocode}
%
% Lua\TeX\ has more writes (and 18 is safe here).
% \changes{v1.4b}{2021/06/09}{Use \cs{@latex@note@no@line} to display the information}
%    \begin{macrocode}
  \chardef\reserved@c\ifx\directlua\@undefined 15 \else 127 \fi%
  \openin\@inputcheck\q@curr@file \space %
  \ifeof\@inputcheck%
    \@latex@note@no@line%
        {Writing file `\@currdir\@curr@file'}%
%    \end{macrocode}
%
% \changes{v1.0y}{1997/10/10}
%         {\cs{reserved@c} not \cs{verbatim@out} to save a csname}
%    \begin{macrocode}
    \ch@ck7\reserved@c\write\relax%
    \immediate\openout\reserved@c\q@curr@file\relax%
  \else%
%    \end{macrocode}
%
% \changes{v1.0y}{1997/10/10}
%         {Use \cs{@gobbletwo}}
% \changes{v1.4b}{2021/06/09}{Use \cs{@latex@note@no@line} to display the information}
%    \begin{macrocode}
    \if@filesw%
      \@latex@note@no@line%
          {File `\@curr@file' already \filec@ntents@where.\MessageBreak%
             Not generating it from this source}%
      \let\write\@gobbletwo%
      \let\closeout\@gobble%
    \else%
%    \end{macrocode}
%    If we are overwriting, we try to make sure that the user is not
%    by mistake overwriting the input file (\cs{jobname}). Of course,
%    this only works for input files ending in \texttt{.tex}. If a
%    different extension is used there is no way to see that we are
%    overwriting ourselves!
% \changes{v1.3y}{2021/03/03}
%         {Fix overwrite check for files with UTF-8 (gh/415)}
%    \begin{macrocode}
      \edef\reserved@b{\detokenize\expandafter{\jobname}}%
      \ifx\@curr@file\reserved@b%
        \@fileswtrue%
      \else%
        \edef\reserved@b{\reserved@b\detokenize{.tex}}%
        \ifx\@curr@file\reserved@b
          \@fileswtrue%
        \fi%
      \fi%
%    \end{macrocode}
%    We allocate a write channel but we open it only if it is
%    (hopefully) safe. If not opened that means we are going to write
%    on the terminal.
% \changes{v1.3g}{2020/01/27}{Fix typo in error message}
% \changes{v1.3j}{2020/02/20}{Fix missing quotes around file name (gh/284)}
% \changes{v1.5f}{2022/11/16}{Introduce key 'nowarn' on filecontents (gh/958)}
% \changes{v1.5f}{2022/11/16}{Do not show "current dir" in message (gh/917)}
%    \begin{macrocode}
      \ch@ck7\reserved@c\write\relax%
      \if@filesw%  % Foul ... trying to overwrite \jobname!
      \@latex@error{Trying to overwrite `\jobname.tex'}{You can't %
        write to the file you are reading from!\MessageBreak%
        Data is written to screen instead.}%
      \else%
        \filec@ntents@warning%
           {Writing or overwriting file `\@curr@file'}%
        \immediate\openout\reserved@c\q@curr@file\relax%
      \fi%
    \fi%
  \fi%
%    \end{macrocode}
%    Closing the \cs{@inputcheck} is done here to avoid having to do
%    this in each branch.
%    \begin{macrocode}
  \closein\@inputcheck%
  \if@tempswa%
%    \end{macrocode}
%
% \changes{v1.0y}{1997/10/10}
%         {\cs{@currenvir} in banner}
%    \begin{macrocode}
    \immediate\write\reserved@c{%
      \@percentchar\@percentchar\space%
          \expandafter\@gobble\string\LaTeX2e file `\@curr@file'^^J%
      \@percentchar\@percentchar\space  generated by the %
        `\@currenvir' \expandafter\@gobblefour\string\newenvironment^^J%
      \@percentchar\@percentchar\space from source `\jobname' on %
         \number\year/\two@digits\month/\two@digits\day.^^J%
      \@percentchar\@percentchar}%
  \fi%
  \let\do\@makeother\dospecials%
%    \end{macrocode}
%    If there are active characters in the upper half (e.g., from
%    \texttt{inputenc}) there would be confusion so we render everything
%    harmless.
% \changes{v1.2f}{2018/03/27}
%         {Use full file name for old release}
%    \begin{macrocode}
  \count@ 128\relax%
  \loop%
    \catcode\count@ 11\relax%
    \advance\count@ \@ne%
    \ifnum\count@<\@cclvi%
  \repeat%
%    \end{macrocode}
%
% \changes{v1.0y}{1997/10/10}
%     {Check for text before or after \cs{end} environment. latex/2636}
%    \begin{macrocode}
  \edef\E{\@backslashchar end\string{\@currenvir\string}}%
  \edef\reserved@b{%
    \def\noexpand\reserved@b%
         ####1\E####2\E####3\relax}%
  \reserved@b{%
    \ifx\relax##3\relax%
%    \end{macrocode}
% There was no |\end{filecontents}|
%    \begin{macrocode}
      \immediate\write\reserved@c{##1}%
    \else%
%    \end{macrocode}
% There was a |\end{filecontents}|, so stop this time.
%    \begin{macrocode}
      \edef^^M{\noexpand\end{\@currenvir}}%
      \ifx\relax##1\relax%
      \else%
%    \end{macrocode}
% Text before the |\end|, write it with a warning.
%    \begin{macrocode}
          \@latex@warning{Writing text `##1' before %
            \string\end{\@currenvir}\MessageBreak
            as last line of \@curr@file}%
        \immediate\write\reserved@c{##1}%
      \fi%
      \ifx\relax##2\relax%
      \else%
%    \end{macrocode}
% Text after the |\end|, ignore it with a warning.
%    \begin{macrocode}
         \@latex@warning{%
           Ignoring text `##2' after \string\end{\@currenvir}}%
      \fi%
    \fi%
    ^^M}%
%    \end{macrocode}
%
%
% \changes{v1.2j}{2018/05/29}{use \cs{csname} not \cs{@undefined}}
%    \begin{macrocode}
  \catcode`\^^L\active%
  \let\L\@undefined%
  \def^^L{\expandafter\ifx\csname L\endcsname\relax\fi ^^J^^J}%
  \catcode`\^^I\active%
  \let\I\@undefined%
  \def^^I{\expandafter\ifx\csname I\endcsname\relax\fi\space}%
  \catcode`\^^M\active%
  \edef^^M##1^^M{%
    \noexpand\reserved@b##1\E\E\relax}}%
\endgroup%
%    \end{macrocode}
%
%
%    \begin{macrocode}
%</2ekernel|latexrelease>
%<latexrelease>\EndIncludeInRelease
%<latexrelease>\IncludeInRelease{2019/10/01}%
%<latexrelease>    {\filec@ntents}{Spaces in file names + optional arg}%
%<latexrelease>
%<latexrelease>\def\filecontents{\@tempswatrue\@fileswtrue
%<latexrelease>  \@ifnextchar[\filec@ntents@opt\filec@ntents
%<latexrelease>}
%<latexrelease>\@namedef{filecontents*}{\@tempswafalse\@fileswtrue
%<latexrelease>  \@ifnextchar[\filec@ntents@opt\filec@ntents
%<latexrelease>}
%<latexrelease>\def\filec@ntents@opt[#1]{%
%<latexrelease>  \edef\@fortmp{\zap@space#1 \@empty}%
%<latexrelease>  \@for\reserved@a:=\@fortmp\do{%
%<latexrelease>    \ifcsname filec@ntents@\reserved@a\endcsname
%<latexrelease>      \csname filec@ntents@\reserved@a\endcsname
%<latexrelease>    \else
%<latexrelease>    \@latex@error{Unknown filecontents option \reserved@a}%
%<latexrelease>       {Valid options are force (or overwrite), nosearch, noheader}%
%<latexrelease>    \fi}%
%<latexrelease>  \filec@ntents
%<latexrelease>}
%<latexrelease>\let\filec@ntents@force\@fileswfalse
%<latexrelease>\let\filec@ntents@overwrite\@fileswfalse  % alternative name
%<latexrelease>\let\filec@ntents@noheader\@tempswafalse
%<latexrelease>\def\filec@ntents@nosearch{%
%<latexrelease>  \let\filec@ntents@checkdir\@currdir
%<latexrelease>  \def\filec@ntents@where{in current directory}}
%<latexrelease>\let\filec@ntents@checkdir\@empty
%<latexrelease>\def\filec@ntents@where{exists on the system}
%<latexrelease>\begingroup%
%<latexrelease>\@tempcnta=1
%<latexrelease>\loop
%<latexrelease>  \catcode\@tempcnta=12  %
%<latexrelease>  \advance\@tempcnta\@ne %
%<latexrelease>\ifnum\@tempcnta<32      %
%<latexrelease>\repeat                  %
%<latexrelease>\catcode`\*=11 %
%<latexrelease>\catcode`\^^M\active%
%<latexrelease>\catcode`\^^L\active\let^^L\relax%
%<latexrelease>\catcode`\^^I\active%
%<latexrelease>\gdef\filec@ntents#1{%
%<latexrelease>  \set@curr@file{\filec@ntents@checkdir#1}%
%<latexrelease>  \edef\q@curr@file{\expandafter\quote@name\expandafter{\@curr@file}}%
%<latexrelease>  \chardef\reserved@c\ifx\directlua\@undefined 15 \else 127 \fi%
%<latexrelease>  \openin\@inputcheck\q@curr@file \space %
%<latexrelease>  \ifeof\@inputcheck%
%<latexrelease>    \@latex@warning@no@line%
%<latexrelease>        {Writing file `\@currdir\@curr@file'}%
%<latexrelease>    \ch@ck7\reserved@c\write\relax%
%<latexrelease>    \immediate\openout\reserved@c\q@curr@file\relax%
%<latexrelease>  \else%
%<latexrelease>    \if@filesw%
%<latexrelease>      \@latex@warning@no@line%
%<latexrelease>          {File `\@curr@file' already \filec@ntents@where.\MessageBreak%
%<latexrelease>             Not generating it from this source}%
%<latexrelease>      \let\write\@gobbletwo%
%<latexrelease>      \let\closeout\@gobble%
%<latexrelease>    \else%
%<latexrelease>      \edef\reserved@a{#1}%
%<latexrelease>      \edef\reserved@a{\detokenize\expandafter{\reserved@a}}%
%<latexrelease>      \edef\reserved@b{\detokenize\expandafter{\jobname}}%
%<latexrelease>      \ifx\reserved@a\reserved@b%
%<latexrelease>        \@fileswtrue%
%<latexrelease>      \else%
%<latexrelease>        \edef\reserved@b{\reserved@b\detokenize{.tex}}%
%<latexrelease>        \ifx\reserved@a\reserved@b
%<latexrelease>          \@fileswtrue%
%<latexrelease>        \fi%
%<latexrelease>      \fi%
%<latexrelease>      \ch@ck7\reserved@c\write\relax%
%<latexrelease>      \if@filesw%  % Foul ... trying to overwrite \jobname!
%<latexrelease>      \@latex@error{Trying to overwrite `\jobname.tex'}{You can't %
%<latexrelease>        write to the file you are reading from!\MessageBreak%
%<latexrelease>        Data is written to screen instead.}%
%<latexrelease>      \else%
%<latexrelease>        \@latex@warning@no@line%
%<latexrelease>           {Writing or overwriting file `\@currdir\@curr@file'}%
%<latexrelease>        \immediate\openout\reserved@c\q@curr@file\relax%
%<latexrelease>      \fi%
%<latexrelease>    \fi%
%<latexrelease>  \fi%
%<latexrelease>  \closein\@inputcheck%
%<latexrelease>  \if@tempswa%
%<latexrelease>    \immediate\write\reserved@c{%
%<latexrelease>      \@percentchar\@percentchar\space%
%<latexrelease>          \expandafter\@gobble\string\LaTeX2e file `\@curr@file'^^J%
%<latexrelease>      \@percentchar\@percentchar\space  generated by the %
%<latexrelease>        `\@currenvir' \expandafter\@gobblefour\string\newenvironment^^J%
%<latexrelease>      \@percentchar\@percentchar\space from source `\jobname' on %
%<latexrelease>         \number\year/\two@digits\month/\two@digits\day.^^J%
%<latexrelease>      \@percentchar\@percentchar}%
%<latexrelease>  \fi%
%<latexrelease>  \let\do\@makeother\dospecials%
%<latexrelease>  \count@ 128\relax%
%<latexrelease>  \loop%
%<latexrelease>    \catcode\count@ 11\relax%
%<latexrelease>    \advance\count@ \@ne%
%<latexrelease>    \ifnum\count@<\@cclvi%
%<latexrelease>  \repeat%
%<latexrelease>  \edef\E{\@backslashchar end\string{\@currenvir\string}}%
%<latexrelease>  \edef\reserved@b{%
%<latexrelease>    \def\noexpand\reserved@b%
%<latexrelease>         ####1\E####2\E####3\relax}%
%<latexrelease>  \reserved@b{%
%<latexrelease>    \ifx\relax##3\relax%
%<latexrelease>      \immediate\write\reserved@c{##1}%
%<latexrelease>    \else%
%<latexrelease>      \edef^^M{\noexpand\end{\@currenvir}}%
%<latexrelease>      \ifx\relax##1\relax%
%<latexrelease>      \else%
%<latexrelease>          \@latex@warning{Writing text `##1' before %
%<latexrelease>             \string\end{\@currenvir}\MessageBreak as last line of \@curr@file}%
%<latexrelease>        \immediate\write\reserved@c{##1}%
%<latexrelease>      \fi%
%<latexrelease>      \ifx\relax##2\relax%
%<latexrelease>      \else%
%<latexrelease>         \@latex@warning{%
%<latexrelease>           Ignoring text `##2' after \string\end{\@currenvir}}%
%<latexrelease>      \fi%
%<latexrelease>    \fi%
%<latexrelease>    ^^M}%
%<latexrelease>  \catcode`\^^L\active%
%<latexrelease>  \let\L\@undefined%
%<latexrelease>  \def^^L{\expandafter\ifx\csname L\endcsname\relax\fi ^^J^^J}%
%<latexrelease>  \catcode`\^^I\active%
%<latexrelease>  \let\I\@undefined%
%<latexrelease>  \def^^I{\expandafter\ifx\csname I\endcsname\relax\fi\space}%
%<latexrelease>  \catcode`\^^M\active%
%<latexrelease>  \edef^^M##1^^M{%
%<latexrelease>    \noexpand\reserved@b##1\E\E\relax}}%
%<latexrelease>\endgroup%
%<latexrelease>\EndIncludeInRelease
%<latexrelease>\IncludeInRelease{0000/00/00}%
%<latexrelease>    {\filec@ntents}{Spaces in file names + optional arg}%
%<latexrelease>
%<latexrelease>\let\filec@ntents@opt        \@undefined
%<latexrelease>\let\filec@ntents@force      \@undefined
%<latexrelease>\let\filec@ntents@overwrite  \@undefined
%<latexrelease>\let\filec@ntents@noheader   \@undefined
%<latexrelease>\let\filec@ntents@nosearch   \@undefined
%<latexrelease>\let\filec@ntents@checkdir   \@undefined
%<latexrelease>\let\filec@ntents@where      \@undefined
%<latexrelease>
%<latexrelease>\begingroup%
%<latexrelease>\@tempcnta=1
%<latexrelease>\loop
%<latexrelease>  \catcode\@tempcnta=12  %
%<latexrelease>  \advance\@tempcnta\@ne %
%<latexrelease>\ifnum\@tempcnta<32      %
%<latexrelease>\repeat                  %
%<latexrelease>\catcode`\*=11 %
%<latexrelease>\catcode`\^^M\active%
%<latexrelease>\catcode`\^^L\active\let^^L\relax%
%<latexrelease>\catcode`\^^I\active%
%<latexrelease>
%<latexrelease>\gdef\filec@ntents#1{%
%<latexrelease>  \openin\@inputcheck#1 %
%<latexrelease>  \ifeof\@inputcheck%
%<latexrelease>    \@latex@warning@no@line%
%<latexrelease>        {Writing file `\@currdir#1'}%
%<latexrelease>    \chardef\reserved@c15 %
%<latexrelease>    \ch@ck7\reserved@c\write%
%<latexrelease>    \immediate\openout\reserved@c#1\relax%
%<latexrelease>  \else%
%<latexrelease>    \closein\@inputcheck%
%<latexrelease>    \@latex@warning@no@line%
%<latexrelease>            {File `#1' already exists on the system.\MessageBreak%
%<latexrelease>             Not generating it from this source}%
%<latexrelease>    \let\write\@gobbletwo%
%<latexrelease>    \let\closeout\@gobble%
%<latexrelease>  \fi%
%<latexrelease>  \if@tempswa%
%<latexrelease>    \immediate\write\reserved@c{%
%<latexrelease>      \@percentchar\@percentchar\space%
%<latexrelease>          \expandafter\@gobble\string\LaTeX2e file `#1'^^J%
%<latexrelease>      \@percentchar\@percentchar\space  generated by the %
%<latexrelease>        `\@currenvir' \expandafter\@gobblefour\string\newenvironment^^J%
%<latexrelease>      \@percentchar\@percentchar\space from source `\jobname' on %
%<latexrelease>         \number\year/\two@digits\month/\two@digits\day.^^J%
%<latexrelease>      \@percentchar\@percentchar}%
%<latexrelease>  \fi%
%<latexrelease>  \let\do\@makeother\dospecials%
%<latexrelease>  \count@ 128\relax%
%<latexrelease>  \loop%
%<latexrelease>    \catcode\count@ 11\relax%
%<latexrelease>    \advance\count@ \@ne%
%<latexrelease>    \ifnum\count@<\@cclvi%
%<latexrelease>  \repeat%
%<latexrelease>  \edef\E{\@backslashchar end\string{\@currenvir\string}}%
%<latexrelease>  \edef\reserved@b{%
%<latexrelease>    \def\noexpand\reserved@b%
%<latexrelease>         ####1\E####2\E####3\relax}%
%<latexrelease>  \reserved@b{%
%<latexrelease>    \ifx\relax##3\relax%
%<latexrelease>      \immediate\write\reserved@c{##1}%
%<latexrelease>    \else%
%<latexrelease>      \edef^^M{\noexpand\end{\@currenvir}}%
%<latexrelease>      \ifx\relax##1\relax%
%<latexrelease>      \else%
%<latexrelease>          \@latex@warning{Writing text `##1' before %
%<latexrelease>             \string\end{\@currenvir}\MessageBreak as last line of #1}%
%<latexrelease>        \immediate\write\reserved@c{##1}%
%<latexrelease>      \fi%
%<latexrelease>      \ifx\relax##2\relax%
%<latexrelease>      \else%
%<latexrelease>         \@latex@warning{%
%<latexrelease>           Ignoring text `##2' after \string\end{\@currenvir}}%
%<latexrelease>      \fi%
%<latexrelease>    \fi%
%<latexrelease>    ^^M}%
%<latexrelease>
%<latexrelease>  \catcode`\^^L\active%
%<latexrelease>  \let\L\@undefined%
%<latexrelease>  \def^^L{\expandafter\ifx\csname L\endcsname\relax\fi ^^J^^J}%
%<latexrelease>  \catcode`\^^I\active%
%<latexrelease>  \let\I\@undefined%
%<latexrelease>  \def^^I{\expandafter\ifx\csname I\endcsname\relax\fi\space}%
%<latexrelease>  \catcode`\^^M\active%
%<latexrelease>  \edef^^M##1^^M{%
%<latexrelease>    \noexpand\reserved@b##1\E\E\relax}}%
%<latexrelease>\endgroup%
%<latexrelease>\EndIncludeInRelease
%<*2ekernel>
%    \end{macrocode}
%
%
%    \begin{macrocode}
\begingroup
\catcode`|=\catcode`\%
\catcode`\%=12
\catcode`\*=11
\gdef\@percentchar{%}
\gdef\endfilecontents{|
  \immediate\closeout\reserved@c
  \def\T##1##2##3{|
  \ifx##1\@undefined\else
    \@latex@warning@no@line{##2 has been converted to Blank ##3e}|
  \fi}|
  \T\L{Form Feed}{Lin}|
  \T\I{Tab}{Spac}|
  \immediate\write\@unused{}}
\global\let\endfilecontents*\endfilecontents
%    \end{macrocode}
%    We no longer prevent the code to be used after begin document (no
%    rollback needed for this change).
%    \begin{macrocode}
%\@onlypreamble\filecontents
%\@onlypreamble\endfilecontents
%\@onlypreamble\filecontents*
%\@onlypreamble\endfilecontents*
\endgroup
%\@onlypreamble\filec@ntents
%    \end{macrocode}
% \end{macro}
% \end{macro}
% \end{environment}
%
%
%
%
%
% \section{Package/class rollback mechanism}
%
%
%
%    \begin{macrocode}
%</2ekernel>
%<*2ekernel|latexreleasefirst>
%    \end{macrocode}
%
% \changes{v1.2d}{2018/02/18}{Introduce rollback concept}
%
%  \begin{macro}{\pkgcls@debug}
%    For testing we have a few extra lines of code that by default do
%    nothing but one can set |\pkgcls@debug| to  |\typeout| to get
%    extra info. Sometime in the future this will be dropped.
%    \begin{macrocode}
%<*tracerollback>
%\let\pkgcls@debug\typeout
\let\pkgcls@debug\@gobble
%</tracerollback>
%    \end{macrocode}
%  \end{macro}
%
%
%  \begin{macro}{\requestedLaTeXdate}
%    The macro (!) |\requestedLaTeXdate| holds the globally requested
%    rollback date (via \texttt{latexrelease}) or zero if no such
%    request was made.
%    \begin{macrocode}
\def\requestedLaTeXdate{0}
%    \end{macrocode}
%  \end{macro}
%
%
%  \begin{macro}{\pkgcls@targetdate}
%  \begin{macro}{\pkgcls@targetlabel}
%  \begin{macro}{\pkgcls@innerdate}
%
%    If a rollback for a package or class is requested then
%    |\pkgcls@targetdate| holds the requested date as a number
%    YYYYMMDD (if there was one, otherwise the value of
%    |\requestedLaTeXdate|) and |\pkgcls@targetlabel| will be
%    empty. If there was a request for a named version then
%    |\pkgcls@targetlabel| holds the version name and
%    |\pkgcls@targetdate| is set to \texttt{1}.
%
%    |\pkgcls@targetdate=0| is used to indicate that there was no
%    rollback request.
%    While loading an old release |\pkgcls@targetdate| is also reset to
%    zero so that |\DeclareRelease| declarations are bypassed.
%
%    In contrast |\pkgcls@innerdate| will always hold the requested
%    date (in a macro not a counter) if there was one, otherwise,
%    e.g., if there was no request or a request to a version name it
%    will contain \TeX{} largest legal number. While loading a file
%    this can be used to provide conditionals that select code based
%    on the request.
%
%
%    \begin{macrocode}
\ifx\pkgcls@targetdate\@undefined
  \newcount\pkgcls@targetdate
\fi
\let\pkgcls@targetlabel\@empty
\def\pkgcls@innerdate{\maxdimen}
%    \end{macrocode}
%  \end{macro}
%  \end{macro}
%  \end{macro}
%
%  \begin{macro}{\pkgcls@candidate}
%  \begin{macro}{\pkgcls@releasedate}
%    When looping through the |\DeclareRelease| declarations we
%    record if the release is the best candidate we have seen so far.
%    This is recorded in |\pkgcls@candidate| and we update it whenever
%    we see a better one.
%
%    In |\pkgcls@releasedate| we keep track of the release date of
%    that candidate.
%    \begin{macrocode}
\let\pkgcls@candidate\@empty
\let\pkgcls@releasedate\@empty
%    \end{macrocode}
%  \end{macro}
%  \end{macro}
%
%  \begin{macro}{\load@onefilewithoptions}
%  \begin{macro}{\@onefilewithoptions}
%    the best place to add the rollback code is at the point where
%    |\@onefilewithoptions| is called to load a single class or
%    package.
%
%    To make things easy we save the old definition as
%    |\load@onefilewithoptions| and then provide a new interface.
%
%    Important: as this code is also unconditionally placed into
%    latexrelease we can only do this name change once otherwise both
%    macros will contain the same code.
%    \begin{macrocode}
\ifx\load@onefilewithoptions\@undefined
 \let\load@onefilewithoptions\@onefilewithoptions
%    \end{macrocode}
%
%    \begin{macrocode}
 \def\@onefilewithoptions#1[#2][#3]#4{%
%    \end{macrocode}
%    First a bit of tracing normally disabled.
%    \begin{macrocode}
%<*tracerollback>
  \pkgcls@debug{--- File loaded request (\noexpand\usepackage or ...)}%
  \pkgcls@debug{\@spaces 1: #1}%
  \pkgcls@debug{\@spaces 2: #2}%
  \pkgcls@debug{\@spaces 3: #3}%
  \pkgcls@debug{\@spaces 4: #4}%
%</tracerollback>
%    \end{macrocode}
%    Three of the arguments are needed later on in error/warning
%    messages so we save them.
% \changes{v1.5l}{2024/06/04}
%         {New argument \cs{pkgcls@ext} (gh/870)}
%    \begin{macrocode}
  \def\pkgcls@name{#1}%                  % for info message
  \def\pkgcls@arg {#3}%                  % for info message
  \edef\pkgcls@ext{%
    \ifx#4\@clsextension document class\else
      \ifx#4\@pkgextension package\else
        file
      \fi
    \fi
  }%                                     % for info message
%    \end{macrocode}
%    then we parse the final optional argument to determine if there
%    is a specific rollback request for the current file. This will
%    set |\pkgcls@targetdate|, |\pkgcls@targetlabel| and
%    |\pkgcls@mindate|.
%    \begin{macrocode}
  \pkgcls@parse@date@arg{#3}%
%    \end{macrocode}
%    When determining the correct release to load we keep track of
%    candidates in |\pkgcls@candidate| and initially we don't have any:
%    \begin{macrocode}
  \let\pkgcls@candidate\@empty
%    \end{macrocode}
%    If we had a rollback request then |#3| may contain data but not
%    necessarily a ``minimal date'' so instead of passing it on we
%    pass on the content of |\pkgcls@mindate|. We need to pass the
%    value not the command, otherwise nested packages may pick up the
%    wrong information.
%   \changes{v1.2h}{2018-04-08}{Pass expanded date}
%    \begin{macrocode}
  \begingroup
  \edef\reserved@a{%
    \endgroup
    \unexpanded{\load@onefilewithoptions#1[#2]}%
    [\pkgcls@mindate]%
    \unexpanded{#4}}%
   \reserved@a
 }
\fi
%    \end{macrocode}
%  \end{macro}
%  \end{macro}
%
%
%  \begin{macro}{\pkgcls@parse@date@arg}
%    The |\pkgcls@parse@date@arg| command parses the second optional
%   argument of |\usepackage|, |\RequirePackage| or |\documentclass|
%   for a rollback request setting the values of |\pkgcls@targetdate|
%   and |\pkgcls@targetlabel|.
%
%   This optional argument has a dual purpose: If it just contains a
%   date string then this means that the package should have at least
%   that date (to ensure that a certain feature is actually available,
%   or a certain bug has been fixed). When the package gets loaded the
%   information in |\Provides...| will then be checked against this
%   request.
%
%   But if it starts with an equal sign followed by a date string or
%   followed by a version name then this means that we should roll
%   back to the state of the package at that date or to the version
%   with the requested name.
%
%   If there was no optional argument or the optional argument
%   does not start with ``\texttt{=}'' then the |\pkgcls@targetdate|
%   is set to the date of the overall rollback request (via
%   \texttt{latexrelease}) or if that was not given it is set to
%   \texttt{0}.
%   In either case |\pkgcls@targetlabel| will be made empty.
%
%   If the argument doesn't start with  ``\texttt{=}'' then it is
%    supposed to be a ``minimal date'' and we therefore save the value
%    in |\pkgcls@mindate|, otherwise this macro is made empty.
%
%   So in summary we have:
%   \begin{flushleft}
%     \hspace*{-1in}\begin{tabular}{ccccc@{}}
%   Input        &  & \cs{pkgcls@targetdate}
%                   & \cs{pkgcls@targetlabel}
%                   & \cs{pkgcls@mindate}\\[3pt]
% \meta{empty}   & $\to$ & \meta{global-rollbackdate-as-number}
%                        & \meta{empty} & \meta{empty}
% \\
% \meta{date}    & $\to$ & \meta{global-rollbackdate-as-number}
%                        & \meta{empty} & \meta{date}
% \\
% \texttt{=}\meta{date} & $\to$ & \meta{date-as-number}
%                        & \meta{empty} & \meta{empty}
% \\
% \texttt{=}\meta{version}& $\to$ & \texttt{1}
%                                 & \meta{version}  & \meta{empty}
% \\
% \meta{other}   & $\to$ & \meta{global-rollbackdate-as-number}
%                        & \meta{empty} & \meta{other}
% \\
%     \end{tabular}
%   \end{flushleft}
%    where \meta{global-rollbackdate-as-number} is a date request given
%    via \texttt{latexrelease} or if there wasn't one \texttt{0}.
%
%    \begin{macrocode}
\def\pkgcls@parse@date@arg #1{%
%    \end{macrocode}
%    If the argument is empty we use the rollback date from
%    \texttt{latexrelease} which has the value of zero if there was no
%    rollback request. The label and the minimal date is made empty in that case.
%    \begin{macrocode}
   \ifx\@nil#1\@nil
     \pkgcls@targetdate\requestedLaTeXdate\relax
     \let\pkgcls@targetlabel\@empty
     \let\pkgcls@mindate\@empty
%    \end{macrocode}
%    Otherwise we parse the argument further, checking for a \texttt{=}
%    as the first character. We append a \texttt{=} at the end so that
%    there is at least one such character in the argument.
%    \begin{macrocode}
   \else
     \pkgcls@parse@date@arg@#1=\@nil\relax
   \fi
 }
%    \end{macrocode}
%    The actual parsing work then happens in |\pkgcls@parse@date@arg@|:
%    \begin{macrocode}
\def\pkgcls@parse@date@arg@#1=#2\@nil{%
%    \end{macrocode}
%    We set |\pkgcls@targetdate| depending on the parsing result; the
%    code is expandable so we can do the parsing as part of the assignment.
%    \begin{macrocode}
  \pkgcls@targetdate
%    \end{macrocode}
%    If a \texttt{=} was in first position then |#1| will be empty. In
%    that case |#2| will be the original argument with a \texttt{=}
%    appended.
%
%    This can be parsed with |\@parse@version|, the trailing character
%    is simply ignored. This macro returns the parsed date as a number
%    (or zero if it wasn't a date) and accepts both YYYY/MM/DD and YYYY-MM-DD
%    formats.
%    \begin{macrocode}
    \ifx\@nil#1\@nil
      \@parse@version0#2//00\@nil\relax
%    \end{macrocode}
%     Whatever is returned is thus assigned to |\pkgcls@targetdate|
%    and therefore we can now test its value. If the value is zero we
%    assume that the remaining argument string represents a version
%    and change |\pkgcls@targetdate| and set |\pkgcls@targetlabel| to
%    the version name (after stripping off the trailing \texttt{=}).
%    \begin{macrocode}
      \ifnum \pkgcls@targetdate=\z@
        \pkgcls@targetdate\@ne
        \def\pkgcls@innerdate{\maxdimen}%
        \pkgcls@parse@date@arg@version#2%
      \else
        \edef\pkgcls@innerdate{\the\pkgcls@targetdate}%
      \fi
      \let\pkgcls@mindate\@empty
    \else
%    \end{macrocode}
%    If |#1| was not empty then there wasn't a \texttt{=} character in
%    first position so we are dealing either with a ``minimum
%    date'' or with some incorrect data. We assume the former and make
%    the following assignments (the first one finishing the assignment
%    of |\pkgcls@targetdate|):
%    \begin{macrocode}
      \requestedLaTeXdate\relax
      \let\pkgcls@targetlabel\@empty
      \def\pkgcls@innerdate{\maxdimen}%
      \def\pkgcls@mindate{#1}%
%    \end{macrocode}
%    If the min-date is after the requested rollback date (if there is
%    any, i.e., if it is not zero) then we have a conflict and
%    therefore issue a warning.
% \changes{v1.2i}{2018/05/08}
%         {Make suspicious rollback a warning not error: github issue 43}
% \changes{v1.5l}{2024/06/04}
%         {New argument \cs{pkgcls@ext} (gh/870)}
%    \begin{macrocode}
      \ifnum \pkgcls@targetdate > \z@
        \ifnum \@parse@version0#1//00\@nil > \pkgcls@targetdate
          \@latex@warning@no@line{Suspicious rollback/min-date date given\MessageBreak
            A minimal date of #1 has been specified for
             \pkgcls@ext\MessageBreak '\pkgcls@name'.\MessageBreak
             But this is in conflict
             with a rollback request to \requestedpatchdate}
        \fi
      \fi
    \fi
}
%    \end{macrocode}
%    Strip off the trailing \texttt{=} and assign the version name to
%    |\pkgcls@targetlabel|.
%    \begin{macrocode}
\def\pkgcls@parse@date@arg@version#1={%
  \def\pkgcls@targetlabel{#1}}
%    \end{macrocode}
%  \end{macro}
%
%  \begin{macro}{\DeclareRelease}
%    First argument is the ``name'' of the release and it can be left empty
%    if one doesn't like to give a name to the release.
%    The second argument is that from which on this release was
%    available (or should be used in case of minor updates).
%    The final argument is the external file name of this release, by
%    convention this should be
%    \meta{pkg/cls-name}\texttt{-}\meta{date}\texttt{.}\meta{extension}
%    but this is not enforced and through this argument one can
%    overwrite it.
%    \begin{macrocode}
\def\DeclareRelease#1#2#3{%
  \ifnum\pkgcls@targetdate>\z@  % some sort of rollback request
%<*tracerollback>
    \pkgcls@debug{---\string\DeclareRelease:}%
    \pkgcls@debug{\@spaces 1: #1}%
    \pkgcls@debug{\@spaces 2: #2}%
    \pkgcls@debug{\@spaces 3: #3}%
%</tracerollback>
%    \end{macrocode}
%    If the date argument |#2| is empty we are dealing with a special
%    release that should be only accessible via its name; a typical
%    use case would be a ``beta'' release. So if we are
%    currently processing a date request we ignore it and otherwise we
%    check if we can match the name and if  so load the corresponding
%    release file.
%    \begin{macrocode}
    \ifx\@nil#2\@nil
      \ifnum\pkgcls@targetdate=\@ne  % named request
        \def\reserved@a{#1}%
        \ifx\pkgcls@targetlabel\reserved@a
          \pkgcls@use@this@release{#3}{}%
%<*tracerollback>
        \else
          \pkgcls@debug{Label doesn't match}%
%</tracerollback>
        \fi
%<*tracerollback>
      \else
        \pkgcls@debug{Date request: ignored}%
%</tracerollback>
      \fi
    \else
%    \end{macrocode}
%    If the value of |\pkgcls@targetdate| is greater than 1 (or in
%    reality greater than something like 19930101) we are dealing with a
%    rollback request to a specific date.
%    \begin{macrocode}
      \ifnum\pkgcls@targetdate>\@ne  % a real request
%    \end{macrocode}
%    So we parse the date of this release to check if it is before or
%    after the request date.
%    \begin{macrocode}
        \ifnum\@parse@version#2//00\@nil
             >\pkgcls@targetdate
%    \end{macrocode}
%    If it is after we have to distinguish between two cases: If there
%    was an earlier candidate we use that one because the other is too
%    late, but if there wasn't one (i.e., if current release is the
%    oldest that exists) we use it as the best choice. However in
%    that case something is wrong (as there shouldn't be a rollback to
%    a date when a package used didn't yet exists). So we make a
%    complained to the user.
%    \begin{macrocode}
          \ifx\pkgcls@candidate\@empty
            \pkgcls@rollbackdate@error{#2}%
            \pkgcls@use@this@release{#3}{#2}%
          \else
            \pkgcls@use@this@release\pkgcls@candidate
                                    \pkgcls@releasedate
          \fi
        \else
%    \end{macrocode}
%    Otherwise, if the release date of this version is before the
%    target rollback and we record it as a candidate. But we don't use
%    it yet as there may be another release which is still before the
%    target rollback.
%    \begin{macrocode}
          \def\pkgcls@candidate{#3}%
          \def\pkgcls@releasedate{#2}%
%<*tracerollback>
          \pkgcls@debug{New candidate: #3}%
%</tracerollback>
        \fi
      \else
%    \end{macrocode}
%    If we end up in this branch we have a named version request. So
%    we check if |\pkgcls@targetlabel| matches the current name and if
%    yes we use this release immediately, otherwise we do nothing as a
%    later declaration may match it.
%    \begin{macrocode}
        \def\reserved@a{#1}%
        \ifx\pkgcls@targetlabel\reserved@a
          \pkgcls@use@this@release{#3}{#2}%
%<*tracerollback>
        \else
          \pkgcls@debug{Label doesn't match}%
%</tracerollback>
        \fi
      \fi
    \fi
  \fi
}
%    \end{macrocode}
%  \end{macro}
%
%  \begin{macro}{\pkgcls@use@this@release}
%    If a certain release has been selected (stored in the external
%    file given in \verb=#1=) we need to input it and afterwards stop
%    reading the current file.
%    \begin{macrocode}
\def\pkgcls@use@this@release#1#2{%
%    \end{macrocode}
%    Before that we record the selection made inside the transcript.
%    \begin{macrocode}
   \pkgcls@show@selection{#1}{#2}%
%    \end{macrocode}
%    We then set the |\pkgcls@targetdate| to zero so that any
%    |\DeclareRelease| or |\DeclareCurrentRelease| in the file we
%    now load are bypassed\footnote{The older release may also have
%    such declarations inside if it was a simply copy of the
%    \texttt{.sty} or \texttt{.cls} file current at that
%    date. Removing these declarations would make the file load a tiny
%    bit faster, but this way it works in any case.} and then we
%    finally load the correct release.
%
%    After loading that file we need to stop reading the current file
%    so we issue |\endinput|. Note that the |\relax| before that is
%    essential to ensure that the |\endinput| is only happening after
%    the file has been fully processed, otherwise it would act after the
%    first line of the |\@@input|!
% \changes{v1.2e}{2018/03/24}
%         {Use full file name for old release}
% \changes{v1.5m}{2024/08/03}
%         {Add selected release to the file list (gh/1413)}
%    \begin{macrocode}
   \pkgcls@targetdate\z@
   \@addtofilelist{#1}%
   \@@input #1\relax
   \endinput
}
%    \end{macrocode}
%  \end{macro}
%
%  \begin{macro}{\pkgcls@show@selection}
%    This command records what selection was made. As that is needed
%    in two places (and it is rather lengthy) it was placed in a
%    separate command. The first argument is the name of the external
%    file that is being loaded and is only needed for debugging.  The
%    second argument is the date that corresponds to this file and it
%    is used as part of the message.
%    \begin{macrocode}
\def\pkgcls@show@selection#1#2{%
%<*tracerollback>
  \pkgcls@debug{Result: use  #1}%
%</tracerollback>
  \GenericInfo
   {\@spaces\@spaces\space}{Rollback for
    \@cls@pkg\space'\@currname' requested ->
    \ifnum\pkgcls@targetdate>\@ne
       date
       \ifnum\requestedLaTeXdate=\pkgcls@targetdate
          \requestedpatchdate
       \else
          \expandafter\@gobble\pkgcls@arg
       \fi.\MessageBreak
%    \end{macrocode}
%    Instead of ``best approximation'' we could say that we have been
%    able to exactly match the date (if it is exact), but that would
%    mean extra tests without much gain, so not done.
%    \begin{macrocode}
       Best approximation is
    \else
       version '\pkgcls@targetlabel'.\MessageBreak
       This corresponds to
    \fi
    \ifx\@nil#2\@nil
       a special release%
    \else
       the release introduced on #2%
    \fi
    \@gobble}%
}
%    \end{macrocode}
%  \end{macro}
%
%  \begin{macro}{\pkgcls@rollbackdate@error}
%    This is called if the requested rollback date is earlier than the
%    earliest known release of a package or class.
%
%    A similar error is given if global rollback date and min-date on
%    a specific package conflict with each other, but that case is
%    happens only once so it is inlined.
% \changes{v1.3u}{2020/11/09}{Change help text because the package may have
%    existed then --- there is just no rollback data (gh/423).}
%    \begin{macrocode}
\def\pkgcls@rollbackdate@error#1{%
  \@latex@error{Suspicious rollback date given}%
     {The \@cls@pkg\space'\@currname'  has no rollback data
      before #1 which\MessageBreak
      is after your requested rollback date --- so
      something may be wrong here.\MessageBreak
      Continue and we use the earliest known release.}}
%    \end{macrocode}
%  \end{macro}
%
%
%  \begin{macro}{\DeclareCurrentRelease}
%    This declares the date (and possible name) of the current version
%    of a package or class.
%    \begin{macrocode}
\def\DeclareCurrentRelease#1#2{%
%    \end{macrocode}
%    First we test if |\pkgcls@targetdate| is greater than zero,
%    otherwise this code is bypassed (as there is no rollback
%    request).
%    \begin{macrocode}
  \ifnum\pkgcls@targetdate>\z@  % some sort of rollback request
%<*tracerollback>
    \pkgcls@debug{---DeclareCurrentRelease}%
    \pkgcls@debug{   1: #1}%
    \pkgcls@debug{   2: #2}%
%</tracerollback>
%    \end{macrocode}
%    If the value is greater than 1 we have to deal with a date
%    request, so we parse |#2| as a date and compare it with
%    |\pkgcls@targetdate|.
%    \begin{macrocode}
    \ifnum\pkgcls@targetdate>\@ne  % a date request
      \ifnum\@parse@version#2//00\@nil
           >\pkgcls@targetdate
%    \end{macrocode}
%    If it is greater that means the release date if this file is
%    later than the requested rollback date. Again we have two cases:
%    If there was a previous candidate release we use that one as the
%    current release is too young, but if there wasn't we have to use
%    this release nevertheless as there isn't any alternative.
%
%    However this case can only happen if there is a
%    |\DeclareCurrentRelease| but no declared older releases (so
%    basically the use of the declaration is a bit dubious).
%    \begin{macrocode}
        \ifx\pkgcls@candidate\@empty
          \pkgcls@rollbackdate@error{#2}%
        \else
          \pkgcls@use@this@release\pkgcls@candidate
                                  \pkgcls@releasedate
        \fi
%    \end{macrocode}
%    Otherwise the current file is the right release, so we record that
%    in the transcript and then carry on.
%    \begin{macrocode}
      \else
        \pkgcls@show@selection{current version}{#2}%
      \fi
    \else % a label request
%    \end{macrocode}
%    Otherwise we have a rollback request to a named version so we
%    check if that fits the current name and if not give an error as
%    this was the last possible opportunity.
%    \begin{macrocode}
      \def\reserved@a{#1}%
      \ifx\pkgcls@targetlabel\reserved@a
        \pkgcls@show@selection{current version}{#2}%
      \else
        \@latex@error{Requested version '\pkgcls@targetlabel' for
          \@cls@pkg\space'\@currname' is unknown}\@ehc
      \fi
    \fi
  \fi
}
%    \end{macrocode}
%  \end{macro}
%
%
%  \begin{macro}{\IfTargetDateBefore}
%    This enables a simple form of conditional code inside a class or
%    package file. If there is a date request and the request date is
%    earlier than the first argument the code in the second argument
%    is processed otherwise the code in the third argument is
%    processed. If there was no date request then we also execute the
%    third argument, i.e., we will get the ``latest'' version of the
%    file.
%
%    Most often the second argument (before-date-code) will be empty.
%    \begin{macrocode}
\DeclareRobustCommand\IfTargetDateBefore[1]{%
  \ifnum\pkgcls@innerdate <%
        \expandafter\@parse@version\expandafter0#1//00\@nil
    \typeout{Exclude code introduced on #1}%
    \expandafter\@firstoftwo
  \else
    \typeout{Include code introduced on #1}%
    \expandafter\@secondoftwo
  \fi
}
%    \end{macrocode}
%  \end{macro}
%
%    \begin{macrocode}
%</2ekernel|latexreleasefirst>
%    \end{macrocode}
%
%
% \section{After Preamble}
% Finally we declare a package that allows all the commands declared
% above to be |\@onlypreamble| to be used after |\begin{document}|.
% \changes{v0.3f}{1994/03/16}
%         {Add pkgindoc package}
% \changes{v1.1a}{1998/03/21}
%         {Correct to new onlypreamble command list}
%    \begin{macrocode}
%<*afterpreamble>
\NeedsTeXFormat{LaTeX2e}
\ProvidesPackage{pkgindoc}
         [2020-08-08 v1.3m Package Interface in Document (DPC)]
\def\reserved@a#1\do\@classoptionslist#2\do\filec@ntents#3\relax{%
  \gdef\@preamblecmds{#1#3}}
\expandafter\reserved@a\@preamblecmds\relax
%</afterpreamble>
%    \end{macrocode}
%
% \Finale
