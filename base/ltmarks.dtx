% \iffalse meta-comment
%
%% File: ltmarks.dtx
%% Copyright (C) 2022-2024
%% Frank Mittelbach, The LaTeX Project
%
% It may be distributed and/or modified under the conditions of the
% LaTeX Project Public License (LPPL), either version 1.3c of this
% license or (at your option) any later version.  The latest version
% of this license is in the file
%
%    https://www.latex-project.org/lppl.txt
%
%
%%% From File: ltmarks.dtx
%
%<*driver>
% \fi
\ProvidesFile{ltmarks.dtx}
             [2024/12/25 v1.1b LaTeX Kernel (Marks)]
% \iffalse
%
\documentclass{l3doc}
\GetFileInfo{ltmarks.dtx}

\EnableCrossrefs
\CodelineIndex
\begin{document}
  \DocInput{ltmarks.dtx}
\end{document}
%</driver>
%
% \fi
%
% \providecommand\hook[1]{\texttt{#1}}
% \providecommand\env[1]{\texttt{#1}}
%
%
%
% \title{The \texttt{ltmarks.dtx} code\thanks{This file has version
%    \fileversion\ dated \filedate, \copyright\ \LaTeX\
%    Project.}}
% \author{^^A
%  Frank Mittelbach, \LaTeX{} Project\thanks
%    {^^A
%      E-mail:
%        \href{mailto:latex-team@latex-project.org}
%          {latex-team@latex-project.org}^^A
%    }^^A
% }
%
% \maketitle
%
%
% \begin{abstract}
%    Marks are used to communicate information about the content of a
%    page to the output routine. For example, in order to construct
%    running headers, the output routine needs information about which
%    section names are present on a page, and this information is
%    passed to it through the mark system. However, marks may also be
%    used for other purposes. This module provides a generalized
%    mechanism for marks of independent classes.
% \end{abstract}
%
% \tableofcontents
%
% ^^A \begin{documentation}
%
% \section{Introduction}
%
%    The \TeX{} engines offer a low-level mark mechanism to
%    communicate information about the content of the current page to
%    the asynchronous operating output routine. It works by placing
%    \cs{mark} commands into the source document. When the material
%    for the current page is assembled in box 255, \TeX{} scans for
%    such marks and sets the commands \cs{topmark}, \cs{firstmark} and
%    \cs{botmark}.  The \cs{firstmark} receives the content of the
%    first \cs{mark} seen in box 255 and \cs{botmark} the content of
%    the last mark seen.  The \cs{topmark} holds the content of the
%    last mark seen on the previous page or more exactly the value of
%    \cs{botmark} from the previous page.  If there are no marks on
%    the current page then all three are made equal to the
%    \cs{botmark} from the previous page.
%
%    This mechanism works well for simple formats (such as plain \TeX)
%    whose output routines are only called to generate pages. It
%    fails, however, in \LaTeX{} (and other more complex formats),
%    because here the output routine is sometimes called without
%    producing a page, e.g., when encountering a float and placing it
%    into one of the float regions. In that case the output routine is
%    called, determines where to place the float, alters the goal for
%    assembling text material (if the float was added to the top or
%    bottom region) and then it resumes collecting textual material.
%
%    As a result the \cs{botmark} gets updated and so \cs{topmark} no
%    longer reflects the situation at the top of the next page when that
%    page is finally boxed.
%
%    Another problem for \LaTeX{} was that it wanted to use several
%    \enquote{independent} marks and in the early implementations of
%    \TeX{} there was only a single \cs{mark} command available.
%    For that reason \LaTeX{} implemented its own mark
%    mechanism where the marks always contained two parts with their
%    own interfaces: \cs{markboth} and \cs{markright} to set marks and
%    \cs{leftmark} and \cs{rightmark} to retrieve them.
%
%    However, this extended mechanism (while supporting scenarios such
%    as chapter/section marks) was far from general. The mark
%    situation at the top of a page (i.e., \cs{topmark}) remained
%    unusable and the two marks offered were not really independent of
%    each other because \cs{markboth} (as the name indicates) was
%    always setting both.
%
%    The new mechanism overcomes both
%    issues:
%    \begin{itemize}
%    \item
%      It provides arbitrarily many, fully independent named marks, that
%      can be allocated and, from that point onwards, used.
%    \item
%      It offers access for each such marks to retrieve its top,
%      first, and bottom values separately.
%    \item
%      Furthermore, the mechanism is augmented to give access to marks
%      in different \enquote{regions} which may not be just full pages.
%    \end{itemize}
%
%
% \section{Design-level and code-level interfaces}
%
%    The interfaces are mainly meant for package developers, but they
%    are usable (with appropriate care) also in the document
%    preamble, for example, when setting up special running headers
%    with \pkg{fancyhdr}, etc. They are therefore available both as
%    CamelCase commands as well as commands for use in the L3
%    programming layer. Both are described together below.
%
% \begin{function}{\NewMarkClass,\mark_new_class:n}
%   \begin{syntax}
%     \cs{NewMarkClass} \Arg{class}
%     \cs{mark_new_class:n} \Arg{class}
%   \end{syntax}
%   Declares a new \meta{class} of marks to be tracked by \LaTeX{}. Each \meta{class}
%   must be declared before it is used.
%
%   Mark classes can only be declared before \verb=\begin{document}=.
% \end{function}
%
% \begin{function}{\InsertMark,\mark_insert:nn}
%   \begin{syntax}
%     \cs{InsertMark} \Arg{class} \Arg{text}
%     \cs{mark_insert:nn} \Arg{class} \Arg{text}
%   \end{syntax}
%   Adds a mark to the current galley for the \meta{class}, containing the
%   \meta{text}.
%
%   It has no effect in places in which you can't place floats, e.g.,
%   a mark inside a box or inside a footnote never shows up anywhere.
%
%   If used in vertical mode it obeys \LaTeX's internal
%   \cs{@nobreak} switch, i.e., it does not introduce a
%   breakpoint if used after a heading. If used in horizontal mode it
%   doesn't handle spacing (like, for example, \cs{index} or
%   \cs{label} does, so it should be attached to material that is
%   typeset.
% \end{function}
%
% \begin{variable}{insertmark}
%   \begin{syntax}
%     \cs{AddToHook} \texttt{\{insertmark\}} \Arg{code}
%   \end{syntax}
%  When marks are inserted, the mark content may need some special
%  treatment, e.g., by default \cs{label}, \cs{index}, and
%  \cs{glossary} do not expand at this time (but only later if and when the
%  mark content is actually used.
%  In order to allow packages to augment or alter this setup there is
%  a public hook \hook{insertmark} that is executed at this point.  It
%  runs in a group so local modification to commands are only applied
%  to the \meta{text} argument of \cs{InsertMark} or \cs{mark_insert:nn}.
% \end{variable}
%
% \begin{function}[EXP]{\TopMark, \FirstMark, \LastMark,
%      \mark_use_top:nn,\mark_use_first:nn,\mark_use_last:nn,}
%   \begin{syntax}
%     \cs{TopMark}   \oarg{region} \Arg{class}
%     \cs{FirstMark} \oarg{region} \Arg{class}
%     \cs{LastMark}  \oarg{region} \Arg{class}
%     \cs{mark_use_top:nn}   \Arg{region} \Arg{class}
%     \cs{mark_use_first:nn} \Arg{region} \Arg{class}
%     \cs{mark_use_last:nn}  \Arg{region} \Arg{class}
%   \end{syntax}
%   These functions expand to the appropriate mark \meta{text} for
%   the given \meta{class} in the specified \meta{region}.
%   The default \meta{region} in the design-level commands is \texttt{page}.
%   Note that with the L3 layer commands there are no
%   optional arguments, i.e., both arguments have to be provided.
%   \begin{texnote}
%     The result is returned within the \tn{unexpanded}
%     primitive (\cs{exp_not:n}), which means that the \meta{text}
%     does not expand further when appearing in an \texttt{x}-type
%     or \texttt{e}-type argument expansion.
%   \end{texnote}
%
%   The \enquote{first} and \enquote{last} marks are
%   those seen first and last in the current region/page, respectively. The
%   \enquote{top} mark is the last mark of the \meta{class} seen
%   in an earlier region, i.e., the \meta{text} what would be \enquote{current} at the
%   very top of the region.
%
%   \noindent\llap{\bfseries Important!\qquad}\indent
%   The commands are only meaningful inside the output routine, in
%   other places their result is (while not random) unpredictable due
%   to the way \LaTeX{} cuts text material into pages.
%   There is, however, one exception: if you produce multiple columns
%   using the \pkg{multicol} package, it is possible to retrieve mark
%   values from the regions \texttt{first-column},
%   \texttt{last-column}, \texttt{mcol-1},
%   \texttt{mcol-2},\ldots\ directly after the environment has
%   ended. This can, for example, be useful if a \env{multicols} has
%   been be used inside a box.
% \end{function}
%
%
%
%   Currently, \meta{region} is one of
%   \texttt{page},
%   \texttt{previous-page},
%   \texttt{column},
%   \texttt{previous-column},
%   \texttt{first-column},
%   \texttt{last-column}, and
%   \texttt{mcol-1} (first column in a \env{multicols}),
%   \texttt{mcol-2} (second column in a \env{multicols}),
%   up to 
%   \texttt{mcol-20} (twentieth column in a \env{multicols}).
%   See section~\ref{sec:regions} for discussion of how these regions
%   behave and how one can make use of them.
%
%
% \begin{function}[EXP]{\IfMarksEqualTF,\IfMarksEqualT,\IfMarksEqualF,\mark_if_eq:nnnnTF,\mark_if_eq:nnnnnnTF}
%   \begin{syntax}
%     \cs{IfMarksEqualTF}    \oarg{region} \Arg{class} \Arg{pos_1} \Arg{pos_2} \Arg{true} \Arg{false}
%     \cs{mark_if_eq:nnnnTF} \Arg{region} \Arg{class} \Arg{pos_1} \Arg{pos_2} \Arg{true} \Arg{false}
%     \cs{mark_if_eq:nnnnnnTF} \Arg{region_1} \Arg{class_1} \Arg{pos_1}
%  \verb=                    = \Arg{region_2} \Arg{class_2} \Arg{pos_2} \Arg{true} \Arg{false}
%   \end{syntax}
%   These conditionals allow you to compare the content of two marks
%   and act based on the result. The commands work in an expansion
%   context, if necessary.
% \end{function}
%
%    It is quite common when programming with marks to need to
%    interrogate conditions such as whether marks have appeared on a
%    previous page, or if there are multiple marks present on the
%    current page, and so on.
%    The tests above allow for the construction of a variety of
%    typical test scenarios, with three examples presented below.
%
%    The first two conditionals cover only the common scenarios.  Both
%    marks are picked up from the same \meta{region} (by default
%    \texttt{page}) and they have to be of the same
%    \meta{class}.\footnote{If an undeclared mark class is used the
%    tests return \emph{true} (not an error).}
%    The \meta{pos\textsubscript{\itshape i}} argument can be either
%    \texttt{top}, \texttt{first}, or \texttt{last}.
%
%    Important to note is that the comparison is not with respect to
%    the textual content of the marks but whether or not they
%    originated from the same \cs{InsertMark} command (or the L3 layer
%    version \cs{mark_insert:nn}).
%
%    If you wish to compare marks across different regions or across
%    different classes, you have to do it using the generic test only
%    available in the L3 programming layer or do it manually, i.e.,
%    get the marks and then compare the values yourself.\footnote{If
%    two undeclared mark classes are compared the result is always
%    \emph{true}; if a declared and an undeclared mark class is used
%    it is always \emph{false}.}
%
% \subsection{Use cases for conditionals}
%
%   However, the basic version is enough for the following typical use cases:
%   \begin{description}
%   \item[Test for at most one mark of class \texttt{myclass} on current
%      page:]
%
%      If the first and last mark in a region are the same then
%      either there was no mark at all, or there was at most one. To test
%      this on the current page:
%\begin{verbatim}
%  \NewMarkClass{myclass}
%  \IfMarksEqualTF{myclass}{first}{last}
%                 { <zero or one mark> }{ <two or more marks> }
%\end{verbatim}
%
%   \item[Test for no mark of class \texttt{myclass} in the previous
%      page:]
%
%      If the top mark is the same as the first mark, there is no mark
%      in the region at all. If we wanted to do this test for the
%      previous page:
%\begin{verbatim}
%  \IfMarksEqualTF[previous-page]{myclass}{top}{first}
%                 { <no marks> }{ <at least one mark> }
%\end{verbatim}
%      Comparing \texttt{top} and \texttt{last} would give you the
%      same result.
%
%   \item[Test for zero, one, or more than one:]
%
%      Combining the two tests from above you can test for zero, one
%      or more than one mark.
%\begin{verbatim}
%  \IfMarksEqualTF{myclass}{top}{first}
%                 { <no marks> }
%                 {\IfMarksEqualTF{myclass}{first}{last}
%                   { <exactly one mark> }{ <more than one mark> }}
%\end{verbatim}
%
% \end{description}
%
% If  you need one of such tests more often (or if you want a separate
% command for it for readability), then consider defining:
%\begin{verbatim}
%  \providecommand\IfNoMarkTF[2][page]{\IfMarksEqualTF[#1]{#2}{first}{last}}
%\end{verbatim}
%
%
%
% \subsection{Understanding regions}
%
%   If a page has just been finished then the region \texttt{page}
%   refers to the current page and \texttt{previous-page}, as the name
%   indicates, refers to the page before the current page. This
%   means you are able to access mark information for the current page
%   as well as for the page before (as long as you are inside the output
%   routine) without the need to explicitly save that information
%   beforehand. The \texttt{page} region is the region that is most
%   often queried, which is why commands like \cs{FirstMark} use that
%   region by default.
%
%   In single column documents the \texttt{column} is the same as the
%   \texttt{page} region, but in two-column documents (if not produced
%   by \env{multicols}), \texttt{column} refers to the current column
%   that just got finished and \text{previous-column} to the one
%   previously finished.  Code for running headers is (in standard
%   \LaTeX{}) evaluated only after both columns have been assembled, which is
%   another way of saying that in that case \texttt{previous-column}
%   refers to the left column and \texttt{column} to the right column.
%   However, to make these somewhat easier to use, there are also aliased names
%   for these two regions: \texttt{first-column} and
%   \texttt{last-column}.\footnote{The region is called \enquote{last-column} not
%   \enquote{second-column} in anticipation of extending the mechanism to
%   multiple columns, where first and last would still make sense.
%   There aren't any
%   \texttt{previous-first-column} and \texttt{previous-last-column}
%   regions to access the corresponding
%   columns from the previous page.}
%
%   Note that you can only look backwards at already processed
%   regions, e.g., in a \texttt{twoside} document finishing a recto
%   (odd, right-hand) page you can access the data from the facing
%   verso (left-hand) page, but if you are finishing a left-hand page
%   you can't integrate data from the upcoming right-hand page. If
%   such a scenario needs to be realized then it is necessary to save
%   the left-hand page temporarily instead of finalizing it, process
%   material for the right-hand page and once both are ready, attach
%   running headers and footers and shipout out both in one
%   go.\footnote{As of now that scenario is not (yet) officially
%   supported but it would be possible to achieve this using the
%   shipout hooks to store the verso page and then on the next shipout
%   use the hook to shipout both with running headers and footers
%   attached.}
%
%   The situation starts getting rather complex if you allow for
%   multiple columns in the way they are supported by the
%   \pkg{multicol} package. In this case you might have a variable
%   number of \enquote{columns} on a single page to be shipped
%   out. And even if not, then a \env{multicols} might start or end in
%   the middle of the page; in either case, the regions \texttt{column} and
%   \texttt{previous-column} become rather meaningless and you should
%   therefore not use them.\footnote{They return something, because
%   they represent the last two columns of the \env{multicols} when
%   you are inside the output routine, but that is obviously of little
%   use.}
%   Instead, the algorithm offers \texttt{mcol-1}, \texttt{mcol-2},
%   \texttt{mcol-3}, etc., to represent the columns in the
%   \env{multicols} on the current page to be shipped out. If there is
%   more than one \env{multicols} on the current page then in the output routine 
%   only the columns of the last one will be accessible.
%
%   These provisions cover, out of the box, a number of layouts and use cases, but
%   obviously not all. However, more cases can be supported by storing away
%   mark information during the processing.  Here is the full
%   algorithm:
%   \begin{itemize}
%   \item

%      The \texttt{column} region is used by the \enquote{current
%      column} that is being built (moving through all columns with
%      \texttt{previous-column} trailing behind (to handle top marks
%      properly).
%
%   \item

%      When the \env{multicols} starts, the \texttt{column} region is
%      cleared, i.e., from that point on it looks as if there have not
%      been any marks so far. This will make sure that the top mark in
%      the first column is always empty.
%
%   \item
%
%      If the \env{multicols} extends beyond the current page, then the
%      material designated for the current page is split into columns.
%      The \texttt{column} region is used to represent each column in
%      turn.

%   \begin{itemize}
%   \item
%
%       First we copy the current data from \texttt{column} to
%      \texttt{previous-column}. Then the mark data from the current
%      column is placed into the \texttt{column} region. Then we alias
%      \texttt{column} to \texttt{mcol-1}.
%
%   \item These steps are repeated for all columns of the \env{multicols}
%      environment.
%
%   \item Finally, the first and the last column of that page is also
%      made available as \texttt{first-column} and
%      \texttt{last-column}, respectively.
%   \end{itemize}
%
%   \item
%      All those marks inside any of the columns are also available in the
%      \texttt{page} region. Thus, if you are interested in the top,
%      first, or last mark of a specific class on the whole page you
%      simply need to query for it in the \texttt{page} region.
%
%   \item
%       If the \env{multicols} continues across several pages then
%      this algorithm above is repeated for each page, except that the \texttt{column}
%      region is not cleared again. This means that the top mark of
%      the first column of the next page will be the last mark of the
%      last column from the previous page.
%
%   \item
%      When the \env{multicols} finishes the remaining material for the
%      current page is balanced to produce columns of roughly equal height.
%
%   \item
%       Again \texttt{column} and \texttt{previous-column} are used
%       while this balancing happens and \texttt{mcol-1},
%       \texttt{mcol-2}, etc., are used to represent the column
%      regions and \texttt{first-column} and \texttt{last-column} are
%      set appropriately.
%
%   \item
%       Then the balanced set of columns is returned back to the page
%      (since there may be space for further material). In addition, all
%      marks inside that material are reinserted so that they become
%      available in the \texttt{page} region.
%
%   \item As a side effect, it is possible (and useful in certain
%      circumstances) to query for mark classes directly after the
%      \env{multicols} has ended without the need to be inside the output
%      routine. The regions that can be queried this way are
%      \texttt{mcol-1}, \texttt{mcol-2}, etc.\ (up to the number of columns
%      the multicol had) and \texttt{first-column} and \text{last-column}.
%
%   \end{itemize}
%
%
%
% \subsection{Debugging mark code}
%
%
% \begin{function}{\DebugMarksOn,\DebugMarksOff,
%                  \mark_debug_on:,\mark_debug_off:}
%   \begin{syntax}
%     \cs{DebugMarksOn} \ldots\ \cs{DebugMarksOff}
%   \end{syntax}
%
%    Commands to turn the debugging of mark code on or off. The
%    debugging output is rather coarse and not really intended for
%    normal use at this point in time.
%
% \end{function}
%
%
%
% \section{Application examples}
%
% If you want to figure out if a break was taken at a specific point,
% e.g., whether a heading appears at the top of the page,
% you can do something like this:
%\begin{verbatim}
% \newcounter{breakcounter}
% \NewMarkClass{break}
% \newcommand\markedbreak[1]{\stepcounter{breakcounter}%
%                            \InsertMark{break}{\arabic{breakcounter}%
%                            \penalty #1\relax
%                            \InsertMark{break}{-\arabic{breakcounter}}
%\end{verbatim}
% To test if the break was taken you can test if
% \verb=\TopMark{break}= is positive (taken) or negative (not taken)
% or zero (there was never any marked break so far).
%  The absolute value can be used to keep track of which break it
% was (with some further coding).
%
%
% \emph{to be extended with additional application examples}
%
%
%
% \section{Legacy \LaTeXe{} interface}
%
%  Here we describe the interfaces that \LaTeXe{} offered since the
%  early nineties and some minor extensions.
%
% \subsection{Legacy design-level and document-level interfaces}
%
% \begin{function}{\markboth, \markright}
%   \begin{syntax}
%     \cs{markboth}  \Arg{left} \Arg{right}
%     \cs{markright} \Arg{right}
%   \end{syntax}
% \LaTeXe{} uses two marks which aren't fully independent. A
%   \enquote{left} mark generated by the first argument of \cs{markboth}
%   and a \enquote{right} mark generated by the second argument of
%   \cs{markboth} or by the only argument of \cs{markright}. The
%   command \cs{markboth} and \cs{markright} are in turn called from
%   heading commands such as \cs{chaptermark} or \cs{sectionmark} and
%   their behavior is controlled by the document class.
%
% For example, in the \cls{article} class with \texttt{twoside} in
%   force the \cs{sectionmark} will issue \cs{markboth} with an empty
%   second argument and \cs{subsectionmark} will issue
%   \cs{markright}. As a result the left mark will contain chapter
%   titles and the right mark subsection titles.
%
% Note, however, that in one-sided documents the standard behavior is
%   that only \cs{markright} is used, i.e., there will only be
%   right-marks but no left marks!
% \end{function}
%
% \begin{function}[EXP]{\leftmark, \rightmark}
%   \begin{syntax}
%     \cs{leftmark}
%     \cs{rightmark}
%   \end{syntax}
%   These functions return the appropriate mark value from the current page
%   and work as before, that is \cs{leftmark} will get the last (!)
%   left mark from the page and \cs{rightmark} the first (!) right
%   mark.
%
%   In other words they work reasonably well if you want to show the
%   section title that is current when you are about to turn the page and
%   also show the first subsection title on the current page (or the last
%   from the previous page if there wasn't one). Other combinations
%   can't be shown using this interface.
%
%   The commands are fully expandable, because this is how they have
%   been always defined in \LaTeX{}. However, this is of course
%   only true if the content of the mark they return is itself
%   expandable and does not contain any fragile material. Given that
%   this can't be guaranteed for arbitrary content, a programmer using
%   them in this way should use \cs{protected@edef} and \emph{not}
%   \cs{edef} to avoid bad surprises as far as this is possible, or use
%   the new interfaces (\cs{TopMark}, \cs{FirstMark}, and \cs{LastMark})
%   which return the \meta{text} in \cs{exp_not:n} to prevent
%   uncontrolled expansion.
% \end{function}
%
%
% \subsection{Legacy interface extensions}
%
%   The new implementation adds three mark classes: \texttt{2e-left},
%   \texttt{2e-right} and \texttt{2e-right-nonempty} and patches
%   \cs{markboth} and \cs{markright} slightly so that they also update
%   these new mark classes, so that the new classes work with existing
%   document classes.
%
%   As a result you can use \verb=\LastMark{2e-left}= and
%   \verb=\FirstMark{2e-right}= instead of \cs{leftmark} and
%   \cs{rightmark}. But more importantly, you can use any of the other
%   retrieval commands to get a different status value from those
%   marks, e.g., \verb=\LastMark{2e-right}= would return the last
%   subsection on the page (instead of the first as
%   returned by \cs{rightmark}).
%
%   The difference between \texttt{2e-right} and
%   \texttt{2e-right-nonempty}  is that the latter will only be updated
%   if the material for the mark is not empty. Thus
%   \verb=\markboth{title}{}= as issued by, say, \cs{sectionmark},
%   sets a \texttt{2e-left} mark with \texttt{title} and a
%   \texttt{2e-right} mark with the empty string but does not add a
%   \texttt{2e-right-nonempty} mark.
%
%   Thus, if you have a section at the start of a page and you would
%   ask for \verb=\FirstMark{2e-right}= you would get an empty string
%   even if there are subsections on that page. But
%   \texttt{2e-right-nonempty} would then give you the first or last subsection
%   on that page. Of course, nothing is simple. If there are no
%   subsections it would tell you the last subsection from an earlier
%   page. We therefore need comparison tools, e.g., if top and
%   first are identical you know that the value is
%   bogus, i.e., a suitable implementation would be
%\begin{verbatim}
% \IfMarksEqualTF{2e-right-nonempty}{top}{first}
%                { <appropriate action if there was no real mark> }
%                {\FirstMark{2e-right-nonempty}}
%\end{verbatim}
%
%
%
% \section{Notes on the mechanism}
%
% In contrast to vanilla \TeX, \eTeX{} extends the mark system to
% allow multiple independent marks. However, it does not solve the
% \cs{topmark} problem which means that \LaTeX{} still needs to manage
% marks almost independently of \TeX{}. The reason for this is that
% the more complex output routine used by \LaTeX{} to handle floats
% (and related structures) means that \tn{topmark(s)} remain
% unreliable. Each time the output routine is fired up, \TeX{} moves
% \tn{botmark} to \tn{topmark}, and while \eTeX{} extends this to
% multiple registers the fundamental concept remains the same. That
% means that the state of marks needs to be tracked by \LaTeX{}
% itself. An early implementation of this package used \TeX{}'s
% \tn{botmark} only to ensure the correct interaction with the output
% routine (this was before the \eTeX{} mechanism was even
% available). However, other than in a prototype implementation for
% \LaTeX3, this package was never made public.
%
% The new implementation now uses \eTeX{}'s marks as they have some
% advantages, because with them we can leave the mark text within the
% galley and only extract the marks during the output routine when we
% are finally shipping out a page or storing away a column for use in
% the next page. That means we do not have to maintain a global data
% structure that we have to keep in sync with informational marks in
% the galley but can rely on everything being in one place and thus
% manipulations (e.g.~reordering of material) will take the marks with
% them without a need for updating a fragile linkage.

% To allow for completely independent marks we use the following
% procedure:
% \begin{itemize}
%   \item
%
%     For every type of marks we allocate a mark class so
%     that in the output routine \TeX{} can calculate for each class
%     the current
%     top, first, and bottom mark independently. For this we use
%     \cs{newmarks}, i.e., one marks register per class.
%
%   \item
%
%     As already mentioned firing up an output routine without
%     shipping out a page means that \TeX's top marks get wrong so it
%     is impossible to rely on \TeX's approach directly. What we do
%     instead is to keep track of the real marks (for the last page or
%     more generally last region) in some global variables.
%
%   \item
%
%     These variables are updated in the output routine at defined
%     places, i.e., when we do real output processing but not if we
%     use special output routines to do internal housekeeping.
%
%   \item
%
%     The trick we use to get correctly updated variables is the
%     following: the material that contains new marks (for example the
%     page to be shipped out) is stored in a box. We then use \TeX{}
%     primitive box splitting functions by splitting off the largest
%     amount possible (which should be the whole box if nothing goes
%     really wrong). While that seems a rather pointless thing to do, it
%     has one important side effect: \TeX{} sets up first and bottom
%     marks for each mark class from the material it has split off. This
%     way we get the first and last marks (if there have been any) from
%     the material in the box.
%
%   \item
%
%     The top marks are simply the last marks from the previous
%     page or region. And if there hasn't been a first or bottom mark in
%     the box then the new top mark also becomes new first and last mark
%     for that class.
%
%   \item
%
%     That mark data is then stored in global token lists for use
%     during the output routine and legacy commands such as
%     \cs{leftmark} or new commands such as \cs{TopMark} simply access
%     the data stored in these token lists.
%  \end{itemize}
%  That's about it in a nutshell. Of course, there are some details to
%  be taken care of---those are discussed in the implementation sections.
%
%
%
%
% \section{Public interfaces for packages such as \pkg{multicol}}
%
% The functions in this section are public so that packages can make
% use of them. However, this must be done with great care, e.g.,
% \cs{mark_update_structure_from_material:nn} updates the global mark
% structure and can therefore be used only in places where such an
% update is meaningful, e.g., in special output routines. Elsewhere, a
% change to the mark structure would break the whole mechanism and
% querying the marks would return incorrect data.
%
% \begin{function}{\mark_update_structure_from_material:nn}
%   \begin{syntax}
%     \cs{mark_update_structure_from_material:nn} \Arg{region} \Arg{material with marks}
%   \end{syntax}
%   Helper function that inspects the marks
%   inside the second argument and assigns new mark values based on
%   that to the \meta{region} given in the first argument.
%   For this it first copies the mark structure from \meta{region} to
%   \texttt{previous-}\meta{region} and then takes all last mark
%   values currently in the region and makes them the new top mark
%   values. Finally it assigns new first and last values for all mark
%   classes based on what was found in the second argument.
%
%   As a consequence, the allowed values for \meta{region} are
%   \texttt{page} and \texttt{column} because only they have
%   \texttt{previous-...} counterparts.
%
%   Another important aspect to keep in mind is that marks are recognized
%   only if they appear on the top level, e.g., if we want to
%   process material stored in boxes we need to put it unboxed (using
%   \cs{unvcopy} etc.)\ into the second argument.
% \end{function}
%
%
%
% \begin{function}{\mark_copy_structure:nn}
%   \begin{syntax}
%     \cs{mark_copy_structure:nn} \Arg{alias} \Arg{source}
%   \end{syntax}
%   Helper function that copies all mark values in the \meta{source}
%   region to \meta{alias}, i.e., make the structures identical. Used
%   to update the \texttt{previous-...} structures inside
%   \cs{mark_update_structure_from_material:nn} and \texttt{first-column} and
%   \texttt{last-column} structures inside the internal commands
%   \cs{__mark_update_singlecol_structures:} or
%   \cs{__mark__update_dblcol_structures:}.
% \end{function}
%
%
%
% \begin{function}{\mark_set_structure_to_err:n}
%   \begin{syntax}
%     \cs{mark_set_structure_to_err:n} \Arg{region}
%   \end{syntax}
%   Helper function that sets all mark values in the \meta{region} to
%   an error message. This is currently used for \texttt{last-column}
%   at times where using marks from it would be questionable/wrong, i.e.,
%   when we have just processed the first column in a two-column document.
% \end{function}
%
%
%
% \begin{function}{\mark_clear_structure:n}
%   \begin{syntax}
%     \cs{mark_clear_structure:n} \Arg{region}
%   \end{syntax}
%   Helper function that sets all mark values in the \meta{region} to
%   empty. This is currently used for \texttt{column} when a multicol
%   environment starts; this is because it wouldn't make sense if the top mark
%   in the first column returned the last mark from a previous
%   multicol (which may have been much earlier, with intermediate
%   material).
% \end{function}
%
%
% \begin{function}{\mark_get_marks_for_reinsertion:nNN}
%   \begin{syntax}
%     \cs{mark_get_marks_for_reinsertion:nNN} \Arg{source}
%           \qquad \meta{token-list-var for collecting first marks}
%           \qquad \meta{token-list-var for collecting last marks}
%   \end{syntax}
%   Helper function for extracting marks that would otherwise get
%    lost, for example when they are hidden inside a box. This helper
%    does not update mark structures and can therefore be used outside
%    the output routine as well.
%
%    It collects all the top-level marks from inside the \meta{source}
%    and then  adds suitable \cs{mark_insert:nn} commands  to each of the two token
%    lists. These token lists can then be executed at the right place
%    to reinsert the marks, e.g., directly after the box. This is, for
%    example, going to be used\footnote{Probably not before 2025, though.}
%    by \pkg{multicol} when a short balanced
%    \env{multicols} is returned to the galley for typesetting.
%
%    If the \meta{source} consists of a single vertical box (plus
%    possibly followed by some glue but nothing else) then the box is
%    unpacked and the top-level marks are collected from its
%    content. However, if it is not a vertical box or there are
%    other data then nothing is unpacked and you have to do the
%    unpacking yourself to get at the marks inside.
%
%    It is quite likely that one only needs a single token list for
%    returning the \cs{mark_insert:nn} statements. If that is the case
%    this command may change to take only two arguments.
% \end{function}
%
%
% \section{Internal functions for the standard output routine of \LaTeX{}}
%
% The functions in this section are tied to the output routine and used in the
% interface to \LaTeXe{} and perhaps at some later time within a new
% output routine
% for \LaTeX. They are not (yet) meant for general use and are
% therefore made internal, even though we already use them in
% \pkg{multicol}. 
% Internal means that \verb|@@| automatically gets
% replaced in the code (and in the documentation) so we have to give
% it a suitable value.
%    \begin{macrocode}
%<@@=mark>
%    \end{macrocode}
%
% \begin{function}{\@@_update_singlecol_structures:}
%   \begin{syntax}
%     \cs{@@_update_singlecol_structures:}
%   \end{syntax}
%   \LaTeXe{} integration function in case we are doing single column
%   layouts.  It assumes that the page content is already stored in
%   \cs{@outputbox} and processes the marks inside that box. It is
%   called as part of \cs{@opcol}.
% \end{function}
%
%
% \begin{function}{\@@_update_dblcol_structures:}
%   \begin{syntax}
%     \cs{@@_update_singlecol_structures:}
%   \end{syntax}
%   \LaTeXe{} integration function mark used when we are doing double
%   column documents. It assumes that the page content is already
%   stored in \cs{@outputbox} and processes the marks inside that
%   box. It then does different post-processing depending on the start
%   of the switch \cs{if@firstcolumn}. If we are in the second column
%   it also has to update page marks, otherwise it only updates column
%   marks. It too is called as part of \cs{@opcol}.
% \end{function}
%
%
% ^^A \end{documentation}
%
%
%
%
%
% \StopEventually{\setlength\IndexMin{200pt}  \PrintIndex  }
%
%
% \section{The Implementation}
%
%
%
%    \begin{macrocode}
%<*2ekernel|latexrelease>
%    \end{macrocode}
%
%    \begin{macrocode}
\ExplSyntaxOn
%    \end{macrocode}
%
%    \begin{macrocode}
%<latexrelease>\NewModuleRelease{2022/06/01}{ltmarks}
%<latexrelease>                 {Marks~handling}
%    \end{macrocode}
%
% \subsection{Allocating new mark classes}
%
%
% \begin{variable}{\g_@@_classes_seq}
%    A list holding all the mark classes that have been declared.
%    \begin{macrocode}
\seq_new:N \g_@@_classes_seq
%    \end{macrocode}
% \end{variable}
%
%
%
%
% \begin{macro}{\mark_new_class:n,\@@_new_class:nn}
%    
%    A mark class is created by initializing a number of data
%    structures.  First, we get a register number to refer to the mark
%    class.  The new mark class is then added to the
%    \cs{g_@@_classes_seq} sequence to be able to easily loop over all
%    classes.  Finally a number of top-level global token lists are
%    declared that hold various versions of the mark for access.
%    \begin{macrocode}
\cs_new_protected:Npn \mark_new_class:n #1
{
  \seq_if_in:NnTF \g_@@_classes_seq {#1}
      {
        \msg_error:nnn { mark } { class-already-defined }
          {#1}
      }
      { \@@_new_class:nn {#1} }
}
%    \end{macrocode}
%    This is only available in the preamble.
% \changes{v1.0c}{2022/05/06}{Wrong command made \cs{@onlypreamble}}
%    \begin{macrocode}
\@onlypreamble \mark_new_class:n
%    \end{macrocode}
%    The internal command carries out the necessary allocations.
%    \begin{macrocode}
\cs_new_protected:Npn \@@_new_class:nn #1
{
%<*trace>
  \@@_debug:n { \iow_term:x { Marks:~new~mark:~#1~\msg_line_context: } }
%</trace>
%    \end{macrocode}
%    Use the \LaTeXe{} interface for now as the L3 programming layer
%    doesn't have one for marks yet.
%    \begin{macrocode}
  \exp_args:Nc \newmarks {c_@@_class_ #1 _mark}
%    \end{macrocode}
%    Remember the new class in the sequence.
%    \begin{macrocode}
  \seq_gput_right:Nn \g_@@_classes_seq {#1}
%    \end{macrocode}
%    \begin{macrocode}
  \@@_init_region:nn {page}{#1}
%    \end{macrocode}
%    For the \texttt{page} region we also keep track of the
%    \texttt{previous-page}.
%    \begin{macrocode}
  \@@_init_region:nn {previous-page}{#1}
%    \end{macrocode}
%    Same game for \texttt{column} and \texttt{previous-column}
%    \begin{macrocode}
  \@@_init_region:nn {column}{#1}
  \@@_init_region:nn {previous-column}{#1}
%    \end{macrocode}
%    But for columns we also allocate token lists for the alias
%    regions \texttt{first-column} and \texttt{last-column}.
%    \begin{macrocode}
  \@@_init_region:nn {first-column}{#1}
  \@@_init_region:nn {last-column}{#1}
%    \end{macrocode}
%    To support multiple columns produced by the \pkg{multicol}
%    package, we preallocate twenty alias regions (since this is the number of columns that
%    \pkg{multicol} supports as a maximum). They are filled by copying
%    the current \texttt{column} into the appropriate \texttt{mcol-...}.  
%    \begin{macrocode}
%fmi  \@@_init_region:nn {mcol}{#1}
%fmi  \@@_init_region:nn {previous-mcol}{#1}
  \@@_init_region:nn {mcol-1}{#1}
  \@@_init_region:nn {mcol-2}{#1}
  \@@_init_region:nn {mcol-3}{#1}
  \@@_init_region:nn {mcol-4}{#1}
  \@@_init_region:nn {mcol-5}{#1}
  \@@_init_region:nn {mcol-6}{#1}
  \@@_init_region:nn {mcol-7}{#1}
  \@@_init_region:nn {mcol-8}{#1}
  \@@_init_region:nn {mcol-9}{#1}
  \@@_init_region:nn {mcol-10}{#1}
  \@@_init_region:nn {mcol-11}{#1}
  \@@_init_region:nn {mcol-12}{#1}
  \@@_init_region:nn {mcol-13}{#1}
  \@@_init_region:nn {mcol-14}{#1}
  \@@_init_region:nn {mcol-15}{#1}
  \@@_init_region:nn {mcol-16}{#1}
  \@@_init_region:nn {mcol-17}{#1}
  \@@_init_region:nn {mcol-18}{#1}
  \@@_init_region:nn {mcol-19}{#1}
  \@@_init_region:nn {mcol-20}{#1}
}
%    \end{macrocode}
% \end{macro}
%
%
%  \begin{macro}{\@@_init_region:nn,\c_@@_empty_tl}
% \changes{v1.1b}{2024/12/03}{Fix inconsistent local/global assignment (gh/1574)}
%
%    For each class (\texttt{\#2}) and region (\texttt{\#1}), we need
%    three token lists: one for top, first, and last. The default value to be
%    returned is \enquote{empty}.
%    \begin{macrocode}
\cs_new_protected:Npn \@@_init_region:nn #1 #2 {
  \tl_new:c { g_@@_#1_top_   #2 _tl }
  \tl_new:c { g_@@_#1_first_ #2 _tl }
  \tl_new:c { g_@@_#1_last_  #2 _tl }
  \tl_gset_eq:cN  { g_@@_#1_top_   #2 _tl } \c_@@_empty_tl
  \tl_gset_eq:cN  { g_@@_#1_first_ #2 _tl } \c_@@_empty_tl
  \tl_gset_eq:cN  { g_@@_#1_last_  #2 _tl } \c_@@_empty_tl
}
%    \end{macrocode}
%
%    All marks will have an identification in the form of a
%    number\footnote{There are a few cases where special
%    identification strings are used, e.g., \texttt{2.09-compat}.} that is
%    incremented each time a mark insertion happens;  therefore the
%    initial empty values should also have such a number, so that data extraction
%    will be uniform.
% \changes{v1.0g}{2024/05/31}{Initialize all marks with an id, use
%    \texttt{0} when a new class is made (gh/1359)}
%    \begin{macrocode}
\tl_const:Nn \c_@@_empty_tl { \@@_value:nn{0}{} }  
%    \end{macrocode}
%  \end{macro}
%
%
% \subsection{Updating mark structures}
%
%
%  \begin{macro}{\l_@@_box,\l_@@_ii_box,\g_@@_tmp_tl,\g_@@_new_top_tl}
%    For some operations we need two temporary private boxes and two
%    private global token lists.
%    \begin{macrocode}
\box_new:N \l_@@_box
\box_new:N \l_@@_ii_box
\tl_new:N  \g_@@_tmp_tl
\tl_new:N  \g_@@_new_top_tl
%    \end{macrocode}
%  \end{macro}
%
%
%
%  \begin{macro}{\@@_extract_and_handle_marks:nn}
% \changes{v1.0e}{2024/01/29}{Macro added}
%    
%    This is the main macro to extract and handle marks inside some
%    vertical material. It is used by
%    \cs{mark_update_structure_from_material:nn} (for updating the mark
%    structure for a region based on the marks found) and by
%    \cs{mark_get_marks_for_reinsertion:nNN} (for extracting marks from
%    some material and prepare for reinserting them later (e.g., out
%    of a box that is placed as a box into the main galley).
%    \begin{macrocode}
\cs_new_protected:Npn \@@_extract_and_handle_marks:nn #1#2 {
%    \end{macrocode}
%    This macro expects code to handle extracted marks in its first argument and
%    vertical material (not boxed or just consisting of a single
%    vertical box) as its second. It
%    extracts top-level mark information from \verb/#2/, stores them
%    as split marks and then calls
%    \verb/#1/ to make use of this information.
%  
%    If it finds a forced break in the material it removes it and then
%    restarts the attempt without it.
%
%    We start with a group to keep most changes local.
%    \begin{macrocode}
    \group_begin:
%    \end{macrocode}
%    
%    Getting the first and last marks out of the material in \verb=#2=
%    is done by putting the material in a box and then doing a split
%    operation to the maximum size possible (which hopefully gets us
%    all of the content).\footnote{With normal column material cut
%    from the main galley we should always get all material in one go,
%    but in certain situations, for example, in a \pkg{multicols}
%    environment that contains some \cs{columnbreak}s a single split
%    operation will not be enough. Thus, this is something we need to
%    handle.}  Because this action is used only to get the mark
%    values, we don't want any underfull box warnings so we (locally)
%    turn those off.
%
%    \begin{macrocode}
      \dim_set_eq:NN \tex_splitmaxdepth:D \c_max_dim
      \int_set_eq:NN \tex_vbadness:D      \c_max_int
      \dim_set_eq:NN \tex_vfuzz:D         \c_max_dim
%    \end{macrocode}
%    
%    There is a further complication: if the material contains infinite
%    shrinking glue then a \tn{vsplit} operation will balk with a
%    low-level error. Now pages or columns, which are our main concern
%    here, can't have such infinite shrinkage if they are cut straight
%    from the galley, however the use of \tn{enlargethispage} actually
%    does add some at the very bottom (and also wraps the whole page
%    into a box by itself, so if we leave it this way then a) we get
%    this error and b) we don't see any marks because they are hidden
%    one level down).
%      
%    Another possible issue are packages or user code that place stray
%    \tn{vbox}es directly into the main galley (an example is
%    \pkg{marginnote} that attaches its marginals in this way). If such
%    boxes end up as the last item on the page we should not unpack
%    them.
%
%    All these issues need to be handled, which is done in
%    \cs{@@_prepare_and_extract:nn}.     
%
%    \begin{macrocode}
      \@@_prepare_and_extract:nn {#1} {#2}
%    \end{macrocode}
%    Once all mark classes have been processed, the data structures are
%    updated and we can close the group, which undoes our local
%    changes and retains only the global ones.
%    \begin{macrocode}
    \group_end:
  }
%    \end{macrocode}
%  \end{macro}
%
%
%  \begin{macro}{\@@_prepare_and_extract:nn}
%
%    This macro does the dirty work. It is not directly integrated in
%    \cs{@@_extract_and_handle_marks:nn} because we may have to call
%    it recursively if we find forced breaks.
%    \begin{macrocode}
\cs_new_protected:Npn \@@_prepare_and_extract:nn #1#2 {
%    \end{macrocode}
%
%    To handle the \cs{enlargethispage} case we do an \tn{unskip} to
%    get rid of any glue that is present at the very end of the material
%    and also check if we have then a \tn{vbox} as the last item and
%    if so unpack that too, but only under certain conditions, see
%    below. All this is temporary done in a group, just for getting
%    the marks out, so it doesn't affect the final page production.
%
%    \begin{macrocode}
  \vbox_set:Nn \l_@@_box
    {
      #2
      \tex_unskip:D
      \box_set_to_last:N \l_@@_box
%    \end{macrocode}
%    After having removed the last box from the current list (if there
%    was one) we check whether the vertical list is now empty. If not,
%    then the last box is definitely not the one from
%    \tn{enlargethispage} and so we can, and should, leave it
%    alone. Otherwise we check if this last box is a \tn{vbox}.
% \changes{v1.0d}{2022/06/01}{Extend the logic for detecting the marks
%    in the box (gh/836)}
%    \begin{macrocode}
      \int_compare:nNnT \tex_lastnodetype:D < 0
        {
          \box_if_vertical:NT \l_@@_box
%    \end{macrocode}
%    If it is, we unpack the box.
%    \begin{macrocode}
            { \vbox_unpack:N \l_@@_box }
        }
%    \end{macrocode}
%    If it wasn't a vbox, it was either an hbox or there was no box.
%    Given that we are only interested in the marks we don't need put
%    it back in that case.
%    \begin{macrocode}
    }
%    \end{macrocode}
%    We are now ready to \cs{vsplit} the box to get at the marks. If
%    the box contains some infinite negative glue the \TeX{} will
%    produce an error complaining about it but it will correctly find
%    the split marks. Given that we can't prevent that error, we
%    hide it from the user and ensure that \TeX{} doesn't stop. The
%    error message still shows in the log, but even that is mitigated
%    as best as possible---see the definition of
%    \cs{@@_vbox_set_split_to_maxdimen:NN} for the tricks employed.
%    \begin{macrocode}
    \@@_vbox_set_split_to_maxdimen:NN \l_@@_ii_box \l_@@_box 
%    \end{macrocode}
%    After splitting we check if there is anything left in
%    \cs{l_@@_box}. If not then the above split has set some split marks
%    that we can then use to finish the extraction:
%    \begin{macrocode}
    \box_if_empty:NTF \l_@@_box
      { #1 }
%    \end{macrocode}
%    If we have a remainder after the split then this means that there
%    was some forced break in the material. We get rid of that by
%    combining the content of the two boxes and restart.
%    \begin{macrocode}
      {
%<*trace>
        \@@_debug:n { \iow_term:x
          { Marks:~ mark~ extraction~needs~ recursion~
            \msg_line_context: } }
%</trace>
        \@@_prepare_and_extract:nn {#1}
           { \vbox_unpack:N \l_@@_ii_box
             \vbox_unpack:N \l_@@_box    }
      }
}
%    \end{macrocode}
%  \end{macro}
%
%
%  \begin{macro}{\@@_vbox_set_split_to_maxdimen:NN}
%
%    Split a box to get at its marks without pausing even if \TeX{} is
%    producing an error message because of infinite negative glue in
%    the box. If there is such an error we ensure that it only shows
%    up in the log but not on the terminal.
%
%    The nice low-level hack by DPC records in the \texttt{.log} that a glue
%    shrinkage error is harmless.
%
%    We disguise \cs{c_max_dim} in an odd looking csname, which then
%    shows up as part of the display of an error message if that error
%    happens. This csname forms part of the error display so what
%    you get is something like
%\begin{verbatim}
%   ! Infinite glue shrinkage found in box being split.
%   <argument> Infinite shrink error above ignored ! 
%   l. ...  }
%\end{verbatim}
%    which hopefully makes it clear that the error is harmless and
%    and should be ignored by the reader of the \texttt{.log}.
%    \begin{macrocode}
\cs_set_eq:cN {Infinite~shrink~error~above~ignored~!}\c_max_dim
%    \end{macrocode}
%
%    The whole definition of \cs{@@_vbox_set_split_to_maxdimen:NN}
%    below is fully expanded, so we have to use a lot of
%    \cs{exp_not:N} commands to prevent expansion where necessary.
%    \begin{macrocode}
\cs_new_protected:Npx \@@_vbox_set_split_to_maxdimen:NN #1#2 {
%    \end{macrocode}
%    We start by saving the current interaction and escape char settings.
%    \begin{macrocode}
  \tl_set:Ne \exp_not:N \l_@@_saved_parameters_tl
     {
       \tex_interactionmode:D
          \exp_not:N \int_use:N \tex_interactionmode:D \scan_stop:
       \tex_escapechar:D
          \exp_not:N \int_use:N \tex_escapechar:D \scan_stop:
     }
%    \end{macrocode}
%    Then we change them so that no escape char is printed in the
%    error message (accounts for the missing backslash in front of
%    \verb/Infinite shrink .../) and we set the interaction to
%    \cs{nonstopmode} so that the error (if any) just goes into
%    the \texttt{.log} file and \TeX{} doesn't stop at that point.
%    \begin{macrocode}
  \tex_escapechar:D     -1 \scan_stop:
  \tex_interactionmode:D 0 \scan_stop:
%    \end{macrocode}
%    Then we do the splitting of the box to \cs{c_max_dim} to get at
%    the marks. This may generate the error we are worried about,
%    i.e., if the box contains infinite negative glue. However, \TeX{}
%    makes this glue finite and continues, which means we get our split
%    marks which is really all we care about.
%    \begin{macrocode}
  \tex_setbox:D #1 \tex_vsplit:D #2 to
%    \end{macrocode}
%    The \cs{use:n} may seem pointless, and it is to some extent, but
%    we need it to get our  disguised \cs{c_max_dim} displayed
%    properly as part of the error message if there is one. Without
%    it, the display would show only part of what we want it to show
%    (try it).
%    \begin{macrocode}
      \exp_not:N \use:n {
        \use:c{Infinite~shrink~error~above~ignored~!}
      }
%    \end{macrocode}
%    Finally, we change the escape char and the interaction mode back
%    to what it was before:
%    \begin{macrocode}
  \exp_not:N \l_@@_saved_parameters_tl
}
%    \end{macrocode}
%  \end{macro}
%
%
%
%  \begin{macro}{\l_@@_saved_parameters_tl}
%    The temporary variable used for resetting escape char and
%    interaction mode.
%    \begin{macrocode}
\tl_new:N \l_@@_saved_parameters_tl
%    \end{macrocode}
%  \end{macro}
%
%  \begin{macro}{\mark_update_structure_from_material:nn}
%
% \changes{v1.0e}{2024/01/29}{Macro renamed}
%
%    This function updates the mark structures of a region. The first
%    argument is the region to update and second argument receives the
%    material that holds the marks. Out of this material we extract
%    the first and last marks for all classes (if there are any) to do
%    the assignments.
%
%    \begin{macrocode}
\cs_new_protected:Npn \mark_update_structure_from_material:nn #1#2 {
  \@@_extract_and_handle_marks:nn
%    \end{macrocode}
%    
%    Once the marks can be extracted we update the structure from the
%    split marks (code in \cs{@@_update_structure_from_splitmarks:n}).
%    \begin{macrocode}
     { \@@_update_structure_from_splitmarks:n {#1} }
     { #2 }
}
%    \end{macrocode}
%  \end{macro}
%
%
%
%  \begin{macro}{\@@_update_structure_from_splitmarks:n}
%    This macro is called after we have done a \cs{tex_vsplit:D}
%    operation and the mark data is in the split marks.
%    \begin{macrocode}
\cs_new_protected:Npn \@@_update_structure_from_splitmarks:n #1 {
%    \end{macrocode}
%    
%    The first thing we do is to copy the current region structure to
%    \texttt{previous-...}; this leaves the current structure
%    untouched so we can update it class by class (as is necessary).
%    \begin{macrocode}
  \mark_copy_structure:nn { previous-#1 } {#1}
%    \end{macrocode}
%    After this action we can get first and last marks of the various
%    classes through \cs{tex_splitfirstmarks:D} and
%    \cs{tex_splitbotmarks:D}. So now we loop over all classes stored in
%    \cs{g_@@_classes_seq}.
%    \begin{macrocode}
  \seq_map_inline:Nn \g_@@_classes_seq
            {
%    \end{macrocode}
%    First action: get the last mark from the previous region, i.e.,
%    \verb=previous-#1=.  But because it is also still inside \verb=#1=,
%    at the moment we use that to construct the name because this is a
%    tiny bit faster. Given that we
%    need this value in various assignments we store it away which
%    avoids unnecessary further csname generations.
%    \begin{macrocode}
              \tl_gset_eq:Nc \g_@@_new_top_tl { g_@@_#1_last_##1_tl }
%    \end{macrocode}
%    This will first of all become the new top mark for the current class.
%    \begin{macrocode}
              \tl_gset_eq:cN { g_@@_#1_top_##1_tl } \g_@@_new_top_tl
%    \end{macrocode}
%    Next action is to get ourselves the new last mark from the
%    material supplied.
%    \begin{macrocode}
              \tl_gset:No \g_@@_tmp_tl
                { \tex_splitbotmarks:D \use:c { c_@@_class_##1_mark } }
%    \end{macrocode}
%    If this mark doesn't exist then obviously neither does the first mark,
%    so both become the last mark from the previous region. We
%    have to be a little careful here: something like
%    \verb=\mark_insert:nn{foo}{}= adds an \enquote{empty} mark that should
%    not be confused with no mark at all. But no mark in our material
%    will result in \cs{g_@@_tmp_tl} being fully empty. This is why we
%    have to make sure that \enquote{empty} from \cs{mark_insert:nn} only
%    appears to be empty when typeset but fails the next test (see below how this
%    is done).
%    \begin{macrocode}
              \tl_if_empty:NTF \g_@@_tmp_tl
                {
                  \tl_gset_eq:cN { g_@@_#1_last_ ##1_tl }
                    \g_@@_new_top_tl
                  \tl_gset_eq:cN { g_@@_#1_first_##1_tl }
                    \g_@@_new_top_tl
                }
%    \end{macrocode}
%    If it wasn't empty, i.e., if it had a real value then we use this
%    value for our new last mark instead.
%    \begin{macrocode}
                {
                  \tl_gset_eq:cN { g_@@_#1_last_##1_tl } \g_@@_tmp_tl
%    \end{macrocode}
%    Because we had a last mark we also have a first mark (which
%    might be the same, but might be not), so we pick that up and
%    assign it to the appropriate token list. This explains why we first
%    checked for the last mark because that makes the processing
%    faster in case there is none.
%    \begin{macrocode}
                  \tl_gset:co { g_@@_#1_first_##1_tl }
                    {
                      \tex_splitfirstmarks:D
                        \use:c { c_@@_class_##1_mark }
                    }
                }
            }
}
%    \end{macrocode}
%  \end{macro}
%
%
%
%  \begin{macro}{\mark_get_marks_for_reinsertion:nNN}
%
%    This function extracts the marks from the material in the first
%    argument but it does not update any the mark structures. Instead,
%    it collects the marks in the token lists given as the second and
%    third argument, in such a way that they can be reinserted by just
%    executing the token lists.\footnote{It is probably enough to
%    collect everything in a single token list as long as we put the
%    first marks first and the last marks last). But for extra
%    flexibility, I currently use 2 token lists. This might change when it is
%    really clear that this is never needed.}
%
%    \begin{macrocode}
\cs_new_protected:Npn \mark_get_marks_for_reinsertion:nNN #1#2#3 {
%    \end{macrocode}
%    First we clear the temporary token lists as we haven't seen any marks yet.
%    \begin{macrocode}
  \tl_gclear:N \g_@@_first_marks_tl
  \tl_gclear:N \g_@@_last_marks_tl
%    \end{macrocode}
%    Then we  extract all top-level marks, thereby filling the token lists
%    with suitable \cs{mark_insert:nn} calls.
%    \begin{macrocode}
  \@@_extract_and_handle_marks:nn
%    \end{macrocode}
%    The first argument holds the code used for filling the token lists and
%    the second holds the material from which all marks should be extracted.
%    \begin{macrocode}
     \@@_get_from_splitmarks:
     { #1 }
%    \end{macrocode}
%    
%    Finally, we copy the updated (or not updated) temporary token
%    lists to the two that have been supplied when the function was
%    called.  By convention \enquote{get} operations return their
%    values in local variables and \cs{@@_extract_and_handle_marks:nn}
%    runs in a group, which is why we have to use global temporary
%    variables for collecting.
%    \begin{macrocode}
  \tl_set_eq:NN #2 \g_@@_first_marks_tl
  \tl_set_eq:NN #3 \g_@@_last_marks_tl
}
%    \end{macrocode}
%  \end{macro}
%
%
%  \begin{macro}{\@@_get_from_splitmarks:}
%    This function is called after we have done a \cs{vsplit} to
%    update the split marks. It loops through all mark classes to find
%    out if there are marks for this class and if so updates the
%    global tls used for collecting.
%    \begin{macrocode}
\cs_new_protected:Npn  \@@_get_from_splitmarks:  {
  \seq_map_inline:Nn \g_@@_classes_seq
            {
%    \end{macrocode}
%    First we to get the last mark for the current class from the
%    material supplied.
%    \begin{macrocode}
              \tl_gset:No \g_@@_tmp_tl
                { \tex_splitbotmarks:D \use:c { c_@@_class_##1_mark } }
%    \end{macrocode}
%    
%    If this mark doesn't exist then obviously first mark doesn't
%    either, so we do nothing (other than issuing some debugging
%    info).

%    We have to be a little careful here: something like
%    \verb=\mark_insert:nn{foo}{}= adds an \enquote{empty} mark that
%    we should not confuse with the case where there is no mark at
%    all.
%
%    When there is no mark at all we get a truly empty
%    \cs{g_@@_tmp_tl} as a result. This is why we have to make sure
%    that an \enquote{empty} mark generated with \cs{mark_insert:nn}
%    only appears to be empty when it is typeset, but fails the next
%    test (see below how this is done).
%    \begin{macrocode}
              \tl_if_empty:NTF \g_@@_tmp_tl
                {
%<*trace>
                  \@@_debug:n { \iow_term:x { Marks:~no~ marks~
                      for~ class~ '##1'~\msg_line_context: } }
%</trace>
                }
%    \end{macrocode}
%    
%    If it wasn't empty, i.e., if it had a real value then we use this
%    value for our new last mark instead. This means we put an
%    appropriate \cs{mark_insert:nn} statement into
%    \cs{g_@@_last_marks_tl}.
%    \begin{macrocode}
                {
%<*trace>
                  \@@_debug:n { \iow_term:x { Marks:~ extract~ last~
                       mark~ for~ class~ '##1'~ =~  \g_@@_tmp_tl } }
%</trace>
                  \tl_gput_right:Ne \g_@@_last_marks_tl
                     { \mark_insert:nn {##1} { \@@_drop_id:o { \g_@@_tmp_tl } } }
%    \end{macrocode}
%    Because we had a last mark we also have a first mark (which might
%    be the same, but might not be), so we pick that up and add it to
%    the \cs{g_@@_first_marks_tl} token list. This explains why we
%    first checked for the last mark because that makes the processing
%    faster in case there is none.
%    \begin{macrocode}
%<*trace>
                  \@@_debug:n { \iow_term:x { Marks:~ extract~ first~
                      mark~ for~ class~ '##1'~ =~
                      \tex_splitfirstmarks:D
                        \use:c { c_@@_class_##1_mark }
                  } }
%</trace>
                  \tl_gput_right:Ne \g_@@_first_marks_tl
                     { \mark_insert:nn {##1}
                       {
%    \end{macrocode}
%    We better drop the id from the returned value otherwise they
%    will accumulate in the marks when reinserted.
%    \begin{macrocode}
                          \@@_drop_id:o {
                            \tex_splitfirstmarks:D
                            \use:c { c_@@_class_##1_mark }
                          }
                       }
                     }
                }
            }
}
%    \end{macrocode}
%  \end{macro}
%
%
%  \begin{macro}{\g_@@_first_marks_tl,\g_@@_last_marks_tl}
%    These are two global temporary variables used in the code above.
%    \begin{macrocode}
\tl_new:N  \g_@@_first_marks_tl
\tl_new:N  \g_@@_last_marks_tl
%    \end{macrocode}
%  \end{macro}
%
%
%  \begin{macro}{\mark_copy_structure:nn}
%    This function copies the structure for one region to  another,
%    e.g., from \texttt{page} to \texttt{previous-page} above,
%    or later from \texttt{column} to \texttt{first-column}, etc.
% \changes{v1.1a}{2024/11/09}{Macro renamed}
%    \begin{macrocode}
\cs_new_protected:Npn \mark_copy_structure:nn #1#2 {
%    \end{macrocode}
%    This requires a simple loop through all mark classes copying the
%    token list from one name to the next.
%    \begin{macrocode}
  \seq_map_inline:Nn \g_@@_classes_seq
    {
      \tl_gset_eq:cc { g_@@_ #1 _top_   ##1 _tl }
                     { g_@@_ #2 _top_   ##1 _tl }
      \tl_gset_eq:cc { g_@@_ #1 _first_ ##1 _tl }
                     { g_@@_ #2 _first_ ##1 _tl }
      \tl_gset_eq:cc { g_@@_ #1 _last_  ##1 _tl }
                     { g_@@_ #2 _last_  ##1 _tl }
    }
}
%    \end{macrocode}
%  \end{macro}
%
%
%  \begin{macro}{\mark_clear_structure:n}
%    This function sets the structure of one region back to an initial
%   state, so that all classes return an empty value if queried.
% \changes{v1.1a}{2024/11/09}{}
%    \begin{macrocode}
\cs_new_protected:Npn \mark_clear_structure:n #1 {
%    \end{macrocode}
%    This requires a simple loop through all mark classes.
%    \begin{macrocode}
  \seq_map_inline:Nn \g_@@_classes_seq
    {
      \tl_gset_eq:cN { g_@@_ #1 _top_   ##1 _tl }
                     \c_@@_empty_tl
      \tl_gset_eq:cN { g_@@_ #1 _first_ ##1 _tl }
                     \c_@@_empty_tl
      \tl_gset_eq:cN { g_@@_ #1 _last_  ##1 _tl }
                     \c_@@_empty_tl
    }
}
%    \end{macrocode}
%  \end{macro}
%
%
%
%  \begin{macro}{\mark_set_structure_to_err:n,\@@_error:n}
%    A slight variation is to install a fixed error message as the
%    value.
%    \begin{macrocode}
\cs_new_protected:Npn \mark_set_structure_to_err:n #1 {
  \seq_map_inline:Nn \g_@@_classes_seq
      {
        \tl_gset:ce { g_@@_ #1 _top_   ##1 _tl } { \@@_value:nn{?}{\@@_error:nn {#1}{?} }}
        \tl_gset:ce { g_@@_ #1 _first_ ##1 _tl } { \@@_value:nn{?}{\@@_error:nn {#1}{?} }}
        \tl_gset:ce { g_@@_ #1 _last_  ##1 _tl } { \@@_value:nn{?}{\@@_error:nn {#1}{?} }}
      }
}
%    \end{macrocode}
%    Given that this is used in only one place, we could hardwire the
%    argument which would be a bit more compact, but who knows,
%    perhaps we end up with another reason to use this error command
%    elsewhere, so for now we keep the argument.
%
%    \begin{macrocode}
\cs_new_protected:Npn \@@_error:nn #1#2 {
  \msg_error:nnnn { mark } { invalid-use } {#1} {#2}
}
%    \end{macrocode}
%  \end{macro}
%
%
%
%
% \subsection{Placing and retrieving marks}
%
%
%
%  \begin{macro}{\mark_insert:nn}
%    This function puts a mark for some \meta{class} at the current point.
%    \begin{macrocode}
\cs_new_protected:Npn \mark_insert:nn #1#2
{
  \seq_if_in:NnTF \g_@@_classes_seq {#1}
      {
%    \end{macrocode}
%    We need to pass the evaluated argument into the mark but protected
%    commands should not expand including those protected using the \cs{protect} approach of
%    \LaTeXe{}. We also  disable \cs{label} and the
%    like.\footnote{Straight copy from \texttt{latex.ltx} but is this
%    even correct? At least a label in a running header makes little
%    sense if it get set several times! Maybe that needs looking at in
%    the 2e kernel.}
%
%    At this point the code eventually should get a public
%    (and a kernel) hook instead of a set of hardwired settings.
%    \begin{macrocode}
        \group_begin:
%    \end{macrocode}
%    Within the group we alter some comments, e.g, \cs{label} or
%    \cs{index}, to do the right at this point. This is done in the
%    kernel hook \cs{@kernel@before@insertmark} which is followed by
%    the public hook \hook{insertmark} that can be used by packages to
%    augment or alter that setup as necessary.
%    \begin{macrocode}
          \@kernel@before@insertmark
          \hook_use:n { insertmark }
          \unrestored@protected@xdef \g_@@_tmp_tl
               {
%    \end{macrocode}
%    To ensure that marks are unique we insert a hidden sequence
%    marker at the beginning of the content of the mark containing the
%    sequence number of the mark.
% \changes{v1.0f}{2024/05/30}{Use sequence marker to make all marks
%    unique on nearby regions (gh/1359)}
%    \begin{macrocode}
                 \@@_value:nn{ \int_use:N\g_@@_int }{#2}
               }
%<*trace>
          \@@_debug:n{ \iow_term:x { Marks:~ set~#1~<-~
              '\tl_to_str:V \g_@@_tmp_tl' ~ \msg_line_context: } }
%</trace>
          \tex_marks:D \use:c { c_@@_class_ #1 _mark }
            {
%    \end{macrocode}
%    Here is the trick to avoid truly empty marks: if the result from
%    the above processing is empty we add something which eventually
%    becomes empty, but not immediately; otherwise we just put
%    \cs{g_@@_tmp_tl} in.
%    \begin{macrocode}
% This is no longer needed with 1.0f
%              \tl_if_empty:NTF \g_@@_tmp_tl
%                { \exp_not:n { \prg_do_nothing: } }
%                { \exp_not:o { \g_@@_tmp_tl } }
              \exp_not:o { \g_@@_tmp_tl }
            }
        \group_end:
%    \end{macrocode}
%    A mark introduces a possible break point and in certain
%    situations that should not happen in vertical mode in \LaTeX{}.
%    This may need some checking and possibly cleanup \ldots.
%    \begin{macrocode}
        \if@nobreak\ifvmode\nobreak\fi\fi
      }
%    \end{macrocode}
%    If the mark class was not known, raise an error.
%    \begin{macrocode}
      {
        \msg_error:nnx { mark } { unknown-class }
          { \tl_to_str:n {#1} }
      }
}
%    \end{macrocode}
%  \end{macro}
%
%
%  \begin{macro}{\@@_value:nn}
%    A hidden marker is placed into every mark added by
%    \cs{mark_insert:nn}. It will not show up in the output but
%    its argument (a counter value that is incremented) makes all
%    marks unique so the test for \enquote{equal} is not fooled by two
%    different marks having the same mark text.
% \changes{v1.0f}{2024/05/30}{Use sequence marker to make all marks
%    unique on nearby regions (gh/1359)}
%    \begin{macrocode}
\cs_new_protected:Npn \@@_value:nn #1#2 { #2 }
%    \end{macrocode}
%  \end{macro}
%
%
% \begin{macro}[int]{\@kernel@before@insertmark}
% \begin{macro}{insertmark}
%    By default \cs{label}, \cs{index}, and \cs{glossary} do nothing
%    when the mark is inserted.
%    \begin{macrocode}
\int_new:N \g_@@_int
\cs_new:Npn \@kernel@before@insertmark {
          \cs_set_eq:NN \label    \scan_stop:
          \cs_set_eq:NN \index    \scan_stop:
          \cs_set_eq:NN \glossary \scan_stop:
%    \end{macrocode}
%    We count each mark and use that to place a hidden marker in front
%    of the mark text. To ensure that there is no overflow (very
%    unlikely but you never know) we restart every 100000 marks. Thus,
%    if somebody puts more than that number of marks on a single page
%    you could construct a scenario in which that approach fails.
% \changes{v1.0f}{2024/05/30}{Use sequence marker to make all marks
%    unique on nearby regions (gh/1359)}
%    \begin{macrocode}
          \int_compare:nNnTF \g_@@_int < {99999}
              { \int_gincr:N \g_@@_int }
              { \int_gzero:N \g_@@_int }
                
}
%    \end{macrocode}
%    The public hook to augment the setup.
%    \begin{macrocode}
\hook_new:n {insertmark}
%    \end{macrocode}
% \end{macro}
% \end{macro}
%
% \begin{macro}[EXP]{\mark_use_top:nn, \mark_use_first:nn, \mark_use_last:nn}
%
%    To retrieve the first, last or top region mark, we grab the
%    appropriate value stored in the corresponding token list variable
%    and pass its contents back. These functions should be used only
%    in output routines and only after \cs{mark_update_structure_from_material:nn} has acted,
%    otherwise their value will be wrong.
%
%    \begin{macrocode}
\cs_new:Npn \mark_use_first:nn #1#2 { \@@_use_check:nnn { g_@@_#1_first_#2_tl } {#1} {#2} }
\cs_new:Npn \mark_use_last:nn #1#2 { \@@_use_check:nnn { g_@@_#1_last_#2_tl } {#1} {#2} }
\cs_new:Npn \mark_use_top:nn #1#2 { \@@_use_check:nnn { g_@@_#1_top_#2_tl } {#1} {#2} }
%    \end{macrocode}
%
%    If used with an unknown class or region these commands will generate an error.
%    If that happens in an expandable context then the error
%    generation is delayed (e.g., if used in a \cs{section}) and
%    happens when the code is finally used in typesetting, e.g., in the
%    TOC or a running header. 
%    If used in a \cs{typeout} you only see something like
%    \verb=\__mark_error:n{page}=.
%    This is not too good, but probably better than low-level errors, I
%    guess, and I don't want to use an expandable error because of the
%    size restrictions in such error messages.
% \changes{v1.1a}{2024/11/09}{}
%    \begin{macrocode}
\cs_new:Npn \@@_use_check:nnn #1#2#3 {
  \tl_if_eq:cNTF {#1} \relax
     { \@@_error:nn {#2} {#3} }
     { \@@_drop_id:v {#1} }
}     
%    \end{macrocode}
%    
%    Each mark starts with an id and while the id does not print it is
%    nevertheless better to remove it when returning the mark, so that
%    downstream manipulation of the data doesn't have to deal with it.
% \changes{v1.0g}{2024/05/31}{Remove the id when returning the mark value (gh/1359)}
%    This is what the \cs{exp_not:o} accomplishes.
%    \begin{macrocode}
\cs_new:Npn \@@_drop_id:n #1 { \exp_not:o { #1 } }
\cs_generate_variant:Nn \@@_drop_id:n { o, v }
%    \end{macrocode}
% \end{macro}
%
%
%
%
% \subsection{Comparing mark values}
%
%
%
%  \begin{macro}[TF,EXP]{\mark_if_eq:nnnn,\mark_if_eq:nnnnnn}
%    Test if in a given region (\verb=#1=) for a given class
%   (\verb=#2=) the marks in position \verb=#3= and \verb=#4= (top,
%   first, or last) are identical
%    \begin{macrocode}
\prg_new_conditional:Npnn \mark_if_eq:nnnn #1#2#3#4 { T , F , TF }
{
  \tl_if_eq:ccTF { g_@@_ #1 _#3_ #2 _tl }
                 { g_@@_ #1 _#4_ #2 _tl }
                 \prg_return_true:
                 \prg_return_false:
}
%    \end{macrocode}
%    The fully general test (with two triplets of the form
%    \meta{region}, \meta{class}, and \meta{position}) is this:
%    \begin{macrocode}
\prg_new_conditional:Npnn \mark_if_eq:nnnnnn #1#2#3#4#5#6 { T , F , TF }
{
  \tl_if_eq:ccTF { g_@@_ #1 _#3_ #2 _tl }
                 { g_@@_ #4 _#6_ #5 _tl }
                 \prg_return_true:
                 \prg_return_false:
}
%    \end{macrocode}
%  \end{macro}
%
%
%
% \subsection{Messages}
%
%    Mark errors are \LaTeX{} kernel errors:
% \changes{v1.0d}{2022/06/01}{Marks are kernel errors}
%    \begin{macrocode}
\prop_gput:Nnn \g_msg_module_type_prop { mark } { LaTeX }
%    \end{macrocode}
%
%    \begin{macrocode}
\msg_new:nnnn { mark } { class-already-defined }
  { Mark~class~'#1'~already~defined }
  {
    \c__msg_coding_error_text_tl
    LaTeX~was~asked~to~define~a~new~mark~class~called~'#1':~
    this~mark~class~already~exists.
    \c__msg_return_text_tl
  }
%    \end{macrocode}
%
%    \begin{macrocode}
\msg_new:nnnn { mark } { unknown-class }
  { Unknown~mark~class~'#1'. }
  {
    \c__msg_coding_error_text_tl
    LaTeX~was~asked~to~manipulate~a~mark~of~class~'#1',~
    but~this~class~of~marks~does~not~exist.
  }
%    \end{macrocode}
%
%    The next error can also happen if the mark class is unknown, so this
%    should perhaps be separated into two different errors.
%    \begin{macrocode}
\msg_new:nnnn { mark } { invalid-use }
  { Mark~region~'#1'~not~usable~or~class~'#2'~unknown }
  {
    \c__msg_coding_error_text_tl
    The~region~'#1'~is~either~not~known~or~data~for~it~
    still~needs~to~be~assembled,~e.g.,~last-column~
    while~building~the~first-column.~
    Also~possible:~the~class~namne~'#2'~is~misspelled.
    \c__msg_return_text_tl
  }
%    \end{macrocode}
%
%
%
%
%
% \subsection{Debugging the mark structures}
%
%  Code and commands in this section are not final, it needs more
%  experimentation to see what kind of tracing information is going to
%  be useful in practice. For now the tracing is mainly meant to be used
%  for code testing and not so much for application testing.
%
% It is quite likely that the
% commands and the behavior of the tracing might change in the
% future once we gained some experience with it.
%
%  \begin{macro}{\g_@@_debug_bool}
%    Holds the current debugging state.
%    \begin{macrocode}
\bool_new:N \g_@@_debug_bool
%    \end{macrocode}
%  \end{macro}
%
%  \begin{macro}{\mark_debug_on:,\mark_debug_off:}
%  \begin{macro}{\@@_debug:n}
%  \begin{macro}{\@@_debug_gset:}
%    Turns debugging on and off by redefining \cs{@@_debug:n}.
%    \begin{macrocode}
\cs_new_eq:NN \@@_debug:n \use_none:n
\cs_new_protected:Npn \mark_debug_on:
  {
    \bool_gset_true:N \g_@@_debug_bool
    \@@_debug_gset:
  }
\cs_new_protected:Npn \mark_debug_off:
  {
    \bool_gset_false:N \g_@@_debug_bool
    \@@_debug_gset:
  }
\cs_new_protected:Npn \@@_debug_gset:
  {
    \cs_gset_protected:Npx \@@_debug:n ##1
      { \bool_if:NT \g_@@_debug_bool {##1} }
  }
%    \end{macrocode}
%  \end{macro}
%  \end{macro}
%  \end{macro}
%
%
%
%  \begin{macro}{\DebugMarksOn,\DebugMarksOff}
%    CamelCase commands for debugging.
%    \begin{macrocode}
\cs_new_eq:NN \DebugMarksOn   \mark_debug_on:
\cs_new_eq:NN \DebugMarksOff  \mark_debug_off:
%    \end{macrocode}
%  \end{macro}
%
%
%
%  \begin{macro}{\@@_class_status:nnn}
%    Shows the mark values across all regions for one mark class
%    (\verb=#2=). 
%
%    The first argument gives some \meta{info} to help
%    in identifying where the command was called, the second is
%    the class and the third holds the number of \texttt{mcol-...} we
%    should display: inside a \env{multicols} environment this will be \cs{col@number},
%    in \LaTeX{}'s normal output routines it will be \texttt{0}. 
% \changes{v1.1a}{2024/11/09}{Add a third argument}
%    \begin{macrocode}
%<*trace>
\cs_new_protected:Npn \@@_class_status:nnn #1#2#3 {
   \typeout{ Marks:~#2~ #1:}
   \@@_region_status:nnn  {#2}{ page~ (previous) }  { previous-page }
   \@@_region_status:nnn  {#2}{ page~ (current)~ }  { page }
   \@@_region_status:nnn  {#2}{ column~ (previous) }{ previous-column }
   \@@_region_status:nnn  {#2}{ column~ (current)~ }{ column }
   \@@_region_status:nnn  {#2}{ column~ (first) }   { first-column }
   \@@_region_status:nnn  {#2}{ column~ (last)~ }   { last-column }
%    \end{macrocode}
%    Then finish  by displaying a subset of the \texttt{mcol-...} regions: none (\texttt{0})
%    in the standard \LaTeX{} output routine and \cs{col@number}
%    within a \env{multicols} environment.
%    \begin{macrocode}
   \int_step_inline:nn {#3}
     {
       \@@_region_status:nnn  {#2}{ column~ (##1)~ }   { mcol-##1 }
     }
}
%    \end{macrocode}
%  \end{macro}
%
%
%  \begin{macro}{\@@_region_status:nnn}
%    
%    Display the top, first, and last mark of a region unless none of
%    them exist or all of them are empty.
%    \begin{macrocode}
\cs_new_protected:Npn \@@_region_status:nnn #1#2#3 {
  \group_begin:
  \cs_set:Npn \@@_value:nn ##1##2{ \exp_not:n{ {##1} ~ ##2 } }
  \tl_if_exist:cT  { g_@@_#3_last_ #1 _tl }   
     {
       \tl_if_eq:cNF { g_@@_#3_last_ #1 _tl } \c_@@_empty_tl
          {
            \typeout{\@spaces #2 =
               ~|~ \use:c { g_@@_#3_top_ #1 _tl }    ~|~ 
                   \use:c { g_@@_#3_first_ #1 _tl }  ~|~
                   \use:c { g_@@_#3_last_ #1 _tl }   ~|
            }
          }
     }
 \group_end:
} 
%    \end{macrocode}
%  \end{macro}
%
%
%  \begin{macro}{\@@_status:nn}
%    Show a snapshot of all mark class values across all regions. The
%    first argument is a string to identify the output, the second
%    argument is the number of \texttt{mcol-...} regions to show. Outside
%    of a \env{multicols} environment this is normally set to 0.
% \changes{v1.1a}{2024/11/09}{Add a second argument}
%    \begin{macrocode}
\cs_new_protected:Npn \@@_status:nn #1#2
  {
    \seq_map_inline:Nn \g_@@_classes_seq
      { \@@_class_status:nnn {#1} {##1} {#2} }
  }
%</trace>
%    \end{macrocode}
%  \end{macro}
%
%
%  \begin{macro}{\ShowMarksAt}
% \changes{v1.0e}{2024/01/29}{Macro added}
%    Debugging helper that displays a snapshot of all known mark
%    structures. The first argument is a text string that is
%    displayed to help identifying when the snapshot was made. The
%    optional second one determines how many \texttt{mcol-...} regions
%    are displayed (by default 4).
%
%    This may not stay like this (or at all), which is why it isn't
%    yet documented as an official command.
%    \begin{macrocode}
\NewDocumentCommand \ShowMarksAt {m O{4} } {
%<*trace>
     \@@_debug:n { \@@_status:nn {#1}{#2} }
%</trace>
}         
%    \end{macrocode}
%  \end{macro}
%
%
% \subsection{Designer-level interfaces}
%
%
% \begin{macro}{\NewMarkClass,\InsertMark}
%    These two are identical to the L3 programming layer commands.
%    \begin{macrocode}
\cs_new_eq:NN  \NewMarkClass \mark_new_class:n
\@onlypreamble \NewMarkClass
%    \end{macrocode}
%
%    \begin{macrocode}
\cs_new_eq:NN \InsertMark  \mark_insert:nn
%    \end{macrocode}
% \end{macro}
%
%
% \begin{macro}[EXP]{\TopMark, \FirstMark, \LastMark}
%    The following commands take an optional argument that defaults to
%    page. There is no checking that the region is actually valid. If
%    not there is simply an empty return.
%    \begin{macrocode}
\NewExpandableDocumentCommand \FirstMark { O{page} m }
                    { \mark_use_first:nn {#1}{#2} }
%    \end{macrocode}
%
%    \begin{macrocode}
\NewExpandableDocumentCommand \LastMark { O{page} m }
                    { \mark_use_last:nn {#1}{#2} }
%    \end{macrocode}
%
%    \begin{macrocode}
\NewExpandableDocumentCommand \TopMark { O{page} m }
                    { \mark_use_top:nn {#1}{#2} }
%    \end{macrocode}
% \end{macro}
%
%
%
%  \begin{macro}[EXP]{\IfMarksEqualTF,\IfMarksEqualT,\IfMarksEqualF}
%  \changes{v1.0h}{2024/10/21}{Define \cs{IfMarksEqualT}, \cs{IfMarksEqualF}}
%    We only provide CamelCase commands for the case with one region
%    (optional) and one class. One could think of also providing a
%    version for the general case with several optional arguments, but
%    use cases for this are most likely rare, so not done yet.
%    \begin{macrocode}
\NewExpandableDocumentCommand \IfMarksEqualTF {O{page}mmm} {
  \mark_if_eq:nnnnTF {#1}{#2}{#3}{#4}
}
\NewExpandableDocumentCommand \IfMarksEqualT {O{page}mmm} {
  \mark_if_eq:nnnnT {#1}{#2}{#3}{#4}
}
\NewExpandableDocumentCommand \IfMarksEqualF {O{page}mmm} {
  \mark_if_eq:nnnnF {#1}{#2}{#3}{#4}
}
%    \end{macrocode}
%  \end{macro}
%
%
%
%
% \section{\LaTeXe{} integration}
%
% \subsection{Core \LaTeXe{} integration}
%
%  \begin{macro}{\@@_update_singlecol_structures:}
%    This command updates the mark structures if we are producing a
%    single column document.
%    \begin{macrocode}
\cs_new_protected:Npn \@@_update_singlecol_structures: {
%    \end{macrocode}
%    First we update the \texttt{page} region (which also updates the
%    \texttt{previous-page}.
%
%    The \cs{@outputbox} is normally in \cs{vbox} in \LaTeX{} but we
%    can't take that for granted (an \pkg{amsmath} test document
%    changed it to an \cs{hbox} just to trip me up) so we are a little
%    careful with unpack now.
%    \begin{macrocode}
  \box_if_vertical:NTF \@outputbox
      {
        \mark_update_structure_from_material:nn {page}
           { \vbox_unpack:N  \@outputbox }
      }
      {
        \mark_update_structure_from_material:nn {page}
           { \hbox_unpack:N  \@outputbox }
      }
%    \end{macrocode}
%    Then we provide the necessary updates for the aliases.
%    \begin{macrocode}
  \mark_copy_structure:nn {previous-column}{previous-page}
  \mark_copy_structure:nn {column}{page}
  \mark_copy_structure:nn {first-column}{page}
  \mark_copy_structure:nn {last-column}{page}
%<*trace>
% move this into status itself?
     \@@_debug:n
         {
           \@@_status:nn
                { in~ OR~ (
                  \legacy_if:nTF {@twoside}
                         { twoside-
                           \int_if_odd:nTF \c@page
                               { odd }{ even }
                         }
                         { oneside }
                         )
                }
                {0}
         }
%</trace>
}
%    \end{macrocode}
%  \end{macro}
%
%  \begin{macro}{\@@_update_dblcol_structures:}
%    This commands handles the updates if we are doing two-column pages.
%    \begin{macrocode}
\cs_new_protected:Npn \@@_update_dblcol_structures: {
%    \end{macrocode}
%    First we update the \texttt{column} and \texttt{previous-column}
%    regions using the material assembled in \cs{@outputbox}.
%    \begin{macrocode}
  \box_if_vertical:NTF \@outputbox
      {
        \mark_update_structure_from_material:nn {column}
           { \vbox_unpack:N  \@outputbox }
      }
      {
        \mark_update_structure_from_material:nn {column}
           { \hbox_unpack:N  \@outputbox }
      }
%    \end{macrocode}
%    How we have to update the alias regions depends on whether or not
%    \cs{@opcol} was called to process the first column or to produce
%    the completed page
%    \begin{macrocode}
  \legacy_if:nTF {@firstcolumn}
    {
%    \end{macrocode}
%    If we are processing the first column then \texttt{column} is our
%    \texttt{first-column} and there is no \texttt{last-column} yet,
%    so we make those an error.
%    \begin{macrocode}
      \mark_copy_structure:nn {first-column}{column}
      \mark_set_structure_to_err:n {last-column}
    }
    {
%    \end{macrocode}
%    If we produce the completed page then the \texttt{first-column}
%    is the same as the new \texttt{previous-column}. However, the
%    structure should already be correct if you think about it
%    (because is was set to \texttt{column} last time which is now the
%    \texttt{previous-column}), thus there is no need to make an update.
%    \begin{macrocode}
%     \mark_copy_structure:nn {first-column}{previous-column}
%    \end{macrocode}
%    However, we now have a proper \texttt{last-column} so we assign that.
%    \begin{macrocode}
      \mark_copy_structure:nn {last-column}{column}
%    \end{macrocode}
%    What now remains doing is to update the \texttt{page} and
%    \texttt{previous-page} regions. For this we have to copy the
%    settings in \texttt{page} into \texttt{previous-page} and then
%    update \texttt{page} such that the top and first marks are taken
%    from the \texttt{first-column} region and the last marks are
%    taken from the \texttt{last-column} region. All this has to be
%    done for all mark classes so we loop over our sequence.
%
%    Note that one loop is needed if we arrange the copy statements in
%    a suitable way.
%    \begin{macrocode}
      \seq_map_inline:Nn \g_@@_classes_seq
        {
%    \end{macrocode}
%    The \texttt{previous-page} updates need to come before the
%    updates for \texttt{page} region because otherwise the values
%    to copy are already overwritten.
%    necessary values.
%    \begin{macrocode}
          \tl_gset_eq:cc { g_@@_previous-page_top_   ##1 _tl }
                         { g_@@_page_top_            ##1 _tl }
          \tl_gset_eq:cc { g_@@_previous-page_first_ ##1 _tl }
                         { g_@@_page_first_          ##1 _tl }
          \tl_gset_eq:cc { g_@@_previous-page_last_  ##1 _tl }
                         { g_@@_page_last_           ##1 _tl }
%    \end{macrocode}
%    To update the \texttt{top} we only have to copy what is in
%    \texttt{first-column}:                      
%    \begin{macrocode}
          \tl_gset_eq:cc { g_@@_page_top_           ##1 _tl }
                         { g_@@_first-column_top_   ##1 _tl }
             
%    \end{macrocode}
%    Updating the \texttt{first} mark for the \texttt{page} region is
%    more complicated. We first have to find out of there is any mark
%    in the first column (this can be done by comparing the \texttt{top} and
%    the \texttt{first} mark of that region). 
% \changes{v1.0f}{2024/05/30}{Correct logic for first mark in page
%    region if first column contains no marks (gh/1359)}
%    \begin{macrocode}
          \tl_if_eq:ccTF { g__@@_first-column_top_   ##1 _tl }
                         { g__@@_first-column_first_ ##1 _tl }
             {
%    \end{macrocode}
%    If there is no mark in the first column we copy the first mark of
%    the last column. If that doesn't contain a mark we still get the
%    right result because the first mark is then equal to the top mark.
%    \begin{macrocode}
               \tl_gset_eq:cc { g_@@_page_first_        ##1 _tl }
                              { g_@@_last-column_first_ ##1 _tl }
             }
             {
%    \end{macrocode}
%    On the other hand, if there is a mark in the first column we copy
%    over the \texttt{first} mark from that column.
%    \begin{macrocode}
               \tl_gset_eq:cc { g_@@_page_first_         ##1 _tl }
                              { g_@@_first-column_first_ ##1 _tl }
             }
%    \end{macrocode}
%    The logic for the \texttt{last} page mark is again simple, we can
%    just copy the value in the \texttt{last} mark of the last column.
%    If that column doesn't contain any marks, then the value in
%    \texttt{last} will be automatically the same as the \texttt{last}
%    from the first column.
%    \begin{macrocode}
          \tl_gset_eq:cc { g_@@_page_last_        ##1 _tl }
                         { g_@@_last-column_last_ ##1 _tl }
        }
    }
%<*trace>
     \@@_debug:n
         {
           \@@_status:nn
              { in~ OR~ (
                \legacy_if:nTF {@twoside}
                       { twoside-
                         \int_if_odd:nTF \c@page
                             { odd }{ even }
                       }
                       { oneside }
                       \space
                       \legacy_if:nTF {@firstcolumn}
                              { first~ }{ second~ }
                              column )
              }
              {0}
         }
%</trace>
}
%    \end{macrocode}
%  \end{macro}
%
%
%
% \subsection{Other \LaTeXe{} output routines}
%
%  This section will cover support for packages that alter the
%  \LaTeX{} output routine (as necessary). The support for
%  \pkg{multicol} (for now) is handled directly in that package.
%
%
%
%
%    \begin{macrocode}
%<@@=>
%    \end{macrocode}
%


%  \begin{macro}[int]{\@expl@@@mark@update@singlecol@structures@@}
%    \begin{macrocode}
\cs_new_eq:NN  \@expl@@@mark@update@singlecol@structures@@
               \__mark_update_singlecol_structures:
%    \end{macrocode}
%  \end{macro}
%
%
%    
%  \begin{macro}[int]{\@expl@@@mark@update@dblcol@structures@@}
%    \begin{macrocode}
\cs_new_eq:NN  \@expl@@@mark@update@dblcol@structures@@
               \__mark_update_dblcol_structures:
%    \end{macrocode}
%  \end{macro}
%
%
%
%
% \subsection{Rollback information}
%
%
%    \begin{macrocode}
%<latexrelease>\IncludeInRelease{0000/00/00}{ltmarks}%
%<latexrelease>                 {Undo~Marks~handling}
%<latexrelease>
%    \end{macrocode}
%    We keep the interface commands around even if we roll back in
%    case they are used in packages that don't roll back. Not likely
%    to do a lot of good, but then there is not much we can do, but
%    this at least they won't give unknown csname errors.
%    \begin{macrocode}
%<latexrelease>\DeclareRobustCommand \NewMarkClass[1]{}
%<latexrelease>\DeclareRobustCommand \InsertMark[2]{}
%<latexrelease>\RenewExpandableDocumentCommand \FirstMark { O{} m } { }
%<latexrelease>\RenewExpandableDocumentCommand \LastMark { O{} m } { }
%<latexrelease>\RenewExpandableDocumentCommand \TopMark { O{} m } { }
%<latexrelease>\RenewExpandableDocumentCommand \IfMarksEqualTF { O{} mmm }{ }
%<latexrelease>
%    \end{macrocode}
%    Same here, this avoided extra roll back code in the OR.
%    \begin{macrocode}
%<latexrelease>\let \@expl@@@mark@update@singlecol@structures@@ \relax
%<latexrelease>\let \@expl@@@mark@update@dblcol@structures@@ \relax
%<latexrelease>
%<latexrelease>
%<latexrelease>\EndModuleRelease
%    \end{macrocode}
%
%    \begin{macrocode}
\ExplSyntaxOff
%    \end{macrocode}
%    \begin{macrocode}
%</2ekernel|latexrelease>
%    \end{macrocode}
%
%    Reset module prefix:
%    \begin{macrocode}
%<@@=>
%    \end{macrocode}
%
%
%
%
%
%%%%%%%%%%%%%%%%%%%%%%%%%%%%%%%%%%%%%%%%%%%%%
\endinput
%%%%%%%%%%%%%%%%%%%%%%%%%%%%%%%%%%%%%%%%%%%%%
%
